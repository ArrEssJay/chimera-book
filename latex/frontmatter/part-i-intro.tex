% Part I Introduction: Foundations of Electromagnetism & Waves
\newpage
\thispagestyle{empty}

\vspace*{3cm}

\begin{center}
{\Large\lorettadisplay\bfseries Part I Introduction}
\end{center}

\vspace{2cm}

The story of modern communications begins not with an invention, but with an act of profound intellectual unification. In the mid-19th century, the disparate phenomena of electricity, magnetism, and light were understood as separate forces of nature. It was James Clerk Maxwell who, through a set of four elegant equations, revealed them to be different manifestations of a single, underlying entity: the electromagnetic field. His work predicted that disturbances in this field would propagate as waves at a speed that, when calculated, was precisely the measured speed of light.

\vspace{1em}

This section of the book returns to these first principles. Before we can encode a single bit or analyse a communication link, we must first understand the fundamental nature of the medium itself. We will explore Maxwell's equations, the resulting physics of wave propagation, the properties of the electromagnetic spectrum, and the essential characteristics of a wave, such as its power and polarisation. These chapters form the physical bedrock upon which all the subsequent layers of theory and technology are built.

\vspace*{\fill}
\newpage
