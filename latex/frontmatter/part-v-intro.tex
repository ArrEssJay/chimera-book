% Part V Introduction: Advanced Coding & Equalisation
\newpage
\thispagestyle{empty}

\vspace*{3cm}

\begin{center}
{\Large\lorettadisplay\bfseries Part V Introduction}
\end{center}

\vspace{2cm}

Having quantified the numerous ways a channel can corrupt a signal, we now turn to the arsenal of techniques developed to fight back. The battle against channel impairments is a story of ingenuity in signal processing. It began with the foundational work of Richard Hamming at Bell Labs in the 1950s, who, frustrated with his relay-based computer's errors, developed the first powerful error-correcting codes. This field culminated in the 1990s with the invention of Turbo and LDPC codes, which brought practical systems to the very brink of the theoretical Shannon limit.

\vspace{1em}

This section details the powerful digital signal processing techniques that enable reliable communication over unreliable channels. We will explore the methods of channel equalisation, which seek to reverse the distorting effects of multipath, and delve deep into the theory and practice of Forward Error Correction (FEC). We will examine the classical block and convolutional codes that were the workhorses for decades, before concluding with the modern, capacity-approaching codes that power today's high-performance systems.

\vspace*{\fill}
\newpage
