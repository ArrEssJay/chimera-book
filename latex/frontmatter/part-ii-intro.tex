% Part II Introduction: Digital Baseband Signals
\newpage
\thispagestyle{empty}

\vspace*{3cm}

\begin{center}
{\Large\lorettadisplay\bfseries Part II Introduction}
\end{center}

\vspace{2cm}

Having established the physical nature of the electromagnetic wave, our focus now shifts from the medium to the message. The second great revolution in communication theory, arriving nearly a century after Maxwell, was the formalisation of ``information'' as a quantifiable, mathematical entity. In his seminal 1948 paper, Claude Shannon laid the groundwork for the digital age, defining the ``bit'' as the fundamental currency of information and establishing the absolute physical limits on how much of it could be transmitted through a noisy channel.

\vspace{1em}

This part of the treatise moves from the physics of the wave to the mathematics of the signal. We will leave the high-frequency carrier wave behind for now and descend into the abstract, low-frequency domain of the baseband signal. Here, we will define the essential metrics used to measure signal quality---Signal-to-Noise Ratio, Bit Error Rate, and the normalised energy ratios---and explore the theoretical limits of communication as defined by Shannon's revolutionary Channel Capacity Theorem. This section provides the universal language used to describe the performance of every digital communication system.

\vspace*{\fill}
\newpage
