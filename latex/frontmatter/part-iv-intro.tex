% Part IV Introduction: The Wireless Channel & Propagation
\newpage
\thispagestyle{empty}

\vspace*{3cm}

\begin{center}
{\Large\lorettadisplay\bfseries Part IV Introduction}
\end{center}

\vspace{2cm}

A signal perfectly formed at the transmitter rarely arrives pristine at the receiver. Once launched from the antenna, the wave embarks on a perilous journey through the wireless channel---the physical space between the transmitter and receiver. The study of propagation is the study of this journey. Early radio pioneers quickly discovered that their signals did not always behave predictably; they faded, were absorbed by the atmosphere, and were blocked by obstacles, leading to the development of statistical channel models pioneered by figures like Lord Rayleigh.

\vspace{1em}

This part of the book confronts the messy reality of the physical world. We will move beyond the idealised vacuum of free space to build a complete and realistic model of the wireless channel. We will quantify the fundamental path loss, analyse the complex effects of multipath fading, explore the different modes of propagation, and account for the profound impact of the atmosphere and weather. This section culminates in the creation of a complete link budget, the master tool of the RF engineer for predicting and ensuring the viability of any wireless link.

\vspace*{\fill}
\newpage
