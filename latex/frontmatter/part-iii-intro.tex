% Part III Introduction: Modulation & Demodulation
\newpage
\thispagestyle{empty}

\vspace*{3cm}

\begin{center}
{\Large\lorettadisplay\bfseries Part III Introduction}
\end{center}

\vspace{2cm}

With an understanding of the physical wave and the abstract digital message, we arrive at the central, practical act of communication: imprinting the latter upon the former. This is the art of modulation. The history of radio is a story of an ever-increasing sophistication in this art, from the simple spark-gap transmitters of Marconi, which were a form of On-Off Keying, to the elegant frequency modulation of Edwin Armstrong, and finally to the complex digital constellations that power our modern world.

\vspace{1em}

This section provides a systematic exploration of the foundational digital modulation schemes. We will examine how information can be encoded by shifting a carrier's amplitude (ASK), frequency (FSK), or phase (PSK). We will then see how these are combined into the highly efficient Quadrature Amplitude Modulation (QAM) that is the workhorse of modern high-speed systems. Each chapter will analyse a scheme's performance, its spectral efficiency, and the trade-offs that make it suitable for a particular application.

\vspace*{\fill}
\newpage
