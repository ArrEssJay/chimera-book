% ==============================================================================
% PREAMBLE - The Chimera Project
% LaTeX configuration, packages, and styling for the book
% ==============================================================================

% ==============================================================================
% FONTS & TYPOGRAPHY
% ==============================================================================
\usepackage{fontspec}
\usepackage{unicode-math}

% Set main fonts with fallbacks
\setmainfont{Minion Pro}[
    Extension = .otf,
    UprightFont = *-Regular,
    BoldFont = *-Bold,
    ItalicFont = *-Italic,
    BoldItalicFont = *-BoldItalic
]

\setsansfont{Myriad Pro}[
    Extension = .otf,
    UprightFont = *-Regular,
    BoldFont = *-Bold,
    ItalicFont = *-Italic,
    BoldItalicFont = *-BoldItalic
]

\setmonofont{Latin Modern Mono}[Scale=0.9]

% Set math font
\setmathfont{Latin Modern Math}

% ==============================================================================
% PAGE LAYOUT
% ==============================================================================
\usepackage[
    paperwidth=7in,
    paperheight=10in,
    textwidth=5in,
    textheight=7.5in,
    top=1.25in,
    bottom=1.25in,
    inner=1in,
    outer=1in,
    headsep=0.25in,
    footskip=0.35in
]{geometry}

% ==============================================================================
% PDF/X COMPLIANCE & COLOR MANAGEMENT
% ==============================================================================
\usepackage[a-1b,pdf17]{pdfx}
\usepackage{xcolor}

% Print version toggle
\newif\ifprintversion
\printversionfalse  % Default: color version

% Define color profiles
\ifprintversion
    % Grayscale for print
    \definecolor{boxblue}{RGB}{100,100,100}
    \definecolor{boxgreen}{RGB}{120,120,120}
    \definecolor{boxyellow}{RGB}{140,140,140}
    \definecolor{boxred}{RGB}{80,80,80}
\else
    % Full color for digital
    \definecolor{boxblue}{RGB}{0,102,204}
    \definecolor{boxgreen}{RGB}{0,153,76}
    \definecolor{boxyellow}{RGB}{255,153,0}
    \definecolor{boxred}{RGB}{204,0,0}
\fi

% ==============================================================================
% GRAPHICS & FIGURES
% ==============================================================================
\usepackage{graphicx}
\usepackage{tikz}
\usetikzlibrary{shapes,arrows,positioning,calc,patterns,decorations.pathreplacing,decorations.markings}
\usepackage{pdfpages}

% Set graphics path
\graphicspath{{assets/}{assets/diagrams/}{assets/artwork/}}

% ==============================================================================
% HYPERLINKS & CROSS-REFERENCES
% ==============================================================================
\usepackage{hyperref}
\hypersetup{
    colorlinks=true,
    linkcolor=blue,
    filecolor=magenta,
    urlcolor=cyan,
    citecolor=green,
    pdftitle={The Chimera Project: Digital Signal Processing},
    pdfauthor={Edited by Rowan Jones},
    pdfsubject={Digital Signal Processing, Communications Engineering},
    pdfkeywords={DSP, Modulation, Wireless Communications, Quantum Biology},
    bookmarksnumbered=true,
    bookmarksopen=true,
    bookmarksopenlevel=2
}

% ==============================================================================
% BIBLIOGRAPHY
% ==============================================================================
\usepackage[
    backend=biber,
    style=ieee,
    sorting=none,
    maxbibnames=99,
    isbn=false,
    doi=true,
    url=true
]{biblatex}
\addbibresource{bibliography.bib}

% ==============================================================================
% MATHEMATICS
% ==============================================================================
\usepackage{amsmath}
\usepackage{amssymb}
\usepackage{mathtools}

% Custom math commands
\newcommand{\vect}[1]{\mathbf{#1}}
\newcommand{\unit}[1]{\,\text{#1}}

% ==============================================================================
% CUSTOM ENVIRONMENTS - Callout Boxes
% ==============================================================================
\usepackage[most]{tcolorbox}

% Non-technical summary box (blue)
\newtcolorbox{nontechnical}{
  colback=boxblue!5!white,
  colframe=boxblue,
  title=Non-Technical Summary,
  fonttitle=\sffamily\bfseries,
  boxrule=1pt,
  sharp corners,
  left=8pt,
  right=8pt,
  top=6pt,
  bottom=6pt,
  toptitle=3pt,
  bottomtitle=3pt,
  before skip=10pt,
  after skip=10pt
}

% Key concept box (green)
\newtcolorbox{keyconcept}{
  colback=boxgreen!5!white,
  colframe=boxgreen,
  title=Key Concept,
  fonttitle=\sffamily\bfseries,
  boxrule=1pt,
  sharp corners,
  left=8pt,
  right=8pt,
  top=6pt,
  bottom=6pt,
  toptitle=3pt,
  bottomtitle=3pt,
  before skip=10pt,
  after skip=10pt
}

% Warning/caution box (yellow)
\newtcolorbox{warning}{
  colback=boxyellow!5!white,
  colframe=boxyellow,
  title=Warning,
  fonttitle=\sffamily\bfseries,
  boxrule=1pt,
  sharp corners,
  left=8pt,
  right=8pt,
  top=6pt,
  bottom=6pt,
  toptitle=3pt,
  bottomtitle=3pt,
  before skip=10pt,
  after skip=10pt
}

% Critical note box (red)
\newtcolorbox{critical}{
  colback=boxred!5!white,
  colframe=boxred,
  title=Critical Note,
  fonttitle=\sffamily\bfseries,
  boxrule=1pt,
  sharp corners,
  left=8pt,
  right=8pt,
  top=6pt,
  bottom=6pt,
  toptitle=3pt,
  bottomtitle=3pt,
  before skip=10pt,
  after skip=10pt
}

% ==============================================================================
% TABLES
% ==============================================================================
\usepackage{array}
\usepackage{longtable}
\usepackage{booktabs}
\usepackage{tabularx}
\usepackage{calc}

% Define \real for pandoc-generated table column widths
\providecommand{\real}[1]{#1}

% ==============================================================================
% LISTS
% ==============================================================================
\usepackage{enumitem}
\setlist{noitemsep, topsep=5pt}

% Pandoc compatibility - define \tightlist for compact lists
\providecommand{\tightlist}{%
  \setlength{\itemsep}{0pt}\setlength{\parskip}{0pt}}

% ==============================================================================
% CHAPTER AND SECTION STYLING
% ==============================================================================
\usepackage{titlesec}

% Chapter formatting
\titleformat{\chapter}[display]
  {\sffamily\bfseries\Large}
  {\chaptertitlename\ \thechapter}{12pt}{\fontsize{24}{28}\selectfont}
\titlespacing*{\chapter}{0pt}{-30pt}{20pt}

% Section formatting
\titleformat{\section}
  {\sffamily\bfseries\large}
  {\thesection}{1em}{}

% Subsection formatting
\titleformat{\subsection}
  {\sffamily\bfseries\normalsize}
  {\thesubsection}{1em}{}

% Subsubsection formatting
\titleformat{\subsubsection}
  {\sffamily\bfseries\normalsize}
  {\thesubsubsection}{1em}{}

% ==============================================================================
% TABLE OF CONTENTS CONFIGURATION
% ==============================================================================

\setcounter{tocdepth}{2}  % Show chapters, sections, subsections
\setcounter{secnumdepth}{3}  % Number up to subsubsections

% TOC formatting
\usepackage{tocloft}

% Part entries in TOC
\renewcommand{\cftpartfont}{\sffamily\bfseries\Large}
\renewcommand{\cftpartpagefont}{\sffamily\bfseries\Large}
\setlength{\cftbeforepartskip}{2em}

% Chapter entries in TOC
\renewcommand{\cftchapfont}{\sffamily\bfseries}
\renewcommand{\cftchappagefont}{\sffamily\bfseries}
\setlength{\cftbeforechapskip}{0.8em}
\renewcommand{\cftchapleader}{\cftdotfill{\cftdotsep}}

% Section entries in TOC
\renewcommand{\cftsecfont}{\sffamily}
\renewcommand{\cftsecpagefont}{\sffamily}
\setlength{\cftbeforesecskip}{0.4em}
\setlength{\cftsecindent}{1.5em}

% Subsection entries in TOC
\renewcommand{\cftsubsecfont}{\normalfont}
\renewcommand{\cftsubsecpagefont}{\normalfont}
\setlength{\cftbeforesubsecskip}{0.2em}
\setlength{\cftsubsecindent}{3em}

% ==============================================================================
% CUSTOM COMMANDS
% ==============================================================================

% Key term highlighting
\newcommand{\keyterm}[1]{\textbf{\textsf{#1}}\index{#1}}

% Paragraph heading (bold sans-serif)
\newcommand{\parhead}[1]{\medskip\noindent\textbf{\textsf{#1:}}\space}

% Logged input command for debugging chapter loading
\newcommand{\loggedinput}[1]{%
  \typeout{LOADING CHAPTER: #1}%
  \input{#1}%
  \typeout{CHAPTER COMPLETE: #1}%
}

% ==============================================================================
% INDEX
% ==============================================================================
\usepackage{makeidx}
\makeindex

% ==============================================================================
% HEADERS AND FOOTERS
% ==============================================================================
\usepackage{fancyhdr}
\pagestyle{fancy}
\fancyhf{}
\fancyhead[LE]{\small\textit{\leftmark}}
\fancyhead[RO]{\small\textit{\rightmark}}
\fancyfoot[C]{\thepage}
\renewcommand{\headrulewidth}{0.4pt}
\renewcommand{\footrulewidth}{0pt}

% Plain style for chapter opening pages
\fancypagestyle{plain}{
  \fancyhf{}
  \fancyfoot[C]{\thepage}
  \renewcommand{\headrulewidth}{0pt}
}

% ==============================================================================
% MISCELLANEOUS
% ==============================================================================

% Better spacing
\usepackage{setspace}
\setstretch{1.1}

% Prevent orphans and widows
\widowpenalty=10000
\clubpenalty=10000

% Better hyphenation
\usepackage[english]{babel}
\hyphenpenation{DSP sig-nal pro-cess-ing mod-u-la-tion}

% Creative Commons icons
\usepackage{ccicons}

% ==============================================================================
% END OF PREAMBLE
% ==============================================================================
