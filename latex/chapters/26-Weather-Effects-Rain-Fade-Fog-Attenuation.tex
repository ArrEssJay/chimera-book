% ==============================================================================
% CHAPTER 26: Weather Effects on Propagation
% ==============================================================================

\chapter{Weather Effects: Rain, Fog, and Clouds}
\label{ch:weather-effects}

\begin{nontechnical}
    \textbf{Weather affects high-frequency radio waves in the same way it affects a beam of light.}

    \parhead{The simple analogy}
    \begin{itemize}
        \item On a clear day, you can see for miles.
        \item In fog or a heavy rainstorm, visibility drops dramatically because the water droplets scatter and absorb the light.
    \end{itemize}
    Radio waves used for satellite TV and 5G mmWave behave just like that light.

    \parhead{The core problem: Rain Fade}
    Raindrops are the perfect size to absorb and scatter microwave signals, especially those above 10 GHz. This effect, known as \keyterm{rain fade}, can weaken a signal by over 99\%.
    \begin{itemize}
        \item \textbf{Why your satellite TV goes out in a thunderstorm:} The dish is trying to receive a very weak signal from a satellite 36,000 km away. The heavy rain in the storm clouds is enough to block the signal completely.
        \item \textbf{Why your 5G might slow down:} High-speed 5G mmWave uses very high frequencies that are also susceptible to rain fade. Your phone might automatically switch to a slower but more reliable 4G frequency during a downpour.
        \item \textbf{Why your FM radio and GPS still work perfectly:} They use much lower frequencies that are almost completely unaffected by rain.
    \end{itemize}

    \parhead{How engineers fix it} Systems that operate at high frequencies must be designed with a \textbf{rain margin}---extra power held in reserve to "burn through" the attenuation. Modern systems also use \textbf{Adaptive Coding and Modulation (ACM)}, automatically switching to a slower, more robust data mode during a fade to keep the link active.
\end{nontechnical}


\section{Overview and Properties}

\subsection{Overview}

While free-space path loss is the primary source of attenuation in a wireless link, atmospheric conditions—especially precipitation—introduce significant and highly variable additional losses. These effects are negligible at lower frequencies but become the dominant source of link outages for systems operating above 10 GHz.

\begin{keyconcept}
    \textbf{Rain attenuation} is the most critical weather-related impairment for microwave and millimetre-wave communication systems. The loss increases dramatically with both \textbf{frequency} and \textbf{rain rate}, and must be accounted for in the link budget with a statistical \textbf{rain margin} to ensure a target level of link availability.
\end{keyconcept}


\subsection{Rain Attenuation}

\paragraph{Physical Mechanism}
At frequencies above 10 GHz, the wavelength of the signal becomes comparable to the size of raindrops (1-5 mm). The raindrops act as lossy dielectrics, absorbing and scattering the radio frequency energy.

\paragraph{The ITU-R Rain Model}
The standard model for predicting rain attenuation is provided by the ITU-R P.838 recommendation. The \keyterm{specific attenuation} (loss in dB per kilometre) is given by a power-law relationship:
\begin{equation}
    \gamma_R = k R^{\alpha} \quad (\text{dB/km})
\end{equation}
where $R$ is the rain rate in mm/hr, and $k$ and $\alpha$ are frequency- and polarisation-dependent coefficients. The total attenuation, $A_{\text{rain}}$, is this value multiplied by the effective path length through the rain.

\begin{table}[H]
    \centering
    \caption{Specific Attenuation (dB/km) for a Heavy Rain Rate (25 mm/hr)}
    \label{tab:rain-attenuation-heavy}
    \begin{tabular}{@{}lc@{}}
        \toprule
        \tableheaderfont Frequency & \tableheaderfont Specific Attenuation ($\gamma_R$) \\
        \midrule
        10 GHz (X-band) & 0.5 dB/km \\
        12 GHz (Ku-band) & 1.0 dB/km \\
        20 GHz (Ka-band) & 3.3 dB/km \\
        30 GHz (Ka-band) & 7.1 dB/km \\
        60 GHz (V-band) & 20.3 dB/km \\
        \bottomrule
    \end{tabular}
\end{table}

\begin{warningbox}
    As the table shows, the attenuation is highly non-linear with frequency. Moving from Ku-band (12 GHz) to Ka-band (30 GHz) increases the specific rain attenuation by a factor of seven for the same rain rate. This has profound implications for system design.
\end{warningbox}


\subsection{Fog, Cloud, and Snow Attenuation}

\paragraph{Fog and Cloud Attenuation}
Fog and clouds are composed of much smaller water droplets than rain. Their primary effect is absorption, which is much less severe than the scattering caused by rain. Fog and cloud attenuation is generally negligible below 30 GHz but can become a factor for EHF and THz systems.

\paragraph{Snow and Ice Attenuation}
Dry snow and ice crystals are nearly transparent to radio waves and cause very little attenuation. Wet snow, however, has a liquid water coating and can cause significant attenuation, comparable to light or moderate rain.


\subsection{Mitigation Techniques}

Engineers use several strategies to ensure link reliability in the presence of weather effects.
\begin{description}
    \item[Link Margin] The most straightforward approach is to design the link with extra power. A \keyterm{rain margin} is the additional dB of signal power available in clear-sky conditions, held in reserve to be "spent" on overcoming rain fade. The required margin is determined statistically based on the local climate and the desired link availability (e.g., 99.9\% or 99.99\%).
    \item[Adaptive Coding and Modulation (ACM)] Modern systems can dynamically change their modulation and coding scheme in response to changing channel conditions. During a rain fade, a system might switch from high-efficiency 64-QAM to robust QPSK, maintaining the link at a lower data rate until the weather passes.
    \item[Site Diversity] For critical ground stations (e.g., satellite gateways), two antennas are placed 10-20 km apart. Since heavy rain cells are often localized, it is unlikely that both sites will be experiencing a deep fade simultaneously. The system can simply switch to whichever site has the clearer path.
    \item[Uplink Power Control (UPC)] For satellite uplinks, the ground station can monitor a beacon signal from the satellite and increase its own transmit power to overcome detected rain fade in real-time.
\end{description}


\begin{workedexample}{Ka-Band Satellite Link Design}
    \parhead{Problem} Determine the required clear-sky link margin for a Ka-band (\qty{20}{GHz}) satellite link in a tropical region to achieve 99.9\% availability.
    \parhead{System Parameters}
    \begin{itemize}
        \item Location: Tropical (ITU Rain Zone E). For 99.9\% availability, this requires designing for a rain rate of $R = \qty{22}{mm/hr}$.
        \item Link Geometry: Effective path length through rain is calculated to be \qty{7}{km}.
        \item Frequency-dependent coefficients for 20 GHz: $k=0.0751, \alpha=1.099$.
    \end{itemize}
    \parhead{Solution}
    \begin{derivationsteps}
        \step Calculate the specific attenuation ($\gamma_R$) for the design rain rate.
        \[ \gamma_R = k R^{\alpha} = 0.0751 \times (22)^{1.099} \approx \qty{2.3}{dB/km} \]
        \step Calculate the total rain attenuation over the effective path length.
        \[ A_{\text{rain}} = \gamma_R \times d_{\text{eff}} = 2.3 \text{ dB/km} \times 7 \text{ km} \approx \qty{16.1}{dB} \]
        \step Add margins for other, smaller effects. Gaseous absorption, cloud attenuation, and scintillation might add another 1.5 dB.
        \[ L_{\text{atmospheric}} = A_{\text{rain}} + L_{\text{other}} = 16.1 + 1.5 = \qty{17.6}{dB} \]
    \end{derivationsteps}
    \parhead{Interpretation} To ensure the link remains operational for 99.9\% of the time, the system must be designed with a \textbf{clear-sky link margin of at least 17.6 dB}. This is a very large margin, highlighting the significant challenge of using Ka-band in tropical regions. Such a link would almost certainly rely on powerful ACM to be economically viable.
\end{workedexample}


\begin{importantbox}[title={Further Reading}]
    Weather effects are a critical component of a complete channel model, especially for high-frequency systems.
    \begin{description}
        \item[Link Budget Analysis] (\Cref{ch:linkbudget}) is where the rain margin and other atmospheric losses are formally incorporated into the end-to-end system calculation.
        \item[Adaptive Modulation and Coding (AMC)] (\Cref{ch:amc}) provides a deep dive into the primary technique used by modern systems to combat time-varying channel conditions like rain fade.
        \item[Satellite Communications] (\Cref{ch:satellite}) explores the specific trade-offs between C-band, Ku-band, and Ka-band in the context of global usability and weather-related availability.
    \end{description}
\end{importantbox}
