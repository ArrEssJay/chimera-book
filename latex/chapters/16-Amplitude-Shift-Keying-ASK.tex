% ==============================================================================
% CHAPTER 16: Amplitude-Shift Keying (ASK)
% ==============================================================================

\chapter{Amplitude-Shift Keying (ASK)}
\label{ch:ask}

\begin{nontechnical}
    \textbf{Amplitude-Shift Keying (ASK) is like using a dimmer switch on a flashlight to send data.} Different brightness levels correspond to different groups of bits.

    \parhead{The simple idea}
    \begin{itemize}
        \item To send "00", set the light to OFF.
        \item To send "01", set the light to DIM.
        \item To send "10", set the light to MEDIUM.
        \item To send "11", set the light to BRIGHT.
    \end{itemize}
    This is an example of 4-ASK, which sends two bits at a time. The simplest form, with just ON and OFF, is called On-Off Keying (OOK).

    \parhead{The problem with amplitude} The main weakness of ASK is its susceptibility to noise and fading. A random fluctuation in signal strength can easily make a "DIM" signal look like an "OFF" one, causing an error. For this reason, most radio systems prefer to use phase (PSK) or frequency (FSK) modulation, which are more robust.

    \parhead{Where it's used} ASK is primarily used in channels that are naturally low-noise and where the hardware is simple. Its two main applications are in \textbf{fiber optic communications}, where a laser's intensity is modulated, and in simple, low-cost \textbf{RFID tags}.
\end{nontechnical}


\section{Overview and Properties}

\subsection{Overview}

\keyterm{Amplitude-Shift Keying (ASK)} is a digital modulation technique where information is encoded by varying the \keyterm{amplitude} of a carrier wave while keeping its frequency and phase constant. It is a one-dimensional modulation scheme, with all constellation points lying on the real (I) axis.

The simplest form of ASK is binary ASK, more commonly known as \keyterm{On-Off Keying (OOK)}, where the presence of a carrier represents a '1' and its absence represents a '0'. Higher-order \keyterm{M-ary ASK (M-ASK)} uses multiple amplitude levels to encode $\log_2(M)$ bits per symbol, thereby increasing spectral efficiency.

\begin{keyconcept}
    While simple to implement, ASK is highly susceptible to noise and channel fading, as these impairments directly corrupt the amplitude in which the information is encoded. For this reason, it is rarely used for high-performance radio communications, which favour more robust phase-based schemes like PSK and QAM. ASK's primary modern applications are in low-noise environments like optical fiber, or in ultra-low-cost systems like RFID.
\end{keyconcept}


\subsection{Mathematical Representation}

The general M-ary ASK waveform is:
\begin{equation}
    s_m(t) = A_m \cos(2\pi f_c t), \quad \text{for } m = 0, 1, \ldots, M-1
\end{equation}
where $A_m$ is one of $M$ distinct amplitude levels. These levels are typically spaced linearly. For example, in 4-ASK, the amplitude levels might be $\{A, 2A, 3A, 4A\}$. In the IQ plane, this corresponds to M points along the positive I-axis.


\subsection{Performance Characteristics}

\paragraph{Bit Error Rate (BER)}
The performance of ASK degrades rapidly as the number of levels, M, increases. This is because for a fixed average power, the distance between adjacent amplitude levels must decrease, making the system more vulnerable to noise. Each doubling of M (which adds one bit per symbol) requires an additional 3--4 dB of $E_b/N_0$ to maintain the same BER.

\paragraph{Power and Spectral Efficiency}
While M-ASK increases spectral efficiency by a factor of $\log_2(M)$, it does so at a significant cost in power efficiency. Furthermore, because it is not a constant-envelope modulation, it requires linear power amplifiers, which are less efficient.

\begin{table}[H]
    \centering
    \caption{Performance Comparison of Modulation Schemes}
    \label{tab:ask-comparison}
    \begin{tabular}{@{}lccc@{}}
        \toprule
        \tableheaderfont Modulation & \tableheaderfont Bits/Symbol & \tableheaderfont Constellation & \tableheaderfont Required $E_b/N_0$ for BER $10^{-5}$ \\
        \midrule
        4-ASK & 2 & 1-Dimensional (4 points) & 13.5 dB \\
        \textbf{QPSK} & 2 & 2-Dimensional (4 points) & \textbf{9.6 dB} \\
        \addlinespace
        16-ASK & 4 & 1-Dimensional (16 points) & 24.0 dB \\
        \textbf{16-QAM} & 4 & 2-Dimensional (16 points) & \textbf{14.5 dB} \\
        \bottomrule
    \end{tabular}
\end{table}

\begin{importantbox}[title={Why QAM is Superior to M-ASK}]
    The table above highlights a fundamental principle: for the same number of bits per symbol, a two-dimensional constellation (like QAM) is always significantly more power-efficient than a one-dimensional constellation (like ASK). By arranging points in a 2D grid instead of a 1D line, the average distance between points can be much greater for the same average power. This geometric advantage is why QAM, not M-ASK, is the standard for high-efficiency RF communications.
\end{importantbox}


\subsection{Primary Applications}

\paragraph{Optical Communications}
The most significant application of ASK is in fiber optic systems, where the intensity (power) of a laser or LED is modulated.
\begin{itemize}
    \item \textbf{OOK (NRZ):} The workhorse of optical networking for decades, used in everything from 10 Gigabit Ethernet to long-haul submarine cables. The simplicity of direct intensity modulation and direct detection with a photodiode is a major advantage.
    \item \textbf{PAM-4 (4-ASK):} As data rates pushed beyond 100 Gbps, increasing the symbol rate became difficult due to hardware limitations. \keyterm{Pulse Amplitude Modulation with 4 levels (PAM-4)} was adopted. It doubles the bit rate for the same symbol rate by using four distinct power levels to encode two bits per symbol. It is the key technology enabling 400G and 800G Ethernet.
\end{itemize}

\paragraph{RFID Systems}
Passive RFID tags use a form of binary ASK called \keyterm{backscatter modulation}. The tag does not generate its own carrier; instead, it modulates the reflection of the carrier wave sent by the reader. By switching its antenna impedance, it can either absorb the reader's signal (representing a '0') or reflect it (representing a '1'). This is effectively OOK, and its extreme simplicity allows the tag to operate without a battery, powered entirely by the reader's RF field.


\begin{workedexample}{PAM-4 in Optical Networking}
    \parhead{Problem} Compare the symbol rate and required SNR for a 100 Gbps link using traditional OOK versus modern PAM-4.
    
    \parhead{Case 1: 100 Gbps using OOK (2-ASK)}
    \begin{itemize}
        \item OOK sends 1 bit per symbol.
        \item To achieve 100 Gbps, the required symbol rate (baud rate) is \textbf{100 Gbaud}.
        \item This requires extremely high-speed, expensive, and power-hungry electronics.
    \end{itemize}

    \parhead{Case 2: 100 Gbps using PAM-4 (4-ASK)}
    \begin{itemize}
        \item PAM-4 sends 2 bits per symbol ($\log_2(4)=2$).
        \item To achieve 100 Gbps, the required symbol rate is $100 / 2 = \textbf{50 Gbaud}$.
        \item This halves the required bandwidth and relaxes the speed requirements on the electronics, making the system cheaper and more power-efficient.
    \end{itemize}
    \parhead{The Trade-off} The BER curve for PAM-4 is shifted approximately 4.8 dB to the right compared to OOK. This means that to achieve the same BER, PAM-4 requires a significantly higher Signal-to-Noise Ratio. In the low-noise environment of an optical fiber, this is an acceptable trade-off to gain the massive advantage of a lower symbol rate.
\end{workedexample}


\begin{importantbox}[title={Further Reading}]
    ASK is a foundational concept that leads directly to more advanced and practical modulation schemes.
    \begin{description}
        \item[On-Off Keying (OOK)] (\Cref{ch:ook}) provides a detailed analysis of the most common form of ASK.
        \item[Quadrature Amplitude Modulation (QAM)] (\Cref{ch:qam}) is the two-dimensional evolution of ASK and is the standard for high-data-rate RF communication.
        \item[Link Budget Analysis] (\Cref{ch:linkbudget}) explores how the higher SNR requirements of M-ASK impact overall system design.
    \end{description}
\end{importantbox}
