% ==============================================================================
% CHAPTER 23: Multipath Propagation & Fading
% ==============================================================================

\chapter{Multipath Propagation \& Fading}
\label{ch:multipath}

\begin{nontechnical}
    \textbf{Multipath is like hearing echoes in a canyon.} When you shout, your voice travels directly to a listener, but it also bounces off the canyon walls, arriving a moment later as echoes. Radio waves do the same thing in a city, bouncing off buildings, the ground, and other obstacles.

    \parhead{The problem with echoes} The receiver hears the original signal plus all the delayed, weaker copies (the echoes) at the same time.
    \begin{itemize}
        \item Sometimes the echoes add up constructively, making the signal stronger.
        \item Sometimes they cancel each other out destructively, making the signal disappear.
    \end{itemize}
    This rapid fluctuation in signal strength as you move through this complex web of reflections is called \textbf{multipath fading}.

    \parhead{Real-world experience}
    \begin{itemize}
        \item \textbf{Dropped cell phone calls} as you walk around a street corner are often caused by you moving into a "dead spot" where multipath signals are cancelling each other out.
        \item \textbf{WiFi dead zones} in your house are locations where reflections from walls cause destructive interference. Moving just a foot or two can dramatically change the signal strength.
    \end{itemize}

    \parhead{How engineers fix it} Modern systems like 4G, 5G, and WiFi are designed specifically to combat multipath. They use techniques like \textbf{MIMO} (multiple antennas to find a path that isn't faded) and \textbf{OFDM} (which breaks a fast data stream into many slow, parallel streams that are less affected by echoes).
\end{nontechnical}


\section{Overview and Properties}

\subsection{Overview}

In any real-world wireless environment, a transmitted signal reaches the receiver via multiple paths due to \keyterm{reflection}, \keyterm{diffraction}, and \keyterm{scattering} off objects in the environment. This phenomenon is known as \keyterm{multipath propagation}. The constructive and destructive interference of these multiple signal copies causes rapid, location-dependent fluctuations in the received signal's amplitude and phase, a process known as \keyterm{multipath fading}.

\begin{keyconcept}
    Multipath fading is the single most significant impairment in mobile and non-line-of-sight (NLOS) wireless channels. It can cause the signal strength to vary by 20--30 dB over distances as small as half a wavelength. Overcoming the effects of fading, rather than just simple path loss, is the primary driver behind the complexity of modern wireless systems like LTE, 5G, and WiFi.
\end{keyconcept}


\subsection{Channel Characterization}

The multipath channel is characterized by its time-varying impulse response, which can be described by two key parameters:

\paragraph{Delay Spread ($\tau_{\text{rms}}$)}
The multipath components arrive at different times, creating a "smearing" of the signal in time. The \keyterm{RMS delay spread} is a measure of the temporal extent of this smearing.
\begin{itemize}
    \item If the symbol period is much longer than the delay spread ($T_s \gg \tau_{\text{rms}}$), the channel is \keyterm{flat-fading}. All frequency components of the signal are affected similarly.
    \item If the symbol period is shorter than the delay spread ($T_s < \tau_{\text{rms}}$), the channel is \keyterm{frequency-selective}. This causes \keyterm{intersymbol interference (ISI)}, as the echoes from one symbol interfere with the next.
\end{itemize}

\paragraph{Doppler Spread ($B_d$)}
Movement of the transmitter, receiver, or objects in the environment causes a frequency shift in each multipath component. The range of these shifts is the \keyterm{Doppler spread}. Its inverse, the \keyterm{coherence time} ($T_c$), describes how long the channel remains effectively static.
\begin{itemize}
    \item If the symbol period is much shorter than the coherence time ($T_s \ll T_c$), the channel is \keyterm{slow-fading}. The channel is constant over many symbols.
    \item If the symbol period is longer than the coherence time ($T_s > T_c$), the channel is \keyterm{fast-fading}. The channel changes during the transmission of a single symbol.
\end{itemize}


\subsection{Statistical Fading Models}

The statistical behaviour of the signal envelope is captured by two primary models:

\paragraph{Rayleigh Fading (NLOS)}
This model applies to non-line-of-sight (NLOS) environments where the received signal is a sum of many scattered components with no single dominant path. It is the worst-case fading scenario, characterized by frequent and deep signal fades. It is the standard model for dense urban and indoor environments.

\paragraph{Rician Fading (LOS)}
This model applies when there is a dominant line-of-sight (LOS) path in addition to scattered components. The strength of the LOS path relative to the scattered paths is defined by the \keyterm{K-factor}. A high K-factor indicates a strong LOS component and less severe fading, approaching an AWGN channel as $K \to \infty$. This model is used for suburban, rural, and open indoor environments.


\subsection{Mitigation Techniques}

Engineers have developed a powerful toolkit to combat the effects of multipath fading:
\begin{description}
    \item[Diversity] This is the most fundamental technique. By providing the receiver with multiple independent copies of the signal, the probability that all copies are simultaneously in a deep fade is dramatically reduced. This can be achieved through space (multiple antennas), frequency (spread spectrum), or time (interleaving and coding).
    \item[OFDM (Orthogonal Frequency-Division Multiplexing)] This technique combats frequency-selective fading by dividing a wideband channel into hundreds or thousands of narrow, flat-fading sub-channels, which are much easier to equalize.
    \item[Channel Equalization] Sophisticated DSP algorithms are used to estimate the channel's impulse response and apply an inverse filter to undo the distortion caused by multipath.
    \item[Adaptive Modulation and Coding (AMC)] The system constantly monitors the channel quality and adapts the modulation and coding scheme to match the current conditions, maximizing throughput during good conditions and ensuring reliability during fades.
\end{description}


\begin{workedexample}{Urban LTE Channel Characterization}
    \parhead{Problem} Characterize the fading environment for a mobile user traveling at 50 km/h in a dense urban environment, using a 4G LTE signal at 2.6 GHz.
    \parhead{System Parameters}
    \begin{itemize}
        \item Channel Model: Dense Urban NLOS $\implies$ Rayleigh Fading.
        \item RMS Delay Spread ($\tau_{\text{rms}}$): Typical value for urban macrocell is $\approx \qty{1}{\mu s}$.
        \item Mobile Velocity ($v$): 50 km/h $\approx$ \qty{13.9}{m/s}.
        \item Carrier Frequency ($f_c$): \qty{2.6}{GHz}.
        \item LTE Symbol Period ($T_s$): Approx. \qty{71.4}{\mu s} (including cyclic prefix).
    \end{itemize}
    \parhead{Analysis}
    \begin{derivationsteps}
        \step Determine if the fading is flat or frequency-selective. First, find the coherence bandwidth.
        \[ B_c \approx \frac{1}{5\tau_{\text{rms}}} = \frac{1}{5 \times 10^{-6}} = \qty{200}{kHz} \]
        The LTE signal bandwidth (e.g., 10 or 20 MHz) is much greater than $B_c$. Therefore, the channel is highly \textbf{frequency-selective}, and will cause severe ISI. This is the primary reason LTE \emph{must} use OFDM.
        
        \step Determine if the fading is fast or slow. First, find the maximum Doppler spread.
        \[ f_d = \frac{v}{\lambda} = \frac{v f_c}{c} = \frac{13.9 \times (2.6 \times 10^9)}{3 \times 10^8} \approx \qty{120}{Hz} \]
        Now, find the coherence time.
        \[ T_c \approx \frac{0.423}{f_d} \approx \frac{0.423}{120} \approx \qty{3.5}{ms} \]
        The symbol period ($T_s \approx \qty{71.4}{\mu s}$) is much smaller than the coherence time ($T_c \approx \qty{3.5}{ms}$). Therefore, the channel is \textbf{slow-fading}, meaning the channel state is constant over many symbols, which allows channel estimation and equalization to be highly effective.
    \end{derivationsteps}
\end{workedexample}


\begin{importantbox}[title={Further Reading}]
    Multipath fading is the central problem that modern wireless communication systems are designed to solve.
    \begin{description}
        \item[Diversity Techniques] (\Cref{ch:diversity}) is the most direct and powerful method for combating the deep fades caused by multipath.
        \item[OFDM] (\Cref{ch:ofdm}) is the modulation scheme of choice for high-data-rate systems in frequency-selective fading channels, such as WiFi and 5G.
        \item[MIMO Systems] (\Cref{ch:mimo}) explains how multiple antennas can be used not only for diversity but also to exploit the multipath environment to create parallel data streams, dramatically increasing capacity.
    \end{description}
\end{importantbox}
