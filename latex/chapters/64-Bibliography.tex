% ==============================================================================
% CHAPTER 64: A Curated Bibliography and Resources
% ==============================================================================

\chapter{A Curated Bibliography and Resources}
\label{ch:bibliography}

\begin{nontechnical}
    This chapter provides a comprehensive, curated collection of references and resources for the practitioner, student, and researcher. It is organised thematically to guide further study into the core domains of digital signal processing, communication theory, and the more specialised topics explored in this book. The selections include foundational textbooks, seminal research papers, key industry standards, and practical software tools that form the bedrock of the field.
\end{nontechnical}

\section{Core Textbooks}

\parhead{Digital Communications and Signal Processing}
\begin{description}
    \item[Proakis, J. G., \& Salehi, M. (2007). \emph{Digital Communications} (5th ed.).] A canonical and comprehensive treatise on modulation, coding, and detection theory. An essential reference for the advanced topics covered in this book.
    \item[Sklar, B. (2001). \emph{Digital Communications: Fundamentals and Applications} (2nd ed.).] An excellent text with a strong practical emphasis, rich with detailed examples and block diagrams that bridge theory and implementation.
    \item[Goldsmith, A. (2005). \emph{Wireless Communications}.] A modern classic focusing on the principles of wireless systems, with in-depth coverage of channel modelling, fading, diversity, and MIMO systems.
    \item[Tse, D., \& Viswanath, P. (2005). \emph{Fundamentals of Wireless Communication}.] Provides a rigorous, information-theoretic perspective on modern wireless communications, particularly MIMO and multi-user systems.
    \item[Oppenheim, A. V., \& Schafer, R. W. (2009). \emph{Discrete-Time Signal Processing} (3rd ed.).] The definitive reference for the fundamental principles of digital signal processing that underpin all of the techniques in this book.
\end{description}

\parhead{Channel Coding Theory}
\begin{description}
    \item[Richardson, T., \& Urbanke, R. (2008). \emph{Modern Coding Theory}.] The authoritative text on modern, capacity-approaching codes, with a deep dive into the theory and practice of LDPC codes.
    \item[Moon, T. K. (2005). \emph{Error Correction Coding: Mathematical Methods and Algorithms}.] A comprehensive and accessible guide to the mathematical foundations and algorithmic implementation of a wide range of error-correcting codes.
\end{description}

\parhead{Electromagnetic Theory and Propagation}
\begin{description}
    \item[Balanis, C. A. (2016). \emph{Antenna Theory: Analysis and Design} (4th ed.).] The standard, comprehensive engineering reference for all aspects of antenna theory, analysis, and practical design.
    \item[Parsons, J. D. (2000). \emph{The Mobile Radio Propagation Channel} (2nd ed.).] A focused and detailed treatment of multipath, fading, and statistical channel models for mobile and terrestrial communication.
\end{description}

\parhead{Quantum Biology and Consciousness}
\begin{description}
    \item[Al-Khalili, J., \& McFadden, J. (2014). \emph{Life on the Edge: The Coming Age of Quantum Biology}.] An accessible and engaging introduction to the field of quantum biology, covering photosynthesis, magnetoreception, and other established phenomena.
    \item[Penrose, R. (1989). \emph{The Emperor's New Mind}.] The foundational work in which Penrose first outlines his arguments for the non-computational nature of consciousness and the necessity of a new quantum physics, which would later evolve into the Orch-OR theory.
\end{description}

\section{Seminal Research Papers}

\begin{description}
    \item[Shannon, C. E. (1948). "A Mathematical Theory of Communication." \emph{Bell System Technical Journal}.] The foundational paper of the entire field of information theory, defining the concepts of entropy, channel capacity, and the fundamental limits of communication.
    \item[Viterbi, A. J. (1967). "Error bounds for convolutional codes and an asymptotically optimum decoding algorithm." \emph{IEEE Trans. on Information Theory}.] Introduced the Viterbi algorithm, the elegant and efficient method for decoding convolutional codes that became a workhorse of digital communications for decades.
    \item[Berrou, C. \textit{et al.} (1993). "Near Shannon limit error-correcting coding and decoding: Turbo-codes." \emph{Proceedings of ICC'93}.] The revolutionary paper that introduced Turbo codes, the first practical codes to achieve performance astonishingly close to the Shannon limit, transforming the field of error correction.
    \item[Frey, A. H. (1962). "Human auditory system response to modulated electromagnetic energy." \emph{Journal of Applied Physiology}.] The first systematic documentation of the microwave auditory effect, establishing the thermoelastic transduction mechanism.
    \item[Hameroff, S., \& Penrose, R. (2014). "Consciousness in the universe: A review of the `Orch OR' theory." \emph{Physics of Life Reviews}.] A comprehensive, peer-reviewed overview of the modern Orch-OR theory, addressing criticisms and summarising supporting evidence.
\end{description}

\section{Key Standards and Specifications}

\begin{description}
    \item[ITU-R Recommendations (P Series)] The International Telecommunication Union provides the essential models for propagation, including atmospheric gas absorption (P.676), rain attenuation (P.618), and terrestrial path loss (P.530). These are the standard models used in professional link budget analysis.
    \item[3GPP Technical Specifications (TS 38 Series)] The 3rd Generation Partnership Project defines the complete standards for 5G New Radio, including the physical layer structure, LDPC and Polar codes, and MIMO procedures.
    \item[IEEE 802.11 Standards] The family of standards that define the physical and MAC layers for all Wi-Fi technologies, specifying the use of OFDM, QAM, LDPC codes, and MIMO.
    \item[DVB-S2X Standard (ETSI EN 302 307)] The specification for modern satellite broadcasting, which provides a masterclass in the application of Adaptive Coding and Modulation (ACM) using a powerful combination of LDPC codes and higher-order modulation (8-PSK, 16/32-APSK).
    \item[GPS Interface Specification (IS-GPS-200)] The official document defining the signal structure of the Global Positioning System, including the frequencies, spreading codes, and data formats for the civilian and military signals.
\end{description}

\section{Software and Online Resources}

\begin{description}
    \item[GNURadio] An open-source software development toolkit that provides signal processing blocks to implement software-defined radios. It is an invaluable tool for hands-on learning and rapid prototyping of communication systems.
    \item[Signal Identification Wiki (sigidwiki.com)] A comprehensive, community-driven database of radio signals, including waterfall plots and demodulated audio. An essential resource for identifying and understanding real-world transmissions.
    \item[Navipedia (European Space Agency)] An extensive online encyclopedia for Global Navigation Satellite Systems (GNSS), providing detailed technical information on the signal structures and operating principles of GPS, Galileo, GLONASS, and BeiDou.
\end{description}