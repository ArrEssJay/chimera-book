% ==============================================================================
% CHAPTER 17: Frequency-Shift Keying (FSK)
% ==============================================================================

\chapter{Frequency-Shift Keying (FSK)}
\label{ch:fsk}

\begin{nontechnical}
    \textbf{Frequency-Shift Keying (FSK) is like sending Morse code with two different musical notes.} A high note represents a '1' and a low note represents a '0'.

    \parhead{The musical analogy} Imagine communicating with a piano. You could agree that the C note represents a `0` and the G note represents a `1`. To send the data "0110", you would play the sequence "C-G-G-C". A receiver can easily distinguish between the two notes, even in a noisy room, because their fundamental pitch (frequency) is different.

    \parhead{Why it's so robust} FSK's key advantage is its immunity to amplitude variations. Noise and fading can make a signal weaker or stronger, but they don't easily change its frequency. This makes FSK extremely reliable in harsh and noisy channel conditions.

    \parhead{Where it's used} Historically, FSK was the backbone of dial-up modems, fax machines, and teletype systems. Today, its robustness and simplicity make it a cornerstone of many modern low-power, wide-area networks (LPWANs) like \textbf{LoRa}, and it is the basis for the \textbf{Bluetooth Low Energy (BLE)} standard.
\end{nontechnical}


\section{Overview and Properties}

\subsection{Overview}

\keyterm{Frequency-Shift Keying (FSK)} is a digital modulation technique where information is encoded by shifting the frequency of a carrier wave between two or more discrete frequencies. In its simplest binary form (BFSK), one frequency, $f_0$, represents a `0' (the "space" frequency) and another, $f_1$, represents a `1' (the "mark" frequency).

\begin{keyconcept}
    FSK is a \textbf{constant-envelope} modulation with excellent immunity to amplitude fluctuations and fading. It can be demodulated with a simple \textbf{non-coherent} receiver, which does not require complex carrier phase synchronization. This combination of robustness and simplicity makes it the preferred choice for many low-cost, low-power, and harsh-environment communication systems.
\end{keyconcept}


\subsection{Mathematical Representation}

The general FSK signal is expressed as a carrier wave whose frequency is switched based on the input data:
\begin{equation}
    s(t) = A \cos\left(2\pi f_k t\right), \quad \text{where } f_k \in \{f_0, f_1, \dots, f_{M-1}\}
\end{equation}
The key parameter that defines the spectral and performance characteristics of FSK is the \keyterm{modulation index}, $h$:
\begin{equation}
    h = \frac{\Delta f}{R_s}
\end{equation}
where $\Delta f$ is the separation between adjacent frequencies and $R_s$ is the symbol rate. For \keyterm{orthogonal FSK}, which allows for optimal detection, the frequencies must be separated by an integer multiple of $R_s/2$.


\subsection{Performance and Spectral Efficiency}

\paragraph{Bit Error Rate (BER)}
The performance of FSK depends on whether coherent or non-coherent detection is used.
\begin{itemize}
    \item \textbf{Coherent FSK:} Requires a synchronized local oscillator. The BER is given by $Q\left(\sqrt{E_b/N_0}\right)$.
    \item \textbf{Non-Coherent FSK:} Uses a simpler envelope detector. The BER is $\frac{1}{2}e^{-E_b/(2N_0)}$.
\end{itemize}
Coherent detection offers a performance advantage of approximately 1-2 dB over non-coherent detection, but at the cost of significantly increased receiver complexity. Compared to BPSK, coherent FSK requires 3 dB more power to achieve the same BER.

\paragraph{Spectral Efficiency}
FSK is not spectrally efficient. The bandwidth required is approximated by Carson's Rule: $B \approx 2(\Delta f + R_s)$. For a typical non-coherent system with $h=1$, the bandwidth is $B \approx 4R_s$, leading to a very low spectral efficiency of around 0.25 bps/Hz. Variants like MSK are much more efficient.


\subsection{Key FSK Variants}

\begin{description}
    \item[Minimum Shift Keying (MSK)] A special form of continuous-phase FSK with a modulation index of $h=0.5$. This is the most spectrally efficient form of FSK, and it can be viewed as a form of offset QPSK with sinusoidal pulse shaping.
    \item[Gaussian FSK (GFSK)] The most common variant in modern systems. Before modulation, the binary data stream is passed through a Gaussian filter. This smooths the sharp frequency transitions, significantly reducing the signal's bandwidth and out-of-band emissions. GFSK is the modulation used in Bluetooth, DECT, and many other wireless standards.
    \item[Multi-Frequency FSK (MFSK)] Uses $M>2$ frequencies to encode $\log_2(M)$ bits per symbol. MFSK is extremely power-efficient in very noisy channels and is used in robust HF radio protocols like MT63.
\end{description}


\begin{workedexample}{LoRa IoT Link Budget Analysis}
    \parhead{Problem} A LoRa sensor in a rural environment needs to transmit data over 5 km. Analyse the link budget to determine its viability.
    \parhead{System Parameters}
    \begin{itemize}
        \item Transmit Power ($P_t$): \qty{14}{dBm} (25 mW)
        \item Antenna Gains ($G_t, G_r$): 2 dBi (Tx), 6 dBi (Rx)
        \item Frequency ($f$): \qty{915}{MHz} (ISM band)
        \item Receiver Noise Figure (NF): \qty{6}{dB}
        \item Bandwidth ($B$): \qty{125}{kHz}
        \item Data Rate ($R_b$): \qty{1}{kbps} (low for long range)
    \end{itemize}
    \parhead{Solution}
    \begin{derivationsteps}
        \step Calculate the Free-Space Path Loss (FSPL) over 5 km.
        \[ \text{FSPL}_{\text{dB}} = 20\log_{10}(d_{\text{km}}) + 20\log_{10}(f_{\text{MHz}}) + 32.45 = 20\log_{10}(5) + 20\log_{10}(915) + 32.45 \approx \qty{105.6}{dB} \]
        \step Calculate the received signal power ($P_r$).
        \[ P_r = P_t + G_t + G_r - \text{FSPL} = 14 + 2 + 6 - 105.6 = \qty{-83.6}{dBm} \]
        \step Calculate the receiver's thermal noise floor in the channel bandwidth.
        \[ P_n = -174 \text{ dBm/Hz} + 10\log_{10}(B) + \text{NF} = -174 + 10\log_{10}(125000) + 6 = -174 + 51 + 6 = \qty{-117}{dBm} \]
        \step Calculate the received SNR.
        \[ \text{SNR} = P_r - P_n = -83.6 - (-117) = \qty{33.4}{dB} \]
    \end{derivationsteps}
    \parhead{Interpretation} The link has an extremely high SNR of 33.4 dB. LoRa technology uses a unique Chirp Spread Spectrum (CSS) modulation, which is a variant of FSK, to achieve a receiver sensitivity as low as -148 dBm. This link would not only be viable but would have an enormous margin, allowing it to operate through significant foliage, buildings, or at much longer ranges. This demonstrates the power of modern FSK-based schemes for long-range, low-power IoT applications.
\end{workedexample}

\begin{importantbox}[title={Further Reading}]
    FSK represents a fundamentally different approach to modulation compared to ASK and PSK.
    \begin{description}
        \item[Minimum Shift Keying (MSK)] (\Cref{ch:msk}) is a spectrally efficient variant of FSK that is of great theoretical and practical importance.
        \item[Bluetooth and LoRa] (\Cref{ch:iot}) provides a deep dive into the practical application of GFSK and Chirp Spread Spectrum in two of the most successful IoT communication standards.
        \item[Bit Error Rate] (\Cref{ch:ber}) provides the detailed performance curves that allow for a direct comparison of FSK's power efficiency against other modulation schemes.
    \end{description}
\end{importantbox}
