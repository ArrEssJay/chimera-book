\section{Chimera DSP Wiki}\label{chimera-dsp-wiki}

Welcome to the \textbf{Chimera Digital Signal Processing Documentation
Wiki}!

This wiki provides a \textbf{comprehensive, first-principles approach}
to understanding wireless communications-\/-\/-from electromagnetic
theory through practical system design to cutting-edge research. Whether
you\textquotesingle re learning DSP fundamentals or exploring quantum
neuromodulation, this resource builds knowledge systematically.

\begin{quote}
\textbf{Note for All Readers}: Most wiki pages include a ``Plain English
Explainer'' section that breaks down complex concepts using everyday
analogies-\/-\/-no engineering background required!
\end{quote}

\begin{center}\rule{0.5\linewidth}{0.5pt}\end{center}

\subsection{\texorpdfstring{ Learning
Path}{ Learning Path}}\label{learning-path}

\textbf{New to wireless communications?} Follow the parts in order:

\textbf{Experienced engineer?} Jump to specific topics using the
navigation below.

\begin{center}\rule{0.5\linewidth}{0.5pt}\end{center}

\subsection{\texorpdfstring{ Part I: Electromagnetic
Fundamentals}{ Part I: Electromagnetic Fundamentals}}\label{part-i-electromagnetic-fundamentals}

\textbf{Build from Maxwell\textquotesingle s equations to antenna
theory}

\begin{itemize}
\tightlist
\item
  {[}{[}Maxwell\textquotesingle s-Equations-\&-Wave-Propagation{]}{]} -
  Foundation of all EM radiation
\item
  {[}{[}Electromagnetic-Spectrum{]}{]} - HF
  \$\textbackslash rightarrow\$ VHF \$\textbackslash rightarrow\$ UHF
  \$\textbackslash rightarrow\$ mmWave \$\textbackslash rightarrow\$ THz
  bands, applications, ionizing vs non-ionizing
\item
  {[}{[}Antenna-Theory-Basics{]}{]} - Gain, directivity, impedance,
  beamwidth, Friis equation
\item
  {[}{[}Wave-Polarization{]}{]} - Linear, circular, elliptical
  polarization, Faraday rotation, GPS RHCP
\item
  {[}{[}Power-Density-\&-Field-Strength{]}{]} - E/H fields, Poynting
  vector, RF safety, link budgets
\end{itemize}

\textbf{Prerequisites}: Basic calculus, physics \textbf{Learning goals}:
Understand EM waves as physical phenomena, antenna basics

\begin{center}\rule{0.5\linewidth}{0.5pt}\end{center}

\subsection{\texorpdfstring{ Part II: RF
Propagation}{ Part II: RF Propagation}}\label{part-ii-rf-propagation}

\textbf{How signals travel through real-world environments}

\begin{itemize}
\tightlist
\item
  {[}{[}Free-Space-Path-Loss-(FSPL){]}{]} - Friis equation and link
  budgets
\item
  {[}{[}Propagation-Modes-(Ground-Wave,-Sky-Wave,-Line-of-Sight){]}{]} -
  HF skywave, VHF LOS, radio horizon, Fresnel zones
\item
  {[}{[}Multipath-Propagation-\&-Fading-(Rayleigh,-Rician){]}{]} -
  Rayleigh/Rician fading, delay spread, Doppler, coherence bandwidth
\item
  {[}{[}Atmospheric-Effects-(Ionospheric,-Tropospheric){]}{]} -
  Ionospheric refraction/absorption,
  O\textbackslash textsubscript\{2\}/H\textbackslash textsubscript\{2\}O
  absorption, ducting, TEC
\item
  {[}{[}Weather-Effects-(Rain-Fade,-Fog-Attenuation){]}{]} - ITU rain
  model, C/Ku/Ka/V-band attenuation, climate zones, mitigation
\end{itemize}

\textbf{Prerequisites}: Part I \textbf{Learning goals}: Predict signal
strength, understand channel impairments

\begin{center}\rule{0.5\linewidth}{0.5pt}\end{center}

\subsection{\texorpdfstring{ Part III: Link Budget \& Channel
Modeling}{ Part III: Link Budget \& Channel Modeling}}\label{part-iii-link-budget-channel-modeling}

\textbf{Connecting transmitters to receivers}

\begin{itemize}
\tightlist
\item
  {[}{[}Link-Loss-vs-Noise{]}{]} - Distinguishing attenuation from
  additive noise
\item
  {[}{[}Signal-to-Noise-Ratio-(SNR){]}{]} - Key quality metric
\item
  {[}{[}Energy-Ratios-(Es-N0-and-Eb-N0){]}{]} - Symbol and bit energy
  ratios
\item
  {[}{[}Complete-Link-Budget-Analysis{]}{]} - System-level power budget,
  margins, availability
\item
  {[}{[}Noise-Sources-\&-Noise-Figure{]}{]} - Thermal noise, amplifier
  noise figure, cascade analysis
\item
  {[}{[}Additive-White-Gaussian-Noise-(AWGN){]}{]} - Fundamental channel
  model
\item
  {[}{[}Channel-Models-(Rayleigh-\&-Rician){]}{]} - Statistical fading
  models for mobile channels
\end{itemize}

\textbf{Prerequisites}: Part II \textbf{Learning goals}: Calculate link
budgets, model channel effects

\begin{center}\rule{0.5\linewidth}{0.5pt}\end{center}

\subsection{\texorpdfstring{ Part IV: Modulation
Theory}{ Part IV: Modulation Theory}}\label{part-iv-modulation-theory}

\textbf{Encoding information onto carriers (simple
\$\textbackslash rightarrow\$ complex)}

\subsubsection{Digital Modulation
Fundamentals}\label{digital-modulation-fundamentals}

\begin{itemize}
\tightlist
\item
  {[}{[}Baseband-vs-Passband-Signals{]}{]} -
  Upconversion/downconversion, IQ modulation, zero-IF receivers
\item
  {[}{[}On-Off-Keying-(OOK){]}{]} - Simplest modulation (carrier on/off)
\item
  {[}{[}Amplitude-Shift-Keying-(ASK){]}{]} - M-ary ASK, PAM-4, power
  efficiency vs spectral efficiency
\item
  {[}{[}Frequency-Shift-Keying-(FSK){]}{]} - Binary \& M-ary frequency
  switching, MSK, GMSK
\item
  {[}{[}Binary-Phase-Shift-Keying-(BPSK){]}{]} - Two-phase modulation,
  coherent detection, 3 dB better than OOK
\end{itemize}

\subsubsection{Advanced Modulation}\label{advanced-modulation}

\begin{itemize}
\tightlist
\item
  {[}{[}What-Are-Symbols{]}{]} - Fundamental building blocks
\item
  {[}{[}QPSK-Modulation{]}{]} - Quadrature Phase-Shift Keying (2
  bits/symbol)
\item
  {[}{[}8PSK-\&-Higher-Order-PSK{]}{]} - 8PSK, 16PSK, spectral
  efficiency vs error performance
\item
  {[}{[}Quadrature-Amplitude-Modulation-(QAM){]}{]} - 16QAM, 64QAM,
  256QAM, optimal 2D constellations
\item
  {[}{[}IQ-Representation{]}{]} - In-phase and Quadrature components
\item
  {[}{[}Constellation-Diagrams{]}{]} - Visualizing modulation schemes
\item
  {[}{[}Spectral-Efficiency-\&-Bit-Rate{]}{]} - Shannon limit,
  bits/sec/Hz, bandwidth-power tradeoff
\end{itemize}

\textbf{Prerequisites}: Part III \textbf{Learning goals}: Choose
modulation schemes, understand tradeoffs (spectral efficiency
vs.~robustness)

\begin{center}\rule{0.5\linewidth}{0.5pt}\end{center}

\subsection{\texorpdfstring{ Part V: Channel Coding \& Error
Control}{ Part V: Channel Coding \& Error Control}}\label{part-v-channel-coding-error-control}

\textbf{Protecting data from channel errors}

\subsubsection{Information Theory}\label{information-theory}

\begin{itemize}
\tightlist
\item
  {[}{[}Shannon\textquotesingle s-Channel-Capacity-Theorem{]}{]} -
  Fundamental limit of communication (C =
  B\$\textbackslash cdot\$log\textbackslash textsubscript\{2\}(1+SNR))
\item
  {[}{[}Hamming-Distance-\&-Error-Detection{]}{]} - Minimum distance,
  error detection/correction capability
\item
  {[}{[}Block-Codes-(Hamming,-BCH,-Reed-Solomon){]}{]} - Linear block
  codes, generator matrix, syndrome decoding
\item
  {[}{[}Convolutional-Codes-\&-Viterbi-Decoding{]}{]} - Trellis codes,
  maximum likelihood decoding
\item
  {[}{[}Turbo-Codes{]}{]} - Iterative decoding, near-Shannon performance
\end{itemize}

\subsubsection{Modern Codes}\label{modern-codes}

\begin{itemize}
\tightlist
\item
  {[}{[}Forward-Error-Correction-(FEC){]}{]} - General FEC concepts
\item
  {[}{[}LDPC-Codes{]}{]} - Low-Density Parity-Check codes (used in
  Chimera)
\item
  {[}{[}Bit-Error-Rate-(BER){]}{]} - Performance metric
\item
  {[}{[}Polar-Codes{]}{]} - Capacity-achieving codes, 5G control
  channels
\end{itemize}

\textbf{Prerequisites}: Part IV \textbf{Learning goals}: Design error
correction schemes, approach Shannon limit

\begin{center}\rule{0.5\linewidth}{0.5pt}\end{center}

\subsection{\texorpdfstring{ Part VI: Practical System
Design}{ Part VI: Practical System Design}}\label{part-vi-practical-system-design}

\textbf{End-to-end wireless systems}

\begin{itemize}
\tightlist
\item
  {[}{[}Signal-Chain-(End-to-End-Processing){]}{]} - Complete TX/RX
  pipeline (Chimera-specific)
\item
  {[}{[}Synchronization-(Carrier,-Timing,-Frame){]}{]} - Carrier
  recovery, symbol timing, frame sync
\item
  {[}{[}Channel-Equalization{]}{]} - ZF, MMSE, DFE, adaptive
  equalization
\item
  {[}{[}Real-World-System-Examples{]}{]} - WiFi 802.11, LTE, DVB-S2, GPS
  detailed analysis
\end{itemize}

\textbf{Prerequisites}: Parts IV-V \textbf{Learning goals}: Design
complete communication systems, debug real-world issues

\begin{center}\rule{0.5\linewidth}{0.5pt}\end{center}

\subsection{\texorpdfstring{ Part VII: Advanced
Topics}{ Part VII: Advanced Topics}}\label{part-vii-advanced-topics}

\textbf{Modern wireless techniques}

\begin{itemize}
\tightlist
\item
  {[}{[}OFDM-\&-Multicarrier-Modulation{]}{]} - Orthogonal
  frequency-division multiplexing, FFT/IFFT, cyclic prefix, PAPR, pilot
  subcarriers
\item
  {[}{[}Spread-Spectrum-(DSSS-FHSS){]}{]} - Direct sequence and
  frequency hopping, processing gain, GPS, Bluetooth, military
  applications
\item
  {[}{[}MIMO-\&-Spatial-Multiplexing{]}{]} - Multiple antennas, spatial
  multiplexing, beamforming, diversity, massive MIMO, WiFi/LTE/5G
\item
  {[}{[}Military-\&-Covert-Communications{]}{]} - LPI/LPD systems, GPS
  M-code, AESA radar, Link 16, FHSS SATCOM, covert channels
\item
  {[}{[}Adaptive-Modulation-\&-Coding-(AMC){]}{]} - Link adaptation, CQI
  feedback, HARQ, Shannon capacity tracking, LTE/5G
\item
  {[}{[}mmWave-\&-THz-Communications{]}{]} - 24-300 GHz propagation,
  beamforming requirements, 5G NR FR2, 6G sub-THz, automotive radar
\end{itemize}

\textbf{Prerequisites}: Part VI \textbf{Learning goals}: Understand
state-of-the-art wireless systems (5G, WiFi 6, satellite, military)

\begin{center}\rule{0.5\linewidth}{0.5pt}\end{center}

\subsection{\texorpdfstring{ Part VIII: Speculative \& Emerging
Topics}{ Part VIII: Speculative \& Emerging Topics}}\label{part-viii-speculative-emerging-topics}

\textbf{Frontier research: Quantum biology meets wireless engineering}

\textbf{Note}: This section explores speculative applications grounded
in cutting-edge research. Content clearly distinguishes established
science from theoretical extrapolation.

\subsubsection{A. Theoretical Framework}\label{a.-theoretical-framework}

\begin{itemize}
\tightlist
\item
  {[}{[}Hyper-Rotational-Physics-(HRP)-Framework{]}{]} - M-theory
  extension: consciousness-matter coupling via quantum coherence
\end{itemize}

\subsubsection{B. THz Technology \&
Biology}\label{b.-thz-technology-biology}

\begin{itemize}
\tightlist
\item
  {[}{[}Terahertz-(THz)-Technology{]}{]} - QCLs, applications,
  propagation, bioeffects
\item
  {[}{[}THz-Propagation-in-Biological-Tissue{]}{]} - Physics of THz wave
  propagation in biological tissue
\item
  {[}{[}THz-Bioeffects-Thermal-and-Non-Thermal{]}{]} - Biological
  effects of THz radiation
\end{itemize}

\subsubsection{C. Quantum Biology \&
Consciousness}\label{c.-quantum-biology-consciousness}

\begin{itemize}
\tightlist
\item
  {[}{[}Microtubule-Structure-and-Function{]}{]} - Microtubule anatomy
  and quantum biology
\item
  {[}{[}Orchestrated-Objective-Reduction-(Orch-OR){]}{]} -
  Penrose-Hameroff quantum consciousness theory
\item
  {[}{[}Quantum-Coherence-in-Biological-Systems{]}{]} - Quantum
  coherence in biology
\item
  {[}{[}THz-Resonances-in-Microtubules{]}{]} - THz frequency resonances
  in microtubules
\end{itemize}

\subsubsection{D. Non-Linear Biological
Demodulation}\label{d.-non-linear-biological-demodulation}

\begin{itemize}
\tightlist
\item
  {[}{[}Non-Linear-Biological-Demodulation{]}{]} - Non-linear biological
  IMD and signal processing
\item
  {[}{[}Intermodulation-Distortion-in-Biology{]}{]} - Non-linear
  biological IMD
\item
  {[}{[}Acoustic-Heterodyning{]}{]} - Acoustic heterodyning in tissue
\item
  {[}{[}Frey-Microwave-Auditory-Effect{]}{]} - Frey effect: microwave
  auditory phenomenon
\item
  {[}{[}Biophysical-Coupling-Mechanism{]}{]} - Quantum coherence
  perturbation mechanism (CHIMERA field)
\end{itemize}

\subsubsection{E. Applied Case Study: HRP-Based THz
Neuromodulation}\label{e.-applied-case-study-hrp-based-thz-neuromodulation}

\begin{itemize}
\tightlist
\item
  {[}{[}AID-Protocol-Case-Study{]}{]} - Rigorous application of HRP
  framework to THz wireless neuromodulation
\end{itemize}

\textbf{Prerequisites}: Parts I-VII + open mind \textbf{Learning goals}:
Apply RF engineering to novel scenarios, practice interdisciplinary
thinking, distinguish speculation from established science

\begin{center}\rule{0.5\linewidth}{0.5pt}\end{center}

\subsection{\texorpdfstring{ Chimera
Implementation}{ Chimera Implementation}}\label{chimera-implementation}

\textbf{How Chimera applies these concepts}

Chimera is a browser-based DSP simulator implementing: -
\textbf{Modulation}: QPSK (see {[}{[}QPSK-Modulation{]}{]}) -
\textbf{Channel}: AWGN (see
{[}{[}Additive-White-Gaussian-Noise-(AWGN){]}{]}) - \textbf{FEC}: LDPC
codes (see {[}{[}LDPC-Codes{]}{]}) - \textbf{Visualization}: Real-time
constellation diagrams, BER analysis - \textbf{Goal}: Learn wireless
communications interactively

\subsubsection{Chimera-Specific Pages}\label{chimera-specific-pages}

\begin{itemize}
\tightlist
\item
  {[}{[}Signal-Chain-(End-to-End-Processing){]}{]} -
  Chimera\textquotesingle s TX/RX pipeline
\end{itemize}

\begin{center}\rule{0.5\linewidth}{0.5pt}\end{center}

\subsection{\texorpdfstring{ Practical Guides (Coming
Soon)}{ Practical Guides (Coming Soon)}}\label{practical-guides-coming-soon}

\begin{itemize}
\tightlist
\item
  Reading the Constellation - Interpreting TX/RX scatter plots
\item
  Understanding BER Curves - Performance analysis
\item
  Tuning Parameters - Optimizing SNR and link loss settings
\item
  Building Your First Link - Step-by-step tutorial
\end{itemize}

\begin{center}\rule{0.5\linewidth}{0.5pt}\end{center}

\subsection{\texorpdfstring{ Recommended
Textbooks}{ Recommended Textbooks}}\label{recommended-textbooks}

\subsubsection{Undergraduate Level}\label{undergraduate-level}

\begin{itemize}
\tightlist
\item
  \textbf{Proakis \& Salehi}, \emph{Digital Communications} (5th ed.)
\item
  \textbf{Haykin}, \emph{Communication Systems} (5th ed.)
\item
  \textbf{Sklar}, \emph{Digital Communications: Fundamentals and
  Applications}
\end{itemize}

\subsubsection{Graduate Level}\label{graduate-level}

\begin{itemize}
\tightlist
\item
  \textbf{Tse \& Viswanath}, \emph{Fundamentals of Wireless
  Communication}
\item
  \textbf{Goldsmith}, \emph{Wireless Communications}
\item
  \textbf{Richardson \& Urbanke}, \emph{Modern Coding Theory}
\end{itemize}

\subsubsection{Quantum Biology (Part
VIII)}\label{quantum-biology-part-viii}

\begin{itemize}
\tightlist
\item
  \textbf{Penrose}, \emph{The Emperor\textquotesingle s New Mind}
  (Orch-OR origins)
\item
  \textbf{Al-Khalili \& McFadden}, \emph{Life on the Edge: The Coming
  Age of Quantum Biology}
\end{itemize}

\textbf{ For complete bibliography}: See {[}{[}Bibliography{]}{]}
(includes 60+ references, standards, and online resources)

\begin{center}\rule{0.5\linewidth}{0.5pt}\end{center}

\subsection{\texorpdfstring{ External
Resources}{ External Resources}}\label{external-resources}

\subsubsection{Signal Databases \&
References}\label{signal-databases-references}

\begin{itemize}
\tightlist
\item
  \href{https://www.sigidwiki.com/wiki/Signal_Identification_Guide}{Signal
  Identification Wiki (sigidwiki)} - Comprehensive RF signal database
\item
  \href{https://gssc.esa.int/navipedia/}{Navipedia (ESA)} - GNSS/GPS
  encyclopedia
  (\href{https://gssc.esa.int/navipedia/index.php?title=GALILEO_Signal_Plan}{Galileo
  Signal Plan})
\item
  \href{https://www.gps.gov/}{GPS.gov} - Official U.S. GPS information
\item
  \href{https://www.radioreference.com/}{RadioReference} - Frequency
  allocations database
\end{itemize}

\subsubsection{Tools \& Software}\label{tools-software}

\begin{itemize}
\tightlist
\item
  \href{https://www.gnuradio.org/}{GNURadio} - Open-source SDR toolkit
\item
  \href{https://www.rfcafe.com/}{RF Café} - RF calculators and
  references
\item
  \href{https://www.dsprelated.com/}{DSP Related} - DSP tutorials and
  articles
\end{itemize}

\subsubsection{Organizations \& Learning}\label{organizations-learning}

\begin{itemize}
\tightlist
\item
  \href{https://www.comsoc.org/}{IEEE Communications Society} -
  Professional organization
\item
  \href{https://wirelesspi.com/}{Wireless Pi} - Educational resources
\item
  \href{https://ocw.mit.edu/courses/6-450-principles-of-digital-communications-i-fall-2006/}{MIT
  OCW: Digital Communications} - Free online course
\end{itemize}

\textbf{ Full resource list}: See {[}{[}Bibliography{]}{]} for 60+
references including standards (ITU-R, 3GPP, IEEE), research papers, and
tools

\begin{center}\rule{0.5\linewidth}{0.5pt}\end{center}

\subsection{\texorpdfstring{ Navigation
Tips}{ Navigation Tips}}\label{navigation-tips}

\textbf{Linear learning}: Follow Parts I \$\textbackslash rightarrow\$
VIII in order (builds knowledge systematically)

\textbf{Topic-based}: Use search or browse sidebar alphabetically

\textbf{Chimera users}: Start with
{[}{[}Signal-Chain-(End-to-End-Processing){]}{]}, then explore
referenced topics

\textbf{Visual learners}: Look for pages with diagrams:
{[}{[}Constellation-Diagrams{]}{]}, {[}{[}IQ-Representation{]}{]},
{[}{[}QPSK-Modulation{]}{]}

\textbf{Theory enthusiasts}: Jump to Parts V (Coding Theory) or VIII
(Quantum Biology)

\begin{center}\rule{0.5\linewidth}{0.5pt}\end{center}

\subsection{\texorpdfstring{ Technical
Appendices}{ Technical Appendices}}\label{technical-appendices}

Full technical documents included in this wiki:

\begin{itemize}
\tightlist
\item
  {[}{[}hrp\_framework\_paper{]}{]} - Hyper-Rotational Physics Framework
  (Jones, 2025) - M-theory extension for consciousness-matter coupling
\item
  {[}{[}aid\_protocol\_v3.1{]}{]} - AID Protocol Technical Specification
  v3.1 - THz neuromodulation system design
\item
  {[}{[}biophysical\_coupling\_mechanism{]}{]} - Quantum Coherence
  Perturbation Mechanism - Detailed coupling physics
\item
  {[}{[}turing\_cage{]}{]} - Turing Cage Analysis - Computational
  neuroscience framework
\end{itemize}

These documents provide the theoretical foundation for Part VIII
(Speculative \& Emerging Topics).

\begin{center}\rule{0.5\linewidth}{0.5pt}\end{center}

\subsection{\texorpdfstring{ Wiki
Development}{ Wiki Development}}\label{wiki-development}

\begin{itemize}
\tightlist
\item
  {[}{[}TODO{]}{]} - Planned wiki pages and future content (16
  placeholder pages tracked)
\item
  {[}{[}Wiki-Maintenance-Report{]}{]} - Wiki maintenance procedures and
  validation tools
\item
  {[}{[}Bibliography{]}{]} - Comprehensive reference list (60+ sources)
\end{itemize}
