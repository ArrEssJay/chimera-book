% =-============================================================================
% CHAPTER 25: Atmospheric Effects on Propagation
% ==============================================================================

\chapter{Atmospheric Effects on Propagation}
\label{ch:atmospheric}

\begin{nontechnical}
    \textbf{Earth's atmosphere acts like a giant, invisible, and ever-changing lens for radio waves.} Just as fog can scatter a flashlight beam and a prism can bend it, the atmosphere profoundly alters the path and strength of a radio signal.

    \parhead{The Two Key Layers}
    \begin{itemize}
        \item \textbf{The Ionosphere (60--1000 km up):} This is a layer of electrically charged gas, created by the sun's radiation. For lower-frequency signals (like AM and shortwave radio), it acts like a giant, curved mirror in the sky, bouncing signals back to Earth and enabling intercontinental communication. For higher-frequency signals (like GPS), it acts like a thick pane of glass, slowing the signal down and causing significant positioning errors if not corrected.
        \item \textbf{The Troposphere (where we live):} This is the lower atmosphere, where weather happens. For high-frequency signals (like satellite TV and 5G), the water vapour, oxygen, and raindrops in the troposphere can absorb and scatter the signal, causing it to weaken. This is why a heavy rainstorm can cause your satellite TV to go out.
    \end{itemize}

    \parhead{Real-world impact}
    \begin{itemize}
        \item \textbf{GPS Accuracy:} The biggest error source for a standard GPS receiver is the delay caused by the ionosphere. Your phone uses a model of the ionosphere to correct for this, reducing the error from over 10 meters to just a few.
        \item \textbf{Rain Fade:} The primary challenge for high-frequency satellite communication is designing the link with enough extra power (a "link margin") to burn through the attenuation caused by heavy rain.
        \item \textbf{AM Radio at Night:} The reason you can hear distant AM stations at night is that the lower, absorbing layers of the ionosphere disappear after sunset, allowing the signals to reach the higher, reflective layers.
    \end{itemize}
\end{nontechnical}


\subsection{Overview}

While the free-space path loss model assumes a vacuum, the Earth's atmosphere introduces a range of complex, frequency-dependent effects that can either enable or impair a communication link. These effects are broadly divided based on the atmospheric layer in which they occur.

\begin{keyconcept}
    The atmosphere has a dual, frequency-dependent nature:
    \begin{itemize}
        \item The \textbf{ionosphere} is critical for enabling long-distance communication below $\sim$30 MHz but is a major source of error for satellite navigation systems.
        \item The \textbf{troposphere} is largely transparent below $\sim$10 GHz but becomes a major source of attenuation and fading for higher-frequency microwave and millimetre-wave systems.
    \end{itemize}
\end{keyconcept}


\subsection{Ionospheric Effects}

The ionosphere is a region of the upper atmosphere that is ionized by solar radiation. This plasma has a refractive index less than one, which causes radio waves to bend.

\paragraph{Refraction and Reflection (Skywave)}
As discussed in \Cref{ch:propagation-modes}, this refraction is the mechanism behind \keyterm{skywave} propagation in the HF band (3-30 MHz). The maximum frequency that can be returned to Earth is the \keyterm{Maximum Usable Frequency (MUF)}, which depends on the electron density of the ionosphere and the angle of the signal path.

\paragraph{Faraday Rotation}
The presence of the Earth's magnetic field makes the ionosphere an anisotropic medium, causing the polarization plane of a linearly polarized wave to rotate. This \keyterm{Faraday rotation} is proportional to $1/f^2$ and can cause severe signal loss if the receiver antenna is not aligned with the rotated polarization.
\begin{warningbox}
    Faraday rotation is the primary reason why satellite communication systems operating at L-band, S-band, and C-band (e.g., GPS, Inmarsat) use \textbf{circular polarization}, which is immune to this effect.
\end{warningbox}

\paragraph{Ionospheric Delay}
The group velocity of a radio wave in a plasma is less than the speed of light in a vacuum. This introduces a propagation delay that is proportional to the \keyterm{Total Electron Content (TEC)} along the path and inversely proportional to the frequency squared ($1/f^2$).
\begin{equation}
    \Delta t = \frac{40.3 \times \text{TEC}}{c f^2} \quad (\text{seconds})
\end{equation}
This delay is the largest source of error in single-frequency satellite navigation systems like GPS. A typical daytime TEC can cause a ranging error of 5-15 meters at the GPS L1 frequency. This error is mitigated in dual-frequency receivers by measuring the delay at two different frequencies and solving for the TEC.


\subsection{Tropospheric Effects}

The troposphere is the neutral, lower part of the atmosphere. Its effects become dominant at frequencies above about 1 GHz.

\paragraph{Atmospheric Absorption}
Gaseous molecules in the atmosphere, primarily oxygen (O$_2$) and water vapour (H$_2$O), have resonant absorption lines that cause signal attenuation.
\begin{itemize}
    \item A strong \textbf{water vapour absorption peak} exists around \qty{22.4}{GHz}.
    \item A very strong \textbf{oxygen absorption peak} exists around \qty{60}{GHz}, causing extreme attenuation of up to 15 dB/km.
\end{itemize}
System designers must operate in the "atmospheric windows" between these absorption peaks.

\paragraph{Rain Attenuation (Rain Fade)}
At frequencies above 10 GHz, raindrops are comparable in size to the signal's wavelength. They absorb and scatter the radio energy, causing significant attenuation known as \keyterm{rain fade}. This is the most critical impairment for satellite links operating at Ku-band (12-18 GHz) and Ka-band (26.5-40 GHz). Link budgets for these systems must include a \keyterm{rain margin} of several dB to ensure high availability during heavy rainfall.

\paragraph{Refraction and Ducting}
Gradients in temperature, pressure, and humidity cause the troposphere's refractive index to decrease with altitude. This causes radio waves to bend slightly downwards, extending the radio horizon beyond the geometric horizon. Under certain conditions, such as a temperature inversion, a \keyterm{tropospheric duct} can form. This duct acts like a waveguide, trapping signals and enabling them to travel hundreds or even thousands of kilometres, far beyond the normal line-of-sight range.


\begin{workedexample}{Satellite Link Atmospheric Loss Budget}
    \parhead{Problem} Calculate the total atmospheric loss for a Ku-band satellite downlink and determine the required clear-sky link margin.
    \parhead{System Parameters}
    \begin{itemize}
        \item Frequency: \qty{12}{GHz} (Ku-band)
        \item Location: A temperate region where heavy rain requires a 5 dB rain margin for 99.9\% availability.
        \item Elevation Angle: 30$^\circ$.
    \end{itemize}
    \parhead{Analysis}
    \begin{derivationsteps}
        \step Calculate the clear-air gaseous absorption. At 12 GHz, the specific attenuation at sea level is about 0.05 dB/km. Over a slant path through the atmosphere, this integrates to a total zenith loss of about 0.4 dB. For a 30$^\circ$ elevation angle, the path length increases.
        \[ L_{\text{gas}} = \frac{L_{\text{zenith}}}{\sin(\theta)} = \frac{0.4 \text{ dB}}{\sin(30^\circ)} = \qty{0.8}{dB} \]
        \step Add other clear-air losses. Tropospheric scintillation (rapid fluctuations) and cloud attenuation might add another 0.5 dB.
        \[ L_{\text{clear-sky}} = L_{\text{gas}} + L_{\text{scint/cloud}} = 0.8 + 0.5 = \qty{1.3}{dB} \]
        \step Add the required rain margin. The problem specifies a 5 dB margin is needed to overcome rain fade for the target availability.
        \[ \text{Total Atmospheric Loss Budget} = L_{\text{clear-sky}} + L_{\text{rain}} = 1.3 + 5.0 = \qty{6.3}{dB} \]
    \end{derivationsteps}
    \parhead{Interpretation} The link budget for this satellite system must have a \textbf{clear-sky margin of at least 6.3 dB}. This means that on a clear day, the received signal must be 6.3 dB stronger than the minimum required for operation. This extra power is held in reserve to be "spent" on overcoming the 5 dB of rain fade during a storm, ensuring the link remains available 99.9% of the time.
\end{workedexample}

\begin{table}[H]
    \centering
    \caption{Summary of Dominant Atmospheric Effects by Frequency Band}
    \label{tab:atmospheric-effects-summary}
    \begin{tabular}{@{}lll@{}}
        \toprule
        \tableheaderfont Frequency Band & \tableheaderfont Dominant Layer & \tableheaderfont Primary Effect(s) \\
        \midrule
        MF / HF (<30 MHz) & Ionosphere & Refraction (Skywave), Absorption (D-Layer) \\
        VHF / UHF (30 MHz - 3 GHz) & Ionosphere & Group Delay (GPS Error), Scintillation \\
        SHF (3 - 30 GHz) & Troposphere & Rain Attenuation, Gaseous Absorption \\
        EHF (30 - 300 GHz) & Troposphere & Severe Gaseous Absorption (O$_2$, H$_2$O), Rain Fade \\
        \bottomrule
    \end{tabular
\end{table}


\begin{importantbox}[title={Further Reading}]
    The atmosphere is the final, and often most unpredictable, component of the wireless channel.
    \begin{description}
        \item[Propagation Modes] (\Cref{ch:propagation-modes}) provides the context for how ionospheric refraction enables the skywave propagation mode.
        \item[Weather Effects] (\Cref{ch:weather}) offers a more detailed quantitative analysis of rain fade and other precipitation-related impairments.
        \item[Link Budget Analysis] (\Cref{ch:linkbudget}) is where these atmospheric loss terms are incorporated into the end-to-end calculation of system performance.
    \end{description}
\end{importantbox}