% ==============================================================================
% CHAPTER 19: 8-PSK and Higher-Order PSK
% ==============================================================================

\chapter{8-PSK \& Higher-Order PSK}
\label{ch:8psk}

\begin{nontechnical}
    \textbf{8-PSK is like upgrading from four hand signals to eight}, allowing you to send 50\% more data with each gesture. The trade-off is that the gestures are now closer together and easier to misinterpret in a noisy environment.

    \parhead{The progression of PSK}
    \begin{itemize}
        \item \textbf{BPSK:} Uses two phase states (0$^\circ$, 180$^\circ$) to send 1 bit per symbol.
        \item \textbf{QPSK:} Uses four phase states (separated by 90$^\circ$) to send 2 bits per symbol.
        \item \textbf{8-PSK:} Uses eight phase states (separated by 45$^\circ$) to send 3 bits per symbol.
    \end{itemize}

    \parhead{The efficiency vs. robustness trade-off} By moving from QPSK to 8-PSK, you increase your data rate by 50\% for the same bandwidth. However, because the constellation points are now closer together, the system is more susceptible to noise. To achieve the same low error rate as QPSK, an 8-PSK system needs a signal that is about 3.5 dB stronger (more than twice the power).

    \parhead{Where it's used} 8-PSK is the perfect choice for \keyterm{bandwidth-limited} channels where power is available but spectrum is scarce. Its most prominent application is in \textbf{satellite television broadcasting (DVB-S2)}, where it allows broadcasters to fit more high-definition channels into a single transponder.
\end{nontechnical}


\subsection{Overview}

\keyterm{8-ary Phase-Shift Keying (8-PSK)} is a digital modulation technique that encodes three bits of information per symbol by shifting the phase of a carrier wave to one of eight equally spaced phase states. It is a \keyterm{constant-envelope} modulation, which makes it highly suitable for use with the non-linear power amplifiers commonly found in satellite transponders.

\begin{keyconcept}
    8-PSK represents a critical trade-off between spectral efficiency and power efficiency. It offers a 50\% increase in data throughput compared to QPSK but requires approximately 3.5 dB more $E_b/N_0$ to achieve the same Bit Error Rate. It is the workhorse modulation for bandwidth-limited satellite links.
\end{keyconcept}


\subsection{Mathematical Representation}

The 8-PSK signal is a carrier wave where the phase, $\phi_m$, is chosen from one of eight values based on the 3-bit input symbol:
\begin{equation}
    s_m(t) = A\cos(2\pi f_c t + \phi_m), \quad \text{where } \phi_m = \frac{m\pi}{4} \text{ for } m = 0, 1, \dots, 7
\end{equation}
In the complex baseband, this corresponds to eight points equally spaced on a circle in the IQ plane. The use of \keyterm{Gray coding} is standard, ensuring that adjacent constellation points (the most likely place for an error to occur) differ by only a single bit, thereby minimizing the bit error rate.


\subsection{Performance Characteristics}

\paragraph{Bit Error Rate (BER)}
The performance of 8-PSK is significantly worse than QPSK for a given $E_b/N_0$. This is a direct consequence of the reduced Euclidean distance between constellation points. The minimum distance for 8-PSK is only about 53\% of the minimum distance for QPSK, leading to a much higher susceptibility to noise.
\begin{equation}
    \text{BER}_{8\text{PSK}} \approx \frac{2}{3}Q\left(\sqrt{\frac{6E_b}{N_0}} \sin\left(\frac{\pi}{8}\right)\right)
\end{equation}

\paragraph{Spectral Efficiency}
By encoding 3 bits per symbol, 8-PSK achieves a spectral efficiency of:
\begin{equation}
    \eta = \frac{R_b}{B} = \frac{3}{1+\alpha} \quad (\text{bps/Hz})
\end{equation}
For a typical $\alpha=0.25$, this yields an efficiency of 2.4 bps/Hz, a 50\% improvement over QPSK's 1.6 bps/Hz.

\begin{table}[H]
    \centering
    \caption{Performance Comparison of PSK Schemes (for BER $10^{-6}$)}
    \label{tab:psk-comparison}
    \begin{tabular}{@{}lccc@{}}
        \toprule
        \tableheaderfont Modulation & \tableheaderfont Bits/Symbol & \tableheaderfont Spectral Eff. ($\eta$) & \tableheaderfont Required $E_b/N_0$ \\
        \midrule
        BPSK & 1 & 0.8 bps/Hz & 10.5 dB \\
        QPSK & 2 & 1.6 bps/Hz & 10.5 dB \\
        \textbf{8-PSK} & 3 & 2.4 bps/Hz & \textbf{14.0 dB} \\
        16-PSK & 4 & 3.2 bps/Hz & 18.5 dB \\
        \bottomrule
    \end{tabular}
    \par\vspace{0.5em}
    \small Note: Spectral efficiency calculated with a typical roll-off factor of $\alpha=0.25$.
\end{table}


\subsection{The Limit of PSK: Why QAM is Preferred for M>8}

As the table above shows, the required power for M-PSK increases dramatically as M gets larger. The constellation points on the unit circle simply get too crowded. For M=16 and beyond, a two-dimensional constellation like \keyterm{Quadrature Amplitude Modulation (QAM)}, which varies both amplitude and phase, provides a much greater Euclidean distance between points for the same average power.
\begin{warningbox}
    For modulation orders greater than 8, \textbf{QAM is almost always superior to PSK} in RF systems. 16-QAM requires approximately 4 dB less power than 16-PSK to achieve the same BER. This is why high-speed systems like WiFi and 5G use 16-QAM, 64-QAM, and 256-QAM, not 16-PSK or higher.
\end{warningbox}


\begin{workedexample}{DVB-S2 Throughput Analysis}
    \parhead{Problem} A satellite operator has a 36 MHz transponder. Compare the maximum data rate they can achieve using QPSK vs. 8-PSK.
    \parhead{System Parameters}
    \begin{itemize}
        \item Transponder Bandwidth: \qty{36}{MHz}
        \item Pulse Shaping Roll-off ($\alpha$): 0.25
        \item FEC Code Rate: 3/4 (meaning 3 information bits for every 4 transmitted bits)
    \end{itemize}
    \parhead{Solution}
    \begin{derivationsteps}
        \step Calculate the maximum symbol rate ($R_s$) the channel can support.
        \[ R_s = \frac{B}{1+\alpha} = \frac{36 \times 10^6}{1.25} = 28.8 \text{ Msymbols/s} \]
        \step Calculate the information bit rate ($R_b$) for QPSK.
        \[ R_b = R_s \times (\text{bits/symbol}) \times (\text{code rate}) = (28.8 \times 10^6) \times 2 \times (3/4) = \textbf{\qty{43.2}{Mbps}} \]
        \step Calculate the information bit rate ($R_b$) for 8-PSK.
        \[ R_b = R_s \times (\text{bits/symbol}) \times (\text{code rate}) = (28.8 \times 10^6) \times 3 \times (3/4) = \textbf{\qty{64.8}{Mbps}} \]
    \end{derivationsteps}
    \parhead{Interpretation} By switching from QPSK to 8-PSK, the satellite operator can increase the data throughput on the same transponder from 43.2 Mbps to 64.8 Mbps---a 50\% increase. This allows them to either offer a higher quality video service or fit more channels into the same bandwidth. This significant gain in spectral efficiency is why 8-PSK is a cornerstone of the DVB-S2 standard, despite its higher SNR requirement.
\end{workedexample}


\begin{importantbox}[title={Further Reading}]
    8-PSK is an important step in the progression towards higher spectral efficiency, and it highlights the critical trade-offs in modulation design.
    \begin{description}
        \item[Quadrature Amplitude Modulation (QAM)] (\Cref{ch:qam}) is the logical next step for achieving higher data rates, and the standard for most modern high-speed systems.
        \item[DVB-S2 Standard] (\Cref{ch:dvbs2}) provides a deep dive into a real-world system that uses adaptive modulation, switching between QPSK, 8-PSK, 16-APSK, and 32-APSK based on channel conditions.
        \item[Link Budget Analysis] (\Cref{ch:linkbudget}) demonstrates how the 3.5 dB SNR penalty of 8-PSK must be accounted for in system design, often requiring a larger receiving antenna or higher transmit power.
    \end{description}
\end{importantbox}