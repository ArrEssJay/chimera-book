% ==============================================================================
% CHAPTER 43: The Signal Chain: End-to-End Processing
% ==============================================================================

\chapter{The Signal Chain: End-to-End Processing}
\label{ch:signal-chain}

\begin{nontechnical}
    \textbf{The signal chain is the complete journey your data takes from sender to receiver.} It's like a highly advanced, automated postal system for information.

    \parhead{The postal analogy}
    \begin{enumerate}
        \item \textbf{Writing and Packaging (Transmitter):} You write a letter (your data). You put it in a special envelope with error-correcting codes written on it (FEC encoding). You write the address in a specific format (modulation).
        \item \textbf{Mailing (Transmission):} The post office (upconverter and antenna) sends your package on its way via aeroplane (radio waves).
        \item \textbf{The Journey (The Channel):} The aeroplane flies through a storm. The package might get wet, jostled, or delayed (noise, fading, and multipath).
        \item \textbf{Receiving and Unpacking (Receiver):} The package arrives. The receiver uses the address to identify it (demodulation), uses the special error-correcting codes to fix any damage (FEC decoding), and finally opens the envelope to read your original letter.
    \end{enumerate}
    
    \parhead{Why it's so complex} Every step in this chain must be performed perfectly. A mistake at any stage—a misread address, a failed error check, a timing error—can cause the entire message to be lost. A modern WiFi signal goes through more than a dozen of these processing stages in a fraction of a second to deliver a single emoji to your screen.
\end{nontechnical}


\section{Overview and Properties}

\subsection{Overview}

The \keyterm{signal chain} is a conceptual model that describes the complete sequence of processing stages a signal undergoes, from the information source at the transmitter to the information sink at the receiver. It provides a systematic framework for understanding how data is prepared for transmission, the impairments it faces during propagation, and the steps required to recover it.

\begin{keyconcept}
    A digital communication system is a cascade of transformations. The transmitter converts abstract bits into a physical, analogue waveform. The channel distorts this waveform and adds noise. The receiver's task is to perform the inverse transformations, combatting the channel's effects to recover the original bits with the lowest possible error rate. Every stage in the chain either adds value (like coding gain) or introduces an impairment (like noise or loss).
\end{keyconcept}

The signal chain can be divided into three main segments: the transmitter, the channel, and the receiver.


\subsection{The Transmitter Chain}

The transmitter's role is to convert a stream of information bits into a robust, high-frequency analogue signal suitable for radiation by an antenna.

\begin{description}
    \item[1. Source Coding (Optional)] Compresses the raw data to remove redundancy (e.g., MP3, JPEG).
    \item[2. Forward Error Correction (FEC) Encoding] Adds structured redundancy to the data stream. This is the "armour" that will allow the receiver to correct errors introduced by the channel.
    \item[3. Modulation] Maps the coded bits to complex symbols ($I+jQ$) on a constellation diagram (e.g., QPSK, 16-QAM).
    \item[4. Pulse Shaping and Upconversion] The baseband IQ signal is filtered to control its bandwidth and then upconverted to the final radio frequency (RF) by an IQ modulator.
    \item[5. Power Amplification] A Power Amplifier (PA) boosts the low-power RF signal to the required level for transmission.
    \item[6. Transmit Antenna] Converts the electrical signal into a propagating electromagnetic wave.
\end{description}


\subsection{The Channel}

The channel is the physical medium through which the signal travels. It is the most unpredictable part of the chain and introduces all the major impairments.
\begin{itemize}
    \item \textbf{Path Loss:} The deterministic weakening of the signal due to geometric spreading (FSPL) and absorption.
    \item \textbf{Multipath Fading:} Constructive and destructive interference from signal reflections, causing rapid fluctuations in signal strength.
    \item \textbf{Noise:} The addition of unwanted, random energy, primarily thermal noise at the receiver front-end.
    \item \textbf{Doppler Shift:} Frequency shifts caused by relative motion between the transmitter and receiver.
\end{itemize}


\subsection{The Receiver Chain}

The receiver's role is to perform the inverse operations of the transmitter, combatting the channel impairments to recover the original data.

\begin{description}
    \item[1. Receive Antenna] Captures the faint electromagnetic wave and converts it back into an electrical signal.
    \item[2. Low-Noise Amplification (LNA)] The first, and most critical, stage of amplification. A high-quality LNA boosts the weak signal while adding the minimum possible amount of its own internal noise.
    \item[3. Downconversion] An IQ demodulator shifts the high-frequency RF signal back down to a low-frequency or zero-IF complex baseband signal ($I+jQ$).
    \item[4. Analogue-to-Digital Conversion (ADC)] The analogue IQ signals are sampled and quantised into a stream of digital numbers.
    \item[5. Synchronisation] A suite of DSP algorithms that perform the critical tasks of frame detection, timing recovery, and carrier frequency/phase recovery.
    \item[6. Equalisation] An adaptive digital filter that compensates for the ISI caused by multipath fading.
    \item[7. Demodulation] The equalised symbols are mapped back to soft bit estimates (LLRs).
    \item[8. FEC Decoding] The FEC decoder uses the redundant parity bits and the soft information from the demodulator to detect and correct bit errors, recovering the original information stream.
\end{description}


\begin{workedexample}{End-to-End WiFi Link Analysis}
    \parhead{Problem} Trace a data packet through a simplified WiFi link and analyse the key transformations.
    \parhead{System Parameters}
    \begin{itemize}
        \item Data: 1000 bits of user data.
        \item Link: 20 MHz channel, 64-QAM, rate-3/4 LDPC code.
        \item Channel: Indoor with multipath, received SNR is 25 dB.
    \end{itemize}
    
    \parhead{Signal Journey}
    \begin{derivationsteps}
        \step \textbf{Transmitter.} The 1000 data bits are encoded by the rate-3/4 LDPC encoder, resulting in $1000 / (3/4) \approx 1333$ coded bits. These are mapped to $1333 / 6 \approx 222$ 64-QAM symbols. These complex symbols are used to modulate the 48 data subcarriers of several OFDM symbols, which are then upconverted to 2.4 GHz and transmitted.
        
        \step \textbf{Channel.} The signal travels 10 metres, incurring about 60 dB of free-space path loss. It reflects off a wall, creating an echo that arrives 20 ns later. The received signal is the sum of the direct path and the weaker, delayed echo, plus the receiver's thermal noise. The final SNR is 25 dB.
        
        \step \textbf{Receiver.} The ADC samples the incoming signal. The synchroniser detects the packet preamble and corrects for frequency and timing offsets. The FFT converts the signal to the frequency domain. The channel equaliser uses the pilot tones to estimate the channel's frequency response and divides it out, correcting for the multipath distortion on each subcarrier. The 222 cleaned 64-QAM symbols are demodulated into soft-bit LLRs. The LDPC decoder takes these $\sim$1333 LLRs and, after several iterations, corrects the few bit errors caused by the residual noise, perfectly recovering the original 1000 data bits. The entire process takes less than a millisecond.
    \end{derivationsteps}
\end{workedexample}


\begin{importantbox}[title={Further Reading}]
    This chapter serves as a high-level map of the entire communication process, connecting the detailed concepts explored in other chapters.
    \begin{description}
        \item[Link Budget Analysis] (\Cref{ch:linkbudget}) is the quantitative application of the signal chain model, assigning a dB value to every gain and loss along the path.
        \item[OFDM] (\Cref{ch:ofdm}) is the practical framework that contains many of these stages, including modulation, IFFT/FFT, and the cyclic prefix for equalisation.
        \item[Forward Error Correction] (\Cref{ch:fec}) and \textbf{Synchronisation} (\Cref{ch:synchronisation}) are the two most critical DSP blocks in the receiver chain, responsible for correcting errors and aligning the receiver, respectively.
    \end{description}
\end{importantbox}
