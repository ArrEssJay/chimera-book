% ==============================================================================
% CHAPTER 21: Free-Space Path Loss (FSPL)
% ==============================================================================

\chapter{Free-Space Path Loss (FSPL)}
\label{ch:fspl}

\begin{nontechnical}
    \textbf{Free-Space Path Loss is like the sound of your voice spreading out as you shout across an open field.} The further away your friend is, the quieter your voice becomes. This isn't because the air "absorbs" the sound, but because the sound energy is spread over a much larger area. Radio waves behave in exactly the same way.

    \parhead{The two golden rules}
    \begin{itemize}
        \item \textbf{Double the distance $\rightarrow$ the signal becomes four times weaker.} This is the famous inverse square law.
        \item \textbf{Double the frequency $\rightarrow$ the signal becomes four times weaker.} This is less intuitive, but it's a critical reason why lower-frequency 4G signals travel further than higher-frequency 5G signals.
    \end{itemize}

    \parhead{A mind-boggling example} The signal from a GPS satellite, 20,000 km away, is weakened by a factor of about one quadrillion (a one followed by 15 zeros) by the time it reaches your phone. The entire science of radio engineering is a battle against this fundamental geometric loss.
\end{nontechnical}


\subsection{Overview}

\keyterm{Free-Space Path Loss (FSPL)} is the loss in signal strength that an electromagnetic wave experiences as it propagates through a direct line-of-sight path in free space. It is a purely geometric effect caused by the spreading of the wave over an ever-increasing spherical surface area. FSPL is not a true "loss" of energy but rather a reduction in power density at the receiver.

\begin{keyconcept}
    FSPL is the largest and most fundamental loss component in any wireless link budget. It increases by \textbf{20 dB per decade of distance} and \textbf{20 dB per decade of frequency}. Understanding and accurately calculating FSPL is the first and most critical step in designing any viable wireless communication system.
\end{keyconcept}


\subsection{The Friis Transmission Equation}

The relationship between transmitted and received power is defined by the \keyterm{Friis transmission equation}:
\begin{equation}
    P_r = P_t G_t G_r \left(\frac{\lambda}{4\pi d}\right)^2
    \label{eq:friis}
\end{equation}
where $P_r$ is the received power, $P_t$ is the transmitted power, $G_t$ and $G_r$ are the linear antenna gains, $\lambda$ is the wavelength, and $d$ is the distance. The term $(\lambda / 4\pi d)^2$ represents the free-space path loss.


\subsection{Practical FSPL Formulas}

By converting the Friis equation to decibels and using common engineering units, we arrive at more practical formulas.

\paragraph{Kilometres and Megahertz}
This is the most common formula used in terrestrial and satellite link budgets.
\begin{equation}
    \text{FSPL}_{\text{dB}} = 20\log_{10}(d_{\text{km}}) + 20\log_{10}(f_{\text{MHz}}) + 32.45
    \label{eq:fspl-eng}
\end{equation}
where $d$ is in kilometres and $f$ is in Megahertz.

\paragraph{Meters and Gigahertz}
This formula is useful for shorter-range systems like WiFi or 5G mmWave.
\begin{equation}
    \text{FSPL}_{\text{dB}} = 20\log_{10}(d_{\text{m}}) + 20\log_{10}(f_{\text{GHz}}) + 20.4
\end{equation}

\begin{warningbox}
    These formulas are valid only under ideal conditions: a direct line-of-sight path in a vacuum with no obstacles, reflections, or atmospheric absorption. In any real-world scenario, the actual path loss will \textbf{always be greater} than the FSPL due to these additional factors. FSPL represents the best-case, theoretical minimum loss.
\end{warningbox}


\begin{workedexample}{FSPL Across Different Scales}
    \parhead{Problem} Calculate the FSPL for three common wireless scenarios.
    
    \parhead{Case 1: WiFi Link}
    \begin{itemize}
        \item Distance: \qty{20}{m}
        \item Frequency: \qty{5}{GHz}
    \end{itemize}
    \[ \text{FSPL} = 20\log_{10}(20) + 20\log_{10}(5) + 20.4 = 26.0 + 14.0 + 20.4 = \textbf{\qty{60.4}{dB}} \]

    \parhead{Case 2: Cellular Link}
    \begin{itemize}
        \item Distance: \qty{2}{km}
        \item Frequency: \qty{1.8}{GHz} (\qty{1800}{MHz})
    \end{itemize}
    \[ \text{FSPL} = 20\log_{10}(2) + 20\log_{10}(1800) + 32.45 = 6.0 + 65.1 + 32.45 = \textbf{\qty{103.6}{dB}} \]
    
    \parhead{Case 3: Geostationary Satellite Link}
    \begin{itemize}
        \item Distance: \qty{36,000}{km}
        \item Frequency: \qty{12}{GHz} (\qty{12000}{MHz})
    \end{itemize}
    \[ \text{FSPL} = 20\log_{10}(36000) + 20\log_{10}(12000) + 32.45 = 91.1 + 81.6 + 32.45 = \textbf{\qty{205.2}{dB}} \]
    \parhead{Interpretation} These examples illustrate the immense range of path loss encountered in wireless systems, from around 60 dB for a short indoor link to over 200 dB for a satellite link—a difference of 140 dB, or a factor of ten trillion in signal power.
\end{workedexample}


\subsection{Real-World Propagation Models}

While FSPL is the theoretical baseline, practical path loss is predicted using more complex empirical or semi-empirical models that account for the environment.
\begin{description}
    \item[Path Loss Exponent Model] A simple modification to the FSPL, where loss increases as $d^n$ instead of $d^2$. The path loss exponent, $n$, is determined experimentally and varies with the environment (e.g., $n \approx 2$ for open space, but can be $n=4$ to $6$ in dense urban or indoor environments).
    \item[Okumura-Hata Model] An empirical model widely used for cellular network planning in urban areas, which adds correction factors to the FSPL for antenna height, city size, and terrain.
    \item[Ray Tracing Models] Sophisticated computer models that simulate the propagation of individual rays, including reflections, diffractions, and scattering from buildings and terrain. These are computationally intensive but provide the most accurate site-specific predictions.
\end{description}


\begin{importantbox}[title={Further Reading}]
    FSPL is the single largest term in any link budget and is the starting point for all practical wireless design.
    \begin{description}
        \item[Link Budget Analysis] (\Cref{ch:linkbudget}) is the full end-to-end calculation where FSPL is combined with transmit power, antenna gains, and receiver sensitivity to determine if a link will work.
        \item[Atmospheric Propagation] (\Cref{ch:propagation-modes}) describes the additional loss mechanisms beyond FSPL, such as rain fade and atmospheric absorption, which are critical for microwave and satellite links.
        \item[Fading Channels] (\Cref{ch:fading}) explains the rapid signal fluctuations caused by multipath reflections, which add a dynamic component on top of the static path loss.
    \end{description}
\end{importantbox}