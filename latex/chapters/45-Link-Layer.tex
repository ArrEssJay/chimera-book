% ==============================================================================
% CHAPTER 45: The Link Layer: Interfacing the Physical Medium
% ==============================================================================

\chapter{The Link Layer: Interfacing the Physical Medium}
\label{ch:link-layer}

\begin{nontechnical}
    Imagine the physical layer of a radio is like a vast, building-wide pneumatic tube system, capable of moving raw pieces of paper from one desk to another. It provides the raw transport, but with no rules. The Link Layer is the corporate mailroom clerk who makes this system useful.

    The clerk's job has four parts. First, they perform \keyterm{Framing}: they take a loose stack of papers (a data packet from the network layer) and put it into a standardised inter-office envelope, adding a header with delivery information and a trailer for verification.

    Second, they handle \keyterm{Addressing}. They write the recipient's specific desk number (their unique \keyterm{MAC address}) on the envelope. This is a local, physical address, distinct from the person's full name and international mailing address (the IP address).

    Third, and most importantly, they manage \keyterm{Media Access Control}. The clerk must look to see if the pneumatic tube is currently in use before sending a new envelope, to avoid collisions. This "listen-before-talk" protocol is the essence of how shared media like the airwaves are managed.

    Finally, the clerk handles \keyterm{Error Control}. They might write a checksum on the envelope. The recipient can recalculate the checksum to see if the envelope was smudged or damaged in transit. If it was, they can request a new copy. This error *detection* and retransmission is the fundamental role of the Link Layer, providing a reliable link over an unreliable physical medium.
\end{nontechnical}

\section{Overview and Properties}

\subsection{Overview: The Role of Layer 2}

In the standard OSI model of networking, the \keyterm{Link Layer} is Layer 2, situated directly above the Physical Layer (Layer 1). While the Physical Layer is concerned with the transmission of raw bits as a physical signal, the Link Layer is responsible for creating a reliable, point-to-point link between two devices on the same physical medium. It takes the chaotic, error-prone service provided by the physical layer and transforms it into a structured and reliable channel for the higher network layers.

\begin{keyconcept}
    The primary responsibility of the Link Layer is to manage the reliable transmission of data \textbf{frames} between adjacent nodes over a shared physical medium. Its core functions are Framing, Addressing, Media Access Control (MAC), and Error Control.
\end{keyconcept}

\subsection{Framing}

The first task of the Link Layer is to partition the raw bitstream from the physical layer into discrete blocks of data called \keyterm{frames}. A typical frame consists of three parts:
\begin{description}
    \item[Header] Contains control information, most importantly the source and destination addresses. It may also include sequence numbers and frame type identifiers.
    \item[Payload] Contains the actual data packet passed down from the network layer (e.g., an IP packet).
    \item[Trailer (or Footer)] Contains error-checking information, most commonly a \keyterm{Frame Check Sequence (FCS)} generated by a Cyclic Redundancy Check (CRC).
\end{description}
The Link Layer adds this header and trailer at the transmitter and strips them off at the receiver, passing only the clean payload up the network stack.

\subsection{Addressing (MAC Addresses)}

To deliver frames on a local, shared network, each device needs a unique identifier. This is the \keyterm{Media Access Control (MAC) address}, a 48-bit hardware address that is permanently burned into the network interface card (e.g., `00:1A:2B:3C:4D:5E`).

It is crucial to distinguish this from an IP address. The MAC address is a local, Layer 2 address used for delivering a frame on a single network segment (e.g., from your laptop to your Wi-Fi router). The IP address is a global, Layer 3 address used for routing a packet across the entire internet.

\subsection{Media Access Control (MAC)}

In most wireless systems, the radio spectrum is a shared medium; if two devices transmit at the same time, their signals will collide and interfere. The \keyterm{Media Access Control (MAC)} protocol is the set of rules that governs "who gets to talk, and when." There are two main approaches:
\begin{description}
    \item[Contention-Based Access] In this distributed approach, there is no central coordinator. Each device contends for access to the medium. The most famous example is \keyterm{Carrier Sense Multiple Access with Collision Avoidance (CSMA/CA)}, used in Wi-Fi. A device will "listen" to the channel to see if it is busy. If it is clear, it waits a random backoff time and then transmits. This is an efficient method for bursty, unpredictable traffic.
    \item[Scheduled (or Controlled) Access] In this centralised approach, a base station or access point acts as a controller, granting each user specific time slots or frequency resources in which to transmit. This avoids collisions entirely and is the method used in cellular systems like 4G and 5G to provide guaranteed quality of service.
\end{description}

\subsection{Error Control: Detection and Retransmission}

While the Physical Layer uses Forward Error Correction (FEC) to *correct* a certain number of bit errors, the Link Layer provides a second, higher level of reliability through error *detection* and *retransmission*.

\paragraph{Error Detection}
The transmitter calculates a \keyterm{Cyclic Redundancy Check (CRC)} over the entire frame and appends it as the Frame Check Sequence (FCS). The receiver recalculates the CRC on the received frame. If the calculated CRC does not match the received FCS, the frame is known to be corrupt and is discarded.

\paragraph{Automatic Repeat Request (ARQ)}
If a frame is successfully received (passes the CRC check), the receiver sends back a short acknowledgement (ACK) frame. If the transmitter does not receive an ACK within a certain timeout period (either because the original frame was lost or the ACK was lost), it assumes the transmission failed and automatically retransmits the frame. This simple stop-and-wait ARQ protocol guarantees reliable delivery. Modern systems use more sophisticated protocols like \keyterm{Hybrid ARQ (HARQ)}, which combines retransmissions with soft-combining at the FEC decoder to achieve extreme reliability and efficiency.

\begin{workedexample}{Calculating Link Layer Efficiency}
    \parhead{Problem}
    Calculate the protocol efficiency for sending a 200-byte data payload over a standard Wi-Fi (802.11) link.
    
    \parhead{Assumptions}
    The total size of the transmitted frame includes the data payload plus the Link Layer overhead.
    \begin{itemize}
        \item Wi-Fi MAC Header: 34 bytes
        \item Frame Check Sequence (FCS/CRC): 4 bytes
        \item Data Payload: 200 bytes
    \end{itemize}

    \parhead{Analysis}
    Protocol efficiency is the ratio of the useful payload data to the total number of bits that must be transmitted for that payload.
    
    \begin{derivationsteps}
        \step \textbf{Calculate the total frame size.}
        \[ \text{Total Frame Size} = \text{Header} + \text{Payload} + \text{FCS} = 34 + 200 + 4 = \mathbf{238~\text{bytes}} \]
        
        \step \textbf{Calculate the protocol efficiency.}
        \[ \text{Efficiency} = \frac{\text{Payload Size}}{\text{Total Frame Size}} = \frac{200}{238} \approx 0.84 \]
        Converting to a percentage, the efficiency is \textbf{84\%}.
    \end{derivationsteps}

    \parhead{Interpretation}
    In this case, 16% of the transmitted data is pure overhead from the Link Layer protocol (header and trailer). This does not even include the additional overhead from the Physical Layer (preambles, guard intervals). For very small packets (like a VoIP packet), this overhead can become much more significant, dominating the transmission and reducing the overall network efficiency. This illustrates the critical trade-offs in packet-based communication system design.
\end{workedexample}

\begin{importantbox}
\section*{Further Reading}
\parhead{Related Concepts and Systems}
The Link Layer is the essential software and protocol layer that makes the Physical Layer's services useful.
\begin{description}
    \item[\Cref{ch:fec} (Forward Error Correction)] It is crucial to distinguish between Layer 1 FEC (which corrects bits) and Layer 2 Error Control (which detects frame errors via CRC and triggers retransmissions via ARQ).
    \item[\Cref{ch:multiple-access} (Multiple Access Schemes)] Provides a detailed analysis of the Media Access Control (MAC) strategies, such as TDMA and OFDMA, that are managed by the Link Layer.
    \item[\Cref{ch:wifi} (Wi-Fi Systems) and \Cref{ch:5g} (5G Systems)] These chapters provide detailed case studies of two of the most sophisticated Link Layer implementations in the world, including the CSMA/CA protocol of Wi-Fi and the scheduled MAC and HARQ protocols of 5G.
    \item[\Cref{ch:signal-chain} (The Signal Chain)] This chapter provides the end-to-end context, showing where the Link Layer's framing and addressing functions fit in relative to the Physical Layer's modulation and coding functions.
\end{description}
\end{importantbox}
