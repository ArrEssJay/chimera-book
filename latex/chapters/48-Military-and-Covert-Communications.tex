% ==============================================================================
% CHAPTER 48: Military & Covert Communications
% ==============================================================================

\chapter{Military \& Covert Communications}
\label{ch:military-covert}

\begin{nontechnical}
    \textbf{Military communication is the art of having a secret conversation in a hostile environment.} Unlike civilian systems that prioritise speed, military systems prioritise survival, security, and stealth.

    \parhead{The four core challenges}
    \begin{itemize}
        \item \textbf{Anti-Jamming (AJ):} Can your signal be heard even when an adversary is actively trying to drown it out with powerful noise?
        \item \textbf{Low Probability of Intercept (LPI):} Can an adversary even tell that your signal is different from random background noise?
        \item \textbf{Low Probability of Detection (LPD):} Can an adversary even detect that you are transmitting at all?
        \item \textbf{Transmission Security (TRANSEC):} If your signal is intercepted, is it encrypted so the message remains secret?
    \end{itemize}

    \parhead{The enabling technology: Spread Spectrum}
    The core technique for achieving these goals is \keyterm{spread spectrum}. Instead of concentrating a signal's power in a narrow, obvious frequency band, the energy is spread thinly over an enormous bandwidth. To an adversary, the signal becomes indistinguishable from the natural background noise of the universe. Only a receiver that knows the exact "spreading code" can gather all the faint pieces of energy and reconstruct the original message.

    \parhead{Real-world examples}
    \begin{itemize}
        \item \textbf{GPS M-Code:} The encrypted military GPS signal, which combines a powerful spread spectrum code with advanced modulation to allow a receiver to lock on even when under heavy jamming.
        \item \textbf{Link 16:} The tactical data network used by NATO forces. It uses \keyterm{frequency hopping}, jumping between frequencies thousands of times per second in a secret, pseudo-random pattern that jammers cannot follow.
        \item \textbf{AESA Radars:} The phased-array radars on modern fighter jets (like the F-22 and F-35) use agile, narrow beams and complex waveforms to make their signals extremely difficult to detect by enemy radar warning receivers.
    \end{itemize}
\end{nontechnical}


\subsection{Overview}

Military communication systems are designed to operate reliably in contested electromagnetic environments. Their design priorities are fundamentally different from commercial systems, focusing on robustness and security over raw spectral efficiency. The primary objectives are \keyterm{Anti-Jamming (AJ)}, \keyterm{Low Probability of Intercept (LPI)}, \keyterm{Low Probability of Detection (LPD)}, and \keyterm{Transmission Security (TRANSEC)}.

\begin{keyconcept}
    The enabling technology for AJ, LPI, and LPD is \textbf{spread spectrum}, which provides a \textbf{processing gain ($G_p$)}. By spreading a low-rate signal over a very wide bandwidth, the signal's power spectral density can be pushed below the thermal noise floor, making it difficult to detect. At the receiver, the despreading process collapses the signal energy back into a narrow band while spreading the energy of any jammer, providing a massive improvement in the effective Signal-to-Noise Ratio.
\end{keyconcept}


\subsection{Key Techniques and Technologies}

\paragraph{Direct-Sequence Spread Spectrum (DSSS)}
As used in GPS, DSSS multiplies the data by a high-rate pseudo-random code. The processing gain is the ratio of the chip rate to the data rate, and can be 40-60 dB. This allows the receiver to recover signals that are tens of decibels below the noise floor.

\paragraph{Frequency-Hopping Spread Spectrum (FHSS)}
Used in systems like Link 16, FHSS rapidly changes the carrier frequency in a pseudo-random pattern known only to the transmitter and receiver. This forces a jammer to either spread its power across the entire hop set (diluting its effectiveness) or attempt to follow the hops (which is made impossible by using fast hop rates and cryptographic hop patterns).

\paragraph{Phased-Array Antennas and Beamforming}
Modern military platforms use \keyterm{Active Electronically Scanned Arrays (AESAs)} with hundreds or thousands of elements. These antennas can form extremely narrow, high-gain beams, steer them instantly without mechanical movement, and even create deep "nulls" in the direction of a known jammer. This spatial filtering provides an additional 20-40 dB of jamming rejection.

\paragraph{Advanced Coding and Modulation}
Military systems use robust modulation schemes (BPSK, MSK) combined with powerful, low-rate Forward Error Correction codes to provide additional \keyterm{coding gain}, further improving the link's resilience to noise and interference.


\begin{workedexample}{Jamming Resistance of a Military Link}
    \parhead{Problem} A tactical DSSS radio link is being subjected to a powerful jammer. Analyse the link's ability to survive.
    \parhead{System Parameters}
    \begin{itemize}
        \item Required $E_b/N_0$ for reliable communication (after all gains): 10 dB.
        \item DSSS Processing Gain ($G_p$): 30 dB.
        \item FEC Coding Gain ($G_c$): 5 dB.
        \item Antenna Spatial Filtering (Nulling): Provides 20 dB of rejection in the jammer's direction.
        \item Jammer-to-Signal Ratio ($J/S$) at the antenna: 50 dB (the jammer is 100,000 times stronger than the desired signal).
    \end{itemize}
    \parhead{Analysis}
    The \keyterm{jamming margin} ($M_J$) determines if the link survives. It is the effective SNR compared to the required SNR.
    \begin{derivationsteps}
        \step \textbf{Calculate the effective J/S ratio after processing.} The total gain from our countermeasures is the sum of the processing gain, coding gain, and spatial filtering gain.
        \[ G_{\text{total}} = G_p + G_c + G_{\text{spatial}} = 30 + 5 + 20 = \qty{55}{dB} \]
        The effective $J/S$ after the receiver applies these gains is:
        \[ (J/S)_{\text{eff}} = (J/S)_{\text{raw}} - G_{\text{total}} = 50 - 55 = \qty{-5}{dB} \]
        
        \step \textbf{Calculate the final link margin.} The effective SNR is now 5 dB \emph{above} the jamming level.
        \[ \text{Link Margin} = \text{Effective SNR} - (\text{Required } E_b/N_0) = 5 \text{ dB} - 10 \text{ dB} = \textbf{\qty{-5}{dB}} \]
        Wait, the calculation should be based on the Jamming Margin formula:
        \[ M_J = G_p + G_{\text{spatial}} - (J/S)_{\text{raw}} - (E_b/N_0)_{\text{req}} = 30 + 20 - 50 - 10 = \qty{-10}{dB} \]
        Let's reconsider. The coding gain applies to the effective SNR.
        \[ (\text{Effective } E_b/N_0) = (\text{Input } E_b/N_0) + G_p + G_{\text{spatial}} + G_c \]
        The "noise" is the jammer, so the input $E_b/J_0$ is -50 dB.
        \[ (\text{Effective } E_b/J_0) = -50 + 30 + 20 + 5 = \qty{5}{dB} \]
        
        \step \textbf{Calculate the margin.}
        \[ \text{Margin} = (\text{Available } E_b/N_0) - (\text{Required } E_b/N_0) = 5 - 10 = \textbf{\qty{-5}{dB}} \]
    \end{derivationsteps}
    
    \parhead{Interpretation} Even with a powerful combination of countermeasures, the link fails with a -5 dB margin. The jammer is too strong. To close the link, the system would need to either increase its transmit power, use a higher-gain antenna, or move to a more advanced code with a higher coding gain. This demonstrates the constant technological arms race that defines electronic warfare.
\end{workedexample}

\begin{table}[H]
    \centering
    \caption{Comparison of Military Communication Systems}
    \label{tab:milcom-systems}
    \begin{tabular}{@{}llll@{}}
        \toprule
        \tableheaderfont System & \tableheaderfont Primary Technique & \tableheaderfont Frequency Band & \tableheaderfont Main Purpose \\
        \midrule
        GPS M-Code & DSSS with BOC & L-Band (1-2 GHz) & Jam-Resistant Navigation \\
        Link 16 & Fast FHSS + TDMA & L-Band (960-1215 MHz) & Tactical Data Networking \\
        MILSTAR / AEHF & FHSS + Processing & Ka/Q-Band (20-50 GHz) & Strategic SATCOM \\
        SINCGARS & Slow FHSS & VHF (30-88 MHz) & Combat Net Radio (Voice) \\
        AESA Radar & Agile Beamforming & X/S-Band (3-12 GHz) & LPI Targeting and Tracking \\
        \bottomrule
    \end{tabular}
\end{table}


\begin{importantbox}[title={Further Reading}]
    Military and covert communications are highly specialised fields that build upon foundational signal processing concepts.
    \begin{description}
        \item[Spread Spectrum] (\Cref{ch:spread-spectrum}) provides the detailed theory of DSSS and FHSS that is the basis for most of the techniques discussed here.
        \item[MIMO Systems] (\Cref{ch:mimo}) explains the principles of beamforming and nulling used in phased-array antennas.
        \item[Forward Error Correction] (\Cref{ch:fec}) describes the powerful coding techniques that provide the essential coding gain needed to close links in jammed environments.
    \end{description}
\end{importantbox}