% ==============================================================================
% CHAPTER 3: Wave Polarization
% ==============================================================================

\chapter{Wave Polarisation}
\label{ch:polarisation}

\begin{nontechnical}
    \textbf{Wave polarisation is like the orientation of a jump rope}---you can shake it up and down (vertical), side-to-side (horizontal), or in circles (circular). For a radio receiver to work, its antenna must be oriented to "catch" the wave in the same way it was sent.

    \parhead{Three main types}
    \begin{itemize}
        \item \textbf{Linear Polarisation} (most common): The wave oscillates in one fixed direction---vertical, horizontal, or at a slant.
        \item \textbf{Circular Polarisation}: The wave rotates in a corkscrew pattern as it travels---either clockwise (right-hand) or counter-clockwise (left-hand).
        \item \textbf{Elliptical Polarisation}: An intermediate state where the wave traces an ellipse.
    \end{itemize}

    \parhead{Real-world examples}
    \begin{itemize}
        \item \textbf{FM Radio:} Uses vertical polarisation, which is why car antennas are typically vertical.
        \item \textbf{WiFi:} Often vertical. Tilting your laptop's screen can sometimes change the signal strength because it misaligns the internal antennas.
        \item \textbf{GPS:} Uses circular polarisation. This is crucial because it works regardless of your phone's orientation and is robust against atmospheric effects that can twist the signal.
    \end{itemize}

    \parhead{Why it matters} A 90° mismatch between the transmitter and receiver antennas can cause over 99\% of the signal to be lost---enough to drop a call or lose GPS lock.
\end{nontechnical}


\subsection{Overview}

\keyterm{Polarisation} describes the geometric orientation of the electric field vector's oscillation as an electromagnetic wave propagates through space. While the wave travels forward, its electric field oscillates in a plane perpendicular to that direction. The specific pattern traced by the tip of the electric field vector in this plane defines the wave's polarisation.

\begin{keyconcept}
    Understanding polarisation is critical for practical system design. The alignment between the transmitting and receiving antennas must match the wave's polarisation to ensure maximum signal transfer. Mismatch results in a predictable and often severe signal loss known as \keyterm{polarisation loss}.
\end{keyconcept}


\subsection{Mathematical Foundation}

The general electric field for a plane wave propagating in the $+z$ direction is the sum of two orthogonal components:
\begin{equation}
    \vec{E}(z,t) = E_x \cos(\omega t - kz + \phi_x)\hat{x} + E_y \cos(\omega t - kz + \phi_y)\hat{y}
    \label{eq:general-efield}
\end{equation}
The polarisation type is determined entirely by the relationship between the amplitudes ($E_x$, $E_y$) and the relative phase difference ($\Delta\phi = \phi_y - \phi_x$). At a fixed point in space (e.g., $z=0$), the tip of the $\vec{E}$ vector traces a shape in the $xy$-plane over time. This shape is known as a Lissajous figure, and it defines the polarisation.


\subsection{Linear Polarisation}

\parhead{Condition} The phase difference between the two components is either $0^\circ$ (in-phase) or $180^\circ$ (anti-phase), so $\Delta\phi = 0$ or $\pi$.
\parhead{Result} The electric field oscillates along a single, fixed line in the transverse plane. The angle of this line, $\theta$, is determined by the relative amplitudes of the components.

\begin{centre}
    \begin{tikzpicture}[scale=1.15, font=\sffamily]
        % Vertical
        \begin{scope}[shift={(0,0)}]
            \node[above] at (0,2.2) {\bfseries Vertical};
            \draw[->, thick] (-1.5,0) -- (1.5,0) node[right] {x};
            \draw[->, thick] (0,-1.5) -- (0,1.5) node[above] {y};
            \draw[<->, ultra thick, diagramprimary] (0,-1.2) -- (0,1.2);
            \node[below=2pt] at (0,-1.5) {$\theta = 90^\circ$};
        \end{scope}
        % Horizontal
        \begin{scope}[shift={(4,0)}]
            \node[above] at (0,2.2) {\bfseries Horizontal};
            \draw[->, thick] (-1.5,0) -- (1.5,0) node[right] {x};
            \draw[->, thick] (0,-1.5) -- (0,1.5) node[above] {y};
            \draw[<->, ultra thick, diagramprimary] (-1.2,0) -- (1.2,0);
            \node[below=2pt] at (0,-1.5) {$\theta = 0^\circ$};
        \end{scope}
        % Slant
        \begin{scope}[shift={(8,0)}]
            \node[above] at (0,2.2) {\bfseries Slant ($+45^\circ$)};
            \draw[->, thick] (-1.5,0) -- (1.5,0) node[right] {x};
            \draw[->, thick] (0,-1.5) -- (0,1.5) node[above] {y};
            \draw[<->, ultra thick, diagramprimary, rotate=45] (-1.2,0) -- (1.2,0);
            \node[below=2pt] at (0,-1.5) {$\theta = 45^\circ$};
        \end{scope}
    \end{tikzpicture}
\end{centre}

\parhead{Applications} Linear polarisation is simple to generate (e.g., with a dipole or monopole antenna) and is widely used in terrestrial broadcasting like AM/FM radio, television, and many WiFi systems.


\subsection{Circular Polarisation}

\parhead{Condition} The amplitudes are equal ($E_x = E_y$) and the phase difference is exactly $\pm90^\circ$ (quadrature phase), so $\Delta\phi = \pm\pi/2$.
\parhead{Result} The tip of the electric field vector traces a perfect circle, rotating once per cycle. The direction of rotation defines its "handedness."

\begin{centre}
    \begin{tikzpicture}[scale=1.2, font=\sffamily]
        % RHCP
        \begin{scope}[shift={(0,0)}]
            \node[above] at (0,2.2) {\bfseries RHCP (Right-Hand)};
            \draw[->, thick] (-1.8,0) -- (1.8,0) node[right] {x};
            \draw[->, thick] (0,-1.8) -- (0,1.8) node[above] {y};
            \draw[very thick, diagramprimary] (0,0) circle (1.3cm);
            \draw[->, very thick, diagramsecondary] (1.3,0) arc (0:-270:1.3cm);
            \node[below=4pt, align=centre] at (0,-1.8) {$\Delta\phi = -90^\circ$\\Clockwise};
        \end{scope}
        % LHCP
        \begin{scope}[shift={(5.5,0)}]
            \node[above] at (0,2.2) {\bfseries LHCP (Left-Hand)};
            \draw[->, thick] (-1.8,0) -- (1.8,0) node[right] {x};
            \draw[->, thick] (0,-1.8) -- (0,1.8) node[above] {y};
            \draw[very thick, diagramprimary] (0,0) circle (1.3cm);
            \draw[->, very thick, diagramsecondary] (1.3,0) arc (0:270:1.3cm);
            \node[below=4pt, align=centre] at (0,-1.8) {$\Delta\phi = +90^\circ$\\Counter-clockwise};
        \end{scope}
    \end{tikzpicture}
\end{centre}

\parhead{Handedness} By convention (IEEE standard), handedness is determined by pointing the thumb of your right or left hand in the direction of wave propagation; if your fingers curl in the direction of the E-field's rotation, that is the handedness.
\begin{itemize}
    \item \keyterm{Right-Hand Circular Polarisation (RHCP):} E-field rotates clockwise when viewed from the receiver.
    \item \keyterm{Left-Hand Circular Polarisation (LHCP):} E-field rotates counter-clockwise.
\end{itemize}
\parhead{Applications} Circular polarisation is critical for satellite communications, including GPS. Its key advantage is that it is immune to the polarisation rotation caused by the ionosphere (\keyterm{Faraday rotation}) and provides robustness against signal cancellation from multipath reflections.


\subsection{Elliptical Polarisation}

\parhead{Condition} The general case. This occurs whenever the amplitudes are unequal ($E_x \neq E_y$) or the phase difference is not a multiple of $90^\circ$.
\parhead{Result} The E-field tip traces an ellipse. This is the most common form of polarisation in the real world, as perfect linear or circular polarisation is an ideal.
\parhead{Quantification} The shape of the ellipse is described by the \keyterm{Axial Ratio (AR)}, the ratio of the major axis to the minor axis. An AR of 1 (or 0 dB) is perfect circular polarisation, while an AR of infinity is perfect linear polarisation. A "good" circularly polarized antenna typically has an AR below 3 dB.


\subsection{Polarisation Loss Factor (PLF)}

When the polarisation of the receiving antenna does not perfectly match the polarisation of the incoming wave, a portion of the signal power is lost. This is quantified by the Polarisation Loss Factor (PLF).

\begin{table}[H]
    \centering
    \caption{Common Polarisation Mismatch Scenarios}
    \label{tab:plf-summary}
    \begin{tabular}{@{}lll@{}}
        \toprule
        \tableheaderfont Transmit Polarisation & \tableheaderfont Receive Polarisation & \tableheaderfont Power Loss \\
        \midrule
        Vertical & Vertical & 0 dB (Perfect Match) \\
        Vertical & Slant at 45$^\circ$ & 3 dB (Half Power) \\
        Vertical & Horizontal & $\infty$ (Complete Null) \\
        \addlinespace
        RHCP & RHCP & 0 dB (Co-polarized) \\
        RHCP & LHCP & $\infty$ (Cross-polarized) \\
        \addlinespace
        RHCP & Linear (any orientation) & 3 dB (Half Power) \\
        \bottomrule
    \end{tabular}
\end{table}

\begin{importantbox}[title={The Critical 3 dB Loss}]
    The case where a linear antenna receives a circularly polarized signal is extremely common. For example, many low-cost GPS devices use a linearly polarized patch antenna to receive the RHCP signal from GPS satellites. This immediately incurs a 3 dB loss (a 50\% reduction in power) before any other factors are considered. This loss must be accounted for in the link budget.
\end{importantbox}


\subsection{Propagation Effects}

The polarisation of a wave is not always preserved as it travels from transmitter to receiver.
\begin{description}
    \item[Faraday Rotation] As a wave passes through the ionosphere, Earth's magnetic field causes its polarisation to rotate. The effect is proportional to $1/f^2$, making it severe at HF frequencies but negligible for satellite communications in the GHz range. Circular polarisation is immune to this effect.
    \item[Rain Depolarization] The non-spherical shape of raindrops can cause circularly or slant-polarized waves to become elliptical, degrading the isolation between orthogonal channels. This is a significant concern for Ku-band and Ka-band satellite systems.
    \item[Reflection] When a wave reflects off a surface, its polarisation can change. A key example is a circularly polarized wave reversing its handedness upon reflection (e.g., RHCP becomes LHCP). This is exploited by circularly polarized antennas to reject multipath signals.
\end{description}


\subsection{Stokes Parameters and the Poincaré Sphere}

For a complete and unambiguous description of any polarisation state (including partially polarized waves), a mathematical framework called the \keyterm{Stokes parameters} is used. These four parameters ($S_0, S_1, S_2, S_3$) capture the total intensity, the preference for horizontal vs. vertical, the preference for $\pm45^\circ$, and the preference for right-hand vs. left-hand circular, respectively.

These parameters can be visualized on the surface of the \keyterm{Poincaré sphere}, an elegant geometric tool where every possible fully polarized state corresponds to a unique point on the sphere's surface.
\begin{itemize}
    \item The North and South poles represent LHCP and RHCP.
    \item The equator represents all possible linear polarizations.
    \item All other points on the surface represent elliptical polarizations.
\end{itemize}

\begin{importantbox}[title={Further Reading}]
    Polarisation is not an isolated topic; it is deeply interconnected with the physical components and environment of a communication system.
    \begin{description}
        \item[Antenna Theory] (\Cref{ch:antenna}) details the practical design of antennas that generate and receive specific polarizations, from simple dipoles to complex helical and patch antennas.
        \item[Atmospheric Propagation] (\Cref{ch:propagation}) provides a deep dive into the physics of Faraday rotation and rain depolarization.
        \item[Multipath Fading] (\Cref{ch:multipath}) explains how polarisation is used to mitigate the destructive effects of signal reflections in wireless channels.
    \end{description}
\end{importantbox}