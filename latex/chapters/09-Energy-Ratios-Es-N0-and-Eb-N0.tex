% ==============================================================================
% CHAPTER 9: Energy Ratios (Es/N0 and Eb/N0)
% ==============================================================================

\chapter{Energy Ratios: \texorpdfstring{$E_s/N_0$ and $E_b/N_0$}{Es/N0 and Eb/N0}}
\label{ch:energy-ratios}

\begin{nontechnical}
    \textbf{Energy ratios like $E_b/N_0$ measure the "quality of each bit" against the background noise.} A higher ratio means a cleaner, more reliable signal.

    \parhead{The radio analogy}
    \begin{itemize}
        \item \textbf{Energy ($E_b$):} The amount of power dedicated to sending a single "1" or "0".
        \item \textbf{Noise ($N_0$):} The constant level of background static or hiss.
        \item \textbf{Ratio ($E_b/N_0$):} A direct measure of how clearly each bit stands out from the static.
    \end{itemize}

    \parhead{The two key metrics}
    \begin{itemize}
        \item \textbf{$E_b/N_0$ (Energy per Bit):} This is the universal standard. It allows engineers to make fair, apples-to-apples comparisons of the performance of different communication systems, regardless of their speed or complexity.
        \item \textbf{$E_s/N_0$ (Energy per Symbol):} This is used for the internal calculations of a specific modulation scheme. A "symbol" can contain multiple bits (e.g., one 16-QAM symbol contains 4 bits).
    \end{itemize}

    \parhead{Why it matters} The required $E_b/N_0$ for a reliable connection is a fundamental design constraint. Deep-space probes use incredibly efficient coding to work with an $E_b/N_0$ near 0 dB, while your WiFi router needs a high $E_b/N_0$ of over 30 dB to achieve its fastest speeds.
\end{nontechnical}


\section{Overview and Properties}

\subsection{Overview}

While the Signal-to-Noise Ratio (SNR) is a useful measure of overall signal quality, it is dependent on bandwidth. To create a universal, normalized metric for comparing the performance of different digital communication systems, engineers use \keyterm{energy ratios}. These ratios compare the energy allocated to a piece of information (a bit or a symbol) to the noise power in a 1 Hz bandwidth.

\begin{keyconcept}
    \textbf{$E_b/N_0$ (Energy per Bit to Noise Density Ratio)} is the most important performance metric in digital communications. It provides a standardized way to evaluate a system's efficiency, independent of data rate or modulation scheme. All theoretical performance limits (like the Shannon limit) and practical system requirements are specified in terms of the required $E_b/N_0$ to achieve a target Bit Error Rate (BER).
\end{keyconcept}


\subsection{Fundamental Definitions}

\paragraph{Energy per Bit ($E_b$)}
The energy allocated to transmit a single information bit. It is the average signal power, $P_s$, divided by the bit rate, $R_b$.
\begin{equation}
    E_b = \frac{P_s}{R_b} \quad (\text{Joules/bit})
\end{equation}

\paragraph{Energy per Symbol ($E_s$)}
The energy allocated to transmit a single modulation symbol. It is related to the bit energy by the number of bits per symbol, $k = \log_2(M)$, where $M$ is the constellation size.
\begin{equation}
    E_s = k \cdot E_b = \frac{P_s}{R_s} \quad (\text{Joules/symbol})
\end{equation}
where $R_s$ is the symbol rate.

\paragraph{Noise Power Spectral Density ($N_0$)}
The noise power present in a 1 Hz bandwidth, given by $N_0 = kT$, where $k$ is Boltzmann's constant and $T$ is the system noise temperature. At room temperature, $N_0 \approx -174$ dBm/Hz.

\subsection{Relationship Between Energy Ratios and SNR}

The energy ratios can be directly related to the carrier SNR ($P_s/P_n$). Since total noise power is $P_n = N_0 B$, we have:
\begin{equation}
    \frac{E_b}{N_0} = \frac{P_s/R_b}{P_n/B} = \frac{P_s}{P_n} \cdot \frac{B}{R_b} = \text{SNR} \cdot \frac{B}{R_b}
\end{equation}
In decibels, this becomes:
\begin{equation}
    \left(\frac{E_b}{N_0}\right)_{\text{dB}} = \text{SNR}_{\text{dB}} + 10\log_{10}\left(\frac{B}{R_b}\right)
\end{equation}
The term $B/R_b$ is related to the spectral efficiency of the system. This equation shows that for a fixed SNR, a more spectrally efficient system (lower $B/R_b$) will have a lower $E_b/N_0$.

\begin{table}[H]
    \centering
    \caption{Conversion Between $E_b/N_0$ and $E_s/N_0$ (in dB)}
    \label{tab:ebno-esno-conversion}
    \begin{tabular}{@{}lcc@{}}
        \toprule
        \tableheaderfont Modulation & \tableheaderfont Bits/Symbol ($k$) & \tableheaderfont Conversion Factor ($10\log_{10}(k)$) \\
        \midrule
        BPSK & 1 & 0.0 dB \\
        QPSK & 2 & 3.0 dB \\
        8-PSK & 3 & 4.8 dB \\
        16-QAM & 4 & 6.0 dB \\
        64-QAM & 6 & 7.8 dB \\
        256-QAM & 8 & 9.0 dB \\
        \bottomrule
    \end{tabular}
    \par\vspace{0.5em}
    \small Note: $\left(E_s/N_0\right)_{\text{dB}} = \left(E_b/N_0\right)_{\text{dB}} + 10\log_{10}(k)$
\end{table}


\subsection{Performance Analysis and the Shannon Limit}

The Bit Error Rate (BER) of any modulation scheme is a direct function of $E_b/N_0$. For example, the BER for coherent BPSK in an AWGN channel is:
\begin{equation}
    \text{BER} = Q\left(\sqrt{\frac{2E_b}{N_0}}\right)
\end{equation}
This allows us to determine the minimum $E_b/N_0$ required to achieve a target BER. For example, a BER of $10^{-5}$ requires an $E_b/N_0$ of approximately 9.6 dB for BPSK or QPSK.

\begin{warningbox}
    \textbf{The Shannon Limit:} The Shannon-Hartley theorem implies a fundamental limit on the minimum $E_b/N_0$ required for error-free communication. As the data rate approaches the channel capacity, the required $E_b/N_0$ approaches its absolute minimum value:
    \[ \left(\frac{E_b}{N_0}\right)_{\min} = \ln(2) \approx -1.59 \text{ dB} \]
    No communication system, no matter how sophisticated its modulation or coding, can achieve reliable communication at an $E_b/N_0$ below this limit.
\end{warningbox}


\begin{workedexample}{Satellite Link Budget Revisited}
    \parhead{Problem} For the satellite link designed in the previous chapter, verify the link margin using an energy ratio approach.
    \parhead{System Parameters}
    \begin{itemize}
        \item Modulation: QPSK (Target BER $10^{-6}$ requires $E_b/N_0 \approx 10.5$ dB)
        \item Data Rate: $R_b = \qty{2}{Mbps}$
        \item Received Power ($P_r$): \qty{-90.5}{dBm}
        \item System Noise Temperature ($T_s$): \qty{150}{K}
    \end{itemize}
    \parhead{Solution}
    \begin{derivationsteps}
        \step Calculate the noise power spectral density ($N_0$) in dBm/Hz.
        \[ N_0 \text{ (W/Hz)} = kT_s = (1.38 \times 10^{-23})(150) = 2.07 \times 10^{-21} \text{ W/Hz} \]
        \[ N_{0, \text{dBm/Hz}} = 10\log_{10}(2.07 \times 10^{-21} / 10^{-3}) = \qty{-176.8}{dBm/Hz} \]
        \step Calculate the available $E_b/N_0$ using the received power and bit rate.
        \[ \left(\frac{E_b}{N_0}\right)_{\text{dB}} = P_{r, \text{dBm}} - N_{0, \text{dBm/Hz}} - 10\log_{10}(R_b) \]
        \[ \left(\frac{E_b}{N_0}\right)_{\text{dB}} = -90.5 - (-176.8) - 10\log_{10}(2 \times 10^6) = 86.3 - 63.0 = \qty{23.3}{dB} \]
        \step Calculate the link margin, including a 3 dB implementation margin.
        \[ \text{Required } E_b/N_0 = 10.5 \text{ (for BER)} + 3.0 \text{ (margin)} = \qty{13.5}{dB} \]
        \[ \text{Link Margin} = (\text{Available}) - (\text{Required}) = 23.3 - 13.5 = \qty{9.8}{dB} \]
    \end{derivationsteps}
    \parhead{Interpretation} The link closes with a healthy margin of 9.8 dB. This margin is available to combat atmospheric fading (especially rain fade at Ku-band), antenna pointing errors, and other real-world impairments, ensuring the link remains reliable.
\end{workedexample}

\begin{importantbox}[title={Further Reading}]
    The concept of $E_b/N_0$ is the linchpin that connects system-level design to theoretical performance.
    \begin{description}
        \item[Bit Error Rate] (\Cref{ch:ber}) is almost exclusively plotted as a function of $E_b/N_0$, providing the performance curves that drive system design.
        \item[Shannon's Channel Capacity] (\Cref{ch:shannon}) uses $E_b/N_0$ to define the absolute theoretical limits of communication.
        \item[Forward Error Correction] (\Cref{ch:fec}) describes how coding techniques can be used to achieve a target BER at a much lower $E_b/N_0$, providing a "coding gain" that is a cornerstone of modern systems.
    \end{description}
\end{importantbox}
