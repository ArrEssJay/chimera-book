% ==============================================================================
% CHAPTER 5: Baseband vs. Passband Signals
% ==============================================================================

\chapter{Baseband vs. Passband Signals}
\label{ch:baseband-passband}

\begin{nontechnical}
    \textbf{Baseband vs. passband is like the difference between sheet music and the sound from a trumpet.}

    \parhead{Musical analogy}
    \begin{itemize}
        \item \textbf{Baseband Signal:} The melody as written on paper. It represents the "pure" information, with frequencies centered around 0 Hz.
        \item \textbf{Passband Signal:} The same melody played on a specific instrument. A flute plays it at a high pitch (a high frequency band), while a tuba plays it at a low pitch (a low frequency band). The melody is the same, but it has been shifted to a different frequency range.
    \end{itemize}

    \parhead{The journey of your voice in a phone call}
    \begin{enumerate}[label=\arabic*.]
        \item Your voice is a \textbf{baseband} audio signal (approx. 300 Hz -- 3.4 kHz).
        \item Your phone's transmitter shifts it to a \textbf{passband} signal (e.g., 1.9 GHz).
        \item The passband signal travels through the air to the cell tower.
        \item The tower's receiver shifts the signal back down to \textbf{baseband} for processing.
        \item The process is reversed to send the signal to the other person's phone.
    \end{enumerate}

    \parhead{Why we need passband signals} An efficient antenna is roughly half a wavelength long. A baseband audio signal at 300 Hz has a wavelength of 1,000 km, requiring an impossible 500 km antenna. A passband WiFi signal at 2.4 GHz has a wavelength of 12.5 cm, requiring a practical 6 cm antenna.
\end{nontechnical}


\section{Overview and Properties}

\subsection{Overview}

The distinction between \keyterm{baseband} and \keyterm{passband} signals is fundamental to all radio communication systems. Baseband signals contain the original information, with a frequency spectrum centered around DC (0 Hz). Passband signals are created by shifting this baseband information up to a high-frequency \keyterm{carrier frequency} ($f_c$) suitable for wireless transmission.

\begin{keyconcept}
    The transformation between baseband and passband domains enables \textbf{efficient wireless transmission} while preserving \textbf{computational simplicity}. This duality is the cornerstone of modern Software-Defined Radio (SDR), where complex signal processing is performed on the low-frequency baseband signal in software, while the high-frequency passband signal interfaces with the physical RF channel.
\end{keyconcept}


\subsection{Signal Definitions}

\paragraph{Baseband Signal}
A baseband signal is one whose frequency content is concentrated around 0 Hz. In digital communications, it is almost always represented as a \keyterm{complex baseband} signal (also known as the complex envelope):
\begin{equation}
    s_{\text{BB}}(t) = s_I(t) + js_Q(t)
\end{equation}
where $s_I(t)$ is the \keyterm{in-phase} component and $s_Q(t)$ is the \keyterm{quadrature} component. This complex representation is the natural language of digital signal processing.

\paragraph{Passband Signal}
A passband signal is a real-world, physical signal whose frequency content is centered around a high-frequency carrier, $f_c$. It is generated from the complex baseband signal via \keyterm{upconversion} or \keyterm{modulation}:
\begin{equation}
    s_{\text{RF}}(t) = \Re\{s_{\text{BB}}(t) \cdot e^{j2\pi f_c t}\} = s_I(t)\cos(2\pi f_c t) - s_Q(t)\sin(2\pi f_c t)
\end{equation}
This process preserves the bandwidth ($B$) of the baseband signal but shifts its spectrum from being centered at 0 Hz to being centered at $f_c$.


\subsection{The Role of Frequency Translation}

Shifting a signal from baseband to passband is a necessary step for three primary reasons:
\begin{description}
    \item[Antenna Efficiency] As discussed, practical antenna dimensions are on the order of the signal's wavelength. Baseband signals have impractically long wavelengths, whereas passband signals in the MHz or GHz range have wavelengths on the scale of centimetres to metres, allowing for small, efficient antennas.
    \item[Spectrum Allocation] Regulatory bodies (like the ITU and FCC) allocate specific passbands to different services (e.g., FM radio, cellular, WiFi). Upconversion allows a signal to be placed in its legally assigned band, enabling multiple services to coexist without interference.
    \item[Frequency Division Multiplexing (FDM)] Multiple independent baseband signals can be upconverted to different, adjacent carrier frequencies and transmitted simultaneously over the same physical medium.
\end{description}


\subsection{The Transceiver Architecture}

A modern wireless transceiver is a physical embodiment of the baseband/passband concept.

\begin{center}
    \begin{tikzpicture}[
        block/.style={rectangle, draw, minimum width=2.2cm, minimum height=0.9cm, font=\sffamily\scriptsize, fill=diagramlight, draw=diagramprimary, thick},
        node distance=1.8cm, font=\scriptsize
    ]
        % Transmit Path
        \node[align=center] (tx_data) {\sffamily Baseband\\Data};
        \node[block, right of=tx_data, node distance=2.5cm] (tx_dsp) {DSP\\Modulator};
        \node[block, right of=tx_dsp, node distance=2.5cm] (tx_dac) {DAC};
        \node[block, right of=tx_dac, node distance=2.5cm] (tx_iq) {IQ\\Upconverter};
        \node[block, right of=tx_iq, node distance=2.5cm] (tx_pa) {PA};
        \node[right of=tx_pa, node distance=2.2cm] (tx_ant) {\faBroadcastTower};

        % Receive Path
        \node[below=3.2cm of tx_data, align=center] (rx_data) {\sffamily Baseband\\Data};
        \node[block, right of=rx_data, node distance=2.5cm] (rx_dsp) {DSP\\Demodulator};
        \node[block, right of=rx_dsp, node distance=2.5cm] (rx_adc) {ADC};
        \node[block, right of=rx_adc, node distance=2.5cm] (rx_iq) {IQ\\Downconverter};
        \node[block, right of=rx_iq, node distance=2.5cm] (rx_lna) {LNA};
        \node[right of=rx_lna, node distance=2.2cm] (rx_ant) {\faBroadcastTower};

        % Local Oscillator
        \node[block, below=1.6cm of tx_iq] (lo) {LO};

        % Connections
        \draw[->,thick] (tx_data) -- (tx_dsp);
        \draw[->,thick] (tx_dsp) -- node[above,font=\tiny] {I/Q} (tx_dac);
        \draw[->,thick] (tx_dac) -- (tx_iq);
        \draw[->,thick] (tx_iq) -- (tx_pa);
        \draw[->,thick] (tx_pa) -- (tx_ant);

        \draw[->,thick] (rx_ant) -- (rx_lna);
        \draw[->,thick] (rx_lna) -- (rx_iq);
        \draw[->,thick] (rx_iq) -- (rx_adc);
        \draw[->,thick] (rx_adc) -- node[above,font=\tiny] {I/Q} (rx_dsp);
        \draw[->,thick] (rx_dsp) -- (rx_data);

        \draw[->,thick] (lo) -- (tx_iq);
        \draw[->,thick] (lo) -- (rx_iq);

        % Domain Boundary
        \draw[dashed,diagramgray,thick] ($(tx_dac.east)!0.5!(tx_iq.west)$) ++(0,1.8) -- ++(0,-5.2);
        \node[above,font=\sffamily\bfseries\small] at ($(tx_data)!0.5!(tx_dac)$) {Baseband Domain (Digital)};
        \node[above,font=\sffamily\bfseries\small] at ($(tx_iq)!0.5!(tx_ant)$) {Passband Domain (Analogue RF)};
    \end{tikzpicture}
\end{center}


\subsection{Receiver Architectures}

\paragraph{Superheterodyne Receiver}
The classic \keyterm{superheterodyne} ("superhet") architecture uses a two-step downconversion process: from the high radio frequency (RF) to a fixed \keyterm{intermediate frequency} (IF), and then from IF to baseband. The key advantage is the high-quality, fixed-frequency IF filter, which provides excellent selectivity and image rejection. This architecture dominated radio design for most of the 20th century.

\paragraph{Direct-Conversion (Zero-IF) Receiver}
Modern transceivers, especially in mobile devices and SDRs, use a \keyterm{direct-conversion} or \keyterm{Zero-IF} architecture. This approach downconverts the RF signal directly to baseband (DC) in a single step. While this drastically reduces component count, cost, and power consumption, it introduces significant design challenges, including DC offsets, flicker noise, and I/Q imbalance, which must be corrected with sophisticated digital signal processing.

\subsection{Practical Impairments}

The ideal mathematical model of frequency translation is disrupted by real-world hardware limitations.
\begin{description}
    \item[Carrier Frequency Offset (CFO)] Mismatches between the transmitter and receiver's local oscillators cause a continuous rotation of the received constellation, which must be tracked and corrected.
    \item[Phase Noise] Random fluctuations (jitter) in the local oscillator's phase introduce noise that spreads energy between constellation points and limits the achievable signal-to-noise ratio.
    \item[I/Q Imbalance] Mismatches in the gain or phase of the I and Q signal paths cause "leakage" from the desired signal into its spectral image, degrading performance.
    \item[DC Offset] In direct-conversion receivers, self-mixing of the local oscillator can create a large, unwanted DC spike at the centre of the baseband spectrum.
\end{description}
Modern DSP algorithms in the baseband domain are almost entirely dedicated to estimating and correcting for these analogue passband impairments.

\begin{importantbox}[title={Further Reading}]
    The baseband/passband duality is the central operating principle of modern communications.
    \begin{description}
        \item[IQ Representation] (\Cref{ch:iq}) provides a deep dive into the mathematics and advantages of using complex baseband signals.
        \item[Modulation Techniques] (e.g., \Cref{ch:qpsk}, \Cref{ch:qam}) are all defined by how information is encoded onto the baseband $s_I(t)$ and $s_Q(t)$ components.
        \item[Synchronization] (\Cref{ch:synchronisation}) covers the essential DSP techniques used to correct for Carrier Frequency Offset and other impairments introduced during frequency translation.
    \end{description}
\end{importantbox}
