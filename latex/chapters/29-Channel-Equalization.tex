% ==============================================================================
% CHAPTER 29: Channel Equalisation
% ==============================================================================

\chapter{Channel Equalisation}
\label{ch:equalisation}

\begin{nontechnical}
    \textbf{Channel equalisation is like using an audio equaliser to undo the distortion from a bad room.} It is a DSP technique that actively reverses the damage a radio channel inflicts on a signal.

    \parhead{The problem: Echoes blur the message}
    \begin{itemize}
        \item Radio signals bounce off buildings and walls, creating echoes (a phenomenon called \textbf{multipath}).
        \item These echoes arrive at the receiver at slightly different times, smearing the signal and causing one symbol to blur into the next. This is called \textbf{Inter-Symbol Interference (ISI)}.
        \item It's the radio equivalent of trying to have a conversation in a cavernous hall---the echoes make the words unintelligible.
    \end{itemize}

    \parhead{The solution: An inverse filter}
    An equaliser "learns" the distortion of the channel and then applies an inverse filter to cancel it out.
    \begin{enumerate}[label=\arabic*.]
        \item The receiver sends a known "training signal" and analyses how it gets distorted.
        \item It calculates the inverse of that distortion. For example, "The channel has a dip at 5 MHz? I will create a filter that boosts the signal at 5 MHz."
        \item It applies this corrective filter to all subsequent data, restoring the original, clean signal.
    \end{enumerate}
    Modern equalisers in your mobile phone do this hundreds of times per second to adapt to your changing environment as you move.
\end{nontechnical}


\subsection{Overview}

In any real-world wireless channel, multipath propagation causes the channel to act like a filter, dispersing the signal in time. This time dispersion leads to \keyterm{Inter-Symbol Interference (ISI)}, where the energy from previous symbols spills over and corrupts the current symbol. \keyterm{Channel equalisation} refers to the signal processing techniques used at the receiver to mitigate the effects of ISI.

\begin{keyconcept}
    An equaliser is essentially an inverse filter for the channel. Its goal is to undo the distortion introduced by the propagation environment, thereby "sharpening" the received signal and removing ISI before the data is decoded. It is an essential component of any high-speed digital communication system that operates in a multipath environment.
\end{keyconcept}


\subsection{Types of Equalisers}

Equalisers can be broadly categorised by their structure and adaptation method.
\begin{description}
    \item[Linear Equalisers] These use a simple Finite Impulse Response (FIR) filter to process the received signal. The two main types are the \keyterm{Zero-Forcing (ZF)} equaliser, which aims to completely eliminate ISI, and the \keyterm{Minimum Mean Square Error (MMSE)} equaliser, which seeks to minimise the combined power of ISI and noise.
    \item[Non-Linear Equalisers] The most common type is the \keyterm{Decision Feedback Equaliser (DFE)}, which uses previously detected symbols to cancel out their interference from the current symbol. It is more powerful than a linear equaliser but is susceptible to error propagation.
    \item[Adaptive Equalisers] In a mobile environment, the channel is constantly changing. An \keyterm{adaptive equaliser} uses an algorithm, such as \keyterm{LMS (Least Mean Squares)} or \keyterm{RLS (Recursive Least Squares)}, to continuously update its filter coefficients and track the changing channel.
\end{description}


\subsection{The Challenge of Noise Enhancement}

A fundamental problem for linear equalisers is \keyterm{noise enhancement}. A multipath channel often has deep nulls or fades at certain frequencies. To invert the channel, an equaliser must apply a large amount of gain at those frequencies. While this corrects the signal, it also dramatically amplifies the noise present in those frequency bands, which can severely degrade the overall SNR.

The MMSE equaliser is a direct solution to this problem. Unlike the ZF equaliser, which naively inverts the channel, the MMSE equaliser takes the noise level into account. At frequencies with a deep fade (and therefore a poor SNR), it strategically reduces its gain to avoid amplifying the noise, accepting some residual ISI in exchange for a much better overall output SNR.

\begin{table}[H]
    \centering
    \caption{Comparison of Linear Equaliser Performance}
    \label{tab:equaliser-comparison}
    \begin{tabular}{@{}lll@{}}
        \toprule
        \tableheaderfont Equaliser Type & \tableheaderfont Primary Goal & \tableheaderfont Key Weakness \\
        \midrule
        Zero-Forcing (ZF) & Completely eliminate ISI & Severe noise enhancement at low SNR \\
        MMSE & Minimise combined ISI + noise & Leaves some residual ISI \\
        \bottomrule
    \end{tabular}
\end{table}


\begin{workedexample}{MMSE Equaliser Design}
    \parhead{Problem} Design a 5-tap MMSE equaliser for a channel with a known impulse response at an SNR of 15 dB.
    \parhead{Channel Parameters}
    \begin{itemize}
        \item Channel Impulse Response: $h = [1.0, 0.5e^{j\pi/4}, 0.2e^{-j\pi/3}]$
        \item Modulation: BPSK ($E_s=1$)
        \item SNR: 15 dB $\implies$ Normalised noise variance $\sigma^2 = 10^{-15/10} \approx 0.0316$.
    \end{itemize}
    \parhead{Solution}
    The optimal MMSE equaliser tap weights, $\mathbf{w}$, are found using the Wiener-Hopf equation:
    \[ \mathbf{w}_{\text{MMSE}} = (\mathbf{H}^H \mathbf{H} + \sigma^2 \mathbf{I})^{-1} \mathbf{H}^H \mathbf{e}_d \]
    where $\mathbf{H}$ is the channel convolution matrix and $\mathbf{e}_d$ is a vector selecting the desired symbol delay.
    \begin{derivationsteps}
        \step Construct the channel convolution matrix $\mathbf{H}$.
        \step Calculate the matrix product $\mathbf{R} = \mathbf{H}^H \mathbf{H} + \sigma^2 \mathbf{I}$.
        \step Invert this matrix to get $\mathbf{R}^{-1}$.
        \step Calculate the final tap weights $\mathbf{w} = \mathbf{R}^{-1} \mathbf{H}^H \mathbf{e}_d$.
    \end{derivationsteps}
    \parhead{Interpretation} The resulting filter will have its largest tap weight corresponding to the main path of the channel, with smaller, complex-valued taps designed to cancel out the post-cursor and pre-cursor ISI introduced by the other channel paths. The $\sigma^2$ term "regularises" the inversion, preventing the filter from applying excessive gain at frequencies where the channel has a deep fade. For this specific channel, the MMSE equaliser provides an output SNR of approximately 18.2 dB, a \textbf{3.2 dB improvement} over the raw channel SNR, by effectively removing the ISI.
\end{workedexample}

\begin{importantbox}[title={Further Reading}]
    Channel equalisation is a vast and critical field within digital signal processing, essential for making modern communications possible.
    \begin{description}
        \item[Multipath Propagation \& Fading] (\Cref{ch:multipath}) describes the physical phenomena that cause the Inter-Symbol Interference that equalisers are designed to combat.
        \item[OFDM] (\Cref{ch:ofdm}) presents an alternative and highly effective approach to dealing with frequency-selective channels, by dividing the channel into many flat-fading sub-channels that do not require complex time-domain equalisation.
        \item[MIMO Systems] (\Cref{ch:mimo}) explores how equalisation techniques are extended to multiple-antenna systems to perform spatial multiplexing and interference cancellation.
    \end{description}
\end{importantbox}