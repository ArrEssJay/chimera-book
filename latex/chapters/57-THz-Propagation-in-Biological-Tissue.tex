% ==============================================================================
% CHAPTER 56: THz Propagation in Biological Tissue
% ==============================================================================

\chapter{THz Propagation in Biological Tissue}
\label{ch:thz-bio-propagation}

\begin{nontechnical}
    \textbf{Terahertz (THz) radiation is a form of "invisible light" that is uniquely sensitive to water.} This single property defines its behaviour in biological tissue and is the key to both its usefulness and its limitations.

    \parhead{The simple analogy: A flashlight in fog}
    Imagine trying to shine a powerful flashlight through a dense fog. The light doesn't travel very far because the tiny water droplets in the fog scatter and absorb the light's energy. THz waves behave in exactly the same way in human tissue. Because our bodies are about 70\% water, tissue acts like a very dense "fog" to THz radiation.

    \parhead{The consequence: Surface-level interaction only}
    This strong absorption means that THz waves cannot penetrate deep into the body.
    \begin{itemize}
        \item At 1 THz, the signal loses over 98\% of its energy after travelling just 100 micrometres (the thickness of a sheet of paper).
        \item This makes THz imaging completely different from X-rays (which pass through soft tissue) or MRI (which can image the entire body). THz is a \textbf{surface-only} technology.
    \end{itemize}

    \parhead{Why this is useful}
    This limitation is also a key feature.
    \begin{itemize}
        \item \textbf{Safety:} Because it is non-ionising and only interacts with the very surface of the skin, it is considered very safe for medical and security applications.
        \item \textbf{High Contrast:} Different types of tissue (e.g., skin cancer vs. healthy skin, or muscle vs. fat) have different water content, and therefore absorb THz radiation differently. This creates a natural, high-contrast image without the need for dyes or ionising radiation.
    \end{itemize}
    This makes THz imaging a powerful tool for applications like detecting early-stage skin cancer, assessing the depth of burns, or non-invasively inspecting the enamel of teeth.
\end{nontechnical}


\section{Overview and Properties}

\subsection{Overview}

The interaction of terahertz (THz) radiation with biological tissue is dominated by the strong absorption spectra of water and the resonant modes of biomolecules. Understanding these interactions is critical for the development of THz applications in medical imaging, spectroscopy, and security screening.

\begin{keyconcept}
    The propagation of THz waves in tissue is primarily limited by the \textbf{strong dielectric absorption of water}, which restricts the penetration depth to the sub-millimetre scale. However, the sensitivity of THz radiation to water content and molecular composition provides a unique source of contrast for high-resolution surface imaging and spectroscopy.
\end{keyconcept}


\subsection{Dielectric Properties of Tissue}

The response of a biological material to an electromagnetic field is described by its complex permittivity, $\epsilon_r(\omega) = \epsilon' - i\epsilon''$.
\begin{itemize}
    \item The real part, $\epsilon'$, determines the refractive index and the speed of the wave in the tissue.
    \item The imaginary part, $\epsilon''$, represents the absorption of energy by the tissue.
\end{itemize}
For biological tissues in the THz range, both $\epsilon'$ and $\epsilon''$ are dominated by the properties of water, which exhibits a strong \keyterm{Debye relaxation} peak around 20 GHz. The tail of this relaxation extends into the THz band, causing the absorption coefficient to increase almost quadratically with frequency.


\subsection{Penetration Depth}

The strong absorption leads to a very shallow penetration depth, which is the distance over which the wave's intensity falls to $1/e$ (about 37\%) of its initial value.

\begin{table}[H]
    \centering
    \caption{Approximate THz Penetration Depth in Hydrated Soft Tissue}
    \label{tab:thz-penetration}
    \begin{tabular}{@{}cc@{}}
        \toprule
        \tableheaderfont Frequency & \tableheaderfont Penetration Depth ($\delta$) \\
        \midrule
        0.1 THz & $\sim$ 1 mm \\
        0.5 THz & $\sim$ 200 $\mu$m \\
        1.0 THz & $\sim$ 50 $\mu$m \\
        2.5 THz & $\sim$ 20 $\mu$m \\
        \bottomrule
    \end{tabular}
\end{table}

This severe limitation is the single most important factor in the design of any THz biomedical system. It makes THz an excellent tool for dermatology and ophthalmology, but unsuitable for imaging internal organs. Tissues with low water content, such as adipose tissue (fat) and bone, are significantly more transparent to THz radiation.

\begin{warningbox}
    Due to the shallow penetration depth, the viability of any proposed mechanism for \textbf{deep brain stimulation} using externally applied THz radiation (such as the speculative EM intermodulation techniques) is highly questionable. The skull and overlying tissues would absorb virtually all of the incident energy before it could reach the cortex.
\end{warningbox}


\begin{workedexample}{Power Deposition in Skin at 1 THz}
    \parhead{Problem} A 1 THz beam with a surface power density of \qty{10}{mW/cm^2} (the IEEE safety limit) is incident on skin. Calculate the power density at a depth of 100 $\mu$m.
    \parhead{Parameters}
    \begin{itemize}
        \item Incident Intensity ($I_0$): \qty{10}{mW/cm^2}.
        \item Depth ($z$): 100 $\mu$m = \qty{0.01}{cm}.
        \item Absorption Coefficient ($\alpha$) at 1 THz in skin: $\approx \qty{200}{cm^{-1}}$.
    \end{itemize}
    \parhead{Analysis}
    The intensity decay is governed by the Beer-Lambert law: $I(z) = I_0 e^{-\alpha z}$.
    \begin{derivationsteps}
        \step \textbf{Calculate the exponent.}
        \[ \alpha z = (200 \text{ cm}^{-1}) \times (0.01 \text{ cm}) = 2.0 \]
        \step \textbf{Calculate the attenuation factor.}
        \[ e^{-\alpha z} = e^{-2} \approx 0.135 \]
        \step \textbf{Calculate the final intensity.}
        \[ I(z=100\mu\text{m}) = I_0 \times 0.135 = 10 \text{ mW/cm}^2 \times 0.135 = \textbf{\qty{1.35}{mW/cm^2}} \]
    \end{derivationsteps}
    \parhead{Interpretation} After penetrating just one-tenth of a millimetre into the skin, the intensity of the THz beam has dropped by 86.5\%. This confirms the surface-localised nature of THz-tissue interactions and is a key reason for its safety profile, as the energy is deposited in the outer, rapidly regenerating layers of the epidermis rather than in sensitive internal organs.
\end{workedexample}


\begin{importantbox}[title={Further Reading}]
    The interaction of THz waves with tissue is a specialised topic at the intersection of electromagnetics and biophysics.
    \begin{description}
        \item[Terahertz (THz) Technology] (\Cref{ch:thz}) describes the hardware used to generate and detect the signals discussed in this chapter.
        \item[The Biophysical Coupling Mechanism] (\Cref{ch:biophysical-coupling}) presents a speculative, non-classical hypothesis for THz-tissue interaction that goes beyond the absorption models discussed here.
        \item[RF Safety] (\Cref{ch:rf-safety}) provides the broader context for the safety standards and the distinction between ionising and non-ionising radiation.
    \end{description}
\end{importantbox}
