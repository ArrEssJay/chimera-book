% ==============================================================================
% CHAPTER 12: Noise Sources & Noise Figure
% ==============================================================================

\chapter{Noise Sources \& Noise Figure}
\label{ch:noise}

\begin{nontechnical}
    \textbf{Noise is the fundamental "static" that every radio receiver must overcome.} Understanding where it comes from and how to minimize it is the most critical aspect of designing a sensitive receiver.

    \parhead{The crowded restaurant analogy}
    \begin{itemize}
        \item \textbf{Signal:} Your friend's voice trying to reach you.
        \item \textbf{Noise:} The unavoidable background chatter of the restaurant.
        \item \textbf{Noise Figure:} How much a poor-quality hearing aid adds its own hiss, making it even harder to understand your friend.
    \end{itemize}

    \parhead{Key insights}
    \begin{enumerate}[label=\arabic*.]
        \item \textbf{Thermal noise is everywhere.} The random motion of electrons in any component above absolute zero creates a baseline of electrical noise. This is the fundamental, inescapable "noise floor" of the universe.
        \item \textbf{The -174 dBm/Hz magic number.} This is the power of that thermal noise floor in a 1 Hz slice of spectrum at room temperature. Every noise calculation in RF engineering starts from this number.
        \item \textbf{Amplifiers add their own noise.} Every active component in a receiver adds its own noise on top of the thermal baseline. The \textbf{Noise Figure (NF)} is a simple metric that tells you how much that component degrades the signal-to-noise ratio. A lower noise figure is better.
        \item \textbf{The first amplifier is the most important.} The Friis formula for cascaded noise proves that the noise figure of the very first component in the receiver chain (the Low-Noise Amplifier, or LNA) has the biggest impact on the entire system's performance. Any noise added at the beginning gets amplified by all subsequent stages. This is why the LNA is always placed as close to the antenna as physically possible.
    \end{enumerate}
\end{nontechnical}


\subsection{Overview}

Noise is any unwanted random energy that interferes with a desired signal, fundamentally limiting the performance of a communication system. The primary source of noise in most well-designed receivers is \keyterm{thermal noise}, an unavoidable phenomenon arising from the thermal agitation of electrons in electronic components.

The quality of a receiver is quantified by its ability to amplify a signal while adding the minimum possible amount of its own internal noise. This is measured by the \keyterm{Noise Figure (NF)} or the equivalent \keyterm{Noise Temperature ($T_e$)}.

\begin{keyconcept}
    The overall noise performance of a receiver is dominated by the noise figure of its first active component, the \textbf{Low-Noise Amplifier (LNA)}. As described by the Friis formula, the high gain of the LNA suppresses the noise contributions of all subsequent stages. Therefore, receiver design is, above all, the art of optimising the "front-end"---the antenna, the feedline, and the LNA.
\end{keyconcept}


\subsection{Thermal Noise}

The power of thermal noise (also known as Johnson-Nyquist noise) in a given bandwidth is:
\begin{equation}
    P_n = kTB
\end{equation}
where $k$ is the Boltzmann constant ($1.38 \times 10^{-23}$ J/K), $T$ is the system noise temperature in Kelvin, and $B$ is the bandwidth in Hz. The \keyterm{noise power spectral density}, or noise power in a 1 Hz bandwidth, is $N_0 = kT$.

At the standard reference temperature of $T_0 = 290$ K (approx. 17$^\circ$C), the thermal noise floor is:
\[ N_0 = (1.38 \times 10^{-23})(290) = 4.0 \times 10^{-21} \text{ W/Hz} = -174 \text{ dBm/Hz} \]
This value is the fundamental starting point for all receiver sensitivity calculations.


\subsection{Noise Figure (NF) and Noise Temperature ($T_e$)}

\paragraph{Noise Figure (NF)}
The Noise Figure of a component quantifies how much it degrades the signal-to-noise ratio from its input to its output. It is defined as:
\begin{equation}
    \text{NF}_{\text{dB}} = \text{SNR}_{\text{in, dB}} - \text{SNR}_{\text{out, dB}}
\end{equation}
An ideal, noiseless component has an NF of 0 dB. A component with an NF of 3 dB will halve the SNR, effectively doubling the noise power.

\paragraph{Noise Temperature ($T_e$)}
An alternative metric, preferred in radio astronomy and satellite communications, is the equivalent noise temperature. It represents the internal noise of a component as if it were generated by a resistor at temperature $T_e$. The two metrics are related by:
\begin{equation}
    T_e = T_0 (10^{\text{NF}_{\text{dB}}/10} - 1)
\end{equation}
A lower noise temperature corresponds to a lower noise figure and better performance.


\subsection{Cascaded Systems and the Friis Formula}

When multiple components are connected in a chain (e.g., LNA $\rightarrow$ cable $\rightarrow$ mixer), their individual noise contributions combine. The total noise factor of the cascade is given by the \keyterm{Friis formula}:
\begin{equation}
    F_{\text{total}} = F_1 + \frac{F_2 - 1}{G_1} + \frac{F_3 - 1}{G_1 G_2} + \cdots
\end{equation}
where $F_i$ and $G_i$ are the linear noise factor and linear power gain of the $i$-th stage, respectively.

\begin{warningbox}
    \textbf{The Three Rules of Low-Noise Design (from Friis's Formula):}
    \begin{enumerate}
        \item \textbf{The first stage dominates.} The noise factor of the first component, $F_1$, adds directly to the total. Therefore, the first active component \emph{must} be a Low-Noise Amplifier.
        \item \textbf{Gain is crucial.} The high gain of the first stage, $G_1$, divides and thus diminishes the noise contributions of all subsequent stages. A high-gain LNA effectively makes the rest of the receiver "invisible" from a noise perspective.
        \item \textbf{Avoid pre-LNA loss.} Any loss (e.g., from a long cable or a filter) before the LNA has a linear gain $G_1 < 1$. This loss adds almost directly to the total system noise figure and permanently degrades the signal. The LNA must be placed as close to the antenna as possible.
    \end{enumerate}
\end{warningbox}

\begin{workedexample}{Receiver Front-End Design Comparison}
    \parhead{Problem} Compare the total system noise figure of two designs: one with a 2 dB loss cable placed before a 1 dB NF LNA, and one with the components reversed. The LNA has 20 dB of gain.
    
    \parhead{Design 1: Cable $\rightarrow$ LNA (Poor Practice)}
    \begin{itemize}
        \item Stage 1 (Cable): Loss = 2 dB $\implies F_1=1.58$, $G_1=0.63$.
        \item Stage 2 (LNA): NF = 1 dB $\implies F_2=1.26$.
    \end{itemize}
    \[ F_{\text{total}} = F_1 + \frac{F_2 - 1}{G_1} = 1.58 + \frac{1.26 - 1}{0.63} = 1.58 + 0.41 = 1.99 \implies \textbf{NF}_{\text{total}} \approx \textbf{3.0 dB} \]
    The total noise figure is dominated by the initial cable loss.

    \parhead{Design 2: LNA $\rightarrow$ Cable (Best Practice)}
    \begin{itemize}
        \item Stage 1 (LNA): NF = 1 dB $\implies F_1=1.26$, $G_1=100$ (20 dB).
        \item Stage 2 (Cable): Loss = 2 dB $\implies F_2=1.58$.
    \end{itemize}
    \[ F_{\text{total}} = F_1 + \frac{F_2 - 1}{G_1} = 1.26 + \frac{1.58 - 1}{100} = 1.26 + 0.0058 = 1.2658 \implies \textbf{NF}_{\text{total}} \approx \textbf{1.0 dB} \]

    \parhead{Interpretation} Placing the LNA first results in a system noise figure that is almost identical to that of the LNA alone. The high gain of the LNA has made the noise contribution of the subsequent cable negligible. This 1.9 dB improvement in system noise figure is a massive gain in performance, equivalent to nearly doubling the size of the receiving antenna.
\end{workedexample}


\subsection{System Noise Temperature and G/T}

In satellite and radio astronomy, the overall system performance is often characterised by the \keyterm{system noise temperature}, $T_{\text{sys}}$. This is the sum of the antenna noise temperature ($T_{\text{ant}}$, which captures noise from the environment the antenna is pointing at) and the receiver's equivalent noise temperature ($T_e$).
\begin{equation}
    T_{\text{sys}} = T_{\text{ant}} + T_e
\end{equation}
The ultimate figure of merit for a ground station is its \keyterm{G/T ratio} ("G over T"), which compares the antenna gain to the total system noise temperature, expressed in dB/K. A higher G/T ratio indicates a more sensitive receiving system.


\begin{importantbox}[title={Further Reading}]
    A deep understanding of noise is essential for designing high-performance receivers.
    \begin{description}
        \item[Link Budget Analysis] (\Cref{ch:linkbudget}) provides the complete framework for calculating receiver sensitivity, incorporating the noise figure and thermal noise floor.
        \item[Signal-to-Noise Ratio (SNR)] (\Cref{ch:snr}) is the direct result of the received signal power competing against the noise floor established by the system's noise figure.
        \item[Antenna Theory] (\Cref{ch:antenna}) explores the concept of antenna noise temperature and how the environment contributes to the overall system noise.
    \end{description}
\end{importantbox}