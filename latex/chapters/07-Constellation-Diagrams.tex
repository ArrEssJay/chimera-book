% ==============================================================================
% CHAPTER 7: Constellation Diagrams
% ==============================================================================

\chapter{Constellation Diagrams}
\label{ch:constellations}

\begin{nontechnical}
    \textbf{A constellation diagram is a visual map of all the possible "hand signals" a radio can use to send data.} Each dot on the map represents a unique signal that encodes a specific group of bits.

    \parhead{The lighthouse analogy} Imagine communicating with a lighthouse. You could send simple messages by turning the light on and off. But to send data faster, you could also vary its \textbf{brightness} (amplitude) and its \textbf{colour} (phase). A constellation diagram is the map of every unique brightness-and-colour combination you've agreed upon.

    \parhead{Real-world adaptation} Your WiFi router does this constantly. In good conditions (high SNR), it uses a complex map with many points, like 64-QAM (64 points, sending 6 bits at a time). When the signal is weak, it switches to a simpler map, like QPSK (4 points, sending only 2 bits at a time), to ensure the message gets through without errors.

    \parhead{Why it matters} More points on the map mean faster data, but the points must be far enough apart to be distinguished from each other, even with noise. The "scatter" of received points on the diagram is the single best indicator of your connection quality.
\end{nontechnical}


\subsection{Overview}

A \keyterm{constellation diagram} is a two-dimensional plot of all possible symbols for a given digital modulation scheme, represented in the complex IQ plane. Each point on the diagram, or \keyterm{constellation point}, corresponds to a unique combination of amplitude and phase that a transmitter can send to represent a specific sequence of bits.

\begin{keyconcept}
    The constellation diagram is the most important diagnostic tool in digital communications. It provides an immediate, intuitive visualization of signal quality. In an ideal channel, received symbols fall precisely on the constellation points. In a real-world channel, noise and impairments cause the received symbols to form a "cloud" or scatter plot. The size and shape of these clouds reveal the type and severity of channel degradation.
\end{keyconcept}


\subsection{Modulation Types}

\paragraph{Phase Shift Keying (PSK)}
In PSK schemes, all constellation points lie on a circle of constant radius (constant amplitude). Information is encoded exclusively in the \keyterm{phase} of the carrier.
\begin{itemize}
    \item \textbf{BPSK (Binary PSK):} The simplest form, with two points at 0$^\circ$ and 180$^\circ$, encoding 1 bit per symbol. It is extremely robust to noise due to the maximum possible separation between points.
    \item \textbf{QPSK (Quadrature PSK):} Uses four points, typically at 45$^\circ$, 135$^\circ$, 225$^\circ$, and 315$^\circ$, encoding 2 bits per symbol. It offers double the spectral efficiency of BPSK for the same bandwidth.
    \item \textbf{Higher-Order PSK (8-PSK, 16-PSK):} Use 8 or 16 points, respectively. As more points are crowded onto the circle, the distance between them decreases rapidly, making these schemes much more susceptible to noise.
\end{itemize}

\paragraph{Quadrature Amplitude Modulation (QAM)}
In QAM schemes, information is encoded in both the \keyterm{amplitude} and \keyterm{phase} of the carrier. Constellation points are typically arranged in a square or rectangular grid.
\begin{itemize}
    \item \textbf{16-QAM:} A 4x4 grid of 16 points, encoding 4 bits per symbol.
    \item \textbf{64-QAM:} An 8x8 grid of 64 points, encoding 6 bits per symbol.
    \item \textbf{256-QAM and 1024-QAM:} Used in high-SNR environments like WiFi 6 and DOCSIS cable modems, encoding 8 and 10 bits per symbol, respectively.
\end{itemize}
QAM offers significantly higher spectral efficiency than PSK for the same number of points, as the grid structure allows for greater separation between symbols. However, it requires a linear power amplifier, as the signal envelope is not constant.


\subsection{Visualizing Channel Impairments}

The shape of the received symbol clouds on a constellation diagram is a powerful diagnostic tool.

\begin{center}
    \begin{tabular}{@{}ll@{}}
        \toprule
        \tableheaderfont Impairment & \tableheaderfont Constellation Appearance \\
        \midrule
        \textbf{AWGN (Noise)} & Symmetrical, circular clouds around each point. \\
        \textbf{Phase Noise} & Points are smeared into arcs around the origin. \\
        \textbf{Carrier Freq. Offset} & The entire constellation rotates continuously. \\
        \textbf{IQ Imbalance} & The constellation becomes skewed, no longer perfectly symmetrical. \\
        \textbf{Amplifier Compression} & The outer points are compressed inward towards the center. \\
        \bottomrule
    \end{tabular}
\end{center}


\subsection{Error Vector Magnitude (EVM)}

\keyterm{Error Vector Magnitude (EVM)} is the standard metric for quantifying the quality of a constellation. It is the root-mean-square (RMS) value of the error vectors, where the error vector is the difference between the ideal constellation point and the actual received symbol. It is typically expressed as a percentage.

\begin{equation}
    \text{EVM}_{\text{RMS}} (\%) = \sqrt{\frac{\frac{1}{N}\sum_{k=1}^{N}|s_k - r_k|^2}{\frac{1}{M}\sum_{m=1}^{M}|s_m|^2}} \times 100\%
\end{equation}
A lower EVM indicates a higher-quality signal with less noise and distortion. For a given modulation scheme, there is a maximum EVM that can be tolerated before the bit error rate becomes unacceptable.

\begin{table}[H]
    \centering
    \caption{Typical EVM Requirements for Common Standards}
    \label{tab:evm-requirements}
    \begin{tabular}{@{}lrc@{}}
        \toprule
        \tableheaderfont Modulation & \tableheaderfont EVM Limit (\%) & \tableheaderfont Approx. SNR (dB) \\
        \midrule
        QPSK & 17.5 & 15 \\
        16-QAM & 12.5 & 18 \\
        64-QAM & 8.0 & 22 \\
        256-QAM & 3.5 & 29 \\
        1024-QAM & 1.8 & 35 \\
        \bottomrule
    \end{tabular}
\end{table}


\subsection{Adaptive Modulation and Coding (AMC)}

Modern wireless systems like 5G and WiFi use \keyterm{Adaptive Modulation and Coding (AMC)} to maximize data throughput. The transmitter and receiver constantly monitor the channel quality (typically by measuring the received SNR or EVM).
\begin{itemize}
    \item In \textbf{good conditions} (high SNR), the system will use a high-order constellation like 256-QAM to achieve very high data rates.
    \item In \textbf{poor conditions} (low SNR), the system will "fall back" to a more robust, lower-order constellation like QPSK to ensure the link remains stable, albeit at a lower data rate.
\end{itemize}
This dynamic adaptation ensures the link is always operating at the highest possible spectral efficiency for the current channel conditions.


\begin{workedexample}{16-QAM Link Budget Analysis}
    \parhead{Problem} A 1 km wireless link operates at 2.4 GHz. Given the system parameters, calculate the available link margin for a 16-QAM signal.
    \parhead{System Parameters}
    \begin{itemize}
        \item Modulation: 16-QAM (BER $10^{-6}$ requires $E_b/N_0 \approx 14.5$ dB)
        \item Bit Rate: \qty{10}{Mbps} $\implies$ Symbol Rate $R_s = 2.5$ Msps
        \item Bandwidth: \qty{3.125}{MHz} (with pulse shaping)
        \item Transmit Power ($P_t$): \qty{30}{dBm} (1 W)
        \item Antenna Gains ($G_t, G_r$): \qty{10}{dBi} each
        \item Receiver Noise Figure (NF): \qty{5}{dB}
    \end{itemize}
    \parhead{Solution}
    \begin{derivationsteps}
        \step Calculate Free-Space Path Loss (FSPL) at 1 km and 2.4 GHz.
        \[ \text{FSPL} = 20\log_{10}(d_{\text{km}}) + 20\log_{10}(f_{\text{MHz}}) + 32.45 = 20\log_{10}(1) + 20\log_{10}(2400) + 32.45 \approx \qty{100.0}{dB} \]
        \step Calculate the received signal power ($P_r$).
        \[ P_r = P_t + G_t + G_r - \text{FSPL} = 30 + 10 + 10 - 100 = \qty{-50}{dBm} \]
        \step Calculate the receiver's thermal noise floor ($N_0$) in dBm/Hz.
        \[ N_0 = -174 \text{ dBm/Hz} + \text{NF} = -174 + 5 = \qty{-169}{dBm/Hz} \]
        \step Calculate the available energy per bit to noise density ratio ($E_b/N_0$).
        \[ E_b/N_0 = P_r - 10\log_{10}(R_b) - N_0 = -50 - 10\log_{10}(10^7) - (-169) = -50 - 70 + 169 = \qty{49}{dB} \]
        \step Calculate the link margin.
        \[ \text{Margin} = (\text{Available } E_b/N_0) - (\text{Required } E_b/N_0) = 49 - 14.5 = \qty{34.5}{dB} \]
    \end{derivationsteps}
    \parhead{Interpretation} The link has a very healthy margin of 34.5 dB. This indicates that the system is robust and can easily tolerate significant fading, interference, or other real-world impairments. It also suggests that a higher-order modulation, such as 64-QAM or 256-QAM, could be used to achieve a much higher data rate over this link.
\end{workedexample}


\begin{importantbox}[title={Further Reading}]
    Constellation diagrams are the visual language of digital modulation. Understanding them is key to mastering the following topics:
    \begin{description}
        \item[Modulation Schemes] (\Cref{ch:bpsk}, \Cref{ch:qpsk}, \Cref{ch:qam}) provides the specific details of how common constellations are constructed and used.
        \item[Channel Impairments] (\Cref{ch:impairments}) offers a deep dive into how phenomena like phase noise and amplifier compression manifest on a constellation diagram.
        \item[Error Correction Coding] (\Cref{ch:fec}) explains the techniques used to correct the symbol errors that occur when noise pushes received samples across decision boundaries.
    \end{description}
\end{importantbox}