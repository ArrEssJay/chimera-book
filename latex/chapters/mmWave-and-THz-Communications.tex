\section{mmWave \& THz Communications}\label{mmwave-thz-communications}

\textbf{Millimeter-wave (mmWave, 24-300 GHz)} and \textbf{Terahertz
(THz, 0.3-10 THz)} communications exploit ultra-high-frequency spectrum
for multi-gigabit wireless links. 5G NR FR2 (24-52 GHz) delivers 20+
Gbps, while future 6G targets sub-THz (100-300 GHz) for 100+ Gbps. These
bands offer massive bandwidth but face severe propagation challenges
requiring advanced beamforming and novel system architectures.

\begin{center}\rule{0.5\linewidth}{0.5pt}\end{center}

\subsection{\texorpdfstring{ For Non-Technical
Readers}{ For Non-Technical Readers}}\label{for-non-technical-readers}

Think of wireless communication like water flowing through pipes of
different sizes:

\textbf{The Water Pipe Analogy}:

\begin{verbatim}
Regular WiFi (2.4 GHz):     [====] Small pipe
                            Flow: 100 Mbps (like a garden hose)

5G mmWave (28 GHz):         [================] Medium pipe
                            Flow: 5 Gbps (like a fire hydrant)

Future 6G (300 GHz):        [================================] Large pipe
                            Flow: 100+ Gbps (like a water main)
\end{verbatim}

\textbf{Why the excitement?} - Download a 4K movie in 3 seconds instead
of 3 minutes - 100 people streaming 4K video in a stadium without
buffering - Wireless connections as fast as fiber optic cables

\textbf{The catch?} Just like a powerful fire hose that only shoots
water in one direction for a short distance: - \textbf{Range}: Works
great up close (50-200 meters) but not across town -
\textbf{Line-of-sight}: Needs clear path---walls, trees, even rain
blocks it - \textbf{Direction}: Works like a flashlight beam, not a
light bulb (requires aiming)

\textbf{Real-world examples you might recognize}: - \textbf{``5G Ultra
Wideband''} (Verizon/AT\&T): This is mmWave---super fast in cities,
slower in suburbs - \textbf{Stadium WiFi}: mmWave lets thousands of fans
upload videos simultaneously - \textbf{Fixed Wireless Internet}: Instead
of cable/fiber to your house, an antenna on your roof beams mmWave from
a nearby tower - \textbf{Self-driving cars}: 77 GHz radar ``sees'' other
vehicles even in fog

\textbf{The tradeoff}:

\begin{verbatim}
Lower frequencies (like FM radio):
 Go far (miles)
 Go through walls
 Slower speeds (like dial-up)

Higher frequencies (mmWave/THz):
 Ultra-fast speeds (like fiber)
 Short range (city block)
 Need clear view (blocked by obstacles)

It's not better or worse---it's choosing the right tool for the job!
\end{verbatim}

\textbf{What this means for you}: - \textbf{Today}: Your phone switches
between regular 5G (wide coverage) and mmWave (speed bursts in cities) -
\textbf{Tomorrow}: Ultra-fast wireless in your home/office, slow when
you walk outside - \textbf{Future (6G)}: Wireless faster than
today\textquotesingle s hardwired internet, but only indoors or short
outdoor distances

\textbf{The technical stuff below explains \emph{how} this magic
works---but you don\textquotesingle t need to understand it to
benefit from it!}

\begin{center}\rule{0.5\linewidth}{0.5pt}\end{center}

\subsection{\texorpdfstring{ Why mmWave \&
THz?}{ Why mmWave \& THz?}}\label{why-mmwave-thz}

\subsubsection{The Spectrum Crunch}\label{the-spectrum-crunch}

\textbf{Sub-6 GHz problem}:

\begin{verbatim}
Available spectrum: ~1 GHz (fragmented across bands)
Demand: Exponential growth (video, AR/VR, IoT)
Result: Spectrum scarcity  congestion

Shannon capacity:
C = B · log(1 + SNR)

To increase C:
- Increase B (bandwidth)  Move to higher frequencies 
- Increase SNR  Limited by power, interference
\end{verbatim}

\textbf{mmWave/THz solution}:

\begin{verbatim}
mmWave (24-52 GHz): 28 GHz bandwidth available (5G FR2)
Sub-THz (100-300 GHz): 200 GHz bandwidth potential (6G)
THz (1-10 THz): Multi-THz bandwidths (research)

Example (100 GHz carrier, 10 GHz BW, SNR = 20 dB):
C = 10 GHz · log(1 + 100) = 66 Gbps

Compare to sub-6 GHz (100 MHz BW):
C = 100 MHz · log(1 + 100) = 660 Mbps

100× more bandwidth  100× higher capacity!
\end{verbatim}

\begin{center}\rule{0.5\linewidth}{0.5pt}\end{center}

\subsection{\texorpdfstring{ Propagation
Characteristics}{ Propagation Characteristics}}\label{propagation-characteristics}

\subsubsection{Path Loss: The Main
Challenge}\label{path-loss-the-main-challenge}

\textbf{Free-space path loss} (FSPL):

\begin{verbatim}
FSPL(dB) = 32.4 + 20·log(f_MHz) + 20·log(d_km)

Example comparisons (d = 100 m):

2.4 GHz (WiFi):
FSPL = 32.4 + 20·log(2400) + 20·log(0.1) = 80 dB

28 GHz (5G mmWave):
FSPL = 32.4 + 20·log(28000) + 20·log(0.1) = 101 dB

300 GHz (sub-THz):
FSPL = 32.4 + 20·log(300000) + 20·log(0.1) = 122 dB

Relative loss:
28 GHz: +21 dB worse than 2.4 GHz
300 GHz: +42 dB worse than 2.4 GHz

Implication: Higher frequency  much shorter range (or need much higher antenna gain)
\end{verbatim}

\begin{center}\rule{0.5\linewidth}{0.5pt}\end{center}

\subsubsection{Atmospheric Absorption}\label{atmospheric-absorption}

\textbf{Oxygen (O\textsubscript{2}) and water vapor
(H\textsubscript{2}O)} absorb mmWave/THz strongly.

\textbf{Absorption peaks}:

\begin{verbatim}
Frequency (GHz) | Attenuation (dB/km at sea level) | Cause
----------------|-----------------------------------|-------
60              | 15 dB/km                         | O resonance
120             | 2 dB/km                          | O 2nd harmonic
183             | 2 dB/km                          | HO resonance
325             | 1 dB/km                          | HO
380-750         | 0.1-1 dB/km                      | Windows (low absorption)
>1 THz          | 10-100 dB/km                     | Multiple molecular resonances

Transmission windows:
- 71-76 GHz, 81-86 GHz (5G FR2 upper band)
- 94 GHz (radar, imaging)
- 130-175 GHz (low absorption)
- 220-325 GHz (6G candidate)
\end{verbatim}

\textbf{Distance implications}:

\begin{verbatim}
Example (100 m link):

28 GHz: 0.1 dB/km × 0.1 km = 0.01 dB (negligible)
60 GHz: 15 dB/km × 0.1 km = 1.5 dB (moderate)
300 GHz: 1 dB/km × 0.1 km = 0.1 dB (low, in window)
1 THz: 50 dB/km × 0.1 km = 5 dB (significant)

Indoor/short-range: Absorption manageable
Outdoor/long-range: Limits reach to <1 km
\end{verbatim}

\textbf{Weather effects}:

\begin{verbatim}
Rain attenuation (ITU-R model):
 = k · R^  dB/km

where R = rain rate (mm/h)

At 28 GHz (heavy rain, 50 mm/h):
  5 dB/km  100 m link: 0.5 dB

At 300 GHz (same rain):
  15 dB/km  100 m link: 1.5 dB

THz: Extremely sensitive to humidity, fog, rain
 Indoor/short-range only in adverse weather
\end{verbatim}

\begin{center}\rule{0.5\linewidth}{0.5pt}\end{center}

\subsubsection{Blockage \& Diffraction}\label{blockage-diffraction}

\textbf{Non-Line-of-Sight (NLOS) problem}:

\begin{verbatim}
mmWave/THz wavelengths:
 = c/f

28 GHz:  = 10.7 mm (1 cm)
300 GHz:  = 1 mm

Diffraction scales with :
- Lower frequencies: Diffract around obstacles (wavelength ~ building size)
- mmWave: Minimal diffraction (wavelength << human body)
- THz: No practical diffraction (wavelength ~ grain of sand)

Blockage:
- Human body: 20-40 dB attenuation (28 GHz)
- Hand: 10-20 dB
- Wall: 30-80 dB (depends on material)
- Foliage: 10-30 dB

Result: Highly directional, LOS-dependent propagation
\end{verbatim}

\textbf{Multipath in mmWave/THz}:

\begin{verbatim}
Sparse multipath environment:
- Few reflections reach receiver (high absorption, blockage)
- Reflections off smooth surfaces (specular, not diffuse)
- Delay spread: Shorter than sub-6 GHz (fewer paths)

Advantage: Simpler channel model (ray-tracing accurate)
Disadvantage: No diversity from multipath  beamforming essential
\end{verbatim}

\begin{center}\rule{0.5\linewidth}{0.5pt}\end{center}

\subsection{\texorpdfstring{ Beamforming: The Enabling
Technology}{ Beamforming: The Enabling Technology}}\label{beamforming-the-enabling-technology}

\textbf{Why beamforming is mandatory}:

\begin{verbatim}
Path loss compensation:
- 28 GHz: 21 dB more loss than 2.4 GHz
- Need: 21 dB+ antenna gain to match range

Beamforming gain:
G(dB) = 10·log(N)  (for N-element array)

Example (64-element array):
G = 10·log(64) = 18 dB

With 256 elements:
G = 10·log(256) = 24 dB

Overcomes path loss + provides spatial selectivity
\end{verbatim}

\begin{center}\rule{0.5\linewidth}{0.5pt}\end{center}

\subsubsection{Analog Beamforming}\label{analog-beamforming}

\textbf{Architecture}:

\begin{verbatim}
Single RF chain  Phase shifters on each antenna element

TX: Data  DAC  Mixer  Power Divider  [Phase Shifters]  Antenna Array
                                           
                                    All elements see same data
                                    Phase shifts steer beam

Advantages:
- Low power (1 RF chain)
- Simple, cost-effective
- High gain (all power focused)

Disadvantages:
- Single beam at a time
- Cannot do MIMO spatial multiplexing
- Fixed beam (hard to adapt dynamically)
\end{verbatim}

\textbf{Phase shift calculation}:

\begin{verbatim}
Desired beam direction: 
Element spacing: d (typically /2)

Phase shift for element n:
 = (2/) · n·d · sin()

Example (28 GHz,  = 30°, d = /2):
 = 10.7 mm
 =  · n · sin(30°) = /2 · n

Element 0: 0°
Element 1: 90°
Element 2: 180°
Element 3: 270°
\end{verbatim}

\begin{center}\rule{0.5\linewidth}{0.5pt}\end{center}

\subsubsection{Hybrid Beamforming}\label{hybrid-beamforming}

\textbf{Compromise}: Analog beamforming per subarray + digital baseband
processing.

\begin{verbatim}
Architecture:
Data streams  [Digital Precoder]  DACs (N_{r} chains)  Mixers 
               [Analog Phase Shifters per subarray]  Antenna Array (N elements)

Where N_{r} << N

Example:
- Total antennas: 256
- RF chains: 16
- Digital precoding: 16 streams (MIMO)
- Analog beamforming: 256/16 = 16 elements per subarray

Benefits:
- Multi-beam capability (N_{r} simultaneous beams)
- MIMO spatial multiplexing (up to N_{r} streams)
- Moderate power/cost (N_{r} RF chains)
\end{verbatim}

\textbf{Precoding}:

\begin{verbatim}
Transmit signal: x = F_analog · F_digital · s

where:
- s: Data streams (N × 1, N  N_{r})
- F_digital: Digital precoder (N_{r} × N)
- F_analog: Analog beamformer (N × N_{r}, phase-only)

Optimization:
Maximize: ||H · F_analog · F_digital||²
Subject to: F_analog has constant-modulus entries (phase-only)

Algorithms: Orthogonal Matching Pursuit (OMP), alternating minimization
\end{verbatim}

\begin{center}\rule{0.5\linewidth}{0.5pt}\end{center}

\subsubsection{Beam Management}\label{beam-management}

\textbf{Challenge}: Narrow beams must be steered to track users.

\textbf{Beam sweeping (initial access)}:

\begin{verbatim}
1. BS transmits sync signals on multiple beam directions
2. UE measures RSRP (Reference Signal Received Power) per beam
3. UE reports best beam index to BS
4. BS selects beam for data transmission

Example (5G NR):
- BS: 64 beam directions (8×8 azimuth/elevation grid)
- Sweep time: 5 ms (one beam per SSB - SS/PBCH Block)
- UE selects best beam (e.g., beam 23)
- Data transmission on beam 23

Beamwidth: ~10° (64-element array at 28 GHz)
\end{verbatim}

\textbf{Beam tracking}:

\begin{verbatim}
Problem: User moves  beam misalignment  link failure

Solutions:
1. Periodic re-sweeping (every 20-100 ms)
2. Predictive tracking:
   - Estimate velocity from Doppler
   - Adjust beam direction proactively
3. Multi-beam transmission:
   - Transmit on 2-3 adjacent beams
   - Handover smoothly as user moves

5G NR: Beam Failure Recovery (BFR)
- UE monitors beam quality (RSRP)
- If below threshold: Trigger beam switch
- Latency: <10 ms for recovery
\end{verbatim}

\begin{center}\rule{0.5\linewidth}{0.5pt}\end{center}

\subsection{\texorpdfstring{ 5G NR FR2
(mmWave)}{ 5G NR FR2 (mmWave)}}\label{g-nr-fr2-mmwave}

\textbf{Frequency Range 2}: 24.25-52.6 GHz

\subsubsection{Frequency Bands}\label{frequency-bands}

\begin{verbatim}
n257: 26.5-29.5 GHz (3 GHz BW)
n258: 24.25-27.5 GHz
n260: 37-40 GHz
n261: 27.5-28.35 GHz

Typical deployment:
- n257 (28 GHz): US carriers (Verizon, AT&T)
- n258 (26 GHz): Europe, Asia
- n260 (39 GHz): US (fixed wireless access)
\end{verbatim}

\begin{center}\rule{0.5\linewidth}{0.5pt}\end{center}

\subsubsection{5G NR mmWave
Specifications}\label{g-nr-mmwave-specifications}

\begin{verbatim}
Bandwidth: 50-400 MHz per carrier
- Typical: 100 MHz (lower latency, easier beam management)
- Maximum: 400 MHz (peak throughput)

Numerology:
- SCS (Subcarrier Spacing): 120 kHz (fast Doppler tolerance)
- Symbol duration: 8.33 s (short, good for mobility)
- Slot: 0.125 ms (8× faster than sub-6 GHz)

Modulation: Up to 256-QAM (spectral efficiency: 7.4 bits/s/Hz)

MIMO: Up to 4 layers (spatial multiplexing with hybrid beamforming)

Peak data rate:
R = BW × Spectral_Eff × MIMO_layers × Aggregation
  = 400 MHz × 7.4 × 4 × 1 = 11.8 Gbps (single carrier)
  
With carrier aggregation (8 carriers):
R = 11.8 × 8 = 94 Gbps (theoretical)

Practical: 2-5 Gbps (typical deployment, moderate SINR)
\end{verbatim}

\begin{center}\rule{0.5\linewidth}{0.5pt}\end{center}

\subsubsection{Applications}\label{applications}

\textbf{Enhanced Mobile Broadband (eMBB)}:

\begin{verbatim}
Use case: Stadiums, airports, malls (high user density)
- 1000+ users per cell
- Aggregate: 20-50 Gbps per gNB
- Per-user: 20-50 Mbps (shared capacity)

Deployment: Small cells (50-200 m range)
- Dense urban: 1 cell per block
- Outdoor-to-indoor: Penetration challenges (require indoor cells)
\end{verbatim}

\textbf{Fixed Wireless Access (FWA)}:

\begin{verbatim}
Use case: Home/business internet (alternative to fiber/cable)
- CPE (Customer Premises Equipment) on roof/window
- LOS to nearby gNB (200-500 m)
- Throughput: 1-3 Gbps (comparable to gigabit fiber)
- Latency: 10-20 ms

Advantage: Rapid deployment (no trenching)
Disadvantage: Weather-sensitive, requires LOS or near-LOS
\end{verbatim}

\textbf{Industrial IoT / URLLC}:

\begin{verbatim}
Use case: Factory automation, robotics
- Latency: 1-5 ms (mini-slot transmission)
- Reliability: 99.999% (5 nines)
- Capacity: 10-100 Mbps per device

Private 5G networks:
- Dedicated spectrum (CBRS, local licensing)
- On-premises gNB (security, low latency)
\end{verbatim}

\begin{center}\rule{0.5\linewidth}{0.5pt}\end{center}

\subsection{\texorpdfstring{ Beyond 5G: Sub-THz
(6G)}{ Beyond 5G: Sub-THz (6G)}}\label{beyond-5g-sub-thz-6g}

\textbf{6G target frequencies}: 100-300 GHz (D-band, G-band)

\subsubsection{Why Sub-THz for 6G?}\label{why-sub-thz-for-6g}

\begin{verbatim}
Bandwidth availability:
- 92-114.25 GHz (WRC-19): 22 GHz continuous
- 130-174.8 GHz: 44 GHz
- 200-260 GHz: 60 GHz (being considered)

Total: 100+ GHz spectrum (vs. 5 GHz for all cellular below 6 GHz!)

Peak data rate (conservative estimate):
BW = 10 GHz, SE = 5 bits/s/Hz, MIMO = 8
R = 10 × 5 × 8 = 400 Gbps

Target: 100 Gbps-1 Tbps (100× faster than 5G)
\end{verbatim}

\begin{center}\rule{0.5\linewidth}{0.5pt}\end{center}

\subsubsection{Sub-THz Challenges}\label{sub-thz-challenges}

\textbf{1. Path Loss}:

\begin{verbatim}
300 GHz FSPL (100 m): 122 dB
Compensation:
- Ultra-massive MIMO: 1024+ elements  30 dB gain
- Dense deployment: 10-50 m cell radius (pico/femto cells)
- Relay/RIS: Intelligent reflecting surfaces
\end{verbatim}

\textbf{2. Hardware Limitations}:

\begin{verbatim}
PA (Power Amplifier):
- 28 GHz: 20-30 dBm per element (mature GaN technology)
- 300 GHz: 5-10 dBm per element (InP, SiGe limited)

Phase shifters:
- 28 GHz: 4-6 bit resolution, low loss
- 300 GHz: 2-3 bit (lossy, expensive)

ADC/DAC:
- Nyquist rate: 2× bandwidth
- 10 GHz BW  20 Gsps ADC/DAC
- Power: 10-100 W per RF chain (prohibitive for mobile)

Solution: Ultra-low-power circuits (sub-threshold, approximate computing)
\end{verbatim}

\textbf{3. Beam Alignment}:

\begin{verbatim}
Beamwidth (1024-element array at 300 GHz):
   / (N·d)  1 mm / (32 × 0.5 mm) = 0.06 rad  3.5°

Challenge: <4° beam  precise alignment required
- Rotation/motion: 10°/s movement  beam misalignment in 0.35 s
- Solution: 100+ Hz beam tracking

Beam switching latency:
- Analog: <1 s (phase shifter settling)
- Digital: 10-100 s (baseband processing)
- Requirement: <1 ms for mobility
\end{verbatim}

\begin{center}\rule{0.5\linewidth}{0.5pt}\end{center}

\subsubsection{6G Candidate
Technologies}\label{g-candidate-technologies}

\textbf{Reconfigurable Intelligent Surface (RIS)}:

\begin{verbatim}
Concept: Passive reflector with electronically tunable elements

Application:
- Coverage extension: Reflect signal around obstacles
- Virtual LOS: Create alternative paths in NLOS
- Energy efficiency: Passive (no power amplifier)

Example:
- RIS: 1024 elements (1m × 1m panel)
- Placement: Building wall
- Reflect 300 GHz signal from BS to blocked UE
- Gain: 20-30 dB (overcome blockage loss)

Status: Research prototypes, not yet standardized
\end{verbatim}

\textbf{Wireless Fiber (WF)}:

\begin{verbatim}
Concept: Short-range (1-10 m), fiber-like data rates

Use case: Wireless backhaul, kiosk downloads, data center links
- Frequency: 300 GHz
- Bandwidth: 20-50 GHz (entire band)
- Data rate: 100-200 Gbps
- Range: <10 m (LOS required)

Advantage: 100× faster than WiFi, no fiber installation
Disadvantage: Ultra-short range, perfect alignment needed
\end{verbatim}

\textbf{OAM (Orbital Angular Momentum) Multiplexing}:

\begin{verbatim}
Concept: Use twisted EM waves (vortex beams) as additional dimension

Orthogonal OAM modes: l = 0, ±1, ±2, ...
- Each mode carries independent data stream
- Separation by phase profile (not frequency)

Capacity:
C = N_OAM × N_MIMO × B × SE

Example (N_OAM = 4, N_MIMO = 8, B = 10 GHz, SE = 5):
C = 4 × 8 × 10 × 5 = 1.6 Tbps

Status: Lab demonstrations, far from practical (alignment critical)
\end{verbatim}

\begin{center}\rule{0.5\linewidth}{0.5pt}\end{center}

\subsection{\texorpdfstring{ Automotive Radar
(mmWave)}{ Automotive Radar (mmWave)}}\label{automotive-radar-mmwave}

\textbf{77-81 GHz radar} for autonomous vehicles.

\subsubsection{System Parameters}\label{system-parameters}

\begin{verbatim}
Frequency: 76-81 GHz (5 GHz bandwidth allocated)
Modulation: FMCW (Frequency-Modulated Continuous Wave)
Range resolution: r = c / (2·BW) = 3 cm (for 5 GHz BW)
Velocity resolution: Doppler shift
Angular resolution: Beamforming (MIMO radar)

Performance:
- Detection range: 200+ m (long-range radar)
- Velocity: ±70 m/s (Doppler)
- Angle: ±60° (wide FoV for short-range, ±10° for long-range)
- Update rate: 10-20 Hz

Applications:
- Adaptive Cruise Control (ACC)
- Collision avoidance
- Blind-spot detection
- Parking assistance
\end{verbatim}

\textbf{MIMO radar}:

\begin{verbatim}
Virtual array: N_TX × N_RX elements
- Physical: 3 TX, 4 RX = 12 elements
- Virtual: 3 × 4 = 12 unique TX-RX pairs (phase centers)
- Angular resolution: Equivalent to 12-element receive array

Imaging:
- Range-Doppler map (2D)
- Range-Angle map (2D)
- 3D point cloud (range-azimuth-elevation)

Example (Bosch 5th gen):
- TX: 3 antennas
- RX: 4 antennas
- Virtual: 12 elements
- Angular resolution: 1° (azimuth)
\end{verbatim}

\begin{center}\rule{0.5\linewidth}{0.5pt}\end{center}

\subsection{\texorpdfstring{ Link Budget Example (28
GHz)}{ Link Budget Example (28 GHz)}}\label{link-budget-example-28-ghz}

\begin{verbatim}
System: 5G FR2 mmWave (28 GHz, 100 MHz BW)

Transmitter (gNB):
- TX power per element: 23 dBm (200 mW)
- Number of elements: 64
- Total TX power: 23 + 10·log(64) = 41 dBm
- Analog beamforming gain: 18 dB (64 elements, single beam)
- EIRP: 41 + 18 = 59 dBm

Path:
- Distance: 100 m
- FSPL: 32.4 + 20·log(28000) + 20·log(0.1) = 101 dB
- Atmospheric absorption: 0.01 dB (negligible)
- Blockage margin: 10 dB (foliage, wall)
- Total loss: 111 dB

Receiver (UE):
- RX antenna gain: 10 dB (16-element array)
- Noise figure: 7 dB
- Thermal noise: -174 + 10·log(100 MHz) + 7 = -87 dBm

Received signal:
RX power = 59 - 111 + 10 = -42 dBm

SNR:
SNR = -42 - (-87) = 45 dB

Throughput (Shannon):
C = 100 MHz × log(1 + 10^(45/10)) = 100 MHz × 15 = 1.5 Gbps

Practical (256-QAM, rate-5/6, 75% efficiency):
R = 100 MHz × 7.4 × 0.75 = 555 Mbps

Margin: 45 - 20 (required SNR for 256-QAM) = 25 dB 
\end{verbatim}

\begin{center}\rule{0.5\linewidth}{0.5pt}\end{center}

\subsection{\texorpdfstring{ Python Example: mmWave Path Loss
Calculator}{ Python Example: mmWave Path Loss Calculator}}\label{python-example-mmwave-path-loss-calculator}

\begin{Shaded}
\begin{Highlighting}[]
\ImportTok{import}\NormalTok{ numpy }\ImportTok{as}\NormalTok{ np}

\KeywordTok{def}\NormalTok{ mmwave\_path\_loss(freq\_ghz, distance\_m, rain\_rate\_mm\_h}\OperatorTok{=}\DecValTok{0}\NormalTok{):}
    \CommentTok{"""}
\CommentTok{    Calculate mmWave path loss including atmospheric effects.}
\CommentTok{    }
\CommentTok{    Args:}
\CommentTok{        freq\_ghz: Frequency (GHz)}
\CommentTok{        distance\_m: Distance (meters)}
\CommentTok{        rain\_rate\_mm\_h: Rain rate (mm/h, optional)}
\CommentTok{    }
\CommentTok{    Returns:}
\CommentTok{        Total path loss (dB)}
\CommentTok{    """}
    \CommentTok{\# Free{-}space path loss}
\NormalTok{    fspl }\OperatorTok{=} \FloatTok{32.4} \OperatorTok{+} \DecValTok{20}\OperatorTok{*}\NormalTok{np.log10(freq\_ghz}\OperatorTok{*}\DecValTok{1000}\NormalTok{) }\OperatorTok{+} \DecValTok{20}\OperatorTok{*}\NormalTok{np.log10(distance\_m}\OperatorTok{/}\DecValTok{1000}\NormalTok{)}
    
    \CommentTok{\# Atmospheric absorption (simplified model)}
    \ControlFlowTok{if}\NormalTok{ freq\_ghz }\OperatorTok{\textless{}} \DecValTok{30}\NormalTok{:}
\NormalTok{        attenuation\_db\_km }\OperatorTok{=} \FloatTok{0.1}
    \ControlFlowTok{elif}\NormalTok{ freq\_ghz }\OperatorTok{\textless{}} \DecValTok{100}\NormalTok{:}
\NormalTok{        attenuation\_db\_km }\OperatorTok{=} \FloatTok{0.5} \OperatorTok{+} \FloatTok{0.05} \OperatorTok{*}\NormalTok{ (freq\_ghz }\OperatorTok{{-}} \DecValTok{30}\NormalTok{)}
    \ControlFlowTok{else}\NormalTok{:}
\NormalTok{        attenuation\_db\_km }\OperatorTok{=} \DecValTok{4} \OperatorTok{+} \FloatTok{0.02} \OperatorTok{*}\NormalTok{ (freq\_ghz }\OperatorTok{{-}} \DecValTok{100}\NormalTok{)}
    
\NormalTok{    atmospheric\_loss }\OperatorTok{=}\NormalTok{ attenuation\_db\_km }\OperatorTok{*}\NormalTok{ (distance\_m }\OperatorTok{/} \DecValTok{1000}\NormalTok{)}
    
    \CommentTok{\# Rain attenuation (ITU{-}R model)}
    \ControlFlowTok{if}\NormalTok{ rain\_rate\_mm\_h }\OperatorTok{\textgreater{}} \DecValTok{0}\NormalTok{:}
\NormalTok{        k }\OperatorTok{=} \FloatTok{0.0001} \OperatorTok{*}\NormalTok{ freq\_ghz}\OperatorTok{**}\DecValTok{2}
\NormalTok{        alpha }\OperatorTok{=} \FloatTok{1.0}
\NormalTok{        rain\_loss }\OperatorTok{=}\NormalTok{ k }\OperatorTok{*}\NormalTok{ rain\_rate\_mm\_h}\OperatorTok{**}\NormalTok{alpha }\OperatorTok{*}\NormalTok{ (distance\_m }\OperatorTok{/} \DecValTok{1000}\NormalTok{)}
    \ControlFlowTok{else}\NormalTok{:}
\NormalTok{        rain\_loss }\OperatorTok{=} \DecValTok{0}
    
\NormalTok{    total\_loss }\OperatorTok{=}\NormalTok{ fspl }\OperatorTok{+}\NormalTok{ atmospheric\_loss }\OperatorTok{+}\NormalTok{ rain\_loss}
    
    \BuiltInTok{print}\NormalTok{(}\SpecialStringTok{f"Frequency: }\SpecialCharTok{\{}\NormalTok{freq\_ghz}\SpecialCharTok{\}}\SpecialStringTok{ GHz, Distance: }\SpecialCharTok{\{}\NormalTok{distance\_m}\SpecialCharTok{\}}\SpecialStringTok{ m"}\NormalTok{)}
    \BuiltInTok{print}\NormalTok{(}\SpecialStringTok{f"  FSPL: }\SpecialCharTok{\{}\NormalTok{fspl}\SpecialCharTok{:.1f\}}\SpecialStringTok{ dB"}\NormalTok{)}
    \BuiltInTok{print}\NormalTok{(}\SpecialStringTok{f"  Atmospheric: }\SpecialCharTok{\{}\NormalTok{atmospheric\_loss}\SpecialCharTok{:.2f\}}\SpecialStringTok{ dB"}\NormalTok{)}
    \BuiltInTok{print}\NormalTok{(}\SpecialStringTok{f"  Rain: }\SpecialCharTok{\{}\NormalTok{rain\_loss}\SpecialCharTok{:.2f\}}\SpecialStringTok{ dB"}\NormalTok{)}
    \BuiltInTok{print}\NormalTok{(}\SpecialStringTok{f"  Total: }\SpecialCharTok{\{}\NormalTok{total\_loss}\SpecialCharTok{:.1f\}}\SpecialStringTok{ dB"}\NormalTok{)}
    
    \ControlFlowTok{return}\NormalTok{ total\_loss}

\KeywordTok{def}\NormalTok{ beamforming\_gain(n\_elements, beamwidth\_deg}\OperatorTok{=}\VariableTok{None}\NormalTok{):}
    \CommentTok{"""}
\CommentTok{    Calculate antenna array gain.}
\CommentTok{    }
\CommentTok{    Args:}
\CommentTok{        n\_elements: Number of antenna elements}
\CommentTok{        beamwidth\_deg: Optional 3dB beamwidth (degrees)}
\CommentTok{    }
\CommentTok{    Returns:}
\CommentTok{        Gain (dB)}
\CommentTok{    """}
\NormalTok{    gain\_db }\OperatorTok{=} \DecValTok{10} \OperatorTok{*}\NormalTok{ np.log10(n\_elements)}
    
    \ControlFlowTok{if}\NormalTok{ beamwidth\_deg:}
        \CommentTok{\# Approximate directivity from beamwidth}
\NormalTok{        directivity }\OperatorTok{=} \DecValTok{41253} \OperatorTok{/}\NormalTok{ (beamwidth\_deg}\OperatorTok{**}\DecValTok{2}\NormalTok{)}
\NormalTok{        gain\_from\_bw }\OperatorTok{=} \DecValTok{10} \OperatorTok{*}\NormalTok{ np.log10(directivity)}
        \BuiltInTok{print}\NormalTok{(}\SpecialStringTok{f"Array gain (element count): }\SpecialCharTok{\{}\NormalTok{gain\_db}\SpecialCharTok{:.1f\}}\SpecialStringTok{ dB"}\NormalTok{)}
        \BuiltInTok{print}\NormalTok{(}\SpecialStringTok{f"Gain from beamwidth: }\SpecialCharTok{\{}\NormalTok{gain\_from\_bw}\SpecialCharTok{:.1f\}}\SpecialStringTok{ dB"}\NormalTok{)}
\NormalTok{        gain\_db }\OperatorTok{=} \BuiltInTok{max}\NormalTok{(gain\_db, gain\_from\_bw)}
    
    \ControlFlowTok{return}\NormalTok{ gain\_db}

\CommentTok{\# Example: 5G mmWave link budget}
\BuiltInTok{print}\NormalTok{(}\StringTok{"=== 5G mmWave Link Budget ===}\CharTok{\textbackslash{}n}\StringTok{"}\NormalTok{)}

\NormalTok{freq }\OperatorTok{=} \DecValTok{28}  \CommentTok{\# GHz}
\NormalTok{distance }\OperatorTok{=} \DecValTok{100}  \CommentTok{\# meters}
\NormalTok{tx\_power\_dbm }\OperatorTok{=} \DecValTok{23}  \CommentTok{\# dBm per element}
\NormalTok{n\_tx\_elements }\OperatorTok{=} \DecValTok{64}
\NormalTok{n\_rx\_elements }\OperatorTok{=} \DecValTok{16}

\NormalTok{path\_loss }\OperatorTok{=}\NormalTok{ mmwave\_path\_loss(freq, distance, rain\_rate\_mm\_h}\OperatorTok{=}\DecValTok{0}\NormalTok{)}
\NormalTok{tx\_gain }\OperatorTok{=}\NormalTok{ beamforming\_gain(n\_tx\_elements)}
\NormalTok{rx\_gain }\OperatorTok{=}\NormalTok{ beamforming\_gain(n\_rx\_elements)}

\NormalTok{eirp }\OperatorTok{=}\NormalTok{ tx\_power\_dbm }\OperatorTok{+}\NormalTok{ tx\_gain}
\NormalTok{rx\_power }\OperatorTok{=}\NormalTok{ eirp }\OperatorTok{{-}}\NormalTok{ path\_loss }\OperatorTok{+}\NormalTok{ rx\_gain}

\NormalTok{noise\_figure }\OperatorTok{=} \DecValTok{7}  \CommentTok{\# dB}
\NormalTok{bandwidth\_mhz }\OperatorTok{=} \DecValTok{100}
\NormalTok{thermal\_noise }\OperatorTok{=} \OperatorTok{{-}}\DecValTok{174} \OperatorTok{+} \DecValTok{10}\OperatorTok{*}\NormalTok{np.log10(bandwidth\_mhz }\OperatorTok{*} \FloatTok{1e6}\NormalTok{) }\OperatorTok{+}\NormalTok{ noise\_figure}

\NormalTok{snr }\OperatorTok{=}\NormalTok{ rx\_power }\OperatorTok{{-}}\NormalTok{ thermal\_noise}

\BuiltInTok{print}\NormalTok{(}\SpecialStringTok{f"}\CharTok{\textbackslash{}n}\SpecialStringTok{Link Budget:"}\NormalTok{)}
\BuiltInTok{print}\NormalTok{(}\SpecialStringTok{f"  EIRP: }\SpecialCharTok{\{}\NormalTok{eirp}\SpecialCharTok{:.1f\}}\SpecialStringTok{ dBm"}\NormalTok{)}
\BuiltInTok{print}\NormalTok{(}\SpecialStringTok{f"  RX power: }\SpecialCharTok{\{}\NormalTok{rx\_power}\SpecialCharTok{:.1f\}}\SpecialStringTok{ dBm"}\NormalTok{)}
\BuiltInTok{print}\NormalTok{(}\SpecialStringTok{f"  Noise: }\SpecialCharTok{\{}\NormalTok{thermal\_noise}\SpecialCharTok{:.1f\}}\SpecialStringTok{ dBm"}\NormalTok{)}
\BuiltInTok{print}\NormalTok{(}\SpecialStringTok{f"  SNR: }\SpecialCharTok{\{}\NormalTok{snr}\SpecialCharTok{:.1f\}}\SpecialStringTok{ dB"}\NormalTok{)}

\NormalTok{capacity\_gbps }\OperatorTok{=}\NormalTok{ (bandwidth\_mhz }\OperatorTok{*}\NormalTok{ np.log2(}\DecValTok{1} \OperatorTok{+} \DecValTok{10}\OperatorTok{**}\NormalTok{(snr}\OperatorTok{/}\DecValTok{10}\NormalTok{))) }\OperatorTok{/} \DecValTok{1000}
\BuiltInTok{print}\NormalTok{(}\SpecialStringTok{f"  Shannon capacity: }\SpecialCharTok{\{}\NormalTok{capacity\_gbps}\SpecialCharTok{:.2f\}}\SpecialStringTok{ Gbps"}\NormalTok{)}
\end{Highlighting}
\end{Shaded}

\begin{center}\rule{0.5\linewidth}{0.5pt}\end{center}

\subsection{\texorpdfstring{ Summary
Comparison}{ Summary Comparison}}\label{summary-comparison}

{\def\LTcaptype{} % do not increment counter
\begin{longtable}[]{@{}llll@{}}
\toprule\noalign{}
Aspect & Sub-6 GHz & mmWave (24-52 GHz) & Sub-THz (100-300 GHz) \\
\midrule\noalign{}
\endhead
\bottomrule\noalign{}
\endlastfoot
\textbf{Bandwidth} & 100 MHz & 400 MHz-2 GHz & 10-50 GHz \\
\textbf{Peak Rate} & 1 Gbps & 10 Gbps & 100+ Gbps \\
\textbf{Range} & 1-5 km & 100-500 m & 10-100 m \\
\textbf{Propagation} & NLOS-friendly & LOS-preferred & LOS-only \\
\textbf{Mobility} & Excellent & Good & Limited \\
\textbf{Beamforming} & Optional & Mandatory & Ultra-massive \\
\textbf{Applications} & Wide-area & Dense urban, FWA & Indoor, backhaul \\
\end{longtable}
}

\begin{center}\rule{0.5\linewidth}{0.5pt}\end{center}

\subsection{\texorpdfstring{ Further
Reading}{ Further Reading}}\label{further-reading}

\subsubsection{Textbooks}\label{textbooks}

\begin{itemize}
\tightlist
\item
  \textbf{Rappaport et al.}, \emph{Millimeter Wave Wireless
  Communications} - Comprehensive mmWave treatment
\item
  \textbf{Akyildiz et al.}, \emph{Terahertz Band Communication} - THz
  fundamentals
\item
  \textbf{Rangan et al.}, \emph{Millimeter-Wave Cellular Wireless
  Networks} - 5G mmWave
\end{itemize}

\subsubsection{Key Papers}\label{key-papers}

\begin{itemize}
\tightlist
\item
  \textbf{Rappaport et al.} (2013): ``Millimeter Wave Mobile
  Communications for 5G'' - Seminal 5G mmWave paper
\item
  \textbf{Alsharif et al.} (2020): ``Sixth Generation (6G) Wireless
  Networks'' - 6G vision including THz
\item
  \textbf{ITU-R P.676}: Atmospheric attenuation models
  (O\textsubscript{2},
  H\textsubscript{2}O)
\end{itemize}

\subsubsection{Standards}\label{standards}

\begin{itemize}
\tightlist
\item
  \textbf{3GPP TS 38.104}: 5G NR Base Station radio
  transmission/reception (FR2 specs)
\item
  \textbf{IEEE 802.11ad/ay}: WiGig 60 GHz mmWave WiFi
\item
  \textbf{IEEE 802.15.3d}: 100 Gbps WPAN (THz band)
\end{itemize}

\subsubsection{Related Topics}\label{related-topics}

\begin{itemize}
\tightlist
\item
  {[}{[}MIMO-\&-Spatial-Multiplexing{]}{]} - Beamforming foundations
\item
  {[}{[}OFDM-\&-Multicarrier-Modulation{]}{]} - mmWave uses OFDM
\item
  {[}{[}Adaptive-Modulation-\&-Coding-(AMC){]}{]} - Critical for
  variable mmWave channels
\item
  {[}{[}Atmospheric-Effects-(Ionospheric,-Tropospheric){]}{]} -
  Propagation background
\item
  {[}{[}Terahertz-(THz)-Technology{]}{]} - THz-specific content (quantum
  cascade lasers, imaging)
\item
  {[}{[}Real-World-System-Examples{]}{]} - 5G NR deployments
\end{itemize}

\begin{center}\rule{0.5\linewidth}{0.5pt}\end{center}

\textbf{Summary}: mmWave (24-300 GHz) and THz (0.3-10 THz) offer massive
bandwidth ($100\times$ more than sub-6 GHz) enabling
multi-gigabit to terabit wireless. 5G NR FR2 (24-52 GHz) delivers 2-10
Gbps with 100-500 m range using massive MIMO beamforming (64-256
elements). Path loss increases 20-40 dB vs.~sub-6 GHz, requiring
directional antennas and dense small-cell deployment. Atmospheric
absorption (O\textsubscript{2} at 60 GHz,
H\textsubscript{2}O at 183 GHz) and rain attenuation
limit range. Blockage (human body 20-40 dB, walls 30-80 dB) makes LOS
critical. Beamforming is mandatory (analog or hybrid) for coverage.
Applications: urban hotspots, fixed wireless access, industrial IoT. 6G
targets sub-THz (100-300 GHz) for 100 Gbps-1 Tbps with ultra-massive
MIMO (1024+ elements), intelligent surfaces (RIS), and 10-50 m cell
radius. Automotive radar (77-81 GHz) uses FMCW for 3 cm range
resolution. mmWave/THz = ultra-high bandwidth, ultra-short range,
ultra-directional---requires paradigm shift in network architecture.
