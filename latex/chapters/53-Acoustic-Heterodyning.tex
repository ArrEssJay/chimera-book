% ==============================================================================
% CHAPTER 49: Acoustic Heterodyning
% ==============================================================================

\chapter{Acoustic Heterodyning}
\label{ch:acoustic-heterodyning}

\begin{nontechnical}
    \textbf{Acoustic heterodyning is like creating a "ghost" sound that only appears where two invisible ultrasound beams cross.}

    \parhead{The simple idea}
    \begin{itemize}
        \item Ultrasound is a very high-pitched sound that humans cannot hear (like a dog whistle).
        \item If you aim two ultrasound beams with slightly different frequencies (e.g., 40,000 Hz and 41,000 Hz) so that they intersect in mid-air, a fascinating non-linear effect occurs.
        \item At the point of intersection, a new, audible sound is generated at the \emph{difference} between the two original frequencies (in this case, 1,000 Hz).
    \end{itemize}

    \parhead{The "beam of sound"}
    Because ultrasound is high-frequency, it can be focused into a very narrow, laser-like beam. The audible sound that is generated inherits this directionality. The result is a highly focused "beam of sound" that can be aimed at a single person in a room, without disturbing others nearby.

    \parhead{Real-world applications}
    \begin{itemize}
        \item \textbf{Museums and Galleries:} To provide audio commentary for a specific exhibit that is only audible to the person standing directly in front of it.
        \item \textbf{Directional Advertising:} In retail stores to deliver a marketing message to a single shopper in an aisle.
        \item \textbf{Medical Imaging:} A related principle, harmonic imaging, is used in modern ultrasound scanners to get clearer pictures with fewer artefacts.
    \end{itemize}
\end{nontechnical}


\section{Overview and Properties}

\subsection{Overview}

\keyterm{Acoustic heterodyning}, also known as a \keyterm{parametric acoustic array}, is a phenomenon in non-linear acoustics where two high-frequency, collimated sound waves (ultrasound) interact within a medium (like air or water) to generate new frequencies. The most important of these are the sum and difference frequencies. By choosing two ultrasonic primary frequencies, $f_1$ and $f_2$, that are close together, an audible difference frequency, $f_{\Delta} = |f_1 - f_2|$, can be generated.

\begin{keyconcept}
    The key advantage of acoustic heterodyning is that the generated low-frequency, audible sound inherits the high directionality of the primary high-frequency ultrasound beams. This allows for the creation of a highly focused "beam of sound" from a physically small transducer, an effect that would require a massive, impractical antenna array using conventional linear acoustics.
\end{keyconcept}


\subsection{Physical Principle}

The phenomenon is governed by the non-linearity of the propagation medium, described by the \keyterm{Westervelt equation}. The non-linear term in the wave equation acts as a source term that is proportional to the square of the acoustic pressure. When two primary waves, $p_1$ and $p_2$, are present, this squared term contains a cross-product, $2p_1p_2\cos(\omega_1 t)\cos(\omega_2 t)$, which through trigonometric identities, produces sum ($\omega_1+\omega_2$) and difference ($\omega_1-\omega_2$) frequency components.

The conversion efficiency is very low, typically on the order of 1\%, meaning that powerful ultrasonic transducers are required to generate an audible sound of a reasonable volume.


\subsection{Key Applications}

\begin{description}
    \item[Parametric Loudspeakers] This is the most common commercial application. By modulating the primary ultrasonic beams with an audio signal, a highly directional and private audio field can be created. This is used in museums, trade shows, and public spaces where targeted audio is desired without creating ambient noise.
    
    \item[Underwater Sonar] In sonar, a parametric array can generate a highly directional, low-frequency beam that can penetrate deep into sediment for sub-bottom profiling, something that would require an enormous transducer using linear acoustics.
    
    \item[Medical Harmonic Imaging] A related principle is used in diagnostic ultrasound. The body's tissue is a non-linear medium. When a transducer sends in a fundamental frequency ($f_0$), the tissue itself generates harmonics ($2f_0, 3f_0, \dots$). By filtering for the second harmonic ($2f_0$), imagers can create images with significantly reduced clutter and improved contrast.
    
    \item[Neuromodulation (Experimental)] Research is ongoing into the use of two intersecting, focused ultrasound beams to generate a localised, low-frequency acoustic stimulation deep within the brain. The goal is to non-invasively modulate neural activity at a specific target location.
\end{description}


\begin{workedexample}{Parametric Speaker Design}
    \parhead{Problem} Design a simple parametric speaker to create a directional 2 kHz audio tone.
    \parhead{Design Choices}
    \begin{itemize}
        \item To generate a \qty{2}{kHz} difference frequency, we can choose two ultrasonic primary frequencies, for example $f_1 = \qty{40}{kHz}$ and $f_2 = \qty{42}{kHz}$.
        \item We use a 10 cm diameter transducer array, which is physically small and portable.
    \end{itemize}
    \parhead{Analysis}
    \begin{derivationsteps}
        \step \textbf{Calculate the Beamwidth.} The beamwidth is determined by the high-frequency primary beams. At 40 kHz in air, the wavelength is $\lambda = c/f \approx \qty{8.6}{mm}$. The approximate half-power beamwidth is:
        \[ \theta_{\text{HPBW}} \approx \frac{70 \lambda}{D} = \frac{70 \times 8.6 \text{ mm}}{100 \text{ mm}} \approx \textbf{6.0 degrees} \]
        This is an extremely narrow, "spotlight"-like beam. At a distance of 5 meters, the audio spot would only be about 52 cm wide.
        
        \step \textbf{Assess the Power Requirement.} The amplitude of the generated 2 kHz tone is proportional to the product of the primary wave amplitudes and the square of their frequency. However, the conversion efficiency is very low. To generate a modest sound pressure level of 74 dB (equivalent to conversational speech) at 5 meters, the ultrasonic transducers would need to generate an extremely intense, inaudible field of around 120-130 dB. This requires a significant amount of power and carefully designed transducers.
    \end{derivationsteps}
    \parhead{Interpretation} The analysis confirms the core trade-off of acoustic heterodyning: it allows for the creation of a highly directional, low-frequency sound beam from a small aperture, but at the cost of very low efficiency and high power consumption.
\end{workedexample}


\begin{importantbox}[title={Further Reading}]
    Acoustic heterodyning is a fascinating application of non-linear wave physics with parallels in the RF domain.
    \begin{description}
        \item[Mixers and Intermodulation] (\Cref{ch:mixers}) describes the electronic equivalent of acoustic heterodyning, where non-linear diodes are used to generate sum and difference frequencies in RF circuits.
        \item[Antenna Theory] (\Cref{ch:antenna}) provides the context for understanding directivity and beamwidth, highlighting why generating a 6$^\circ$ beam at 2 kHz with linear acoustics would require an impractically large antenna (over 1.5 meters in diameter).
    \end{description}
\end{importantbox}
