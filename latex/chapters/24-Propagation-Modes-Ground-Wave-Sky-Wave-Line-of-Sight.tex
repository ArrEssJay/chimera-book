% ==============================================================================
% CHAPTER 24: Propagation Modes
% ==============================================================================

\chapter{Propagation Modes}
\label{ch:propagation-modes}

\begin{nontechnical}
    \textbf{Radio waves travel in three main ways}, depending on their frequency. Think of it like trying to get a message across a valley: you can shout and have it follow the ground, bounce it off the clouds, or use a laser pointer.

    \parhead{1. Ground Wave (Following the Ground)}
    \begin{itemize}
        \item \textbf{How it works:} Low-frequency waves hug the Earth's surface and can even bend around the curve of the horizon.
        \item \textbf{Who uses it:} AM radio stations, which is why you can hear them from hundreds of kilometres away, far beyond the horizon.
    \end{itemize}

    \parhead{2. Sky Wave (Bouncing off the Sky)}
    \begin{itemize}
        \item \textbf{How it works:} Medium-frequency waves shoot up into the sky, reflect off the ionosphere (a charged layer in the upper atmosphere), and bounce back down to Earth, thousands of kilometres away.
        \item \textbf{Who uses it:} Shortwave radio broadcasters and amateur radio operators, who use this "sky-bounce" to talk to people on other continents.
    \end{itemize}

    \parhead{3. Line-of-Sight (Like a Laser Pointer)}
    \begin{itemize}
        \item \textbf{How it works:} High-frequency waves travel in a straight line. If there is an obstacle like a building or a mountain in the way, the signal is blocked.
        \item \textbf{Who uses it:} Almost everything else: FM radio, TV, mobile phones, WiFi, and satellite communications. This is why cell towers and TV antennas are placed up high, and why your satellite dish needs a clear view of the sky.
    \end{itemize}
\end{nontechnical}


\section{Overview and Properties}

\subsection{Overview}

The mechanism by which an electromagnetic wave travels from a transmitter to a receiver is known as its \keyterm{propagation mode}. The dominant mode is determined almost entirely by the signal's frequency and its interaction with the Earth's surface and atmosphere. Understanding these modes is essential for predicting radio coverage, designing antenna systems, and selecting appropriate frequencies for a given application.

\begin{keyconcept}
    The three primary propagation modes are \textbf{Ground Wave}, \textbf{Sky Wave}, and \textbf{Line-of-Sight (LOS)}. Each mode operates in a distinct frequency range and has unique characteristics that define its range, reliability, and application.
\end{keyconcept}


\subsection{Ground Wave Propagation}

\parhead{Mechanism}
At low and medium frequencies (LF/MF, roughly 30 kHz to 3 MHz), vertically polarized radio waves can travel along the surface of the Earth, following its curvature. This \keyterm{ground wave} propagation is possible because the wave induces currents in the conductive surface of the Earth, which slows the wavefront near the ground and causes it to tilt forward and "hug" the terrain.

\parhead{Characteristics}
\begin{itemize}
    \item \textbf{Frequency Dependent:} Attenuation increases dramatically with frequency. It is only effective below about 3 MHz.
    \item \textbf{Terrain Dependent:} Performance is excellent over conductive surfaces like seawater but poor over dry, sandy, or frozen ground.
    \item \textbf{Polarisation:} Only vertically polarized waves can propagate effectively.
    \item \textbf{Range:} Can achieve stable, predictable coverage over hundreds of kilometres, well beyond the horizon.
\end{itemize}

\parhead{Applications}
This mode is ideal for long-range, over-the-horizon broadcasting and navigation.
\begin{itemize}
    \item \textbf{AM Radio Broadcasting (540-1700 kHz):} Ground wave provides the primary daytime coverage area for AM stations.
    \item \textbf{Maritime Communication:} Its excellent performance over seawater makes it ideal for ship-to-shore communication.
    \item \textbf{Aviation Beacons (NDBs):} Non-Directional Beacons in the LF band provide reliable homing signals for aircraft.
\end{itemize}


\subsection{Sky Wave Propagation}

\parhead{Mechanism}
In the High Frequency (HF) band (3-30 MHz), radio waves are not absorbed by the atmosphere but are instead refracted (bent) by the \keyterm{ionosphere}, a layer of ionised plasma in the upper atmosphere (60-400 km altitude). If the angle of incidence and frequency are correct, the wave will be bent back down towards the Earth, landing thousands of kilometres away. This process can repeat in multiple "hops," enabling global communication.

\parhead{Characteristics}
\begin{itemize}
    \item \textbf{Dynamic:} The ionosphere's density and height change dramatically with the time of day, season, and the 11-year solar cycle.
    \item \textbf{Frequency Dependent:} There is a \keyterm{Maximum Usable Frequency (MUF)} above which waves will penetrate the ionosphere and escape into space, and a \keyterm{Lowest Usable Frequency (LUF)} below which waves are absorbed.
    \item \textbf{Night vs. Day:} At night, the lower absorbing layers of the ionosphere disappear, allowing lower frequencies (including the AM broadcast band) to propagate via sky wave.
\end{itemize}

\parhead{Applications}
\begin{itemize}
    \item \textbf{Shortwave Broadcasting:} International broadcasters like the BBC World Service use sky wave to reach global audiences.
    \item \textbf{Amateur (Ham) Radio:} Enables hobbyists to communicate with people on other continents using relatively low power.
    \item \textbf{Over-the-Horizon (OTH) Radar:} Military radar systems bounce HF signals off the ionosphere to detect aircraft and ships thousands of kilometres away.
\end{itemize}


\subsection{Line-of-Sight (LOS) Propagation}

\parhead{Mechanism}
At frequencies above about 30 MHz (VHF and higher), radio waves are no longer significantly reflected by the ionosphere or guided by the ground. They travel in a straight line, just like visible light. For a link to be established, there must be a direct, unobstructed path between the transmitter and receiver.

\parhead{Characteristics}
\begin{itemize}
    \item \textbf{Range Limitation:} The maximum range is limited by the curvature of the Earth, known as the \keyterm{radio horizon}. The formula for this range (in km) is approximately $d \approx 4.12(\sqrt{h_t} + \sqrt{h_r})$, where $h_t$ and $h_r$ are the antenna heights in meters.
    \item \textbf{Obstruction:} Any significant obstacle—such as a building, mountain, or even dense foliage—can block or severely attenuate the signal.
    \item \textbf{Fresnel Zone:} For a high-quality link, a clear ellipsoidal volume around the direct path, known as the first \keyterm{Fresnel Zone}, must be kept free of obstructions.
\end{itemize}

\parhead{Applications}
LOS is the propagation mode for virtually all modern high-data-rate communication systems.
\begin{itemize}
    \item \textbf{FM Radio and Television:} Use very tall broadcast towers to maximize their line-of-sight coverage area.
    \item \textbf{Cellular Communications:} While often used in NLOS urban environments (relying on reflections and diffractions), the underlying system is designed around LOS principles.
    \item \textbf{WiFi, Bluetooth, 5G mmWave:} Short-range LOS systems where walls and even people can be significant obstacles.
    \item \textbf{Satellite Communications:} The ultimate LOS link, requiring a perfectly clear path from the ground dish to the satellite in the sky.
\end{itemize}

\begin{table}[H]
    \centering
    \caption{Summary of Primary Propagation Modes}
    \label{tab:propagation-summary}
    \begin{tabular}{@{}llll@{}}
        \toprule
        \tableheaderfont Mode & \tableheaderfont Frequency Band & \tableheaderfont Mechanism & \tableheaderfont Key Application \\
        \midrule
        Ground Wave & LF, MF (30 kHz -- 3 MHz) & Follows Earth's curvature & AM Radio \\
        Sky Wave & HF (3 -- 30 MHz) & Refraction off ionosphere & Shortwave Radio \\
        Line-of-Sight & VHF, UHF, SHF+ (>30 MHz) & Direct, straight-line path & Cellular, WiFi, Satellite \\
        \bottomrule
    \end{tabular}
\end{table}

\begin{workedexample}{VHF Marine Communication Range}
    \parhead{Problem} Calculate the maximum communication range between two ships at sea using VHF marine radio.
    \parhead{System Parameters}
    \begin{itemize}
        \item Frequency: \qty{156}{MHz} (VHF, so it is Line-of-Sight).
        \item Ship 1 Antenna Height ($h_t$): \qty{15}{m} above sea level.
        \item Ship 2 Antenna Height ($h_r$): \qty{10}{m} above sea level.
    \end{itemize}
    \parhead{Solution}
    \begin{derivationsteps}
        \step Identify the limiting factor. For a VHF link over the ocean, the maximum range is determined by the radio horizon.
        \step Apply the radio horizon formula.
        \[ d_{\text{horizon}} \approx 4.12(\sqrt{h_t} + \sqrt{h_r}) \]
        \[ d_{\text{horizon}} \approx 4.12(\sqrt{15} + \sqrt{10}) = 4.12(3.87 + 3.16) \approx \textbf{\qty{29.0}{km}} \]
    \end{derivationsteps}
    \parhead{Interpretation} The maximum reliable communication distance between the two ships is approximately 29 km (about 15.6 nautical miles). Beyond this distance, the curvature of the Earth will block the direct line-of-sight path. To communicate over longer distances, the ships would need to switch to an HF radio (using sky wave) or a satellite communication system.
\end{workedexample}

\begin{importantbox}[title={Further Reading}]
    The propagation mode is the first and most fundamental aspect of the channel that dictates system design.
    \begin{description}
        \item[Free-Space Path Loss (FSPL)] (\Cref{ch:fspl}) is the baseline attenuation model for Line-of-Sight links.
        \item[Multipath Propagation \& Fading] (\Cref{ch:multipath}) describes the complex effects of reflections and scattering that occur within a Line-of-Sight environment.
        \item[Atmospheric Effects] (\Cref{ch:atmospheric}) provides a deep dive into the physics of the ionosphere that enables sky wave propagation, as well as atmospheric absorption that affects high-frequency LOS links.
    \end{description}
\end{importantbox}
