% ==============================================================================
% CHAPTER 44: Real-World System Examples
% ==============================================================================

\chapter{Real-World System Examples}
\label{ch:real-world-systems}

\begin{nontechnical}
    \textbf{Real-world communication systems are like different types of vehicles}, each one expertly designed for a specific purpose. There is no single "best" vehicle; a Formula 1 car is perfect for a racetrack but useless for a mountain expedition. Similarly, there is no single "best" wireless standard.

    \parhead{The vehicles of wireless communication}
    \begin{itemize}
        \item \textbf{WiFi} is the \textbf{sports car}: incredibly fast, but with a limited range and a thirst for power. Perfect for high-speed data in your home or office.
        \item \textbf{4G/5G Cellular} is the \textbf{family saloon}: a versatile all-rounder, balancing speed, range, and efficiency to provide reliable connectivity almost everywhere.
        \item \textbf{GPS} is the \textbf{specialised survey aircraft}: it doesn't carry much data (just a tiny navigation message), but it performs its one task—pinpoint positioning—with incredible precision across the entire globe.
        \item \textbf{Bluetooth} is the \textbf{electric scooter}: extremely low-power and designed for short-range personal connections, like linking your phone to your headphones.
        \item \textbf{LoRaWAN} is the \textbf{ultra-endurance tractor}: very slow, but incredibly robust, with a massive range and the ability to run for years on a single battery. Perfect for agricultural sensors in remote fields.
    \end{itemize}

    \parhead{The engineering trade-offs}
    Each of these systems makes a different set of engineering compromises, trading off data rate vs. range, power consumption vs. complexity, and robustness vs. efficiency. This chapter explores how the theoretical concepts we've discussed are combined in the real world to build these specialised and highly optimised systems.
\end{nontechnical}


\subsection{Overview}

This chapter provides a comparative analysis of several key modern communication systems. By examining their physical layer parameters, we can see how the fundamental concepts of modulation, coding, multiple access, and synchronisation are integrated to meet a specific set of design goals.

\begin{keyconcept}
    Every real-world system represents a set of engineering trade-offs. There is no single standard that simultaneously maximises data rate, range, power efficiency, and spectral efficiency. The design of each system is a masterclass in optimisation, tailored for its intended application and operating environment.
\end{keyconcept}


\subsection{System Comparison}

The following table provides a high-level summary of the systems we will analyse, highlighting their fundamentally different design objectives.

\begin{table}[H]
    \centering
    \caption{High-Level Comparison of Modern Wireless Systems}
    \label{tab:system-comparison}
    \begin{tabularx}{\textwidth}{@{}lXXX@{}}
        \toprule
        \tableheaderfont System & \tableheaderfont Primary Design Goal & \tableheaderfont Key Technology & \tableheaderfont Defining Limitation \\
        \midrule
        WiFi 6 (802.11ax) & Max Throughput & MIMO-OFDM, 1024-QAM & Short Range \\
        5G NR & Coverage \& Capacity & Massive MIMO, OFDMA & Complexity \\
        DVB-S2X & Bandwidth Efficiency & ACM, LDPC/BCH & High Latency \\
        GPS & Global Positioning & DSSS (High Proc. Gain) & Very Low Data Rate \\
        Bluetooth LE & Low Power & GFSK, Frequency Hopping & Short Range \\
        LoRaWAN & Long Range \& Battery Life & Chirp Spread Spectrum & Very Low Data Rate \\
        \bottomrule
    \end{tabularx}
\end{table}


\subsection{WiFi (802.11ax / WiFi 6)}

\parhead{Objective} To provide extremely high data rates (multi-gigabit) over a short range (a single home or office) in a dense, high-interference environment.
\parhead{Key Technologies}
\begin{itemize}
    \item \textbf{OFDMA (Orthogonal Frequency-Division Multiple Access):} A multi-user version of OFDM that allows a single channel to be subdivided and allocated to multiple users simultaneously, dramatically improving efficiency in environments with many devices.
    \item \textbf{High-Order Modulation (1024-QAM):} Enables the transmission of 10 bits per symbol, achieving very high spectral efficiency in high-SNR conditions.
    \item \textbf{MU-MIMO (Multi-User MIMO):} Allows an access point with multiple antennas to talk to multiple devices at the same time, increasing overall network capacity.
    \item \textbf{LDPC Coding:} Provides powerful, efficient error correction.
\end{itemize}
\parhead{Trade-off} The pursuit of maximum throughput means WiFi uses very wide channels (80/160 MHz) and complex modulation schemes that require a high SNR, inherently limiting its effective range.


\subsection{5G New Radio (NR)}

\parhead{Objective} To provide a flexible and scalable framework for three distinct use cases: enhanced Mobile Broadband (eMBB), Ultra-Reliable Low-Latency Communication (URLLC), and massive Machine-Type Communication (mMTC).
\parhead{Key Technologies}
\begin{itemize}
    \item \textbf{Massive MIMO:} Base stations with a very large number of antennas (64-256) use beamforming to focus energy towards individual users, dramatically improving spectral efficiency and coverage.
    \item \textbf{Flexible Numerology:} The OFDM parameters (like subcarrier spacing) can be dynamically changed to optimise for different scenarios—narrow spacing for coverage, wide spacing for low latency.
    \item \textbf{Advanced Channel Codes:} Employs high-performance \keyterm{LDPC codes} for data channels and \keyterm{Polar codes} for control channels.
\end{itemize}
\parhead{Trade-off} 5G's incredible flexibility and performance come at the cost of immense complexity in both the hardware (massive antenna arrays) and the software (resource scheduling, beam management).


\subsection{GPS (Global Positioning System)}

\parhead{Objective} To provide a globally available, highly reliable positioning and timing signal that can be received by low-cost devices with small antennas, even when the signal is far below the thermal noise floor.
\parhead{Key Technologies}
\begin{itemize}
    \item \textbf{Direct-Sequence Spread Spectrum (DSSS):} The 50 bps navigation message is spread over a 2 MHz bandwidth using a 1023-chip Gold code. This provides a massive \textbf{43 dB of processing gain}.
    \item \textbf{Code Division Multiple Access (CDMA):} Each satellite transmits on the same frequency but uses a unique spreading code, allowing a receiver to distinguish them.
    \item \textbf{BPSK Modulation:} The most power-efficient and robust modulation scheme, ideal for an extremely power-limited link.
\end{itemize}
\parhead{Trade-off} The entire system is optimised for signal detection and ranging at the expense of data rate. The information throughput of the civilian GPS signal is negligible (50 bps), but its timing precision is on the order of nanoseconds.


\subsection{LoRaWAN}

\parhead{Objective} To provide ultra-long-range (multi-kilometre), low-power connectivity for IoT devices that only need to send small amounts of data infrequently (e.g., a few bytes per hour). Battery life is the primary design driver.
\parhead{Key Technologies}
\begin{itemize}
    \item \textbf{Chirp Spread Spectrum (CSS):} A proprietary modulation technique where the signal is a "chirp"—a sinusoid that sweeps across a wide frequency band. This provides a large processing gain, similar to DSSS, allowing the receiver to pull signals out from far below the noise floor.
    \item \textbf{Adaptive Data Rate (ADR):} The network server remotely controls the "spreading factor" (the duration of the chirp) for each device. Devices close to a gateway use a fast chirp (low spreading factor), saving airtime and power. Devices far away use a slow chirp (high spreading factor), trading data rate for the extra processing gain needed to close the link.
\end{itemize}
\parhead{Trade-off} LoRa achieves its incredible range and battery life (often 5-10 years) by sacrificing data rate. At its longest-range settings, the data rate can be as low as a few hundred bits per second, and network regulations (duty cycles) limit each device to a very small number of transmissions per day.


\begin{workedexample}{The Data Rate vs. Range Trade-off}
    \parhead{Problem} Compare the design choices of WiFi and LoRa to illustrate the fundamental trade-off between throughput and range.
    \parhead{Analysis}
    \begin{itemize}
        \item \textbf{WiFi (High Throughput):} To achieve 600 Mbps in a 20 MHz channel, WiFi needs a spectral efficiency of 30 bps/Hz. It achieves this with 4x4 MIMO and 64-QAM. According to the Shannon limit, this requires a very high SNR of around 25-30 dB. The system is optimised for speed in a high-SNR, short-range environment.
        \item \textbf{LoRa (Long Range):} To achieve a 10 km range with very low transmit power, LoRa must operate with a very low SNR, often as low as -20 dB (the signal is 100 times weaker than the noise). To close the link, it uses a massive processing gain from CSS and an extremely low data rate (e.g., 250 bps). The spectral efficiency is incredibly low (<0.01 bps/Hz), but the link is exceptionally robust.
    \end{itemize}
    \parhead{Interpretation} WiFi and LoRa represent opposite ends of the design spectrum. WiFi sacrifices robustness for extremely high spectral efficiency. LoRa sacrifices spectral efficiency for extreme robustness and power efficiency. Neither is "better"; they are simply optimised for different, non-overlapping applications.
\end{workedexample}


\begin{importantbox}[title={Further Reading}]
    This chapter serves as a practical synthesis of the entire book. To understand the details behind these systems, the following chapters are essential.
    \begin{description}
        \item[OFDM and MIMO] (\Cref{ch:ofdm}, \Cref{ch:mimo}) are the two key technologies that enable the high throughput of modern WiFi and 5G.
        \item[Spread Spectrum] (\Cref{ch:spread-spectrum}) explains the principles of processing gain that are fundamental to the operation of GPS and LoRa.
        \item[Adaptive Modulation and Coding] (\Cref{ch:amc}) is the "brain" that allows systems like 5G and LoRa to dynamically trade speed for range.
        \item[Link Budget Analysis] (\Cref{ch:linkbudget}) is the tool used to perform the quantitative trade-offs discussed for each of these systems.
    \end{description}
\end{importantbox}