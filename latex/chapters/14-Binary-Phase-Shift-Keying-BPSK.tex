% ==============================================================================
% CHAPTER 14: Binary Phase-Shift Keying (BPSK)
% ==============================================================================

\chapter{Binary Phase-Shift Keying (BPSK)}
\label{ch:bpsk}

\begin{nontechnical}
    \textbf{BPSK is like Morse code, but instead of turning a signal on and off, it flips the wave upside-down.} This simple flip, a 180$^\circ$ phase shift, is used to represent a "1" versus a "0".

    \parhead{The simple idea}
    \begin{itemize}
        \item To send a \textbf{bit 0}, transmit a normal carrier wave.
        \item To send a \textbf{bit 1}, transmit the same wave, but inverted (phase-shifted by 180$^\circ$).
    \end{itemize}

    \parhead{Why it's so robust} A receiver just has to decide if the wave it sees is "normal" or "flipped." This is a much easier decision to make in a noisy environment than trying to detect if a signal is merely on or off. This makes BPSK extremely power-efficient.

    \parhead{Real-world use} Its robustness makes BPSK the workhorse for \keyterm{power-limited} channels where every decibel of signal power counts. It is used extensively in deep-space missions (like the Voyager probes), satellite telemetry, and was the foundational modulation for the original GPS signal. The trade-off is its low speed: it can only send one bit at a time.
\end{nontechnical}


\subsection{Overview}

\keyterm{Binary Phase-Shift Keying (BPSK)} is the simplest and most power-efficient form of digital phase modulation. It encodes one bit of information per symbol by shifting the phase of a carrier wave between two states, separated by 180$^\circ$.

\begin{keyconcept}
    BPSK offers the best performance (lowest Bit Error Rate for a given $E_b/N_0$) of any binary modulation scheme. Its constellation has the maximum possible Euclidean distance between its two points. This makes it the optimal choice for \textbf{power-limited} channels, where signal strength is the primary constraint.
\end{keyconcept}


\subsection{Mathematical Representation}

The BPSK signal can be represented as a carrier wave whose phase, $\phi_n$, is switched based on the input data bit:
\begin{equation}
    s(t) = A \cos(2\pi f_c t + \phi_n), \quad \text{where } \phi_n = \begin{cases} 0^\circ & \text{for bit `0'} \\ 180^\circ & \text{for bit `1'} \end{cases}
\end{equation}
Using the identity $\cos(\theta + 180^\circ) = -\cos(\theta)$, this is more commonly expressed as antipodal signaling:
\begin{equation}
    s(t) = d_n \cdot A \cos(2\pi f_c t)
\end{equation}
where $d_n$ is a bipolar symbol from the set $\{-1, +1\}$. This shows that BPSK is equivalent to amplitude modulation with a bipolar baseband signal.

\paragraph{IQ Representation}
In the complex baseband, BPSK is a purely real signal, occupying only the I-axis of the constellation diagram.
\begin{itemize}
    \item \textbf{I (In-phase):} $I_n = d_n \cdot A = \pm A$
    \item \textbf{Q (Quadrature):} $Q_n = 0$
\end{itemize}
The two constellation points are located at $(+A, 0)$ and $(-A, 0)$.


\subsection{Modulation and Demodulation}

\paragraph{Modulation}
A BPSK modulator is straightforward. The incoming binary data stream is first converted into a bipolar Non-Return-to-Zero (NRZ) signal ($d_n \in \{-1, +1\}$). This baseband signal is then simply multiplied by the carrier wave, $\cos(2\pi f_c t)$.

\paragraph{Coherent Demodulation}
Demodulation is performed by a \keyterm{coherent detector}. The incoming RF signal is multiplied by a local oscillator that is perfectly synchronized in frequency and phase with the original carrier.
\begin{equation}
    r(t) \cdot \cos(2\pi f_c t) = (d_n A \cos(2\pi f_c t)) \cdot \cos(2\pi f_c t) = \frac{A d_n}{2} (1 + \cos(4\pi f_c t))
\end{equation}
A low-pass filter removes the double-frequency term ($2f_c$), leaving a DC voltage proportional to $d_n$. A simple slicer then makes a decision: if the voltage is positive, the bit is a `0'; if it's negative, the bit is a `1'.

\begin{warningbox}
    \textbf{Carrier recovery is critical.} The performance of coherent BPSK relies on the receiver's ability to generate a local oscillator that is an exact replica of the transmitter's carrier. Any phase error, $\phi_e$, will reduce the detected signal voltage by a factor of $\cos(\phi_e)$, degrading the SNR. This is typically achieved using a Phase-Locked Loop (PLL) such as a Costas Loop or a squaring loop.
\end{warningbox}

\paragraph{Differential BPSK (DPSK)}
To simplify receiver design, \keyterm{Differential BPSK (DPSK)} encodes information in the phase \emph{transitions} between symbols, not their absolute phase. A `1' is represented by a 180$^\circ$ phase change, and a `0' by no change. This allows for a much simpler non-coherent receiver but comes at the cost of an approximately 3 dB performance penalty compared to coherent BPSK.


\subsection{Performance and Spectral Efficiency}

\paragraph{Bit Error Rate (BER)}
The theoretical BER for coherent BPSK in an AWGN channel is the best possible for any binary scheme:
\begin{equation}
    \text{BER} = Q\left(\sqrt{\frac{2E_b}{N_0}}\right)
\end{equation}
To achieve a BER of $10^{-5}$, BPSK requires an $E_b/N_0$ of approximately 9.6 dB.

\paragraph{Spectral Efficiency}
As a binary modulation, BPSK encodes 1 bit per symbol. Its spectral efficiency, $\eta$, after accounting for practical pulse shaping (with a roll-off factor $\alpha$), is:
\begin{equation}
    \eta = \frac{R_b}{B} = \frac{1}{1+\alpha} \quad \text{(bps/Hz)}
\end{equation}
For a typical $\alpha=0.35$, the spectral efficiency is approximately 0.74 bps/Hz. This is relatively low, making BPSK unsuitable for bandwidth-limited applications.


\begin{workedexample}{BPSK Link Budget for a Deep Space Probe}
    \parhead{Problem} A deep space probe transmits data back to Earth. Determine if the link will close for a target BER of $10^{-6}$.
    \parhead{System Parameters}
    \begin{itemize}
        \item Transmit Power ($P_t$): \qty{20}{W} (\qty{43}{dBm})
        \item Transmit Antenna Gain ($G_t$): \qty{48}{dBi} (high-gain dish)
        \item Distance ($r$): 1 Astronomical Unit (AU) $\approx 1.5 \times 10^8$ km
        \item Frequency ($f$): \qty{8.4}{GHz} (X-band)
        \item Receive Antenna: \qty{70}{m} DSN dish ($G_r \approx \qty{74}{dBi}$)
        \item System Noise Temperature ($T_s$): \qty{25}{K} (cryogenically cooled LNA)
        \item Data Rate ($R_b$): \qty{100}{kbps}
    \end{itemize}
    \parhead{Solution}
    \begin{derivationsteps}
        \step Calculate the Free-Space Path Loss (FSPL).
        \[ \text{FSPL}_{\text{dB}} \approx 20\log_{10}(d_{\text{km}}) + 20\log_{10}(f_{\text{MHz}}) + 32.45 = 20\log_{10}(1.5 \times 10^8) + 20\log_{10}(8400) + 32.45 \approx \qty{281.3}{dB} \]
        \step Calculate the received signal power ($P_r$).
        \[ P_r = P_t + G_t + G_r - \text{FSPL} = 43 + 48 + 74 - 281.3 = \qty{-116.3}{dBm} \]
        \step Calculate the noise power spectral density ($N_0$).
        \[ N_0 \text{ (dBm/Hz)} = 10\log_{10}(k T_s / 10^{-3}) = 10\log_{10}(1.38 \times 10^{-23} \times 25 / 10^{-3}) = \qty{-184.6}{dBm/Hz} \]
        \step Calculate the available $E_b/N_0$.
        \[ E_b/N_0 = P_r - 10\log_{10}(R_b) - N_0 = -116.3 - 10\log_{10}(100 \times 10^3) - (-184.6) = -116.3 - 50 + 184.6 = \qty{18.3}{dB} \]
        \step Determine the link margin. For a BER of $10^{-6}$, BPSK requires $E_b/N_0 \approx 10.5$ dB.
        \[ \text{Margin} = (\text{Available}) - (\text{Required}) = 18.3 - 10.5 = \qty{7.8}{dB} \]
    \end{derivationsteps}
    \parhead{Interpretation} The link closes with a healthy margin of 7.8 dB. This demonstrates how BPSK, combined with massive antennas and cryogenically cooled receivers, enables communication across interplanetary distances, despite the colossal path loss.
\end{workedexample}


\begin{importantbox}[title={Further Reading}]
    BPSK is the foundational building block for more complex phase modulation schemes.
    \begin{description}
        \item[Quadrature Phase-Shift Keying (QPSK)] (\Cref{ch:qpsk}) can be understood as two independent BPSK systems operating in parallel on the I and Q axes, doubling the spectral efficiency.
        \item[Carrier Recovery] (\Cref{ch:synchronisation}) provides a deep dive into the practical implementation of the Costas and squaring loops required for coherent BPSK demodulation.
        \item[Bit Error Rate] (\Cref{ch:ber}) explores the BPSK performance curve in detail and compares it to other modulation formats.
    \end{description}
\end{importantbox}