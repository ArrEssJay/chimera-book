% ==============================================================================
% CHAPTER 57: THz Bioeffects: Thermal and Non-Thermal
% ==============================================================================

\chapter{THz Bioeffects: Thermal and Non-Thermal}
\label{ch:thz-bioeffects}

\begin{nontechnical}
    \textbf{Terahertz (THz) radiation is a form of "invisible light" that is fundamentally non-ionising}, meaning it does not have enough energy per photon to break chemical bonds or damage DNA, unlike X-rays or gamma rays. Its interaction with biological tissue is therefore a subject of great interest for both safety and potential therapeutic applications.

    \parhead{The two types of effects}
    \begin{itemize}
        \item \textbf{Thermal Effects (Proven and Understood):} This is the main effect. THz energy is strongly absorbed by water molecules in the skin, causing them to vibrate faster, which we perceive as heat. At the low power levels used in applications like airport scanners, this heating is minuscule (less than one-thousandth of a degree) and completely safe. All international safety standards are based on preventing significant thermal effects.
        \item \textbf{Non-Thermal Effects (Speculative and Controversial):} This is the idea that THz waves could affect cells \emph{without} heating them, perhaps by resonating with the natural vibrational frequencies of proteins or other biomolecules, like a specific musical note can make a wine glass vibrate. While there have been some intriguing lab results, there is currently \textbf{no conclusive evidence} for any significant non-thermal effects at safe power levels. Most scientists remain sceptical, attributing many reported effects to subtle, localised heating.
    \end{itemize}

    \parhead{The bottom line on safety}
    Based on decades of research, all established scientific and regulatory bodies (like ICNIRP and the WHO) conclude that the only proven health risk from THz exposure is excessive heating. As long as exposure is kept below the established safety limits (which are very conservative), THz technology is considered safe.
\end{nontechnical}


\section{Overview and Properties}

\subsection{Overview}

The interaction of terahertz (THz) radiation with biological systems is a critical area of study, underpinning both the safety standards for emerging technologies and the potential for new medical applications. These interactions are broadly categorised into two types: \keyterm{thermal effects}, which are well-understood and form the basis of current safety guidelines, and \keyterm{non-thermal effects}, which remain speculative and are the subject of ongoing research and debate.

\begin{keyconcept}
    The dominant bioeffect of THz radiation is \textbf{thermal heating}, caused by the strong absorption of THz energy by water molecules in superficial tissue layers. All established safety standards are designed to limit this temperature rise to less than 1$^\circ$C. While non-thermal effects have been hypothesised, there is currently no conclusive, reproducible evidence for their existence at physiologically relevant, non-thermal power levels.
\end{keyconcept}


\subsection{Thermal Effects}

\paragraph{Mechanism}
When THz radiation enters tissue, its energy is absorbed, primarily by water molecules, causing them to vibrate and rotate more vigorously. This increased kinetic energy manifests as a rise in temperature. The process is governed by the principles of dielectric heating and is highly localised to the surface due to the shallow penetration depth of THz waves.

\paragraph{Modelling}
The temperature rise, $\Delta T$, can be accurately predicted using the bioheat equation, which balances the deposited power from the THz beam against the tissue's thermal conductivity and perfusion (blood flow). For a given intensity, the temperature rise is limited and predictable.

\begin{warningbox}
    The key safety parameter is not just power, but \textbf{power density} (W/m$^2$) and \textbf{exposure duration}. International safety guidelines, such as those from ICNIRP, set conservative limits (e.g., 10 mW/cm$^2$ for occupational exposure) that are designed to keep any potential temperature increase well below 1$^\circ$C, preventing any possibility of thermal damage.
\end{warningbox}


\subsection{Non-Thermal Effects (Hypothesised)}

A non-thermal effect is defined as a biological response that occurs without a measurable change in temperature. Several mechanisms have been proposed, but none have been definitively proven in vivo.

\paragraph{Resonant Absorption}
This hypothesis suggests that THz frequencies could match the natural vibrational or rotational frequencies of specific biomolecules (like proteins or DNA), causing them to resonate. This resonant energy transfer could potentially alter their conformation and function. The primary challenge to this theory is that in the warm, wet environment of a cell, these resonances are extremely broad and heavily damped, making a sharp, specific resonant interaction unlikely.

\paragraph{Cell Membrane Effects}
Some studies have suggested that THz fields can alter cell membrane permeability or influence ion channels. However, distinguishing these effects from highly localised, microthermal effects near the membrane remains a significant experimental challenge.

\paragraph{Quantum Coherence}
As discussed in \Cref{ch:microtubules}, the most speculative theories propose that THz radiation could interact with hypothesised quantum coherent states in structures like microtubules. This remains a fringe area of research with no experimental validation.


\begin{workedexample}{Safety Compliance of a THz Scanner}
    \parhead{Problem} An airport security scanner uses a 0.5 THz beam with an average power of 5 mW, spread over a 10 cm$^2$ area. Does it comply with general public safety standards?
    
    \parhead{Analysis}
    \begin{derivationsteps}
        \step \textbf{Calculate the average power density.}
        \[ S_{\text{avg}} = \frac{\text{Power}}{\text{Area}} = \frac{5 \text{ mW}}{10 \text{ cm}^2} = \textbf{\qty{0.5}{mW/cm^2}} \]
        
        \step \textbf{Compare to the safety limit.} The ICNIRP guideline for public exposure in this frequency range is \textbf{\qty{2}{mW/cm^2}}.
        
        \step \textbf{Calculate the safety margin.}
        \[ \text{Exposure Level} = \frac{0.5 \text{ mW/cm}^2}{2.0 \text{ mW/cm}^2} = 25\% \]
        The device operates at only 25\% of the public safety limit.
    \end{derivationsteps}
    
    \parhead{Interpretation} The scanner is well within international safety standards. The thermal effect is negligible; the calculated temperature rise on the skin surface from this exposure would be a tiny fraction of a degree, far below what could be perceived or cause any biological harm.
\end{workedexample}

\begin{table}[H]
    \centering
    \caption{Summary of THz Bioeffect Mechanisms}
    \label{tab:bioeffect-summary}
    \begin{tabularx}{\textwidth}{@{}XXXX@{}}
        \toprule
        \tableheaderfont Effect Type & \tableheaderfont Mechanism & \tableheaderfont Scientific Status & \tableheaderfont Basis of Safety Standards \\
        \midrule
        \textbf{Thermal} & Dielectric Heating & \textbf{Established} & \textbf{Yes} \\
        Non-Thermal & Molecular Resonance & Controversial / Unproven & No \\
        Non-Thermal & Membrane Effects & Inconclusive & No \\
        Non-Thermal & Quantum Coherence & Highly Speculative & No \\
        \bottomrule
    \end{tabularx}
\end{table}


\begin{importantbox}[title={Further Reading}]
    The study of THz bioeffects is an active and important area of research that informs the safe development of new technologies.
    \begin{description}
        \item[THz Propagation in Biological Tissue] (\Cref{ch:thz-bio-propagation}) explains the underlying physics of why thermal effects are dominant and localised to the surface.
        \item[The Frey Effect] (\Cref{ch:frey-effect}) provides a case study of a well-established, non-thermal (but thermoelastic) effect that occurs at lower microwave frequencies.
        \item[RF Safety] (\Cref{ch:rf-safety}) provides the broader regulatory context and explains the principles behind the setting of international exposure limits.
    \end{description}
\end{importantbox}
