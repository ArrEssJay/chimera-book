% ==============================================================================
% CHAPTER 45: Antenna Theory Basics
% ==============================================================================

\chapter{Antenna Theory Basics}
\label{ch:antenna}

\begin{nontechnical}
    \textbf{An antenna is a transducer that acts like a gateway between the electrical world of circuits and the electromagnetic world of radio waves.}

    \parhead{The loudspeaker analogy}
    \begin{itemize}
        \item A \textbf{loudspeaker} takes an electrical signal and converts it into sound waves that travel through the air.
        \item A \textbf{microphone} does the reverse, capturing sound waves and converting them back into an electrical signal.
    \end{itemize}
    An antenna does the exact same thing, but for radio waves instead of sound. The same antenna is used for both transmitting and receiving.

    \parhead{The funnel effect: Gain}
    Antennas can be designed to focus energy.
    \begin{itemize}
        \item An \textbf{omnidirectional antenna} (like the simple stick on a WiFi router) spreads its energy in all directions, like a bare lightbulb.
        \item A \textbf{directional antenna} (like a satellite dish) focuses all its energy into a narrow, powerful beam, like the reflector in a flashlight. This focusing effect is called \textbf{gain}.
    \end{itemize}

    \parhead{The fundamental trade-off}
    There is a direct trade-off between gain and coverage. An omnidirectional antenna can talk to devices in any direction but has a short range. A high-gain satellite dish has an enormous range but must be pointed with pinpoint accuracy. The choice of antenna is always a compromise based on the application.
\end{nontechnical}


\section{Overview and Properties}

\subsection{Overview}

An \keyterm{antenna} is a transducer designed to transmit and receive electromagnetic waves. It provides the critical interface between a guided wave propagating on a transmission line and a free-space wave propagating through the channel. The principle of \keyterm{reciprocity} dictates that an antenna's properties (such as its radiation pattern and impedance) are identical for both transmission and reception.

The design of an antenna is fundamentally governed by its relationship to the wavelength, $\lambda$, of the signal it is intended to operate with.

\begin{keyconcept}
    The two most important parameters of an antenna are its \textbf{gain} and its \textbf{radiation pattern}. Gain quantifies how well the antenna focuses energy in a particular direction, while the radiation pattern provides a complete 3D map of this focusing effect. These two properties, combined with the antenna's impedance and bandwidth, determine its suitability for a given application.
\end{keyconcept}


\subsection{Key Antenna Parameters}

\paragraph{Radiation Pattern}
The radiation pattern is a graphical representation of the antenna's radiated power in three-dimensional space. Key features include the \keyterm{main lobe} (the direction of maximum radiation), \keyterm{sidelobes} (smaller, unwanted lobes of radiation), and \keyterm{nulls} (directions of zero radiation).

\paragraph{Gain and Directivity}
\keyterm{Directivity} measures an antenna's ability to concentrate radiated power in a particular direction, compared to an ideal isotropic radiator. \keyterm{Gain} is the more practical metric; it is the directivity reduced by the antenna's own internal losses (e.g., ohmic losses in the metal). Gain is expressed in \keyterm{dBi} (decibels relative to isotropic).

\paragraph{Beamwidth}
The \keyterm{Half-Power Beamwidth (HPBW)} is the angular width of the main lobe, measured between the two points where the power has dropped to half (-3 dB) of its peak value. A high-gain antenna has a narrow beamwidth, while a low-gain antenna has a wide beamwidth.

\paragraph{Polarisation}
Polarisation describes the orientation of the electric field vector of the radiated wave. The most common types are \keyterm{linear} (vertical or horizontal) and \keyterm{circular} (Right-Hand or Left-Hand). For maximum signal transfer, the transmitting and receiving antennas must have the same polarisation.

\paragraph{Impedance and SWR}
The \keyterm{input impedance} of an antenna must be matched to the impedance of the transmission line (typically 50 $\Omega$) to ensure maximum power transfer. The quality of this match is measured by the \keyterm{Standing Wave Ratio (SWR)}. An SWR of 1:1 is a perfect match.

\paragraph{Bandwidth}
The bandwidth is the range of frequencies over which the antenna's performance (gain, SWR, pattern) remains within acceptable limits.


\subsection{Common Antenna Types}

\begin{table}[H]
    \centering
    \caption{Comparison of Common Antenna Types}
    \label{tab:antenna-types}
    \begin{tabularx}{\textwidth}{@{}lXXX@{}}
        \toprule
        \tableheaderfont Antenna Type & \tableheaderfont Typical Gain & \tableheaderfont Radiation Pattern & \tableheaderfont Primary Application(s) \\
        \midrule
        Dipole ($\lambda/2$) & 2.15 dBi & Omnidirectional (toroid) & Base for many designs, FM radio \\
        Monopole ($\lambda/4$) & 3-5 dBi & Omnidirectional (hemisphere) & Vehicle antennas, handheld radios \\
        Patch (Microstrip) & 6-9 dBi & Directional (broadside) & Mobile phones, GPS, WiFi \\
        Yagi-Uda & 10-15 dBi & Highly Directional (end-fire) & TV reception, point-to-point links \\
        Parabolic Dish & 30-60 dBi & Extremely Directional & Satellite, microwave backhaul \\
        Phased Array & 15-30+ dBi & Electronically Steerable Beam & 5G, radar, satellite comms \\
        \bottomrule
    \end{tabularx}
\end{table}


\begin{workedexample}{Satellite Dish Gain Calculation}
    \parhead{Problem} Calculate the gain and beamwidth of a 1.2-meter satellite dish operating in the Ku-band.
    \parhead{System Parameters}
    \begin{itemize}
        \item Dish Diameter ($D$): \qty{1.2}{m}.
        \item Frequency ($f$): \qty{12.5}{GHz}.
        \item Wavelength ($\lambda$): $c/f = (3\times10^8)/(12.5\times10^9) = \qty{0.024}{m}$.
        \item Aperture Efficiency ($\eta$): 65\% (typical for a dish).
    \end{itemize}
    \parhead{Solution}
    \begin{derivationsteps}
        \step \textbf{Calculate the Antenna Gain.} The gain of a parabolic dish is given by:
        \[ G = \eta \left(\frac{\pi D}{\lambda}\right)^2 = 0.65 \times \left(\frac{\pi \times 1.2}{0.024}\right)^2 \approx 16032 \text{ (linear)} \]
        Converting to dBi:
        \[ G_{\text{dBi}} = 10\log_{10}(16032) \approx \textbf{\qty{42.0}{dBi}} \]
        
        \step \textbf{Calculate the Half-Power Beamwidth.} The beamwidth of a dish is approximated by:
        \[ \theta_{\text{HPBW}} \approx \frac{70 \lambda}{D} = \frac{70 \times 0.024}{1.2} = \textbf{1.4 degrees} \]
    \end{derivationsteps}
    \parhead{Interpretation} The 1.2-meter dish provides a massive gain of 42 dBi, meaning it concentrates power over 15,000 times more effectively than an isotropic radiator. However, this comes at the cost of an extremely narrow 1.4$^\circ$ beamwidth. This demonstrates why satellite dishes require precise alignment and are unsuitable for mobile applications.
\end{workedexample}


\begin{importantbox}[title={Further Reading}]
    The antenna is the critical interface between the electronic system and the physical channel.
    \begin{description}
        \item[Link Budget Analysis] (\Cref{ch:linkbudget}) is where antenna gain ($G_t$ and $G_r$) is used to overcome the massive losses of the channel.
        \item[Wave Polarisation] (\Cref{ch:polarisation}) provides a deep dive into the E-field orientation that antennas are designed to launch and receive.
        \item[MIMO Systems] (\Cref{ch:mimo}) explains how using multiple, cleverly arranged antennas can be used to multiply the capacity of a wireless link.
    \end{description}
\end{importantbox}
