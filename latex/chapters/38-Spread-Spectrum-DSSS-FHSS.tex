% ==============================================================================
% CHAPTER 38: Spread Spectrum (DSSS & FHSS)
% ==============================================================================

\chapter{Spread Spectrum Techniques}
\label{ch:spread-spectrum}

\begin{nontechnical}
    \textbf{Spread spectrum is like hiding a whisper in a hurricane.} Instead of transmitting a signal in a narrow, obvious frequency band, it deliberately spreads the signal's energy over a massive bandwidth, making it look like random background noise.

    \parhead{The counter-intuitive idea}
    While conventional radio tries to be as spectrally efficient as possible, spread spectrum "wastes" bandwidth to gain three powerful advantages:
    \begin{itemize}
        \item \textbf{Stealth and Security:} The signal is so faint in any given frequency slice that it is incredibly difficult to detect or intercept.
        \item \textbf{Anti-Jamming:} An adversary trying to jam the signal can only block a small fraction of the wide bandwidth. The receiver can still recover the message from the unjammed portions.
        \item \textbf{Multiple Access (CDMA):} Many users can transmit in the same wide band at the same time, each using a unique "spreading code." A receiver tuned to a specific code can pull its desired signal out of the combined transmissions of everyone else.
    \end{itemize}

    \parhead{The two main methods}
    \begin{itemize}
        \item \textbf{Direct-Sequence (DSSS):} The data is multiplied by a very fast pseudo-random code (a "chipping sequence"), which spreads its energy over a wide band. This is the technology that allows \textbf{GPS} to work with signals that are thousands of times weaker than the thermal noise floor.
        \item \textbf{Frequency-Hopping (FHSS):} The transmitter rapidly hops between hundreds or thousands of different frequencies in a pseudo-random pattern known only to the receiver. This is the technology that allows \textbf{Bluetooth} to coexist with WiFi and other devices in the crowded 2.4 GHz band.
    \end{itemize}
    
    \parhead{A piece of history} The concept of frequency hopping was co-invented during World War II by actress Hedy Lamarr and composer George Antheil for use in radio-guided torpedoes, though it was not implemented until decades later.
\end{nontechnical}


\section{Overview and Properties}

\subsection{Overview}

\keyterm{Spread spectrum} techniques are a class of modulation where the bandwidth of the transmitted signal is deliberately made much wider than the minimum required to transmit the information. This is achieved by multiplying the data by a high-rate \keyterm{spreading code} or by rapidly hopping its carrier frequency. This process introduces a large amount of redundancy, which provides a \keyterm{processing gain} at the receiver.

\begin{keyconcept}
    \textbf{Processing Gain ($G_p$)} is the fundamental metric of a spread spectrum system. It is the ratio of the spread bandwidth to the information bandwidth, $G_p = B_{\text{spread}} / B_{\text{info}}$. This gain allows the receiver to recover signals that may be far below the thermal noise floor, reject strong interference, and separate multiple users sharing the same spectrum (CDMA).
\end{keyconcept}


\subsection{Direct-Sequence Spread Spectrum (DSSS)}

In DSSS, a narrowband data signal is multiplied by a high-rate, pseudo-random noise (PN) sequence, also known as a \keyterm{chipping sequence}. Each data bit is encoded by a full sequence of many "chips."

\paragraph{Spreading and Despreading}
The spreading operation convolves the narrow spectrum of the data with the wide spectrum of the PN sequence. At the receiver, the incoming signal is multiplied by the exact same, synchronised PN sequence.
\begin{itemize}
    \item The desired signal, when multiplied by the code a second time, is "despread" back to its original narrow bandwidth.
    \item Any interference or noise, which is uncorrelated with the PN sequence, is spread out over the full bandwidth.
\end{itemize}
A narrow bandpass filter can then remove most of the spread interference, resulting in a dramatic improvement in the effective Signal-to-Noise Ratio, equal to the processing gain.

\paragraph{Code Division Multiple Access (CDMA)}
If each user is assigned a unique and orthogonal (or near-orthogonal) PN sequence, multiple users can transmit on the same frequency at the same time. A receiver tuned to a specific user's code can despread their desired signal while seeing all other users' signals as random noise. This was the basis for 2G and 3G cellular technologies like IS-95 and UMTS. A major challenge in CDMA is the \keyterm{near-far problem}, which requires precise power control to prevent users close to the base station from overwhelming users who are further away.


\subsection{Frequency-Hopping Spread Spectrum (FHSS)}

In FHSS, the carrier frequency of the signal is rapidly changed in a pseudo-random pattern known to both the transmitter and receiver. The signal "hops" between many different channels within a wide frequency band.

\paragraph{Fast vs. Slow Hopping}
\begin{itemize}
    \item \textbf{Slow Hopping (SFH):} Several data symbols are transmitted on each frequency hop. This is simpler to synchronise.
    \item \textbf{Fast Hopping (FFH):} The carrier hops multiple times during the transmission of a single data symbol. This provides more robustness against narrowband jamming and fading.
\end{itemize}
The \keyterm{hop rate} and the number of available channels determine the system's processing gain and its ability to avoid interference.

\begin{keyconcept}
    The primary advantage of FHSS is its ability to \textbf{avoid interference}. If a particular frequency channel is jammed or experiencing a deep fade, the system simply hops to another clear channel on the next time interval. This makes it extremely robust in crowded and contested spectrum, which is why it was chosen for \textbf{Bluetooth}.
\end{keyconcept}

\begin{table}[H]
    \centering
    \caption{Comparison of DSSS and FHSS}
    \label{tab:dsss-vs-fhss}
    \begin{tabular}{@{}lll@{}}
        \toprule
        \tableheaderfont Characteristic & \tableheaderfont DSSS & \tableheaderfont FHSS \\
        \midrule
        Spreading Method & Multiplication by a fast code & Hopping between frequencies \\
        Primary Advantage & Processing gain (noise rejection) & Interference avoidance \\
        Multiple Access & CDMA (Code Division) & FH-CDMA or TDMA/FDMA \\
        Near-Far Problem & Severe & Minimal \\
        Synchronisation & Difficult (chip-level) & Moderate (hop-level) \\
        Primary Application & GPS, 3G Cellular & Bluetooth, Military Radio \\
        \bottomrule
    \end{tabular}
\end{table}


\begin{workedexample}{GPS Signal Processing Gain}
    \parhead{Problem} The civilian GPS L1 C/A signal is received at the Earth's surface with a power of approximately -156 dBW, while the thermal noise in the signal's bandwidth is about -133 dBW. Explain how the signal can be detected.

    \parhead{System Parameters}
    \begin{itemize}
        \item Received Signal Power ($P_s$): \qty{-156}{dBW}.
        \item Noise Power ($P_n$): \qty{-133}{dBW}.
        \item Spreading Code (Chip) Rate ($R_c$): 1.023 Mcps.
        \item Data Rate ($R_b$): 50 bps.
    \end{itemize}

    \parhead{Analysis}
    \begin{derivationsteps}
        \step \textbf{Calculate the input SNR.} This is the ratio of the signal power to the noise power before any processing.
        \[ \text{SNR}_{\text{in}} = P_s - P_n = -156 - (-133) = \textbf{\qty{-23}{dB}} \]
        The signal is 23 dB \emph{weaker} than the noise, meaning the noise power is 200 times stronger than the signal power. The signal is completely buried in the noise.

        \step \textbf{Calculate the Processing Gain ($G_p$).} This is the ratio of the chip rate to the data rate.
        \[ G_p = 10\log_{10}\left(\frac{R_c}{R_b}\right) = 10\log_{10}\left(\frac{1.023 \times 10^6}{50}\right) \approx \textbf{\qty{43.1}{dB}} \]
        
        \step \textbf{Calculate the output SNR.} After despreading in the receiver's correlator, the effective SNR is improved by the processing gain.
        \[ \text{SNR}_{\text{out}} = \text{SNR}_{\text{in}} + G_p = -23 + 43.1 = \textbf{\qty{+20.1}{dB}} \]
    \end{derivationsteps}
    
    \parhead{Interpretation} The DSSS processing provides a massive 43.1 dB of gain. This transforms an impossibly weak signal, buried 23 dB below the noise floor, into a strong signal with an SNR of over 20 dB at the demodulator. This high output SNR allows for a very low Bit Error Rate, ensuring reliable reception of the navigation data.
\end{workedexample}


\begin{importantbox}[title={Further Reading}]
    Spread spectrum is a powerful technique that finds application in many specialised areas of communication.
    \begin{description}
        \item[CDMA] (\Cref{ch:cdma}) provides a deep dive into Code Division Multiple Access, the multi-user application of DSSS that powered 3G cellular networks.
        \item[GPS Systems] (\Cref{ch:gps}) explores the full architecture of the Global Positioning System, which relies entirely on DSSS for its operation.
        \item[IoT Wireless Standards] (\Cref{ch:iot}) discusses how FHSS is used in Bluetooth and other low-power standards to achieve robust communication in the crowded ISM bands.
    \end{description}
\end{importantbox}
