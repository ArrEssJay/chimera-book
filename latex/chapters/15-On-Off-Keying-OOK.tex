% ==============================================================================
% CHAPTER 15: On-Off Keying (OOK)
% ==============================================================================

\chapter{On-Off Keying (OOK)}
\label{ch:ook}

\begin{nontechnical}
    \textbf{On-Off Keying (OOK) is the simplest possible way to send digital data.} It is the radio equivalent of using a flashlight for Morse code.

    \parhead{The simple idea}
    \begin{itemize}
        \item To send a \textbf{bit 1}, turn the radio signal ON.
        \item To send a \textbf{bit 0}, turn the radio signal OFF.
    \end{itemize}

    \parhead{Where it's used} Your car key fob, garage door opener, and wireless doorbell almost certainly use OOK. When you press the button, a tiny transmitter simply turns a carrier wave on and off to send the code. It is also the basis for passive RFID tags and many simple optical communication systems.

    \parhead{The trade-off} OOK is incredibly simple and cheap to implement, requiring minimal components and power. However, it is not very efficient. It requires about 3 dB more power (twice the power) than a slightly more complex scheme like BPSK to achieve the same reliability in a noisy environment.
\end{nontechnical}


\section{Overview and Properties}

\subsection{Overview}

\keyterm{On-Off Keying (OOK)} is a form of \keyterm{Amplitude-Shift Keying (ASK)} in which binary data is represented by the presence (for a '1') or absence (for a '0') of a carrier wave. Its simplicity and low cost have made it a ubiquitous modulation scheme for low-data-rate, short-range, and power-sensitive applications since the dawn of radio.

\begin{keyconcept}
    The primary advantage of OOK is the simplicity of its demodulator. It can be detected with a \textbf{non-coherent envelope detector}, which does not require a complex carrier recovery circuit. This makes the receiver hardware extremely simple and inexpensive, which is the driving factor for its use in mass-market consumer electronics.
\end{keyconcept}


\subsection{Mathematical Representation}

The OOK signal is a carrier wave multiplied by a unipolar data signal, $b_k \in \{0, 1\}$:
\begin{equation}
    s(t) = b_k \cdot A \cos(2\pi f_c t)
\end{equation}
In the IQ plane, OOK is a one-dimensional modulation that occupies two points on the positive I-axis: one at the origin (for bit `0') and one at amplitude $A$ (for bit `1'). This is in contrast to BPSK, which uses antipodal points $\pm A$. This suboptimal constellation spacing is the primary reason for OOK's 3 dB performance penalty compared to BPSK.


\subsection{Modulation and Demodulation}

\paragraph{Modulation}
An OOK modulator can be as simple as a single RF switch or transistor controlled by the binary data stream, turning a continuous wave (CW) oscillator on and off.

\paragraph{Demodulation}
The most common OOK demodulator is an \keyterm{envelope detector}. The process is as follows:
\begin{description}
    \item[1. Bandpass Filtering] An initial filter isolates the desired signal frequency and removes out-of-band noise.
    \item[2. Envelope Detection] The signal is rectified (e.g., using a diode) and then low-pass filtered to extract the amplitude envelope.
    \item[3. Slicing] The recovered envelope is compared against a fixed decision threshold (typically set at $A/2$). If the envelope voltage is above the threshold, the bit is decoded as a `1'; otherwise, it is a `0'.
\end{description}


\subsection{Performance and Spectral Efficiency}

\paragraph{Bit Error Rate (BER)}
The theoretical BER for non-coherent OOK in an AWGN channel is:
\begin{equation}
    \text{BER} \approx \frac{1}{2}e^{-E_b/(2N_0)}
\end{equation}
To achieve a BER of $10^{-5}$, non-coherent OOK requires an $E_b/N_0$ of approximately 13.5 dB. For comparison, coherent BPSK requires only 9.6 dB, highlighting OOK's inferior power efficiency.

\paragraph{Spectral Efficiency}
Due to the sharp on-off transitions, OOK has a relatively wide power spectrum. Its spectral efficiency is poor, typically less than 0.5 bps/Hz, making it unsuitable for bandwidth-limited applications.


\begin{workedexample}{Passive RFID Link Budget Analysis}
    \parhead{Problem} Analyse the forward link (reader to tag) for a passive UHF RFID system to determine its maximum theoretical range.
    \parhead{System Parameters}
    \begin{itemize}
        \item Reader Transmit Power ($P_t$): \qty{1}{W} (\qty{30}{dBm})
        \item Reader Antenna Gain ($G_t$): \qty{6}{dBi}
        \item Tag Antenna Gain ($G_{tag}$): \qty{2}{dBi}
        \item Frequency ($f$): \qty{915}{MHz}
        \item Tag Activation Sensitivity ($P_{\text{tag,min}}$): \qty{-18}{dBm} (minimum power required to power up the tag's chip)
    \end{itemize}
    \parhead{Solution}
    \begin{derivationsteps}
        \step Calculate the transmit EIRP (Effective Isotropic Radiated Power).
        \[ \text{EIRP} = P_t + G_t = 30 \text{ dBm} + 6 \text{ dBi} = \qty{36}{dBm} \]
        \step The power received by the tag is given by the Friis transmission equation. We need to find the maximum allowable path loss (PL).
        \[ P_{\text{tag,min}} = \text{EIRP} + G_{tag} - \text{PL}_{\max} \]
        \[ -18 \text{ dBm} = 36 \text{ dBm} + 2 \text{ dBi} - \text{PL}_{\max} \implies \text{PL}_{\max} = 36 + 2 - (-18) = \qty{56}{dB} \]
        \step Use the Free-Space Path Loss formula to find the distance corresponding to this path loss.
        \[ \text{PL}_{\text{dB}} = 20\log_{10}(d) + 20\log_{10}(f) - 27.55 \quad (\text{for } d \text{ in meters, } f \text{ in MHz}) \]
        \[ 56 = 20\log_{10}(d) + 20\log_{10}(915) - 27.55 \]
        \[ 56 = 20\log_{10}(d) + 59.2 - 27.55 \implies 20\log_{10}(d) = 24.35 \implies d = 10^{(24.35/20)} \approx \qty{16.5}{meters} \]
    \end{derivationsteps}
    \parhead{Interpretation} The theoretical maximum range at which the tag can be powered up is 16.5 meters. In reality, factors like multipath fading, polarisation mismatch, and obstruction losses will reduce this range significantly, typically to between 5 and 10 meters, which aligns with the performance of commercial UHF RFID systems. This calculation demonstrates that the forward link power is the primary limiting factor in passive RFID range.
\end{workedexample}


\begin{importantbox}[title={Further Reading}]
    OOK is the simplest member of the amplitude modulation family and serves as a crucial baseline for understanding more advanced schemes.
    \begin{description}
        \item[Amplitude-Shift Keying (ASK)] (\Cref{ch:ask}) is the direct M-ary generalization of OOK, using multiple, non-zero amplitude levels to encode more bits per symbol.
        \item[Binary Phase-Shift Keying (BPSK)] (\Cref{ch:bpsk}) provides a direct comparison, illustrating how a simple change from unipolar to bipolar signaling yields a significant 3 dB performance improvement.
        \item[Envelope Detection] (\Cref{ch:envelope}) provides a detailed analysis of the non-coherent receiver architecture that makes OOK so simple and cost-effective.
    \end{description}
\end{importantbox}
