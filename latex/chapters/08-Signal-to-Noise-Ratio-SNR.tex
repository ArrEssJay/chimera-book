% ==============================================================================
% CHAPTER 8: Signal-to-Noise Ratio (SNR)
% ==============================================================================

\chapter{Signal-to-Noise Ratio (SNR)}
\label{ch:snr}

\begin{nontechnical}
    \textbf{Signal-to-Noise Ratio (SNR) is like trying to have a conversation.} In a quiet library (high SNR), you can hear every word clearly. In a loud nightclub (low SNR), the background noise overwhelms the voice, and you can barely understand anything.

    \parhead{The simple idea}
    \begin{itemize}
        \item \textbf{Signal} is the information you want to receive (a voice, a data stream).
        \item \textbf{Noise} is the random, unwanted interference that corrupts the signal (static, hiss).
        \item \textbf{SNR} is a simple measure of how much stronger the signal is than the noise.
    \end{itemize}

    \parhead{Real-world examples}
    \begin{itemize}
        \item The "bars" on your mobile phone are a direct indicator of SNR. Five bars means a strong signal and low noise (high SNR), while one bar means a weak signal that is close to the noise floor (low SNR).
        \item A high SNR for your WiFi connection allows for high-order modulations like 256-QAM, enabling fast data rates. As you move away from the router and the SNR drops, your device will automatically switch to a more robust, lower-speed modulation like QPSK.
    \end{itemize}
\end{nontechnical}


\subsection{Overview}

The \keyterm{Signal-to-Noise Ratio (SNR)} is the fundamental metric used to quantify the quality of a communication link. It is a dimensionless ratio that compares the power of the desired signal to the power of the background noise within the same bandwidth. A high SNR is a prerequisite for reliable, high-speed communication.

\begin{keyconcept}
    SNR is the single most important parameter in determining the performance of a communication system. It directly dictates the maximum achievable data rate, as described by the \textbf{Shannon-Hartley theorem}, and the bit error rate (BER) for any given modulation scheme. Every aspect of RF system design, from transmit power to antenna gain and receiver sensitivity, is ultimately aimed at maximising the SNR at the receiver.
\end{keyconcept}


\subsection{Mathematical Definitions}

\paragraph{Linear and Logarithmic Forms}
SNR is defined as the ratio of signal power ($P_s$) to noise power ($P_n$). In practice, it is almost always expressed in decibels (dB).
\begin{align}
    \text{SNR}_{\text{linear}} &= \frac{P_s}{P_n} \\
    \text{SNR}_{\text{dB}} &= 10\log_{10}\left(\frac{P_s}{P_n}\right) = P_{s, \text{dBm}} - P_{n, \text{dBm}}
\end{align}
The decibel scale is essential for handling the enormous dynamic range of signals in communication systems and for simplifying link budget calculations.

\paragraph{Energy Ratios ($E_b/N_0$ and $E_s/N_0$)}
In digital communications, it is often more useful to work with normalized energy ratios, which are independent of bandwidth and data rate.
\begin{itemize}
    \item \keyterm{$E_b/N_0$} (Energy per Bit to Noise Density Ratio): The most fundamental metric, used to compare the performance of different modulation and coding schemes.
    \item \keyterm{$E_s/N_0$} (Energy per Symbol to Noise Density Ratio): Directly related to the geometry of a constellation diagram.
\end{itemize}
These ratios are related to the carrier SNR by the following equations:
\begin{align}
    \frac{E_s}{N_0} &= \text{SNR} \cdot \frac{B}{R_s} \\
    \frac{E_b}{N_0} &= \frac{E_s}{N_0} \cdot \frac{1}{\log_2(M)}
\end{align}
where $B$ is the bandwidth, $R_s$ is the symbol rate, and $M$ is the modulation order.

\begin{warningbox}
    It is critical to be precise when discussing SNR. An "$E_b/N_0$ of 10 dB" is a very different requirement from a "carrier SNR of 10 dB". For QPSK, which carries 2 bits per symbol, the required $E_s/N_0$ is 3 dB higher than the required $E_b/N_0$.
\end{warningbox}


\subsection{Noise Fundamentals}

The ultimate limit on any communication system is \keyterm{thermal noise}, caused by the random thermal motion of electrons in a conductor. The power of this noise is uniformly distributed across all frequencies.
\parhead{Noise Power Spectral Density ($N_0$)}
The noise power in a 1 Hz bandwidth is given by $N_0 = kT$, where $k$ is the Boltzmann constant and $T$ is the system noise temperature in Kelvin. At room temperature (290 K), the thermal noise floor is approximately:
\[ N_0 \approx -174 \text{ dBm/Hz} \]
\parhead{Total Noise Power ($P_n$)}
The total noise power in a receiver is the noise density multiplied by the receiver's bandwidth, $B$, and its \keyterm{Noise Figure} (NF), which accounts for the additional noise generated by the receiver's own electronics.
\begin{equation}
    P_{n, \text{dBm}} = -174 \text{ dBm/Hz} + 10\log_{10}(B) + \text{NF}_{\text{dB}}
\end{equation}


\subsection{SNR and Bit Error Rate (BER)}

For any given modulation scheme, the probability of a bit error is a direct function of $E_b/N_0$. For BPSK, the simplest modulation, the relationship is given by:
\begin{equation}
    \text{BER}_{\text{BPSK}} = Q\left(\sqrt{\frac{2E_b}{N_0}}\right)
\end{equation}
where $Q(\cdot)$ is the Gaussian Q-function. This exponential relationship shows that small improvements in SNR can lead to dramatic reductions in BER.

\begin{table}[H]
    \centering
    \caption{Typical $E_b/N_0$ Requirements for a BER of $10^{-5}$ (Uncoded)}
    \label{tab:snr-ber-reqs}
    \begin{tabular}{@{}lc@{}}
        \toprule
        \tableheaderfont Modulation Scheme & \tableheaderfont Required $E_b/N_0$ (dB) \\
        \midrule
        BPSK & 9.6 \\
        QPSK & 9.6 \\
        8-PSK & 14.0 \\
        16-QAM & 14.5 \\
        64-QAM & 18.8 \\
        \bottomrule
    \end{tabular}
\end{table}


\subsection{Shannon's Channel Capacity Theorem}

The theoretical upper limit on the data rate that can be transmitted over a channel is given by the \keyterm{Shannon-Hartley theorem}:
\begin{equation}
    C = B \log_2(1 + \text{SNR})
\end{equation}
where $C$ is the channel capacity in bits per second, $B$ is the bandwidth in Hz, and SNR is the linear signal-to-noise ratio. This theorem establishes a fundamental trade-off between bandwidth and SNR. To achieve a higher data rate, one must either increase the bandwidth or improve the SNR. Modern modulation and coding techniques are designed to operate as close to this theoretical limit as possible.

\begin{workedexample}{Satellite Downlink SNR Calculation}
    \parhead{Problem} Calculate the received SNR for a typical geostationary satellite TV downlink.
    \parhead{System Parameters}
    \begin{itemize}
        \item Transmit EIRP: \qty{53}{dBW}
        \item Distance (GEO): \qty{36,000}{km}
        \item Frequency: \qty{12}{GHz} (Ku-band)
        \item Receiver G/T (Gain-to-Noise-Temperature): \qty{15}{dB/K}
        \item Bandwidth: \qty{36}{MHz}
    \end{itemize}
    \parhead{Solution}
    \begin{derivationsteps}
        \step Calculate the Free-Space Path Loss (FSPL).
        \[ \text{FSPL} = 20\log_{10}(d_{\text{km}}) + 20\log_{10}(f_{\text{MHz}}) + 32.45 = 20\log_{10}(36000) + 20\log_{10}(12000) + 32.45 \approx \qty{205.6}{dB} \]
        \step Calculate the received power ($P_r$) at the antenna. The Friis equation in this form is $P_r = \text{EIRP} + G_r - \text{FSPL}$. Since we have G/T, we first find the Carrier-to-Noise Density Ratio ($C/N_0$).
        \[ C/N_0 \text{ (dB-Hz)} = \text{EIRP} - \text{FSPL} + G/T - 10\log_{10}(k) = 53 - 205.6 + 15 - (-228.6) \approx \qty{91.0}{dB-Hz} \]
        \step Calculate the Carrier-to-Noise Ratio (CNR or SNR) in the channel bandwidth.
        \[ \text{SNR} = C/N = C/N_0 - 10\log_{10}(B) = 91.0 - 10\log_{10}(36 \times 10^6) = 91.0 - 75.6 = \qty{15.4}{dB} \]
    \end{derivationsteps}
    \parhead{Interpretation} The received SNR of 15.4 dB is excellent for a satellite link. It is well above the threshold required for robust QPSK or 8-PSK reception with forward error correction, ensuring high-quality video even with some margin for rain fade.
\end{workedexample}

\begin{importantbox}[title={Further Reading}]
    SNR is the central metric that connects nearly all other topics in this book.
    \begin{description}
        \item[Link Budget Analysis] (\Cref{ch:linkbudget}) is the complete process of calculating the final SNR of a system, accounting for all gains and losses.
        \item[Bit Error Rate] (\Cref{ch:ber}) explores the mathematical relationship between SNR and the probability of errors for various modulation schemes.
        \item[Forward Error Correction] (\Cref{ch:fec}) describes the coding techniques used to achieve reliable communication at much lower SNR values than would be possible with uncoded transmission.
        \item[Constellation Diagrams] (\Cref{ch:constellations}) provides the visual language for understanding how noise affects a signal and leads to errors.
    \end{description}
\end{importantbox}