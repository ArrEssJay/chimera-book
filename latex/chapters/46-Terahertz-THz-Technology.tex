% ==============================================================================
% CHAPTER 46: Terahertz (THz) Technology
% ==============================================================================

\chapter{Terahertz (THz) Technology}
\label{ch:thz}

\begin{nontechnical}
    \textbf{Terahertz (THz) waves are a form of "invisible light" that sits on the electromagnetic spectrum between microwaves and the infrared radiation we feel as heat.} For decades, this region was known as the "THz gap" because it was notoriously difficult to generate and detect.

    \parhead{Unique properties}
    THz radiation has a unique set of properties that make it incredibly useful for specific applications:
    \begin{itemize}
        \item Like X-rays, it can \textbf{see through} many common materials like clothing, paper, cardboard, and plastic.
        \item Unlike X-rays, it is \textbf{completely non-ionising} and safe for biological tissue. A THz photon has a million times less energy than an X-ray photon.
        \item It is \textbf{strongly absorbed by water and metal}. This is both a key feature and a major limitation.
    \end{itemize}

    \parhead{Real-world applications}
    \begin{itemize}
        \item \textbf{Airport Security Scanners:} The primary use today. THz scanners can detect concealed, non-metallic objects under clothing without the health risks associated with X-ray backscatter systems.
        \item \textbf{Pharmaceutical Quality Control:} THz can "see" through pill coatings to verify the chemical composition and integrity of the tablet inside.
        \item \textbf{Future 6G Communications:} The vast, unused bandwidth in the THz spectrum offers the potential for future wireless data rates of over 1 Terabit-per-second (Tbps).
    \end{itemize}
    
    \parhead{The big limitation: Water}
    The strong absorption by water means THz waves are blocked by rain, fog, and even atmospheric humidity, limiting their use for long-range communication. It also means they cannot penetrate deep into the human body (which is mostly water), making them safe but limiting their medical use to surface imaging (e.g., skin cancer detection).
\end{nontechnical}


\subsection{Overview}

\keyterm{Terahertz (THz) technology} encompasses the generation, manipulation, and detection of electromagnetic waves in the frequency range from approximately 0.1 to 10 THz. This region, often called the \keyterm{THz gap}, lies between the upper end of what is achievable with conventional electronics and the lower end of what is accessible with conventional optics. Recent advances, particularly the development of the \keyterm{Quantum Cascade Laser (QCL)}, have opened up this part of the spectrum for a wide range of scientific and commercial applications.

\begin{keyconcept}
    THz radiation offers a unique combination of properties: the penetration capabilities of microwaves and the high-resolution potential of optics, all while being non-ionising and safe for biological tissue. Its primary limitations are the strong atmospheric absorption by water vapour and the relative immaturity of the source and detector technology.
\end{keyconcept}


\subsection{Propagation Characteristics}

\paragraph{Atmospheric Attenuation}
The dominant impairment for THz propagation is absorption by water vapour molecules, which have strong rotational absorption lines throughout the THz band.
\begin{itemize}
    \item Below 1 THz, "windows" of relatively low attenuation exist, but loss is still on the order of 10-50 dB/km.
    \item Above 1 THz, the attenuation quickly rises to hundreds or thousands of dB/km, making long-distance terrestrial communication impossible.
\end{itemize}
This confines most practical THz communication systems to indoor, short-range (<100 m), or space-based applications.

\paragraph{Material Interaction}
THz waves can penetrate many common dielectric materials but are strongly reflected by metals and absorbed by polar molecules, especially water. This makes THz ideal for non-destructive inspection (e.g., checking package contents) and security screening.


\subsection{Generation and Detection}

\paragraph{THz Sources}
\begin{itemize}
    \item \textbf{Quantum Cascade Lasers (QCLs):} The most important modern source. These are semiconductor lasers that can be engineered to emit coherent, high-power (mW to >1W) THz radiation. Most high-performance QCLs require cryogenic cooling.
    \item \textbf{Photoconductive Antennas (PCAs):} Used in \keyterm{THz Time-Domain Spectroscopy (THz-TDS)}. An ultrafast femtosecond laser illuminates a semiconductor antenna, generating a broadband THz pulse.
    \item \textbf{Electronic Sources:} High-frequency electronic devices like Gunn diodes and Schottky diode frequency multipliers can generate power up to ~1 THz, but their efficiency drops off rapidly.
\end{itemize}

\paragraph{THz Detectors}
Detecting low-power THz radiation is challenging. Common methods include cryogenically cooled \keyterm{bolometers} (which measure a temperature change), Schottky diodes, and coherent detection using a PCA as a receiver.


\subsection{Key Applications}

\begin{table}[H]
    \centering
    \caption{Major Applications of Terahertz Technology}
    \label{tab:thz-applications}
    \begin{tabular}{@{}lll@{}}
        \toprule
        \tableheaderfont Application Area & \tableheaderfont Key THz Property & \tableheaderfont Example Use Case \\
        \midrule
        Security Imaging & Penetrates clothing, non-ionising & Airport body scanners for concealed object detection. \\
        Spectroscopy & Unique molecular spectral "fingerprints" & Pharmaceutical quality control, explosives detection. \\
        Communications & Huge available bandwidth (>10 GHz) & Future 6G systems targeting >100 Gbps data rates. \\
        Medical Imaging & High contrast for water content & Non-invasive skin cancer margin detection. \\
        Non-Destructive Testing & Penetrates packaging, ceramics, plastics & Inspection of aerospace composites, art conservation. \\
        \bottomrule
    \end{tabular}
\end{table}


\begin{workedexample}{THz Security Scanner Link Budget}
    \parhead{Problem} Assess the feasibility of a THz imaging system for security screening at a distance of 3 meters.
    \parhead{System Parameters}
    \begin{itemize}
        \item Frequency: \qty{300}{GHz} (0.3 THz).
        \item Transmit Power ($P_t$): \qty{10}{mW} (\qty{10}{dBm}).
        \item Antenna Gains ($G_t, G_r$): 30 dBi each (high-gain horn antennas).
        \item Receiver Noise Figure (NF): 10 dB.
        \item Bandwidth ($B$): 1 GHz (for fast imaging).
    \end{itemize}
    \parhead{Solution}
    \begin{derivationsteps}
        \step \textbf{Calculate Transmit EIRP.}
        \[ \text{EIRP} = P_t + G_t = 10 \text{ dBm} + 30 \text{ dBi} = \qty{40}{dBm} \]
        
        \step \textbf{Calculate Path Loss.} At 300 GHz, atmospheric absorption is about 1 dB/km, which is negligible over 3 meters. The dominant loss is FSPL.
        \[ \text{FSPL} = 20\log_{10}(3 \text{ m}) + 20\log_{10}(300 \text{ GHz}) + 20.4 \approx 9.5 + 49.5 + 20.4 = \qty{79.4}{dB} \]
        This does not account for the reflection loss from the target, which can be 20-30 dB. Let's assume a total path loss of \qty{105}{dB}.
        
        \step \textbf{Calculate Received Power.}
        \[ P_r = \text{EIRP} - L_{\text{total}} + G_r = 40 - 105 + 30 = \qty{-35}{dBm} \]
        
        \step \textbf{Calculate Receiver Sensitivity.}
        \[ P_{\text{min}} = (\text{Noise Floor}) + \text{SNR}_{\text{req}} = (-174 + 10\log_{10}(10^9) + 10) + 15 = (-174+90+10) + 15 = \qty{-59}{dBm} \]
        (Assuming a required SNR of 15 dB for a clear image).
        
        \step \textbf{Calculate Link Margin.}
        \[ \text{Margin} = P_r - P_{\text{min}} = -35 - (-59) = \textbf{\qty{24}{dB}} \]
    \end{derivationsteps}
    \parhead{Interpretation} The link closes with a very healthy margin of 24 dB. This demonstrates that for short-range applications like security screening, even with the high path loss at THz frequencies, modern sources and detectors can provide more than enough power to create high-quality images.
\end{workedexample}


\begin{importantbox}[title={Further Reading}]
    THz technology sits at the frontier of the electromagnetic spectrum, building on principles from both RF and optics.
    \begin{description}
        \item[Atmospheric Effects] (\Cref{ch:atmospheric}) explains in detail the physics of the water vapour and oxygen absorption lines that so profoundly limit THz propagation.
        \item[Antenna Theory] (\Cref{ch:antenna}) discusses the challenges of designing and fabricating antennas at sub-millimetre wavelengths, where techniques begin to merge with optical lens design.
        \item[Future Wireless (6G)] (\Cref{ch:6g}) will explore how the vast bandwidth available in the THz spectrum is envisioned to be used for the next generation of wireless communication.
    \end{description}
\end{importantbox}