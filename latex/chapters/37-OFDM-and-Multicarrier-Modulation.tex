% ==============================================================================
% CHAPTER 37: OFDM & Multicarrier Modulation
% ==============================================================================

\chapter{OFDM \& Multicarrier Modulation}
\label{ch:ofdm}

\begin{nontechnical}
    \textbf{OFDM is the genius trick that makes modern high-speed wireless communication possible.} Instead of trying to send a huge amount of data down a single, fast "lane" where one pothole (interference) can cause a crash, OFDM splits the data across thousands of slow, narrow lanes.

    \parhead{The highway analogy}
    \begin{itemize}
        \item \textbf{Single-Carrier (Old Method):} One massive truck trying to drive at 200 km/h on a single-lane road. It's very fast, but extremely vulnerable to any problem on that road.
        \item \textbf{OFDM (Modern Method):} One thousand small delivery vans, each driving at a very slow, safe speed in its own dedicated lane. The total amount of cargo delivered is the same, but the system is incredibly robust. If one lane is blocked, the other 999 keep moving.
    \end{itemize}

    \parhead{Why it's so robust against echoes (multipath)}
    The main problem in wireless is echoes bouncing off walls and buildings, which corrupt fast signals. Because each OFDM "van" is moving so slowly, the echoes from one van arrive long before the next van in that lane even leaves. A special "guard interval" (the Cyclic Prefix) is used to simply ignore the echoes, completely eliminating their effect.

    \parhead{Where it's used}
    OFDM is the foundational technology for almost every modern high-speed communication standard:
    \begin{itemize}
        \item All modern \textbf{WiFi} (802.11a/g/n/ac/ax/be).
        \item \textbf{4G (LTE)} and \textbf{5G} cellular networks.
        \item Digital television broadcasting (\textbf{DVB-T}).
        \item Wired broadband like \textbf{ADSL} and VDSL.
    \end{itemize}
\end{nontechnical}


\subsection{Overview}

\keyterm{Orthogonal Frequency-Division Multiplexing (OFDM)} is a multicarrier modulation technique that has become the dominant physical layer for high-data-rate wireless and wireline systems. It combats the primary challenge of high-speed communication—\keyterm{intersymbol interference (ISI)} caused by multipath fading—by dividing a single high-rate data stream into a multitude of low-rate streams, each transmitted on a separate, closely spaced subcarrier.

\begin{keyconcept}
    The genius of OFDM lies in its ability to transform a single, wideband, \textbf{frequency-selective} channel (which is difficult to equalise) into hundreds or thousands of parallel, narrowband, \textbf{flat-fading} sub-channels. Each sub-channel can be equalised with a simple one-tap (single complex multiplication) equaliser, dramatically reducing receiver complexity and enabling reliable communication in harsh multipath environments.
\end{keyconcept}


\subsection{The Principle of Orthogonality}

The key to OFDM's spectral efficiency is that the subcarriers are mathematically \keyterm{orthogonal}, allowing them to be packed together so tightly that their spectra overlap. The subcarrier spacing, $\Delta f$, is chosen to be the inverse of the useful symbol duration, $T_u$. This ensures that at the peak of each subcarrier, the response of all other subcarriers is exactly zero, eliminating \keyterm{inter-carrier interference (ICI)}.

This entire process of modulation and demodulation is implemented with extreme efficiency using the \keyterm{Inverse Fast Fourier Transform (IFFT)} at the transmitter and the \keyterm{Fast Fourier Transform (FFT)} at the receiver.


\subsection{The Cyclic Prefix (CP)}

To combat the ISI caused by multipath delay spread, a \keyterm{guard interval} is inserted at the beginning of each OFDM symbol. In OFDM, this guard interval is a copy of the \emph{end} of the symbol, known as the \keyterm{Cyclic Prefix (CP)}.
\begin{itemize}
    \item The CP must be longer than the channel's maximum expected delay spread.
    \item It ensures that as long as the echoes from the previous symbol arrive within this guard time, they do not corrupt the current symbol.
    \item It also cleverly preserves the orthogonality of the subcarriers by making the linear convolution of the channel appear as a circular convolution, which is easily invertible in the frequency domain.
\end{itemize}
The trade-off is a reduction in spectral efficiency, as the CP carries no new information. A typical CP overhead is between 10\% and 25\%.


\subsection{Peak-to-Average Power Ratio (PAPR)}

The primary disadvantage of OFDM is its high \keyterm{Peak-to-Average Power Ratio (PAPR)}. When many subcarriers happen to align in phase, their amplitudes can sum constructively, creating a large peak in the time-domain signal that is many times higher than the average power. This requires the transmitter's power amplifier to be highly linear and to operate with a significant "back-off" from its saturation point, which reduces its power efficiency. This is a critical design challenge, especially for battery-powered mobile devices.


\begin{table}[H]
    \centering
    \caption{Key OFDM Parameters in Modern Standards}
    \label{tab:ofdm-standards}
    \begin{tabular}{@{}llll@{}}
        \toprule
        \tableheaderfont Standard & \tableheaderfont FFT Size (N) & \tableheaderfont Subcarrier Spacing & \tableheaderfont Primary Use \\
        \midrule
        WiFi 5 (802.11ac) & 64 -- 512 & 312.5 kHz & WLAN \\
        4G (LTE) & 128 -- 2048 & 15 kHz & Mobile Broadband \\
        5G NR & 512 -- 4096 & 15, 30, 60, 120 kHz & Mobile, mmWave, URLLC \\
        DVB-T2 & 2k, 8k, 16k, 32k & 1 -- 4 kHz & Digital Television \\
        \bottomrule
    \end{tabular}
\end{table}


\begin{workedexample}{WiFi 6 (802.11ax) Data Rate Calculation}
    \parhead{Problem} Calculate the maximum theoretical data rate for a single spatial stream in a WiFi 6 system.
    \parhead{System Parameters}
    \begin{itemize}
        \item Channel Bandwidth: \qty{160}{MHz}.
        \item Modulation: 1024-QAM (10 bits per symbol).
        \item FEC Code Rate: 5/6.
        \item OFDM Parameters: FFT size $N=2048$, with 1960 data subcarriers. Symbol duration $T_{\text{sym}} = \qty{13.6}{\mu s}$ (including guard interval).
    \end{itemize}
    \parhead{Solution}
    \begin{derivationsteps}
        \step Calculate the number of information bits per OFDM symbol.
        \[ \text{Bits/Symbol} = (\text{Data Subcarriers}) \times (\text{Bits/Mod Symbol}) \times (\text{Code Rate}) \]
        \[ \text{Bits/Symbol} = 1960 \times 10 \times (5/6) \approx 16333 \text{ bits} \]
        \step Calculate the total data rate.
        \[ \text{Data Rate} = \frac{\text{Bits/Symbol}}{T_{\text{symbol}}} = \frac{16333 \text{ bits}}{13.6 \times 10^{-6} \text{ s}} \approx 1.2 \times 10^9 \text{ bps} = \textbf{\qty{1.2}{Gbps}} \]
    \end{derivationsteps}
    \parhead{Interpretation} A single stream of WiFi 6 can achieve a data rate of 1.2 Gbps. A modern 8x8 MIMO router can theoretically support an aggregate throughput of up to $8 \times 1.2 = \qty{9.6}{Gbps}$, as widely advertised. This incredible throughput is made possible by the combination of wide bandwidth, high-order QAM, and the robustness of the underlying OFDM physical layer.
\end{workedexample}


\begin{importantbox}[title={Further Reading}]
    OFDM is the physical layer that enables most of the advanced communication techniques discussed in this book.
    \begin{description}
        \item[Multipath Fading] (\Cref{ch:multipath}) describes the core problem of ISI that OFDM was designed to solve.
        \item[MIMO Systems] (\Cref{ch:mimo}) explains how multiple-antenna systems are layered on top of OFDM to create parallel spatial streams, dramatically increasing capacity.
        \item[5G and WiFi Systems] (\Cref{ch:5g}, \Cref{ch:wifi}) provide detailed case studies of how OFDM and its variant, OFDMA, are implemented in the world's most successful wireless standards.
    \end{description}
\end{importantbox}