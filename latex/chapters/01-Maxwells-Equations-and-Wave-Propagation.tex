% ==============================================================================
% CHAPTER 1: Maxwell's Equations & Wave Propagation - Professional Revision
% ==============================================================================

\chapter{Maxwell's Equations \& Wave Propagation}
\label{ch:maxwell}

\begin{nontechnical}
    Maxwell's Equations are the fundamental laws that explain how every wireless device works—from a car radio to a deep-space probe. They describe how electric and magnetic fields can create each other, forming a wave that travels through space at the speed of light.

    \parhead{The Four Fundamental Rules}
    \begin{enumerate}
        \item Electric charges create electric fields.
        \item Magnetic field lines always form closed loops (there are no magnetic monopoles).
        \item A changing magnetic field creates an electric field (the principle behind electric generators).
        \item An electric current or a changing electric field creates a magnetic field (the principle behind antennas).
    \end{enumerate}

    \parhead{The Breakthrough}
    Maxwell's genius was in adding the final term to the fourth equation. This single addition predicted the existence of waves travelling at exactly the measured speed of light, proving that light itself is an electromagnetic wave. Every wireless technology is a practical application of this profound discovery.
\end{nontechnical}

\section{Overview}

\keyterm{Maxwell's Equations} are a set of four partial differential equations that form the foundation of classical electromagnetism, classical optics, and electric circuits. Formulated by James Clerk Maxwell in the 1860s, they unify the previously separate phenomena of electricity, magnetism, and light into a single, coherent framework.

\begin{keyconcept}
    The most profound prediction of Maxwell's Equations is the existence of self-propagating \textbf{electromagnetic waves}. The theory demonstrated that these waves travel at a constant speed, $c$, which, when calculated from the fundamental physical constants of electricity ($\epsilon_0$) and magnetism ($\mu_0$), was found to be precisely the measured speed of light. This was the definitive proof that light is an electromagnetic phenomenon.
\end{keyconcept}

\section{The Four Equations in Differential Form}

\subsection{Gauss's Law for Electricity}
This law states that an electric field diverges from electric charges.
\begin{equation}
    \nabla \cdot \mathbf{E} = \frac{\rho}{\epsilon_0}
    \label{eq:gauss_electric}
\end{equation}
where:
\begin{description}
    \item[$\mathbf{E}$] is the electric field vector (\unit{V/m}).
    \item[$\rho$] is the electric charge density (\unit{C/m^3}).
    \item[$\epsilon_0$] is the permittivity of free space.
\end{description}

\subsection{Gauss's Law for Magnetism}
This law states that there are no magnetic monopoles; magnetic field lines always form closed loops.
\begin{equation}
    \nabla \cdot \mathbf{B} = 0
    \label{eq:gauss_magnetic}
\end{equation}
where $\mathbf{B}$ is the magnetic field vector (Tesla).

\subsection{Faraday's Law of Induction}
This law describes how a time-varying magnetic field creates (induces) a circulating electric field.
\begin{equation}
    \nabla \times \mathbf{E} = -\frac{\partial \mathbf{B}}{\partial t}
    \label{eq:faraday}
\end{equation}

\subsection{Ampère-Maxwell Law}
This law states that magnetic fields can be generated by two sources: by an electric current ($\mathbf{J}$) and by a time-varying electric field (the "displacement current").
\begin{equation}
    \nabla \times \mathbf{B} = \mu_0 \mathbf{J} + \mu_0 \epsilon_0 \frac{\partial \mathbf{E}}{\partial t}
    \label{eq:ampere_maxwell}
\end{equation}
where $\mu_0$ is the permeability of free space and $\mathbf{J}$ is the current density (\unit{A/m^2}).

\begin{warningbox}
    Maxwell's addition of the \keyterm{displacement current} term ($\mu_0 \epsilon_0 \frac{\partial \mathbf{E}}{\partial t}$) was the critical step. Without it, the equations do not predict the existence of propagating waves. It establishes the perfect symmetry between electric and magnetic fields: a changing B-field creates an E-field, and a changing E-field creates a B-field.
\end{warningbox}

\section{The Electromagnetic Wave}

In free space, where there are no charges ($\rho=0$) and no currents ($\mathbf{J}=0$), Maxwell's equations can be combined to derive the electromagnetic wave equation.

\begin{workedexample}{Derivation of the Wave Equation}
    \parhead{Objective}
    Derive the wave equation for the electric field $\mathbf{E}$ in a vacuum.
    
    \parhead{Derivation}
    \begin{derivationsteps}
        \step Start by taking the curl of Faraday's Law (\cref{eq:faraday}):
          \[ \nabla \times (\nabla \times \mathbf{E}) = -\frac{\partial}{\partial t}(\nabla \times \mathbf{B}) \]
        \step Substitute the Ampère-Maxwell Law (\cref{eq:ampere_maxwell}) with $\mathbf{J}=0$:
          \[ \nabla \times (\nabla \times \mathbf{E}) = -\mu_0 \epsilon_0 \frac{\partial^2 \mathbf{E}}{\partial t^2} \]
        \step Apply the vector identity $\nabla \times (\nabla \times \mathbf{A}) = \nabla(\nabla \cdot \mathbf{A}) - \nabla^2 \mathbf{A}$. Since $\rho=0$, Gauss's Law gives $\nabla \cdot \mathbf{E} = 0$. This simplifies the identity to yield the final \keyterm{wave equation}:
          \begin{equation}
              \nabla^2 \mathbf{E} = \mu_0 \epsilon_0 \frac{\partial^2 \mathbf{E}}{\partial t^2}
              \label{eq:wave_equation}
          \end{equation}
    \end{derivationsteps}
    \parhead{Conclusion}
    This is the standard form of a wave equation, where the propagation speed $v$ is given by $v^2 = 1/(\mu_0 \epsilon_0)$. The same derivation can be performed for the magnetic field $\mathbf{B}$.
\end{workedexample}

\begin{figure}[H]
    \centering
    \begin{tikzpicture}[scale=1.2, x=1.5cm, y=1.2cm, z=0.7cm, font=\sffamily\small]
        % Axes
        \draw[->, thick] (0,0,0) -- (4,0,0) node[below left] {z (Propagation)};
        \draw[->, thick] (0,0,0) -- (0,1.5,0) node[below right] {y (E-field)};
        \draw[->, thick] (0,0,0) -- (0,0,1.5) node[above] {x (B-field)};

        % E-field (sine wave in y-z plane)
        \draw[color=diagramprimary, thick, variable=\t, domain=0:3.8, samples=100] 
            plot (\t, {sin(\t*180/pi*2)}, 0) node[right, black] {$\mathbf{E}$};
        
        % B-field (sine wave in x-z plane)
        \draw[color=diagramsecondary, thick, variable=\t, domain=0:3.8, samples=100] 
            plot (\t, 0, {sin(\t*180/pi*2)}) node[above, black] {$\mathbf{B}$};

        % Wavelength annotation
        \draw[<->, dashed] (0,-1.2,0) -- (3.14159,-1.2,0) node[midway, below] {Wavelength, $\lambda$};
        
        % Field vectors at a point
        \begin{scope}[shift={(pi/4,0,0)}]
            \draw[->, ultra thick, diagramprimary] (0,0,0) -- (0,{sin(pi/2)},0);
            \draw[->, ultra thick, diagramsecondary] (0,0,0) -- (0,0,{sin(pi/2)});
        \end{scope>
    \end{tikzpicture}
    \caption{A linearly polarised transverse electromagnetic (TEM) wave, showing the orthogonal electric ($\mathbf{E}$) and magnetic ($\mathbf{B}$) fields propagating in the z-direction.}
    \label{fig:tem_wave}
\end{figure}

\begin{importantbox}
    \section*{Further Reading}
    Maxwell's Equations are the physical bedrock upon which all subsequent topics are built. A firm grasp of these principles is essential for understanding:
    \begin{description}
        \item[The Electromagnetic Spectrum (\Cref{ch:spectrum})] which categorises the solutions to the wave equation by frequency.
        \item[Wave Polarisation (\Cref{ch:polarisation})] which describes the geometric orientation of the electric field vector $\mathbf{E}$.
        \item[Antenna Theory (\Cref{ch:antenna})] which explains the practical mechanisms for generating and receiving these waves.
    \end{description}
\end{importantbox}