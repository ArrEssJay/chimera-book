% ==============================================================================
% CHAPTER 1: Maxwell's Equations & Wave Propagation
% ==============================================================================

\chapter{Foundations of Communication Theory} % Assuming this is Part I, Chapter 1
\label{ch:foundations}

\section{Maxwell's Equations \& Wave Propagation}
\label{sec:maxwell}

\begin{nontechnical}
    \parhead{Maxwell's Equations explain how every wireless device works}---from your phone to GPS satellites to radio telescopes.
    \parhead{Simple idea} Electric and magnetic fields can create each other. When they do this in just the right way, they form a wave that travels through space at the speed of light.

    \parhead{The four rules}
    \begin{enumerate}[label=\arabic*.]
        \item Electric charges create electric fields.
        \item Magnetic field lines always form closed loops (no magnetic monopoles).
        \item Changing magnetic fields create electric fields (how generators work).
        \item Electric currents and changing electric fields create magnetic fields (how antennas work).
    \end{enumerate}

    \parhead{Maxwell's breakthrough} He realised these four rules predict waves travelling at exactly the speed of light---proving that light itself is an electromagnetic wave.

    \parhead{Real impact} Every wireless technology (WiFi, Bluetooth, 5G, GPS, radar, satellite TV) relies on electromagnetic waves described by these equations.
\end{nontechnical}

\subsection{Overview}

\keyterm{Maxwell's Equations} are the fundamental laws of electromagnetism, unifying electricity, magnetism, and optics into a single coherent framework. Formulated by James Clerk Maxwell in 1865, these four elegant equations describe how electric and magnetic fields interact and propagate through space.

\begin{keyconcept}
    Maxwell's Equations predict that electromagnetic disturbances propagate as \textbf{waves} at the speed of light, proving that light itself is an electromagnetic phenomenon. This unification of optics with electromagnetic theory represents one of the greatest achievements in physics.
\end{keyconcept}

\subsection{The Four Maxwell's Equations}

\subsubsection{Differential Form (Local Description)}

\paragraph{1. Gauss's Law (Electric)}
States that \keyterm{electric charges create electric fields}:
\begin{equation}
    \nabla \cdot \mathbf{E} = \frac{\rho}{\epsilon_0}
    \label{eq:gauss-electric}
\end{equation}
where:
\begin{description}[\dinfont, style=unboxed, labelwidth=1.5cm]
    \item[$\mathbf{E}$] is the electric field vector (\unit{V/m}).
    \item[$\rho$] is the charge density (\unit{C/m^3}).
    \item[$\epsilon_0$] is the permittivity of free space (\SI{8.854e-12}{F/m}).
\end{description}
\parhead{Physical meaning} Electric field lines originate from positive charges and terminate on negative charges. The divergence of $\mathbf{E}$ at any point is proportional to the charge density there.

\paragraph{2. Gauss's Law for Magnetism}
States that \keyterm{no magnetic monopoles exist}:
\begin{equation}
    \nabla \cdot \mathbf{B} = 0
    \label{eq:gauss-magnetic}
\end{equation}
where $\mathbf{B}$ is the magnetic field vector (Tesla or \unit{Wb/m^2}).
\parhead{Physical meaning} Magnetic field lines always form closed loops---there are no isolated north or south poles.

\paragraph{3. Faraday's Law of Induction}
States that \keyterm{changing magnetic fields create electric fields}:
\begin{equation}
    \nabla \times \mathbf{E} = -\frac{\partial \mathbf{B}}{\partial t}
    \label{eq:faraday}
\end{equation}
\parhead{Physical meaning} A time-varying magnetic field induces a circulating electric field. This is the principle behind electrical generators and transformers.

\paragraph{4. Ampère-Maxwell Law}
States that \keyterm{electric currents and changing electric fields create magnetic fields}:
\begin{equation}
    \nabla \times \mathbf{B} = \mu_0 \mathbf{J} + \mu_0 \epsilon_0 \frac{\partial \mathbf{E}}{\partial t}
    \label{eq:ampere-maxwell}
\end{equation}
where $\mu_0$ is the permeability of free space (\SI{4\pi e-7}{H/m}), $\mathbf{J}$ is the current density (\unit{A/m^2}), and the final term is the \keyterm{displacement current}.

\begin{keyconcept}
    \textbf{Maxwell's Critical Insight:} The displacement current term, $\mu_0 \epsilon_0 \partial \mathbf{E}/\partial t$, was missing from Ampère's original law. Maxwell added it for mathematical consistency—this single term is what allows the equations to predict the existence of electromagnetic waves.
\end{keyconcept}

\subsection{Derivation of the Wave Equation}

\begin{workedexample}{Deriving the Wave Equation in a Vacuum}
In free space (no charges, no currents), Maxwell's equations simplify.
\begin{enumerate}
    \item Take the curl of Faraday's Law (\cref{eq:faraday}):
          \[ \nabla \times (\nabla \times \mathbf{E}) = -\frac{\partial}{\partial t}(\nabla \times \mathbf{B}) \]
    \item Substitute the simplified Ampère-Maxwell Law ($\mathbf{J}=0$):
          \[ \nabla \times (\nabla \times \mathbf{E}) = -\mu_0 \epsilon_0 \frac{\partial^2 \mathbf{E}}{\partial t^2} \]
    \item Apply the vector identity $\nabla \times (\nabla \times \mathbf{A}) = \nabla(\nabla \cdot \mathbf{A}) - \nabla^2 \mathbf{A}$. In a vacuum, Gauss's Law ($\rho=0$) gives $\nabla \cdot \mathbf{E} = 0$. This simplifies the identity, yielding the \keyterm{wave equation}:
          \begin{equation}
              \nabla^2 \mathbf{E} = \mu_0 \epsilon_0 \frac{\partial^2 \mathbf{E}}{\partial t^2}
              \label{eq:wave-E}
          \end{equation}
\end{enumerate}
\end{workedexample}

\subsection{Wave Speed and the Speed of Light}
The general form of a wave equation is $\nabla^2 f = (1/v^2) \partial^2 f/\partial t^2$, where $v$ is the wave propagation speed. Comparing this to \cref{eq:wave-E}, we see that $1/v^2 = \mu_0 \epsilon_0$. The speed of the wave is therefore:
\begin{equation}
v = \frac{1}{\sqrt{\mu_0 \epsilon_0}} \approx \SI{2.998e8}{m/s} = c
\label{eq:speed-of-light}
\end{equation}

\begin{keyconcept}
    \textbf{Maxwell's Triumph:} The predicted speed of electromagnetic waves, calculated from the fundamental constants of electricity ($\epsilon_0$) and magnetism ($\mu_0$), was precisely equal to the measured speed of light ($c$). This was the definitive proof that \textbf{light is an electromagnetic wave}.
\end{keyconcept}

\begin{importantbox}
    For a deeper understanding of the practical applications of these foundational concepts, the following chapters are essential:
    \begin{itemize}
        \item \textbf{\Cref{ch:spectrum}:} The Electromagnetic Spectrum
        \item \textbf{\Cref{ch:antenna}:} Antenna Theory Basics
        \item \textbf{\Cref{ch:polarisation}:} Wave Polarisation
    \end{itemize}
\end{importantbox}