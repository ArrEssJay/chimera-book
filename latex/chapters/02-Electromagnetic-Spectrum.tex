% ==============================================================================
% CHAPTER 2: The Electromagnetic Spectrum
% ==============================================================================

\chapter{The Electromagnetic Spectrum}
\label{ch:spectrum}

\begin{nontechnical}
    \parhead{The electromagnetic spectrum is like a piano keyboard}---but instead of sound notes, it's frequencies of light and radio waves.

    \parhead{The big picture}
    \begin{itemize}
        \item \textbf{Low notes (low frequency):} Radio, AM/FM, WiFi, microwaves.
        \item \textbf{Middle notes:} Infrared (heat from a fire), visible light (the colours we see).
        \item \textbf{High notes (high frequency):} Ultraviolet, X-rays, gamma rays.
    \end{itemize}

    \parhead{It's all the same thing} Radio waves, WiFi, light, and X-rays are all electromagnetic waves---they just have different frequencies.

    \parhead{Why frequency matters} Low frequencies travel far and penetrate buildings but need big antennas. High frequencies can carry more data and use small antennas but are more easily blocked.
\end{nontechnical}

\subsection{Overview}

The \keyterm{electromagnetic (EM) spectrum} encompasses all frequencies of electromagnetic radiation, from extremely low frequency (ELF) radio waves to ultra-high energy gamma rays, spanning more than 20 orders of magnitude. All EM waves travel at the speed of light in a vacuum and obey Maxwell's equations.

\begin{keyconcept}
    The electromagnetic spectrum is fundamentally unified---radio waves, visible light, and gamma rays are all the same phenomenon, differing only in their frequency and wavelength. This allows us to apply common principles across vastly different applications.
\end{keyconcept}

\subsection{Fundamental Relationships}

The relationship between frequency ($f$), wavelength ($\lambda$), and the speed of light ($c$) is the cornerstone of spectrum analysis:
\begin{equation}
    c = \lambda f
    \label{eq:wavelength}
\end{equation}
This inverse relationship means that as frequency increases, wavelength decreases. The energy of a single photon is directly proportional to its frequency, a concept from quantum mechanics given by the Planck-Einstein relation:
\begin{equation}
    E = h f = \frac{hc}{\lambda}
\end{equation}
where $h$ is Planck's constant. This energy determines whether radiation is \keyterm{ionising} (able to break chemical bonds, like X-rays) or \keyterm{non-ionising} (unable to, like radio waves). The threshold between them occurs in the ultraviolet region at a photon energy of approximately \SI{10}{eV}.

\subsection{Radio Frequencies (RF): \qty{3}{kHz} to \qty{300}{GHz}}
This portion of the spectrum is the workhorse of all modern wireless communications. The descriptions below are organised by standard frequency band designations.

\begin{description}[font=\normalfont, style=unboxed, labelwidth=1cm]
    \item[ELF (3--30 Hz)] \textbf{Extremely Low Frequency.} Used for submarine communication due to its ability to penetrate seawater. Wavelengths are measured in thousands of kilometres.
    \item[VLF (3--30 kHz)] \textbf{Very Low Frequency.} Used for long-range navigation and time signals.
    \item[LF (30--300 kHz)] \textbf{Low Frequency.} Used for AM longwave radio and aviation beacons (NDBs).
    \item[MF (300 kHz -- 3 MHz)] \textbf{Medium Frequency.} Home to the standard AM broadcast band and maritime communication.
    \item[HF (3--30 MHz)] \textbf{High Frequency.} The shortwave band, enabling global communication via ionospheric reflection (skywave).
    \item[VHF (30--300 MHz)] \textbf{Very High Frequency.} Includes FM radio, broadcast television, and air traffic control. Propagation is almost exclusively line-of-sight.
    \item[UHF (300 MHz -- 3 GHz)] \textbf{Ultra High Frequency.} The most crowded part of the spectrum, containing mobile phones, GPS, WiFi, and Bluetooth.
    \item[SHF (3--30 GHz)] \textbf{Super High Frequency.} Used for satellite communications (Ku-band), 5G mid-band, and 5~GHz WiFi.
    \item[EHF (30--300 GHz)] \textbf{Extremely High Frequency.} Known as the \keyterm{millimetre wave (mmWave)} band, it is used for high-capacity 5G and automotive radar.
\end{description}

\subsection{High-Energy Frequencies: >\qty{750}{THz}}
These frequencies carry enough energy per photon to be ionising, posing a risk to biological tissue.

\begin{description}[font=\normalfont, style=unboxed, labelwidth=1cm]
    \item[UV (750 THz -- 30 PHz)] \textbf{Ultraviolet.} Marks the transition to ionising radiation. Used for sterilisation and semiconductor manufacturing.
    \item[X-ray (30 PHz -- 30 EHz)] \textbf{X-Rays.} Highly ionising radiation used for medical imaging and security screening.
    \item[Gamma ($\gamma$)] \textbf{>30 EHz.} The most energetic and penetrating form of radiation, used in radiotherapy and astronomy.
\end{description}

\subsection{Spectrum Summary Table}

% Re-formatted with tabularx to ensure the table fits within the text width.
% The 'X' column type allows text to wrap automatically.
\begin{table}[H]
    \centering
    \caption{The Electromagnetic Spectrum at a Glance}
    \label{tab:spectrum-summary}
    \begin{tabularx}{\textwidth}{@{}l l l X l@{}}
        \toprule
        \tableheaderfont Band & \tableheaderfont Frequency & \tableheaderfont Wavelength & \tableheaderfont Key Applications & \tableheaderfont Ionising? \\
        \midrule
        Radio (RF) & \qty{3}{kHz} -- \qty{300}{GHz} & > \qty{1}{mm} & Communications, Radar & No \\
        Infrared (IR) & \qty{0.3} -- \qty{430}{THz} & \qty{1}{mm} -- \qty{700}{nm} & Thermal, Fiber Optics & No \\
        Visible & \qty{430} -- \qty{750}{THz} & \qty{700} -- \qty{400}{nm} & Human Vision, Optics & No \\
        Ultraviolet (UV) & \qty{750}{THz} -- \qty{30}{PHz} & \qty{400} -- \qty{10}{nm} & Sterilisation, Lithography & Transition \\
        X-ray & \qty{30}{PHz} -- \qty{30}{EHz} & \qty{10} -- \qty{0.01}{nm} & Medical Imaging & Yes \\
        Gamma Ray & > \qty{30}{EHz} & < \qty{0.01}{nm} & Radiotherapy, Astronomy & Yes \\
        \bottomrule
    \end{tabularx}
\end{table}

\begin{importantbox}
    A deep understanding of the electromagnetic spectrum is critical for system design. The following chapters build directly on these concepts:
    \begin{description}
        \item[Propagation Characteristics (\Cref{ch:atmospheric})] provides a detailed analysis of how signals in different bands interact with the atmosphere, including absorption, reflection, and ducting.
        \item[Antenna Theory (\Cref{ch:antenna})] explores the practical design of antennas that are resonant and efficient at specific frequencies, from massive VLF towers to tiny mmWave arrays.
        \item[Link Budget Analysis (\Cref{ch:linkbudget})] provides the full framework for calculating system performance, combining frequency-dependent path loss with antenna gain, power, and receiver sensitivity.
    \end{description}
\end{importantbox}