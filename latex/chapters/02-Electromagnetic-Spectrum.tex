% ==============================================================================
% CHAPTER 2: The Electromagnetic Spectrum
% ==============================================================================

\chapter{The Electromagnetic Spectrum}
\label{ch:spectrum}

\begin{nontechnical}
    \textbf{The electromagnetic spectrum is like a piano keyboard}---but instead of sound notes, it's frequencies of light and radio waves.

    \parhead{The big picture}
    \begin{itemize}
        \item \textbf{Low notes (low frequency):} Radio, AM/FM, WiFi, microwaves.
        \item \textbf{Middle notes:} Infrared (heat from a fire), visible light (the colours we see).
        \item \textbf{High notes (high frequency):} Ultraviolet, X-rays, gamma rays.
    \end{itemize}

    \parhead{It's all the same thing} Radio waves, WiFi, light, and X-rays are all electromagnetic waves---they just have different frequencies.

    \parhead{Why frequency matters} Low frequencies travel far and penetrate buildings but need big antennas. High frequencies can carry more data and use small antennas but are more easily blocked.
\end{nontechnical}

\subsection{Overview}

The \keyterm{electromagnetic (EM) spectrum} encompasses all frequencies of electromagnetic radiation, from extremely low frequency (ELF) radio waves to ultra-high energy gamma rays, spanning more than 20 orders of magnitude. All EM waves travel at the speed of light in a vacuum and obey Maxwell's equations.

\begin{keyconcept}
    The electromagnetic spectrum is fundamentally unified---radio waves, visible light, and gamma rays are all the same phenomenon, differing only in their frequency and wavelength. This allows us to apply common principles across vastly different applications.
\end{keyconcept}

\subsection{Fundamental Relationships}

The relationship between frequency ($f$), wavelength ($\lambda$), and the speed of light ($c$) is the cornerstone of spectrum analysis:
\begin{equation}
    c = \lambda f
    \label{eq:wavelength}
\end{equation}
This inverse relationship means that as frequency increases, wavelength decreases. The energy of a single photon is directly proportional to its frequency, a concept from quantum mechanics given by the Planck-Einstein relation:
\begin{equation}
    E = h f = \frac{hc}{\lambda}
\end{equation}
where $h$ is Planck's constant. This energy determines whether radiation is \keyterm{ionizing} (able to break chemical bonds, like X-rays) or \keyterm{non-ionizing} (unable to, like radio waves). The threshold between them occurs in the ultraviolet region at a photon energy of approximately 10~eV.

\subsection{Radio Frequencies (RF): \qty{3}{kHz} to \qty{300}{GHz}}

This portion of the spectrum is the workhorse of all modern wireless communications.

\begin{spectrumband}{ELF}{Extremely Low Frequency: 3--30 Hz}
    \item[Wavelength:] 100,000--10,000 km.
    \item[Applications:] Submarine communication (due to its ability to penetrate seawater), geophysical surveys.
    \item[Propagation:] Travels within the Earth-ionosphere waveguide with extremely low attenuation.
\end{spectrumband}

\begin{spectrumband}{VLF}{Very Low Frequency: 3--30 kHz}
    \item[Wavelength:] 100--10 km.
    \item[Applications:] Long-range navigation (e.g., LORAN), time signals, lightning detection.
    \item[Propagation:] Primarily by ground wave and reflection from the ionosphere.
\end{spectrumband}

\begin{spectrumband}{LF}{Low Frequency: 30--300 kHz}
     \item[Wavelength:] 10--1 km.
    \item[Applications:] AM radio (longwave), non-directional beacons (NDB) for aviation, some RFID systems.
    \item[Propagation:] Reliable ground wave propagation, stable day and night.
\end{spectrumband}

\begin{spectrumband}{MF}{Medium Frequency: 300 kHz -- 3 MHz}
    \item[Wavelength:] 1 km -- 100 m.
    \item[Applications:] AM radio (the standard broadcast band), maritime communication.
    \item[Propagation:] Ground wave during the day; ionospheric reflection (\keyterm{skywave}) at night allows for long-distance reception.
\end{spectrumband}

\begin{spectrumband}{HF}{High Frequency: 3--30 MHz}
    \item[Wavelength:] 100--10 m.
    \item[Applications:] Shortwave radio, amateur radio ("ham"), over-the-horizon radar, long-distance aviation communication.
    \item[Propagation:] Relies on refraction from the ionosphere, enabling global, intercontinental communication.
\end{spectrumband}

\begin{spectrumband}{VHF}{Very High Frequency: 30--300 MHz}
    \item[Wavelength:] 10--1 m.
    \item[Applications:] FM radio (88--108 MHz), television broadcast, air traffic control, marine communications.
    \item[Propagation:] Almost exclusively line-of-sight (LOS), though occasional atmospheric effects like tropospheric ducting can extend range.
\end{spectrumband}

\begin{spectrumband}{UHF}{Ultra High Frequency: 300 MHz -- 3 GHz}
    \item[Wavelength:] 1 m -- 10 cm.
    \item[Applications:] The most crowded part of the spectrum. Includes television, mobile phones (GSM, LTE, 5G), GPS, WiFi (2.4 GHz), Bluetooth, and microwave ovens.
    \item[Propagation:] Line-of-sight, with moderate ability to penetrate buildings. Rain attenuation is minimal.
\end{spectrumband}

\begin{spectrumband}{SHF}{Super High Frequency: 3--30 GHz}
    \item[Wavelength:] 10--1 cm.
    \item[Applications:] Satellite communications (C-band, Ku-band), radar systems, 5G mid-band, WiFi (5 and 6 GHz), point-to-point microwave links.
    \item[Propagation:] Strictly line-of-sight. Signals are significantly attenuated by rain (\keyterm{rain fade}) and atmospheric gases.
\end{spectrumband}

\begin{spectrumband}{EHF}{Extremely High Frequency: 30--300 GHz}
    \item[Wavelength:] 1 cm -- 1 mm. Also known as the \keyterm{millimetre wave (mmWave)} band.
    \item[Applications:] mmWave 5G, automotive radar (77 GHz), high-capacity wireless backhaul, radio astronomy.
    \item[Propagation:] Suffers from severe attenuation due to rain, foliage, and atmospheric absorption, especially from oxygen at 60 GHz.
\end{spectrumband}

\begin{calloutbox}{The 60 GHz Oxygen Absorption Anomaly}
    Around 60 GHz, oxygen molecules (O$_2$) in the atmosphere have a resonant absorption peak, causing extreme signal attenuation of approximately \qty{15}{dB/km}. While this severely limits range, it is intentionally exploited for security. Signals are contained within a room or short link, preventing eavesdropping. This is the basis for technologies like WiGig (IEEE 802.11ad).
\end{calloutbox}

\subsection{Optical Frequencies: \qty{300}{GHz} to \qty{30}{PHz}}

This region, spanning from the Terahertz gap to Ultraviolet, is governed by the principles of optics.

\begin{spectrumband}{THz}{Terahertz: 0.3--10 THz}
    \item[Wavelength:] 1 mm -- 30 $\mu$m.
    \item[Status:] Often called the "THz gap," as it was historically difficult to generate and detect these frequencies.
    \item[Applications:] Security imaging (can see through clothing but is non-ionizing), spectroscopy, biomedical sensing, and is a major research area for future 6G wireless systems.
\end{spectrumband}

\begin{spectrumband}{IR}{Infrared: 10--430 THz}
    \item[Wavelength:] 30 $\mu$m -- 700 nm.
    \item[Divisions:] Split into Far-IR (thermal radiation from objects at room temperature), Mid-IR, and Near-IR.
    \item[Applications:] Thermal imaging, fiber optic communications (which use the 1550 nm window in NIR), remote controls, and medical imaging.
\end{spectrumband}

\begin{spectrumband}{Visible}{Visible Light: 430--750 THz}
    \item[Wavelength:] 700 nm (red) to 400 nm (violet).
    \item[Properties:] This is the narrow band of frequencies to which the human eye is sensitive. It is non-ionizing.
    \item[Applications:] Human vision, optical communications (Li-Fi, free-space optics), photography, and photovoltaics.
\end{spectrumband}

\subsection{High-Energy Frequencies: >\qty{750}{THz}}

These frequencies carry enough energy per photon to be ionizing, posing a risk to biological tissue.

\begin{spectrumband}{UV}{Ultraviolet: 750 THz -- 30 PHz}
    \item[Wavelength:] 400 nm -- 10 nm.
    \item[Properties:] This band marks the transition to ionizing radiation. Earth's ozone layer blocks the most harmful components.
    \item[Applications:] Sterilization (UVC light is germicidal), fluorescence microscopy, and photolithography for semiconductor manufacturing.
\end{spectrumband}

\begin{spectrumband}{X-ray}{X-Rays: 30 PHz -- 30 EHz}
    \item[Wavelength:] 10 nm -- 0.01 nm.
    \item[Properties:] Highly ionizing radiation capable of penetrating soft tissue but blocked by denser materials like bone and metal.
    \item[Applications:] Medical imaging (radiography, CT scans), security screening, and X-ray crystallography to determine molecular structures.
\end{spectrumband}

\begin{spectrumband}{Gamma}{$\gamma$-Rays: >30 EHz}
    \item[Wavelength:] < 0.01 nm.
    \item[Properties:] The most energetic and penetrating form of electromagnetic radiation.
    \item[Sources:] Emitted during radioactive decay, nuclear reactions, and by extreme astrophysical objects like pulsars and black holes.
    \item[Applications:] Cancer therapy (radiotherapy), sterilization of medical equipment, and gamma-ray astronomy.
\end{spectrumband}

% A final summary table using booktabs for a professional look
\subsection{Spectrum Summary Table}
\begin{table}[H]
    \centering
    \caption{The Electromagnetic Spectrum at a Glance}
    \label{tab:spectrum-summary}
    \begin{tabularx}{\textwidth}{@{}llXXc@{}}
        \toprule
        \tableheaderfont Band & \tableheaderfont Frequency & \tableheaderfont Wavelength & \tableheaderfont Key Applications & \tableheaderfont Ionizing? \\
        \midrule
        Radio (RF) & 3 kHz -- 300 GHz & > 1 mm & Communications, Radar & No \\
        Infrared (IR) & 0.3 -- 430 THz & 1 mm -- 700 nm & Thermal, Fiber Optics & No \\
        Visible & 430 -- 750 THz & 700 -- 400 nm & Human Vision, Optics & No \\
        Ultraviolet (UV) & 750 THz -- 30 PHz & 400 -- 10 nm & Sterilization, Lithography & Transition \\
        X-ray & 30 PHz -- 30 EHz & 10 -- 0.01 nm & Medical Imaging & Yes \\
        Gamma Ray & > 30 EHz & < 0.01 nm & Radiotherapy, Astronomy & Yes \\
        \bottomrule
    \end{tabularx}
\end{table}

\begin{importantbox}[title={Further Reading}]
    A deep understanding of the electromagnetic spectrum is critical for system design. The following chapters build directly on these concepts:
    \begin{description}
        \item[Propagation Characteristics] (\Cref{ch:propagation}) provides a detailed analysis of how signals in different bands interact with the atmosphere, including absorption, reflection, and ducting.
        \item[Antenna Theory] (\Cref{ch:antenna}) explores the practical design of antennas that are resonant and efficient at specific frequencies, from massive VLF towers to tiny mmWave arrays.
        \item[Link Budget Analysis] (\Cref{ch:linkbudget}) provides the full framework for calculating system performance, combining frequency-dependent path loss with antenna gain, power, and receiver sensitivity.
    \end{description}
\end{importantbox}