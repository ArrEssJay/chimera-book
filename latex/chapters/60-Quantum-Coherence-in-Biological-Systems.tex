% ==============================================================================
% CHAPTER 59: Quantum Coherence in Biological Systems
% ==============================================================================
\chapter{Quantum Coherence in Biological Systems}
\label{ch:quantum-coherence-in-biological-systems}
\begin{nontechnical}
Imagine a coin spinning in the air. Until it lands, it is in a state that is neither heads nor tails, but a superposition of both possibilities at once. \keyterm{Quantum coherence} is the property that allows a particle to exist in such a superposition of multiple states simultaneously, with a well-defined phase relationship between them that allows these states to interfere with each other like waves.
code
Code
For decades, it was assumed that such delicate quantum effects could only survive in the ultra-cold, isolated environment of a laboratory. A living organism is warm, wet, and chaotic—seemingly the worst possible place for quantum coherence to exist. Yet, nature has surprised us. It is now established that plants use quantum coherence to transfer energy with near-perfect efficiency during photosynthesis. There is also strong evidence that migratory birds use a quantum compass in their eyes to navigate using the Earth's magnetic field.

This raises a profound question: if quantum effects are functional in plants and birds, could they also play a role in the human brain? This chapter explores the frontier of quantum biology. We will begin with the fundamental challenge—the "warm, wet problem" of decoherence—before examining the established science of quantum effects in photosynthesis and avian navigation. Finally, we will delve into the speculative but compelling hypothesis that quantum coherence in the brain's microtubules could be the physical basis of consciousness itself.
\end{nontechnical}
\subsection{Overview}
\keyterm{Quantum coherence} is the property of a quantum system where its constituent parts maintain a definite phase relationship, allowing the system to exhibit wave-like interference and non-classical correlations. The study of this phenomenon in biological systems challenges the long-held assumption that the thermal environment of a living cell is too disruptive for such delicate quantum states to survive long enough to perform a functional role.
While quantum effects in processes like photosynthesis are now well-established, the extension of these principles to explain higher-order biological functions, such as neural processing and consciousness, remains a highly speculative but active area of research. This chapter provides a survey of the field, from the fundamental principles and established examples to the speculative theories that motivate much of the research at the intersection of quantum physics and neuroscience.
\begin{keyconcept}
Mounting experimental evidence demonstrates that non-trivial quantum coherence can be maintained and exploited by biological systems at physiological temperatures. The central debate now focuses on the extent of these effects: are they confined to specific molecular processes, or do they scale up to play a significant role in macroscopic functions like cognition?
\end{keyconcept}
\subsection{The Challenge of Decoherence in Biology}
\paragraph{The Warm, Wet Problem}
The primary obstacle to any theory of biological quantum effects is \keyterm{decoherence}. A quantum system's coherence is destroyed through its interaction with the environment. In a biological cell at 310~K, a quantum state is subject to a constant barrage of thermal fluctuations and collisions with water molecules. Classical estimates of the decoherence timescale, (\tau_d), under these conditions are in the femtosecond-to-picosecond range, whereas most biological processes unfold on a millisecond timescale or slower. This apparent mismatch suggests that any quantum coherence would be destroyed long before it could influence a biological function.
\paragraph{Vibronic Coupling and Thermal Coherence}
Recent advances in theoretical quantum chemistry are providing a potential solution to this problem. It is now understood that \keyterm{vibronic coupling}—a strong interaction between a molecule's electronic states and its vibrational modes (phonons)—can create robust, hybrid quantum states. Theoretical frameworks such as the Vibronic Exciton-Thermal Fluctuation Correlation Canon (VE-TFCC) show that these vibronic states can maintain coherence even at thermal equilibrium. In essence, the quantum system becomes "dressed" by its interaction with specific, structured vibrations of its environment, creating protected subspaces that are shielded from the random thermal noise. The implication is that if biological molecules can exhibit strong vibronic coupling, they may be able to sustain quantum coherence despite the warm temperature.
\subsection{Established Examples of Functional Quantum Effects}
\paragraph{Photosynthetic Light Harvesting}
The first definitive evidence for functional quantum coherence in biology was discovered in the Fenna-Matthews-Olson (FMO) protein complex in green sulphur bacteria. Using two-dimensional electronic spectroscopy, scientists observed long-lived quantum "beating" at room temperature, indicating that absorbed light energy exists as a delocalised \keyterm{exciton} in a coherent superposition across multiple chromophore molecules. This quantum coherence allows the system to perform a "quantum walk," simultaneously sampling all possible energy transfer pathways to find the most efficient route to the photosynthetic reaction centre, achieving near-100% energy transfer efficiency. The observed coherence time of (\sim)600~fs is orders of magnitude longer than simple decoherence models would predict, a longevity attributed to the highly structured protein environment.
\paragraph{Avian Magnetoreception}
There is strong evidence that migratory birds navigate using a quantum compass based on the \keyterm{Radical Pair Mechanism}. A protein called cryptochrome in the bird's retina is excited by blue light, creating a pair of molecules with entangled electron spins. The Earth's weak magnetic field influences the evolution of this spin state, altering the final chemical products of the reaction. The bird's brain can then interpret the yield of these chemical products as a map of the magnetic field lines. For this mechanism to work, the quantum coherence of the electron spin entanglement must be preserved for several microseconds, a remarkably long time in a biological context.
\subsection{Speculative Extensions to Neural Systems}
\paragraph{Microtubules as a Quantum Substrate}
The most prominent—and controversial—theory of quantum effects in the brain is the \keyterm{Orchestrated Objective Reduction (Orch-OR)} hypothesis of Penrose and Hameroff. This theory proposes that neuronal microtubules are the site of quantum computations that give rise to consciousness. As discussed in \Cref{ch:thz-resonances-microtubules}, the hypothesis is that the quasi-crystalline structure of microtubules can support coherent THz-frequency phonons, and that strong vibronic coupling within tubulin proteins could protect these states from decoherence long enough for computation to occur.
\begin{warningbox}
The extension of quantum principles to explain consciousness is highly speculative and is not part of mainstream neuroscience. The primary objection remains the immense challenge of decoherence. While the theoretical possibility of thermal coherence exists, there is currently no direct experimental evidence for functionally relevant, long-lived quantum coherence in the neurons of a living brain.
\end{warningbox}
\paragraph{The Open Question of Timescale}
For quantum coherence to be relevant to consciousness, as proposed by Orch-OR, it would need to be sustained for tens of milliseconds (the timescale of neural gamma oscillations). This is many orders of magnitude longer than the coherence times observed in photosynthesis. Whether any biological structure, even one as highly ordered as a microtubule, can provide the necessary shielding to achieve this remains the central, unresolved question in the field. Advancing our understanding will require new experimental techniques capable of probing for quantum effects directly within a living neuron without destroying the state in the process of measuring it.
\begin{importantbox}
\section*{Further Reading}
\parhead{Foundational Concepts}
For a detailed analysis of the microtubule structure that is hypothesised to support these quantum effects, see \Cref{ch:microtubules}. A full description of the consciousness theory that relies on this phenomenon is provided in \Cref{ch:orch-or}. The specific vibrational modes that may play a role in protecting coherence are detailed in \Cref{ch:thz-resonances-microtubules}.
\parhead{Key Research Papers}
\begin{description}
\item[Engel, G. S. \textit{et al.} (2007) \textit{Nature}.] The seminal paper demonstrating long-lived quantum coherence in a photosynthetic complex at cryogenic temperatures.
\item[Collini, E. \textit{et al.} (2010) \textit{Nature}.] The follow-up study that confirmed the existence of photosynthetic quantum coherence at physiological, room-temperature conditions.
\item[Hore, P. J. & Mouritsen, H. (2016) \textit{Annu. Rev. Biophys.}] A comprehensive review of the radical pair mechanism and the strong experimental evidence supporting its role in avian magnetoreception.
\item[Bao, J. \textit{et al.} (2024) \textit{J. Chem. Theory Comput.}] Details modern theoretical work suggesting strong vibronic coupling can maintain thermal coherent states at physiological temperatures, providing a possible solution to the "warm, wet problem".
\end{description}
\parhead{Critical Assessments and Sceptical Viewpoints}
\begin{description}
\item[Tegmark, M. (2000) \textit{Phys. Rev. E}.] A highly cited critique that calculates the decoherence time for quantum states in microtubules to be on the order of femtoseconds, arguing that the neural environment is far too noisy for functional quantum computation.
\item[Koch, C. & Hepp, K. (2006) \textit{Nature}.] A broader critique of quantum consciousness theories from the perspective of mainstream neuroscience, outlining the significant biological and physical hurdles such theories must overcome.
\end{description}
\end{importantbox}