% ==============================================================================
% CHAPTER 22: Channel Models (Rayleigh and Rician Fading)
% ==============================================================================

\chapter{Channel Models: Rayleigh \& Rician Fading}
\label{ch:fading}

\begin{nontechnical}
    \textbf{Channel models are like flight simulators for radio engineers.} They are sophisticated computer simulations that replicate real-world radio environments, allowing engineers to test a communication system's performance in a virtual city, on a highway, or indoors, long before any hardware is built.

    \parhead{The problem with the real world} Transmitted radio signals don't just travel in a straight line. They bounce off buildings, get absorbed by trees, and scatter in all directions. This effect, called \textbf{multipath}, causes the received signal strength to fluctuate wildly over very short distances.

    \parhead{The two main "flight simulators"}
    \begin{itemize}
        \item \textbf{Rayleigh Fading Model (No Line-of-Sight):} This simulates a dense urban environment where there is no direct path between the transmitter and receiver. The signal arrives as a chaotic jumble of reflected copies. This causes severe, rapid signal fluctuations ("deep fades") and is the worst-case scenario for mobile communications.
        \item \textbf{Rician Fading Model (Line-of-Sight + Reflections):} This simulates a more open environment, like a suburban street or a rural area, where there is one strong, direct path plus several weaker reflections. The signal is much more stable than in a Rayleigh channel, with less severe fading.
    \end{itemize}

    \parhead{Why it matters} When engineers designed the 4G LTE standard, they used these models to run millions of simulations. This ensured that your phone would work reliably whether you were in a dense city (a Rayleigh environment) or on an open highway (a Rician environment).
\end{nontechnical}


\section{Overview and Properties}

\subsection{Overview}

While the AWGN channel provides a crucial theoretical baseline, real-world wireless channels are far more complex. Signals propagate through multiple paths due to reflection, diffraction, and scattering, an effect known as \keyterm{multipath}. The constructive and destructive interference of these multiple paths at the receiver causes rapid fluctuations in the received signal's amplitude and phase, a phenomenon called \keyterm{fading}.

\keyterm{Channel models} are statistical representations of these fading effects. By simulating the channel, engineers can predict system performance, design robust modulation and coding schemes, and develop the necessary countermeasures like equalization and diversity.

\begin{keyconcept}
    The two most important statistical models for fading are the \textbf{Rayleigh} and \textbf{Rician} distributions. Rayleigh fading models the severe signal fluctuations in non-line-of-sight (NLOS) environments, while Rician fading models the more stable conditions of a channel with a dominant line-of-sight (LOS) path.
\end{keyconcept}


\subsection{The Fading Channel Model}

For a \keyterm{flat-fading} channel (where the signal bandwidth is much smaller than the channel's coherence bandwidth), the received signal is modelled as the transmitted signal multiplied by a time-varying complex channel gain, $h(t)$:
\begin{equation}
    r(t) = h(t) \cdot s(t) + n(t)
\end{equation}
The statistical properties of the envelope, $|h(t)|$, determine the type of fading.


\subsection{Rayleigh Fading}

\parhead{Physical Model}
Rayleigh fading occurs in rich scattering, \keyterm{non-line-of-sight (NLOS)} environments. The received signal is the sum of a large number of independent, randomly phased reflected and scattered paths, with no single path being dominant. By the Central Limit Theorem, the resulting complex gain $h(t)$ is a zero-mean complex Gaussian random process.

\parhead{Statistical Properties}
The envelope of the signal, $r = |h(t)|$, follows the Rayleigh probability density function:
\begin{equation}
    p(r) = \frac{r}{\sigma^2} \exp\left(-\frac{r^2}{2\sigma^2}\right), \quad r \geq 0
\end{equation}
A key characteristic of the Rayleigh distribution is that the probability of the signal dropping into a "deep fade" (near zero amplitude) is high, making it a very challenging channel.

\parhead{Doppler Spectrum}
The movement of the receiver or objects in the environment causes a frequency shift in each multipath component, known as the Doppler effect. The range of these shifts is called the \keyterm{Doppler spread}, and its maximum value is $f_d = v/\lambda$. The inverse of the Doppler spread is the \keyterm{coherence time}, $T_c$, which is the time duration over which the channel is approximately constant.


\subsection{Rician Fading}

\parhead{Physical Model}
Rician fading occurs when there is a \keyterm{dominant line-of-sight (LOS)} path in addition to weaker scattered paths. This is typical of rural, suburban, or indoor open-plan environments.

\parhead{Statistical Properties}
The presence of the stable LOS component means the channel gain, $h(t)$, is a complex Gaussian random process with a non-zero mean. The envelope follows the Rician distribution. The character of the Rician channel is defined by the \keyterm{K-factor}:
\begin{equation}
    K = \frac{\text{Power in the LOS component}}{\text{Power in the scattered components}}
\end{equation}
\begin{itemize}
    \item As $K \to 0$ ($-\infty$ dB), the LOS path disappears, and the channel becomes Rayleigh.
    \item As $K \to \infty$, the scattered paths become negligible, and the channel approaches a simple, non-fading AWGN channel.
\end{itemize}
Typical values for K in wireless links range from 6 to 15 dB. A higher K-factor means less severe fading and better link performance.


\subsection{Performance Impact}

\begin{warningbox}
    \textbf{The Fading Penalty:} Fading has a devastating impact on performance. To achieve a target BER of $10^{-5}$ with QPSK, a Rayleigh fading channel can require \textbf{20-30 dB more average SNR} than a simple AWGN channel. Overcoming this "fading margin" is the primary driver for a vast array of techniques in modern wireless design, including error correction coding, interleaving, and diversity.
\end{warningbox}


\begin{workedexample}{4G LTE Urban Channel Simulation}
    \parhead{Problem} Assess the performance of an uncoded QPSK signal in a simulated urban 4G LTE channel.
    \parhead{System Parameters}
    \begin{itemize}
        \item Carrier Frequency: \qty{2.6}{GHz} (LTE Band 7)
        \item Mobile Velocity: \qty{30}{km/h} (8.3 m/s)
        \item Channel Model: Rayleigh fading (representing a dense urban NLOS environment).
        \item Required BER for uncoded QPSK: $10^{-3}$, which needs an AWGN SNR of approx. 7 dB.
    \end{itemize}
    \parhead{Analysis}
    \begin{derivationsteps}
        \step Calculate the maximum Doppler spread.
        \[ f_d = \frac{v}{\lambda} = \frac{v f_c}{c} = \frac{8.3 \times (2.6 \times 10^9)}{3 \times 10^8} \approx \qty{72}{Hz} \]
        \step Calculate the channel's coherence time.
        \[ T_c \approx \frac{1}{2f_d} \approx \frac{1}{2 \times 72} \approx \qty{6.9}{ms} \]
        This is a "slow-fading" channel for LTE, as the channel state is constant over several OFDM symbols.
        \step Estimate the required average SNR in the fading channel. To achieve the same BER of $10^{-3}$, a Rayleigh channel requires approximately 18 dB more SNR than an AWGN channel.
        \[ \text{Required SNR}_{\text{Rayleigh}} \approx \text{Required SNR}_{\text{AWGN}} + \text{Fading Margin} = 7 \text{ dB} + 18 \text{ dB} = \qty{25}{dB} \]
    \end{derivationsteps}
    \parhead{Interpretation} An enormous 25 dB average SNR is required to achieve even marginal performance. This is why uncoded transmission is never used in mobile systems. LTE employs a combination of powerful \textbf{Turbo coding}, \textbf{OFDM}, and \textbf{MIMO diversity} to overcome this massive fading penalty, allowing it to operate reliably at much lower average SNRs (typically 5-15 dB).
\end{workedexample}


\begin{importantbox}[title={Further Reading}]
    Fading is the central challenge of mobile wireless communications. Understanding these models is key to the following topics.
    \begin{description}
        \item[Diversity Techniques] (\Cref{ch:diversity}) describes the fundamental methods (space, time, frequency) used to combat the destructive effects of deep fades.
        \item[OFDM] (\Cref{ch:ofdm}) explains how a frequency-selective fading channel can be cleverly transformed into a set of parallel, flat-fading sub-channels, which are much easier to manage.
        \item[Channel Equalization] (\Cref{ch:equalization}) covers the DSP algorithms used to estimate and invert the effects of the channel gain, $h(t)$, to recover the original signal.
    \end{description}
\end{importantbox}
