% ==============================================================================
% CHAPTER 6: IQ Representation
% ==============================================================================

\chapter{IQ Representation}
\label{ch:iq}

\begin{nontechnical}
    \textbf{IQ representation is like describing a location on a map using X and Y coordinates}. It provides a two-dimensional way to pinpoint the exact state of a radio signal at any given moment.

    \parhead{The map analogy}
    \begin{itemize}
        \item The \textbf{I (In-phase)} component is the East-West (X) coordinate.
        \item The \textbf{Q (Quadrature)} component is the North-South (Y) coordinate.
        \item Together, the (I, Q) pair gives a unique position, which represents both the signal's strength (\textbf{amplitude}) and its timing (\textbf{phase}).
    \end{itemize}

    \parhead{Why two dimensions?} A simple radio signal can only vary its amplitude (getting stronger or weaker). By adding a second, perpendicular dimension (the Q channel), we can vary amplitude and phase simultaneously. This allows us to encode significantly more information onto the same radio wave, which is the foundation of all modern high-speed data transmission like 4G, 5G, and WiFi.

    \parhead{Real-world use} Every smartphone, WiFi chip, and Software-Defined Radio (SDR) processes signals internally using IQ data. It is the universal language of digital communications.
\end{nontechnical}


\section{Overview and Properties}

\subsection{Overview}

\keyterm{IQ Representation}, also known as \keyterm{complex baseband representation}, is the fundamental mathematical framework for describing and processing modulated signals in modern communications. It allows any bandpass signal to be represented as two orthogonal baseband components:
\begin{itemize}
    \item \textbf{I (In-phase):} The component that modulates a cosine carrier, $\cos(2\pi f_c t)$.
    \item \textbf{Q (Quadrature):} The component that modulates a sine carrier, $\sin(2\pi f_c t)$, which is phase-shifted by 90$^\circ$ from the cosine.
\end{itemize}
This decomposition simplifies analysis and hardware/software implementation by moving all complex processing from the high-frequency passband domain to the low-frequency baseband domain.

\begin{keyconcept}
    By representing a signal with two orthogonal components, IQ representation allows us to control both the \textbf{amplitude} and \textbf{phase} of a carrier wave independently. This two-dimensional control is the key to all modern high-order modulation schemes like QPSK and QAM.
\end{keyconcept}


\subsection{Mathematical Description}

A general passband signal can be expressed in terms of its time-varying amplitude $A(t)$ and phase $\phi(t)$:
\begin{equation}
    s(t) = A(t) \cos[2\pi f_c t + \phi(t)]
\end{equation}
Using trigonometric identities, this can be expanded into the canonical IQ form:
\begin{equation}
    s(t) = \underbrace{A(t)\cos[\phi(t)]}_{I(t)} \cos(2\pi f_c t) - \underbrace{A(t)\sin[\phi(t)]}_{Q(t)} \sin(2\pi f_c t)
\end{equation}
This gives us the definition of the baseband \keyterm{in-phase} and \keyterm{quadrature} components:
\begin{align}
    I(t) &= A(t)\cos[\phi(t)] \\
    Q(t) &= A(t)\sin[\phi(t)]
\end{align}
The passband signal is the result of multiplying these two baseband signals with orthogonal carriers and summing them.

\paragraph{Complex Baseband Representation}
The IQ components are most powerfully represented as a single complex number, known as the \keyterm{complex envelope} or complex baseband signal:
\begin{equation}
    s_{\text{BB}}(t) = I(t) + jQ(t)
\end{equation}
The original passband signal can be perfectly reconstructed from this complex representation:
\begin{equation}
    s(t) = \Re\{s_{\text{BB}}(t) \cdot e^{j2\pi f_c t}\}
\end{equation}
Conversely, the amplitude and phase can be recovered from the IQ components:
\begin{align}
    A(t) &= |s_{\text{BB}}(t)| = \sqrt{I^2(t) + Q^2(t)} \\
    \phi(t) &= \arg[s_{\text{BB}}(t)] = \arctan\left(\frac{Q(t)}{I(t)}\right)
\end{align}

\subsection{Modulation and Demodulation}

\paragraph{IQ Modulator (Transmitter)}
An IQ modulator translates the complex baseband signal to a real passband signal.
\begin{description}
    \item[1. Symbol Mapping] Digital bits are mapped to discrete complex values ($I+jQ$) that represent points on a constellation diagram.
    \item[2. Pulse Shaping & DAC] The discrete symbols are converted into smooth analogue waveforms, $I(t)$ and $Q(t)$, by digital-to-analogue converters (DACs).
    \item[3. Upconversion] The $I(t)$ waveform is mixed with a cosine carrier, and the $Q(t)$ waveform is mixed with a sine carrier.
    \item[4. Combination] The two resulting signals are summed to produce the final RF passband signal.
\end{description}

\paragraph{IQ Demodulator (Receiver)}
An IQ demodulator performs the reverse operation, recovering the complex baseband signal from the passband signal.
\begin{description}
    \item[1. Downconversion] The incoming RF signal is split and mixed with both cosine and sine waves from a local oscillator.
    \item[2. Low-Pass Filtering] This removes the high-frequency mixing products, leaving only the desired baseband $I(t)$ and $Q(t)$ signals.
    \item[3. ADC & DSP] The analogue $I(t)$ and $Q(t)$ are sampled by analogue-to-digital converters (ADCs), and the resulting digital IQ stream is then passed to the DSP for demodulation, error correction, and decoding.
\end{description}

\begin{warningbox}
    \textbf{Phase coherence is essential.} The receiver's local oscillator must be perfectly synchronized in both frequency and phase with the transmitter's carrier. Any phase error will cause the I and Q components to "leak" into each other, a phenomenon known as \keyterm{IQ cross-talk}, which severely degrades performance.
\end{warningbox}

\subsection{Constellation Diagrams}

A \keyterm{constellation diagram} is the fundamental tool for visualizing a digital modulation scheme. It is a scatter plot of all possible transmitted symbols on the 2D IQ plane.

\begin{center}
    \begin{tikzpicture}[scale=1.5]
        % Axes
        \draw[->] (-2.5,0) -- (2.5,0) node[right] {\sffamily\small $I$ (Real)};
        \draw[->] (0,-2.5) -- (0,2.5) node[above] {\sffamily\small $Q$ (Imaginary)};

        % Grid
        \draw[very thin, diagramgray!50] (-2,-2) grid[step=0.5] (2,2);

        % QPSK Constellation
        \node[circle, fill=diagramprimary, inner sep=2pt, label={[font=\sffamily\small]above right:01}] at (1.414,1.414) {};
        \node[circle, fill=diagramprimary, inner sep=2pt, label={[font=\sffamily\small]above left:00}] at (-1.414,1.414) {};
        \node[circle, fill=diagramprimary, inner sep=2pt, label={[font=\sffamily\small]below left:10}] at (-1.414,-1.414) {};
        \node[circle, fill=diagramprimary, inner sep=2pt, label={[font=\sffamily\small]below right:11}] at (1.414,-1.414) {};
        
        \node[above,font=\sffamily\bfseries] at (0,2.5) {QPSK Constellation Diagram};
    \end{tikzpicture}
\end{center}

In the presence of noise, received samples do not fall perfectly on the constellation points but form a "cloud" around each one. A symbol error occurs if the noise is large enough to push a sample into the decision region of a different symbol.

\begin{keyconcept}
    The \textbf{Euclidean distance} between points on a constellation diagram is a direct measure of a modulation scheme's robustness to noise. Schemes with greater separation between points will have a lower bit error rate (BER) for a given signal-to-noise ratio (SNR).
\end{keyconcept}

\begin{workedexample}{16-QAM System Design}
    \parhead{Problem} Determine the key parameters for a 16-QAM system designed to transmit \qty{10}{Mbps}.
    \parhead{Analysis}
    \begin{itemize}
        \item \textbf{Bits per Symbol:} 16-QAM has 16 points, so it encodes $\log_2(16) = 4$ bits per symbol.
        \item \textbf{Symbol Rate:} To achieve a bit rate of \qty{10}{Mbps}, the required symbol rate is $R_s = (\qty{10}{Mbps}) / 4 = 2.5$ Msymbols/s.
        \item \textbf{Bandwidth:} Using a typical raised-cosine filter with a roll-off factor of $\alpha = 0.35$, the occupied bandwidth is $B = R_s(1 + \alpha) = 2.5 \times 1.35 = \qty{3.375}{MHz}$.
        \item \textbf{Spectral Efficiency:} The efficiency is $\eta = R_b / B = 10 / 3.375 \approx 2.96$ bps/Hz.
        \item \textbf{SNR Requirement:} To achieve a BER of $10^{-5}$, 16-QAM typically requires an $E_b/N_0$ of approximately 12 dB.
    \end{itemize}
    \parhead{Conclusion} 16-QAM offers high spectral efficiency, but at the cost of a higher required SNR compared to simpler schemes like QPSK. It is best suited for high-quality, high-SNR channels.
\end{workedexample}

\begin{importantbox}[title={Further Reading}]
    IQ representation is the language used to describe nearly all other digital modulation concepts.
    \begin{description}
        \item[Modulation Schemes] (\Cref{ch:bpsk}, \Cref{ch:qpsk}, \Cref{ch:qam}) are all defined by how they map bits to specific points on the IQ plane.
        \item[Constellation Diagrams] (\Cref{ch:constellations}) explores the visualization of these schemes and the impact of noise and other impairments.
        \item[Synchronization] (\Cref{ch:sync}) covers the critical DSP algorithms for estimating and correcting carrier frequency and phase offsets in the received IQ signal.
        \item[Software-Defined Radio] (\Cref{ch:sdr}) provides a practical look at how modern radios are built almost entirely around the processing of IQ data in software.
    \end{description}
\end{importantbox}
