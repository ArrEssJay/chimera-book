% ==============================================================================
% CHAPTER 4: Power Density & Field Strength
% ==============================================================================

\chapter{Power Density \& Field Strength}
\label{ch:power-density}

\begin{nontechnical}
    \textbf{Power density is like measuring sunlight intensity}---how much energy hits each square meter. \textbf{Field strength} is like measuring the "force" of the electromagnetic wave at a point.

    \parhead{Simple analogy} Imagine standing near a campfire:
    \begin{itemize}
        \item \textbf{Field strength} is how hot the air feels on your skin.
        \item \textbf{Power density} is the total heat energy hitting your body per square meter.
    \end{itemize}

    \parhead{Real-world comparison}
    \begin{itemize}
        \item \textbf{Sunlight:} \qty{1000}{W/m^2} (strong enough to power solar panels).
        \item \textbf{WiFi router at \qty{1}{m}:} \qty{0.00001}{W/m^2} (100 million times weaker).
        \item \textbf{Cell tower at \qty{100}{m}:} \qty{0.0001}{W/m^2} (10 million times weaker).
    \end{itemize}

    \parhead{Key insight} Double your distance from a source and you get \emph{one quarter} of the power density. This inverse square law is why WiFi works well at \qty{50}{m} but not at \qty{200}{m}, and why satellites \qty{36,000}{km} away need enormous transmit power.
\end{nontechnical}

\subsection{Overview}

\keyterm{Power density} and \keyterm{field strength} are the fundamental parameters used to quantify the intensity of electromagnetic radiation at a given point in space. Their accurate calculation is essential for link budget analysis, antenna design, radar systems, and RF safety compliance.

\begin{keyconcept}
    In the far field of a radiating source, power density ($S$) is directly proportional to the square of the electric field strength ($E$). Their relationship is governed by the impedance of free space, $\eta_0 \approx 377\,\Omega$:
    \[
        S = \frac{E_{\text{rms}}^2}{\eta_0} \quad (\text{W/m}^2)
    \]
    Power density follows the \textbf{inverse square law}, decaying as $1/r^2$ with distance from the source.
\end{keyconcept}

\subsection{Field Strength Definitions}

\paragraph{Electric Field Strength (E)}
The \keyterm{electric field}, $\vec{E}$, is a vector field that describes the force exerted on a stationary positive test charge, $q$, at any point in space. It is defined as the force per unit charge:
\begin{equation}
    \vec{E} = \frac{\vec{F}}{q} \quad (\text{V/m})
\end{equation}
For a sinusoidal plane wave, the instantaneous field is $E(z,t) = E_0 \cos(\omega t - kz + \phi)$, where $E_0$ is the peak amplitude. In RF engineering, we almost always use the \keyterm{Root Mean Square (RMS)} value for consistency with power calculations:
\begin{equation}
    E_{\text{rms}} = \frac{E_0}{\sqrt{2}}
\end{equation}

\paragraph{Magnetic Field Strength (H)}
The \keyterm{magnetic field strength}, $\vec{H}$, describes the "magnetising force" within a magnetic field, distinct from the magnetic flux density $\vec{B}$. In the far field, the E-field and H-field are related by the impedance of free space:
\begin{equation}
    \frac{E}{H} = \eta_0 = \sqrt{\frac{\mu_0}{\epsilon_0}} \approx 377~\Omega
\end{equation}

\subsection{Near Field vs. Far Field}

The space around an antenna is divided into two primary regions.
\begin{description}
    \item[Near Field] The region close to the antenna where the fields are complex and reactive. Here, $E/H \neq 377\,\Omega$, and the field strength decays rapidly (often as $1/r^2$ or $1/r^3$). This region is primarily of interest for antenna design and certain specialised applications like RFID.
    \item[Far Field] The region far from the antenna, beginning at the Fraunhofer distance, $r > 2D^2/\lambda$ (where $D$ is the antenna's largest dimension). Here, the wave behaves as a simple plane wave, $E/H = 377\,\Omega$, and power density follows the inverse square law. \textbf{All link budget and communication calculations are performed in the far field.}
\end{description}

\begin{workedexample}{WiFi Far Field Boundary}
    \parhead{Given} A WiFi antenna operating at \qty{2.4}{GHz} ($\lambda \approx \qty{12.5}{cm}$) with a largest dimension of $D = \qty{5}{cm}$.
    \parhead{Calculate} The Fraunhofer distance, which marks the beginning of the far field.
    \parhead{Solution}
    \[
        r_{\text{far}} = \frac{2D^2}{\lambda} = \frac{2 \times (0.05)^2}{0.125} = \frac{0.005}{0.125} = 0.04 \text{ m} = \qty{4}{cm}
    \]
    \parhead{Interpretation} The far field for a typical WiFi antenna begins only a few centimetres away. This confirms that nearly all practical WiFi usage occurs in the far field, validating the use of far-field equations for analysis.
\end{workedexample}

\subsection{Power Density and the Poynting Vector}

The \keyterm{Poynting vector}, $\vec{S}$, describes the magnitude and direction of energy flow in an electromagnetic field, measured in watts per square meter (W/m$^2$). It is defined by the cross product:
\begin{equation}
    \vec{S} = \vec{E} \times \vec{H}
\end{equation}
In the far field, where $\vec{E}$ and $\vec{H}$ are perpendicular, the time-averaged magnitude of the Poynting vector, which we call the power density, is:
\begin{equation}
    S = \langle |\vec{S}| \rangle = \frac{E_{\text{rms}}^2}{\eta_0} = \frac{E_{\text{rms}}^2}{377}
\end{equation}

\subsection{The Inverse Square Law}

For an \keyterm{isotropic radiator} (an idealised antenna that radiates power equally in all directions), the power $P_t$ is spread uniformly over the surface of a sphere of radius $r$. The power density at that distance is therefore:
\begin{equation}
    S = \frac{P_t}{4\pi r^2}
\end{equation}
This is the fundamental \keyterm{inverse square law}: power density is inversely proportional to the square of the distance. A directional antenna with a linear gain of $G$ focuses this power, so the power density in the main beam is:
\begin{equation}
    S = \frac{P_t G}{4\pi r^2} = \frac{\text{EIRP}}{4\pi r^2}
\end{equation}
where \keyterm{EIRP} (Effective Isotropic Radiated Power) is simply $P_t \times G$.

\begin{warningbox}
    In all link budget equations, antenna gain ($G$) must be expressed in \textbf{linear units}, not decibels. To convert from dBi to a linear value, use the formula: $G_{\text{linear}} = 10^{(G_{\text{dBi}}/10)}$.
\end{warningbox}

\begin{workedexample}{Satellite Downlink Analysis}
    \parhead{Problem} Calculate the power density and E-field strength on the ground from a geostationary satellite.
    \parhead{Given}
    \begin{itemize}
        \item Satellite EIRP: \qty{50}{dBW}
        \item Distance to Earth: $r = \qty{36,000}{km}$
    \end{itemize}
    \parhead{Solution}
    \begin{derivationsteps}
        \step Convert EIRP from logarithmic to linear units.
        \[ \text{EIRP}_{\text{linear}} = 10^{(50/10)} = 10^5 \text{ W} = \qty{100}{kW} \]
        \step Calculate the power density on the ground using the inverse square law.
        \[ S = \frac{\text{EIRP}}{4\pi r^2} = \frac{10^5}{4\pi (3.6 \times 10^7)^2} \approx 6.1 \times 10^{-12} \text{ W/m}^2 \]
        \step Convert power density to RMS electric field strength.
        \[ E_{\text{rms}} = \sqrt{S \times 377} = \sqrt{(6.1 \times 10^{-12}) \times 377} \approx 1.5 \times 10^{-6} \text{ V/m} = \qty{1.5}{\mu V/m} \]
    \end{derivationsteps}
    \parhead{Interpretation} The signal from a GEO satellite is incredibly weak by the time it reaches the ground. The power density is on the order of picowatts per square meter, and the field strength is in the microvolts-per-meter range. This necessitates the use of large, high-gain receiving dishes and very sensitive low-noise amplifiers.
\end{workedexample}

\subsection{RF Safety Standards}

To protect the public from potential thermal effects of RF radiation, regulatory bodies like the FCC (in the US) and ICNIRP (internationally) set strict limits on power density and field strength.

\begin{table}[H]
    \centering
    \caption{ICNIRP 2020 General Public Exposure Limits (Time-Averaged)}
    \label{tab:icnirp-limits}
    \begin{tabular}{@{}ll@{}}
        \toprule
        \tableheaderfont Frequency Range & \tableheaderfont Power Density Limit (W/m$^2$) \\
        \midrule
        10 -- 400 MHz & 2 \\
        400 -- 2000 MHz & $f/200$ (where $f$ is in MHz) \\
        2 -- 300 GHz & 10 \\
        \bottomrule
    \end{tabular}
\end{table}

\begin{importantbox}[title={Safety in Context}]
    At WiFi frequencies (\qty{2.4}{GHz}), the ICNIRP limit is \qty{10}{W/m^2}. As calculated in the `nontechnical` box, the power density even a few meters from a typical router is in the microwatts per square meter range---many thousands of times below the safety limit. The inverse square law is the primary reason that RF exposure from consumer devices drops to negligible levels very quickly with distance.
\end{importantbox}


\begin{importantbox}[title={Further Reading}]
    Power density and field strength are the foundational metrics for designing and analysing any wireless system.
    \begin{description}
        \item[Free-Space Path Loss] (\Cref{ch:fspl}) formalises the inverse square law into the core equation used in all link budget calculations.
        \item[Antenna Theory] (\Cref{ch:antenna}) provides the tools to understand and calculate antenna gain ($G$), a critical multiplier for power density in a desired direction.
        \item[Link Budget Analysis] (\Cref{ch:linkbudget}) brings all these concepts together---transmit power, antenna gains, and path loss---to predict the final received signal power.
    \end{description}
\end{importantbox}