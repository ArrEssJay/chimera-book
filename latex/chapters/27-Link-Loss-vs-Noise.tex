% ==============================================================================
% CHAPTER 27: Link Loss vs. Noise
% ==============================================================================

\chapter{Link Loss vs. Noise}
\label{ch:loss-vs-noise}

\begin{nontechnical}
    \textbf{Link loss and noise are the two fundamental enemies of any wireless signal.} Understanding the difference is like knowing the difference between someone whispering from far away versus someone shouting in a noisy room.

    \parhead{Link Loss (The Signal Gets Weaker)}
    \begin{itemize}
        \item \textbf{What it is:} The predictable weakening of a signal as it travels over a distance.
        \item \textbf{Analogy:} Shouting across a football field. Your voice isn't being interfered with, it's just spreading out and getting quieter with distance.
        \item \textbf{The Fix:} Increase the initial power (shout louder) or use a megaphone (a high-gain antenna).
    \end{itemize}

    \parhead{Noise (Interference is Added)}
    \begin{itemize}
        \item \textbf{What it is:} Random, unavoidable electrical static that gets added to the signal at the receiver.
        \item \textbf{Analogy:} Trying to have a conversation in a crowded, noisy restaurant. Even if the person is speaking loudly, the background chatter can make them impossible to understand.
        \item \textbf{The Fix:} You can't get rid of the background noise. Your only options are to speak much louder than the noise, or to use error correction (like asking "did you say X or Y?").
    \end{itemize}

    \parhead{The bottom line: It's all about the ratio} The final quality of a connection depends on the \textbf{Signal-to-Noise Ratio (SNR)}. Link loss reduces the "S", while noise increases the "N". A good connection requires the received signal to be significantly stronger than the receiver's internal noise floor.
\end{nontechnical}


\section{Overview and Properties}

\subsection{Overview}

In any communication system, the signal that arrives at the receiver is a degraded version of the one that was transmitted. This degradation is caused by two fundamentally different mechanisms: \keyterm{link loss} and \keyterm{additive noise}.

\begin{keyconcept}
    \textbf{Link loss is a deterministic, multiplicative effect} that attenuates the signal's power. \textbf{Noise is a random, additive effect} that corrupts the signal by adding unwanted energy. Both degrade performance, but they are independent phenomena. A link can have very low loss but be unusable due to high noise, or be noise-free but unusable due to extreme loss.
\end{keyconcept}


\subsection{The System Model}

The journey of a signal through a wireless channel can be modelled as a two-step process: first, the signal is attenuated by the link loss, and second, noise is added.
\begin{equation}
    r(t) = \frac{s(t)}{\sqrt{L}} + n(t)
\end{equation}
where $s(t)$ is the transmitted signal, $L$ is the linear link loss factor, and $n(t)$ is the additive noise. The received signal power is thus $P_r = P_t / L$, and the final Signal-to-Noise Ratio is SNR $= P_r / P_n$.


\subsection{Link Loss}

\parhead{Definition}
Link loss (or path loss) is the sum of all deterministic power reductions in a communication link. It is a predictable quantity that can be calculated in a \keyterm{link budget}.

\parhead{Components}
The total link loss is the sum of several factors, expressed in decibels:
\begin{itemize}
    \item \textbf{Free-Space Path Loss (FSPL):} The largest component, caused by the geometric spreading of the wave. It is proportional to the square of the distance and the square of the frequency.
    \item \textbf{Atmospheric and Weather Losses:} Attenuation from atmospheric gases, rain, and fog, which are significant above 10 GHz.
    \item \textbf{Other Losses:} Includes losses in cables, connectors, and from antenna misalignment.
\end{itemize}
The antenna gains ($G_t$ and $G_r$) are positive terms in the link budget that help to overcome these losses.

\parhead{Mitigation}
Since link loss is deterministic, it is countered by increasing the power budget of the system:
\begin{itemize}
    \item Increasing transmit power ($P_t$).
    \item Using higher-gain antennas ($G_t, G_r$).
    \item Reducing the distance or using a lower frequency (to decrease FSPL).
\end{itemize}


\subsection{Noise}

\parhead{Definition}
Noise is any random, unwanted energy that is added to the signal, primarily at the receiver front-end. It is fundamentally unpredictable and is described by its statistical properties.

\parhead{Components}
\begin{itemize}
    \item \textbf{Thermal Noise:} The primary, unavoidable source of noise, caused by the thermal motion of electrons in electronic components. Its power is given by $P_n = kTB$.
    \item \textbf{Receiver Noise Figure (NF):} A measure of the additional noise generated internally by the receiver's own amplifiers and mixers.
    \item \textbf{External Noise:} Interference from other man-made transmitters or natural sources like lightning, which can raise the effective noise floor.
\end{itemize}

\parhead{Mitigation}
Noise power cannot be reduced by simple amplification (which amplifies the signal and noise together). Instead, it is mitigated by:
\begin{itemize}
    \item \textbf{Reducing Bandwidth ($B$):} The narrowest possible filter is used to let the signal through while blocking out-of-band noise.
    \item \textbf{Low-Noise Design:} Using a high-quality Low-Noise Amplifier (LNA) with a low noise figure as the first component in the receiver chain.
    \item \textbf{Error Correction Coding (FEC):} Adds redundancy to the data to allow the receiver to detect and correct the bit errors caused by noise.
\end{itemize}


\begin{table}[H]
    \centering
    \caption{Comparison of Link Loss and Noise}
    \label{tab:loss-vs-noise}
    \begin{tabular}{@{}lll@{}}
        \toprule
        \tableheaderfont Characteristic & \tableheaderfont Link Loss & \tableheaderfont Noise \\
        \midrule
        Nature & Deterministic & Random (Stochastic) \\
        Effect on Signal & Multiplicative (attenuates) & Additive (corrupts) \\
        Source & Geometry, Environment & Physics (Thermal), Electronics \\
        Primary Mitigation & More Power / Higher Gain & Better Receiver / Coding \\
        Link Budget Term & A large negative dB value & The "Noise Floor" (e.g., -114 dBm) \\
        \bottomrule
    \end{tabular}
\end{table}


\begin{workedexample}{WiFi Link Analysis}
    \parhead{Problem} Analyse a 100-meter WiFi link to determine the final SNR, showing the distinct contributions of link loss and noise.
    \parhead{System Parameters}
    \begin{itemize}
        \item Transmit Power ($P_t$): \qty{20}{dBm} (100 mW).
        \item Antenna Gains ($G_t, G_r$): 2 dBi each.
        \item Frequency: \qty{2.4}{GHz}.
        \item Bandwidth ($B$): \qty{20}{MHz}.
        \item Receiver Noise Figure (NF): \qty{7}{dB}.
    \end{itemize}
    \parhead{Solution}
    \begin{derivationsteps}
        \step \textbf{Calculate the Link Loss.} The primary component is FSPL.
        \[ \text{FSPL} = 20\log_{10}(100 \text{ m}) + 20\log_{10}(2.4 \text{ GHz}) + 20.4 = 40 + 7.6 + 20.4 = \qty{68.0}{dB} \]
        The received power is the transmitted power minus the loss, plus the antenna gains.
        \[ P_r = P_t + G_t + G_r - \text{FSPL} = 20 + 2 + 2 - 68.0 = \qty{-44.0}{dBm} \]
        The total link loss has reduced the signal power from +20 dBm to -44.0 dBm.
        
        \step \textbf{Calculate the Noise Floor.} This depends only on the receiver's bandwidth and noise figure.
        \[ P_n = -174 \text{ dBm/Hz} + 10\log_{10}(B) + \text{NF} = -174 + 10\log_{10}(20 \times 10^6) + 7 = -174 + 73 + 7 = \qty{-94.0}{dBm} \]
        
        \step \textbf{Calculate the Final SNR.} This is the ratio of the received signal power to the noise floor.
        \[ \text{SNR} = P_r - P_n = -44.0 - (-94.0) = \textbf{\qty{50.0}{dB}} \]
    \end{derivationsteps}
    \parhead{Interpretation} The link loss determined the final received power, while the receiver's characteristics determined the noise floor. The final SNR of 50 dB is excellent, indicating that the received signal is 100,000 times stronger than the noise, allowing for very high-order QAM and high data rates.
\end{workedexample}


\begin{importantbox}[title={Further Reading}]
    The interplay between link loss and noise is the central theme of system-level design.
    \begin{description}
        \item[Link Budget Analysis] (\Cref{ch:linkbudget}) is the full, formal process of calculating all link loss and noise terms to predict the final SNR.
        \item[Free-Space Path Loss (FSPL)] (\Cref{ch:fspl}) and \textbf{Weather Effects} (\Cref{ch:weather-effects}) provide the details for calculating the dominant link loss components.
        \item[Noise Figure] (\Cref{ch:noise}) explains in detail how to calculate the noise floor of a receiver.
        \item[Signal-to-Noise Ratio (SNR)] (\Cref{ch:snr}) is the ultimate result of this calculation, and the primary input for determining the system's performance.
    \end{description}
\end{importantbox}
