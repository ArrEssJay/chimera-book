% ==============================================================================
% CHAPTER 62: Formula Reference Card
% ==============================================================================

\chapter{Formula Reference Card}
\label{ch:formula-reference-card}

\begin{nontechnical}
    This chapter serves as a consolidated quick-reference guide for the practical engineer. It contains the essential formulae used throughout the field of digital signal processing and communications system design, organised for clarity and ease of use. The sections cover the core calculations required for link budget analysis, from fundamental wave propagation and antenna parameters to receiver noise performance and signal quality metrics. Each formula is presented with clear definitions of its variables and cross-references to the chapters where the underlying theory is discussed in detail. This reference is designed not to replace a thorough understanding of the concepts, but to act as a practical companion for design, analysis, and rapid calculation.
\end{nontechnical}

\subsection{Overview}

This reference card consolidates the most frequently used formulae in wireless communications system design. For the detailed theoretical background and derivation of these equations, please consult the cross-referenced chapters.

\begin{keyconcept}
    All logarithmic formulae use decibels (dB). Power is typically expressed in dBm (decibels relative to 1~mW) or dBW (decibels relative to 1~W). Antenna gain is expressed in dBi (decibels relative to an isotropic radiator). Ensure all units are consistent before performing calculations.
\end{keyconcept}

\section{Fundamental Propagation Formulae}

\paragraph{Free-Space Path Loss (FSPL)}
The fundamental loss due to the geometric spreading of a wave. This represents the best-case scenario in a vacuum with no obstructions.
\begin{equation}
    \text{FSPL (dB)} = 20\log_{10}(d) + 20\log_{10}(f) + 32.45
\end{equation}
where \(d\) is the distance in kilometres (km) and \(f\) is the frequency in megahertz (MHz).

\paragraph{Friis Transmission Equation}
The master equation for calculating the received power in a link budget.
\begin{equation}
    P_{r} = \text{EIRP} - \text{FSPL} + G_{r} - L_{\text{misc}}
\end{equation}
where all units are in dB, dBm, or dBi:
\begin{description}
    \item[\(P_{r}\)] Received Power (dBm).
    \item[EIRP] Effective Isotropic Radiated Power (dBm), calculated as \(P_{t} + G_{t}\).
    \item[\(G_{r}\)] Receive Antenna Gain (dBi).
    \item[\(L_{\text{misc}}\)] Sum of all other losses (e.g., atmospheric, cable, pointing loss) in dB.
\end{description}

\paragraph{Antenna Gain (Parabolic Dish)}
The gain of a parabolic dish antenna, which is proportional to its area and the square of the frequency.
\begin{equation}
    G \approx \eta \left(\frac{\pi D}{\lambda}\right)^2
\end{equation}
where \(G\) is the linear gain, \(\eta\) is the aperture efficiency (typically 0.55--0.70), \(D\) is the dish diameter, and \(\lambda\) is the wavelength.

\section{Receiver Performance Formulae}

\paragraph{Thermal Noise Power}
The fundamental noise floor of any receiver, caused by the thermal agitation of electrons.
\begin{equation}
    P_n\ (\text{dBm}) = -174 + 10\log_{10}(B) + \text{NF}
\end{equation}
where:
\begin{description}
    \item[\(-174\)] The thermal noise power spectral density in dBm/Hz at room temperature (290~K).
    \item[\(B\)] The receiver's equivalent noise bandwidth in Hertz (Hz).
    \item[NF] The receiver's Noise Figure in dB.
\end{description}

\paragraph{Cascaded Noise Figure (Friis Formula)}
Calculates the total noise figure for a chain of components (e.g., LNA, cable, mixer).
\begin{equation}
    F_{\text{total}} = F_1 + \frac{F_2 - 1}{G_1} + \frac{F_3 - 1}{G_1 G_2} + \dots
\end{equation}
where \(F_i\) and \(G_i\) are the \textit{linear} noise factor and power gain of the \(i\)-th stage, respectively. This formula demonstrates that the noise figure of the first component has the dominant effect on the system's performance.

\section{System Performance Metrics}

\paragraph{Signal-to-Noise Ratio (SNR)}
The fundamental measure of signal quality at the receiver.
\begin{equation}
    \text{SNR (dB)} = P_{r}\ (\text{dBm}) - P_{n}\ (\text{dBm})
\end{equation}
where \(P_r\) is the received signal power and \(P_n\) is the total noise power in the signal bandwidth.

\paragraph{Energy per Bit to Noise Density Ratio (\(E_b/N_0\))}
The normalized SNR, which is the key metric for determining the bit error rate. It relates the carrier SNR to the bit rate and bandwidth.
\begin{equation}
    \frac{E_b}{N_0} \text{ (dB)} = \text{SNR (dB)} - 10\log_{10}\left(\frac{R_b}{B}\right)
\end{equation}
where \(R_b\) is the bit rate (bps) and \(B\) is the bandwidth (Hz).

\paragraph{Shannon Channel Capacity}
The theoretical maximum data rate for error-free communication over a channel with a given bandwidth and SNR.
\begin{equation}
    C = B \log_2(1 + \text{SNR}_{\text{linear}})
\end{equation}
where \(C\) is the capacity in bits per second (bps) and \(\text{SNR}_{\text{linear}}\) is the linear signal-to-noise ratio.

\paragraph{Bit Error Rate (BER) for QPSK}
The approximate bit error rate for Quadrature Phase-Shift Keying in an AWGN channel.
\begin{equation}
    \text{BER} \approx Q\left(\sqrt{\frac{2E_b}{N_0}}\right) = \frac{1}{2}\text{erfc}\left(\sqrt{\frac{E_b}{N_0}}\right)
\end{equation}
where \(Q(\cdot)\) is the Gaussian Q-function and \(\text{erfc}(\cdot)\) is the complementary error function.

\begin{workedexample}{Complete Link Budget for a Point-to-Point Microwave Link}
    \parhead{Problem}
    Design a 10~km microwave backhaul link operating at 18~GHz to carry 100~Mbps of data with a target BER of \(10^{-6}\) using 16-QAM. Determine the required link margin.

    \parhead{System Parameters}
    \begin{itemize}
        \item Transmit Power, \(P_t\): 30~dBm (1~W)
        \item Antenna Gains, \(G_t = G_r\): 42~dBi (1.2~m dishes)
        \item Required \(E_b/N_0\) for 16-QAM at BER \(10^{-6}\): \(\approx\) 14~dB
        \item Bandwidth, \(B\): \(\approx\) 30~MHz (for 100~Mbps with 16-QAM)
        \item Receiver Noise Figure, NF: 4~dB
        \item Miscellaneous Losses (atmospheric, cables, etc.), \(L_{\text{misc}}\): 5~dB
    \end{itemize}

    \parhead{Analysis}
    \begin{derivationsteps}
        \step \textbf{Calculate the Free-Space Path Loss (FSPL) at 10~km and 18,000~MHz.}
        \[ \text{FSPL} = 20\log_{10}(10) + 20\log_{10}(18000) + 32.45 = 20 + 85.1 + 32.45 = \mathbf{137.6~\text{dB}} \]

        \step \textbf{Calculate the Received Power (\(P_r\)).}
        \[ P_r = P_t + G_t + G_r - \text{FSPL} - L_{\text{misc}} = 30 + 42 + 42 - 137.6 - 5 = \mathbf{-28.6~\text{dBm}} \]
        
        \step \textbf{Calculate the Receiver Noise Floor (\(P_n\)).}
        \[ P_n = -174 + 10\log_{10}(30 \times 10^6) + 4 = -174 + 74.8 + 4 = \mathbf{-95.2~\text{dBm}} \]
        
        \step \textbf{Calculate the Received Signal-to-Noise Ratio (SNR).}
        \[ \text{SNR} = P_r - P_n = -28.6 - (-95.2) = \mathbf{66.6~\text{dB}} \]
        
        \step \textbf{Calculate the available \(E_b/N_0\).}
        \[ \frac{E_b}{N_0} = \text{SNR} - 10\log_{10}\left(\frac{100 \times 10^6}{30 \times 10^6}\right) = 66.6 - 10\log_{10}(3.33) = 66.6 - 5.2 = \mathbf{61.4~\text{dB}} \]
        
        \step \textbf{Calculate the Link Margin.}
        \[ \text{Margin} = (\text{Available } E_b/N_0) - (\text{Required } E_b/N_0) = 61.4 - 14 = \mathbf{47.4~\text{dB}} \]
    \end{derivationsteps}

    \parhead{Interpretation}
    The link is viable with an exceptionally healthy margin of 47.4~dB. This large margin is essential to ensure high availability, as it provides the budget to overcome significant rain fade (which can easily exceed 20~dB at 18~GHz), multipath effects, and other real-world impairments.
\end{workedexample}

\begin{importantbox}
\section*{Further Reading}
This reference card provides the "what" of the key formulae in communication system design. The "why" and "how" are detailed in the main body of the manuscript. For in-depth derivations and a qualitative understanding of these concepts, please refer to the following key chapters:
\begin{description}
    \item[\Cref{ch:fspl} (Free-Space Path Loss)] Derives the fundamental equations for signal attenuation with distance and frequency.
    \item[\Cref{ch:noise} (Noise Sources \& Noise Figure)] Explains the origin of thermal noise and the crucial role of the LNA.
    \item[\Cref{ch:snr} (Signal-to-Noise Ratio)] and \Cref{ch:energy-ratios} (\(E_b/N_0\)) Detail the metrics used to quantify signal quality.
    \item[\Cref{ch:ber} (Bit Error Rate)] Explores the relationship between signal quality and the final system performance.
    \item[\Cref{ch:shannon} (Shannon's Channel Capacity)] Discusses the theoretical limits of communication.
    \item[\Cref{ch:linkbudget} (Complete Link Budget Analysis)] Provides a comprehensive, step-by-step guide to applying these formulae in a full system analysis.
\end{description}
\end{importantbox}