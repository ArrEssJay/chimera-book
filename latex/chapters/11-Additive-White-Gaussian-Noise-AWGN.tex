% ==============================================================================
% CHAPTER 11: Additive White Gaussian Noise (AWGN)
% ==============================================================================

\chapter{Additive White Gaussian Noise (AWGN)}
\label{ch:awgn}

\begin{nontechnical}
    \textbf{AWGN is the fundamental "static" or "hiss" that exists in every electronic communication system.} It is the baseline of random interference that every signal must overcome to be heard.

    \parhead{The three key properties}
    \begin{itemize}
        \item \textbf{Additive:} The noise simply adds itself to the desired signal.
        \item \textbf{White:} The noise has equal power across all frequencies, like white light containing all colours.
        \item \textbf{Gaussian:} The random fluctuations in the noise's amplitude follow a perfect bell-curve distribution.
    \end{itemize}

    \parhead{Real-world examples}
    \begin{itemize}
        \item The hiss you hear from an FM radio when it's tuned between stations.
        \item The "snow" on an old analog television with no signal.
        \item The grainy appearance of a photograph taken in very low light, caused by thermal noise in the camera's sensor.
    \end{itemize}

    \parhead{Why it matters} AWGN is caused by the random thermal motion of electrons in electronic components. It is a fundamental property of physics and \textbf{cannot be eliminated}. It represents the absolute minimum noise floor that every communication system must be designed to overcome. All performance metrics, like BER and channel capacity, are first benchmarked against this idealised form of noise.
\end{nontechnical}


\subsection{Overview}

\keyterm{Additive White Gaussian Noise (AWGN)} is the most important and widely used channel model in communication theory. It provides a mathematically tractable yet physically realistic representation of the unavoidable thermal noise present in all receiver electronics.

\begin{keyconcept}
    The AWGN channel is the benchmark against which all modulation and coding schemes are measured. While real-world channels have additional impairments like fading and interference, a system's performance in an AWGN channel defines its theoretical best-case performance.
\end{keyconcept}


\subsection{Mathematical Model}

In an AWGN channel, the received signal $r(t)$ is the linear sum of the transmitted signal $s(t)$ and a random noise process $n(t)$:
\begin{equation}
    r(t) = s(t) + n(t)
\end{equation}
The noise process $n(t)$ is a zero-mean Gaussian random process. Its amplitude follows the Gaussian probability density function:
\begin{equation}
    p(n) = \frac{1}{\sqrt{2\pi\sigma^2}} e^{-\frac{n^2}{2\sigma^2}}
\end{equation}
where $\sigma^2$ is the variance of the noise, which is equal to the average noise power, $P_n$.

The "white" property means that the noise has a constant \keyterm{power spectral density}, $N_0/2$ (for a two-sided spectrum), across all frequencies. The total noise power in a given bandwidth $B$ is therefore $P_n = N_0 B$.


\subsection{Physical Origins}

The primary source of AWGN is \keyterm{thermal noise} (also known as Johnson-Nyquist noise), which is generated by the random thermal agitation of charge carriers (usually electrons) inside any electrical conductor at a temperature above absolute zero. Its power is given by:
\begin{equation}
    P_n = kTB
\end{equation}
where $k$ is Boltzmann's constant, $T$ is the system noise temperature in Kelvin, and $B$ is the bandwidth in Hz. Other electronic noise sources, like shot noise, also contribute and, due to the Central Limit Theorem, their sum tends to have a Gaussian distribution.


\subsection{Impact on Constellation Diagrams}

In the complex baseband domain, AWGN adds an independent Gaussian random variable to both the I and Q components of a transmitted symbol. This causes the received symbols to form a circular "cloud" or scatter plot around each ideal constellation point. The standard deviation ($\sigma$) of this cloud is directly related to the noise power. A higher SNR results in tighter, more distinct clouds, while a lower SNR causes the clouds to expand and overlap, leading to bit errors.

\begin{center}
    \begin{tikzpicture}[scale=1.0]
        % High SNR
        \begin{scope}[shift={(0,0)}]
            \node[above,font=\sffamily\bfseries] at (0,3) {High SNR ($\sim$20 dB)};
            \draw[->] (-2.5,0) -- (2.5,0); \node[right] at (2.5,0) {I};
            \draw[->] (0,-2.5) -- (0,2.5); \node[above] at (0,2.5) {Q};
            \foreach \x/\y in {1.5/1.5, -1.5/1.5, -1.5/-1.5, 1.5/-1.5} {
                \node[circle, fill=diagramprimary, inner sep=2pt] at (\x,\y) {};
                \foreach \i in {1,...,25} {
                    \fill[black,opacity=0.3] (\x+0.1*rand, \y+0.1*rand) circle (0.8pt);
                }
            }
        \end{scope}

        % Low SNR
        \begin{scope}[shift={(7,0)}]
            \node[above,font=\sffamily\bfseries] at (0,3) {Low SNR ($\sim$5 dB)};
            \draw[->] (-2.5,0) -- (2.5,0); \node[right] at (2.5,0) {I};
            \draw[->] (0,-2.5) -- (0,2.5); \node[above] at (0,2.5) {Q};
            \foreach \x/\y in {1.5/1.5, -1.5/1.5, -1.5/-1.5, 1.5/-1.5} {
                \node[circle, fill=diagramprimary, inner sep=2pt] at (\x,\y) {};
                \foreach \i in {1,...,50} {
                    \fill[black,opacity=0.2] (\x+0.6*rand, \y+0.6*rand) circle (0.8pt);
                }
            }
        \end{scope}
    \end{tikzpicture}
\end{center}


\begin{workedexample}{AWGN Link Budget Analysis}
    \parhead{Problem} A satellite ground station receives a BPSK signal. Calculate the expected Bit Error Rate.
    \parhead{System Parameters}
    \begin{itemize}
        \item Received Signal Power ($P_s$): \qty{-120}{dBm}
        \item System Noise Temperature ($T_s$): \qty{150}{K}
        \item Data Rate ($R_b$): \qty{1}{Mbps}
    \end{itemize}
    \parhead{Solution}
    \begin{derivationsteps}
        \step Calculate the noise power spectral density ($N_0$).
        \[ N_0 \text{ (W/Hz)} = kT_s = (1.38 \times 10^{-23})(150) = 2.07 \times 10^{-21} \text{ W/Hz} \]
        \[ N_{0, \text{dBm/Hz}} = 10\log_{10}(2.07 \times 10^{-21} / 10^{-3}) = \qty{-176.8}{dBm/Hz} \]
        \step Calculate the available $E_b/N_0$.
        \[ \left(\frac{E_b}{N_0}\right)_{\text{dB}} = P_{r, \text{dBm}} - 10\log_{10}(R_b) - N_{0, \text{dBm/Hz}} \]
        \[ \left(\frac{E_b}{N_0}\right)_{\text{dB}} = -120 - 10\log_{10}(10^6) - (-176.8) = -120 - 60 + 176.8 = \qty{-3.2}{dB} \]
        \step Calculate the theoretical BER for coherent BPSK.
        \[ \text{BER} = Q\left(\sqrt{2 \cdot 10^{(E_b/N_0)_{\text{dB}}/10}}\right) = Q\left(\sqrt{2 \cdot 10^{-3.2/10}}\right) = Q(\sqrt{0.957}) = Q(0.978) \approx 0.164 \]
    \end{derivationsteps}
    \parhead{Interpretation} The calculated BER of 0.164 (or about 1 in 6 bits being wrong) indicates that the link is unusable in its current state. The received signal is too weak compared to the thermal noise floor. To make this link work, one would need to employ powerful forward error correction (FEC) to provide significant coding gain, increase the receive antenna gain, or reduce the data rate.
\end{workedexample}


\begin{importantbox}[title={Further Reading}]
    The AWGN model is the essential starting point for performance analysis.
    \begin{description}
        \item[Bit Error Rate] (\Cref{ch:ber}) provides the specific performance curves that show how different modulation schemes behave in an AWGN channel.
        \item[Channel Capacity] (\Cref{ch:shannon}) uses the AWGN model to define the absolute maximum data rate that can be transmitted over a channel with a given SNR.
        \item[Fading Channels] (\Cref{ch:fading}) describes more complex channel models, such as Rayleigh and Rician fading, which build upon the AWGN foundation to model the effects of multipath propagation in mobile and indoor environments.
    \end{description}
\end{importantbox}