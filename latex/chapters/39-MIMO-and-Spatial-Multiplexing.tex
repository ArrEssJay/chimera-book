% ==============================================================================
% CHAPTER 39: MIMO & Spatial Multiplexing
% ==============================================================================

\chapter{MIMO \& Spatial Multiplexing}
\label{ch:mimo}

\begin{nontechnical}
    \textbf{MIMO is the magic that makes modern WiFi and 5G so fast.} It's like having multiple, invisible conversations happening at the same time, in the same room, without interfering with each other.

    \parhead{The simple idea: Multiple Antennas}
    MIMO stands for \textbf{M}ultiple-\textbf{I}nput, \textbf{M}ultiple-\textbf{O}utput. It simply means using multiple antennas on both the transmitter (e.g., your WiFi router) and the receiver (e.g., your laptop).

    \parhead{The breakthrough: Spatial Multiplexing}
    In an indoor environment, radio signals bounce off walls, furniture, and people, creating a complex web of paths between the transmitter and receiver. MIMO exploits this.
    \begin{itemize}
        \item A 4-antenna router can send \textbf{four independent data streams} at the same time on the same frequency.
        \item Each stream travels a slightly different path to the 4 antennas on your laptop.
        \item The receiver's DSP chip is powerful enough to solve the "puzzle" of the mixed-up signals, separating them back into the original four streams.
    \end{itemize}
    The result? \textbf{Four times the data rate} compared to a single-antenna system, with no extra power or spectrum required.

    \parhead{Real-world impact}
    The number of "streams" is what determines the speed of your WiFi.
    \begin{itemize}
        \item \textbf{1x1 (No MIMO):} Baseline speed (e.g., 150 Mbps).
        \item \textbf{2x2 MIMO:} Two streams, double the speed (e.g., 300 Mbps).
        \item \textbf{4x4 MIMO:} Four streams, quadruple the speed (e.g., 600 Mbps).
        \item \textbf{8x8 MIMO (WiFi 6/7):} Eight streams, eight times the speed (over 1 Gbps).
    \end{itemize}
    The multiple antennas on your router are not just for show; they are the physical hardware that enables this spatial multiplexing.
\end{nontechnical}


\section{Overview and Properties}

\subsection{Overview}

\keyterm{Multiple-Input, Multiple-Output (MIMO)} is a revolutionary antenna technology that exploits the multipath propagation environment to dramatically increase the data rate and/or reliability of a wireless link. By using multiple antennas at both the transmitter and the receiver, a MIMO system can open up multiple parallel "spatial channels" over the same frequency band.

\begin{keyconcept}
    The capacity of a MIMO channel scales \textbf{linearly} with the minimum of the number of transmit ($N_T$) and receive ($N_R$) antennas, without requiring any additional bandwidth or power. This remarkable result, often expressed as $C \approx \min(N_T, N_R) \cdot B \log_2(1 + \text{SNR})$, represents a fundamental shift from the traditional logarithmic scaling of SISO (Single-Input, Single-Output) systems.
\end{keyconcept}


\subsection{The Three Gains of MIMO}

MIMO technology can be configured to provide three distinct types of benefits:

\begin{description}
    \item[1. Spatial Multiplexing Gain] This is the most famous benefit. It involves sending independent data streams from each transmit antenna simultaneously. A receiver with enough antennas can separate these streams, multiplying the data rate by a factor of up to $\min(N_T, N_R)$. This is the primary technique used to achieve the gigabit-per-second speeds of modern WiFi and 5G.
    
    \item[2. Diversity Gain] In this mode, the same information is sent across all antennas in a structured way (e.g., using a space-time block code like the Alamouti scheme). The receiver combines the multiple copies of the signal. Since it is highly unlikely that all antenna paths will be in a deep fade at the same time, this dramatically improves the link's reliability and robustness against fading. The BER performance improves exponentially with the number of diversity branches.
    
    \item[3. Array Gain] This is the gain achieved by coherently combining the signals from all antennas. For a receiver with $N_R$ antennas, this provides an SNR improvement of up to $10\log_{10}(N_R)$ dB. This is equivalent to having a single, much larger and more sensitive antenna.
\end{description}
Modern systems often use a hybrid approach, dynamically trading off between these three gains based on the channel conditions.


\subsection{MIMO System Model and Detection}

A MIMO system with $N_T$ transmit and $N_R$ receive antennas is described by a simple matrix equation:
\begin{equation}
    \mathbf{y} = \mathbf{H}\mathbf{x} + \mathbf{n}
\end{equation}
where $\mathbf{x}$ is the $N_T \times 1$ vector of transmitted symbols, $\mathbf{y}$ is the $N_R \times 1$ vector of received symbols, $\mathbf{n}$ is the noise vector, and $\mathbf{H}$ is the $N_R \times N_T$ \keyterm{channel matrix}. The element $h_{ij}$ of this matrix is the complex gain from transmit antenna $j$ to receive antenna $i$.

The receiver's task is to solve this system of linear equations to find the transmitted vector $\mathbf{x}$, given the received vector $\mathbf{y}$ and an estimate of the channel matrix $\mathbf{H}$.
\begin{itemize}
    \item \textbf{Zero-Forcing (ZF) Detection:} The receiver computes the pseudo-inverse of the channel matrix. This perfectly cancels all inter-stream interference but can amplify noise in ill-conditioned channels.
    \item \textbf{MMSE Detection:} A more advanced technique that minimises the combined effect of noise and interference, providing better performance than ZF at low SNR.
    \item \textbf{Maximum Likelihood (ML) Detection:} The optimal method, which exhaustively searches all possible transmit vectors to find the one that is closest to the received signal. It is computationally prohibitive for all but the simplest systems.
\end{itemize}


\subsection{Massive MIMO in 5G}

\keyterm{Massive MIMO} is a key technology in 5G, where base stations are equipped with a very large number of antennas (e.g., 64, 128, or 256). This provides several transformative benefits:
\begin{itemize}
    \item \textbf{Channel Hardening:} As the number of antennas becomes large, the random fading of the channel averages out, and the channel starts to behave like a predictable, non-fading one.
    \item \textbf{Favourable Propagation:} The channel vectors for different users become nearly orthogonal, allowing the base station to serve many users simultaneously in the same frequency band with simple linear processing.
    \item \textbf{Energy Efficiency:} The massive array gain allows the base station to focus energy with extreme precision towards each user, dramatically reducing the required transmit power per antenna and improving overall energy efficiency by orders of magnitude.
\end{itemize}


\begin{workedexample}{WiFi 6 Capacity Analysis}
    \parhead{Problem} A WiFi 6 (802.11ax) access point has 8 antennas and a user's device has 2 antennas. Compare the maximum theoretical capacity with a legacy single-antenna system in a 160 MHz channel with an SNR of 30 dB.
    
    \parhead{Analysis}
    \begin{derivationsteps}
        \step \textbf{Calculate the SISO Capacity.} For a single-antenna (SISO) system, the Shannon capacity is:
        \[ C_{\text{SISO}} = B \log_2(1 + \text{SNR}) = (160 \times 10^6) \cdot \log_2(1 + 10^{30/10}) \approx \textbf{\qty{1.6}{Gbps}} \]
        
        \step \textbf{Calculate the MIMO Capacity.} The number of parallel spatial streams is limited by the minimum of the transmit and receive antennas, so $\min(N_T, N_R) = \min(8, 2) = 2$. The MIMO capacity scales linearly with the number of streams.
        \[ C_{\text{MIMO}} \approx \min(N_T, N_R) \times C_{\text{SISO}} = 2 \times 1.6 \text{ Gbps} = \textbf{\qty{3.2}{Gbps}} \]
    \end{derivationsteps}
    
    \parhead{Interpretation} By using just two antennas at the device, the MIMO system can achieve double the theoretical data rate of a single-antenna system under the same conditions. This demonstrates the power of spatial multiplexing. While the access point has 8 antennas, the link is limited by the device's 2 antennas. The access point can use its remaining antennas to serve other users simultaneously (\keyterm{multi-user MIMO}) or for beamforming to improve the signal quality to this user.
\end{workedexample}


\begin{importantbox}[title={Further Reading}]
    MIMO is the enabling technology for virtually all modern high-speed wireless standards.
    \begin{description}
        \item[OFDM] (\Cref{ch:ofdm}) is the modulation scheme that is almost always paired with MIMO. The combination, known as MIMO-OFDM, allows for simple per-subcarrier equalisation of the MIMO channel.
        \item[5G and WiFi Systems] (\Cref{ch:5g}, \Cref{ch:wifi}) provide detailed case studies of how MIMO, Multi-User MIMO (MU-MIMO), and Massive MIMO are implemented in the world's most advanced wireless standards.
        \item[Channel Equalisation] (\Cref{ch:equalisation}) describes the underlying DSP algorithms (ZF, MMSE) used to separate the spatial streams at the MIMO receiver.
    \end{description}
\end{importantbox}
