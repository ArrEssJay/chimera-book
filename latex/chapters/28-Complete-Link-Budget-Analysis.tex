% ==============================================================================
% CHAPTER 28: Complete Link Budget Analysis
% ==============================================================================

\chapter{Complete Link Budget Analysis}
\label{ch:linkbudget}

\begin{nontechnical}
    \textbf{A link budget is the master equation of wireless engineering.} It is a systematic accounting of all the gains and losses a signal experiences on its journey from the transmitter to the receiver. It's like a financial balance sheet for signal power.

    \parhead{The simple accounting}
    \begin{enumerate}
        \item \textbf{Starting Power:} The power from the transmitter's amplifier.
        \item \textbf{Add Gains:} The focusing effect of the transmit and receive antennas. A satellite dish provides a massive gain.
        \item \textbf{Subtract Losses:} This is the biggest part. It includes the immense Free-Space Path Loss, as well as smaller losses from cables, connectors, walls, and weather.
        \item \textbf{The Result (Received Power):} The final power that arrives at the receiver.
    \end{enumerate}

    \parhead{The final check}
    This final received power is then compared to the receiver's \textbf{sensitivity} (the faintest signal it can hear above its own internal noise).
    \begin{itemize}
        \item If \textbf{Received Power > Sensitivity}, the link works. The difference between them is the \textbf{link margin}.
        \item If \textbf{Received Power < Sensitivity}, the link fails.
    \end{itemize}

    \parhead{Why it matters} Every wireless system, from a simple WiFi link to the Voyager 1 space probe, is designed using a link budget. It is the tool engineers use to answer the most fundamental question: "Will this link work?" A healthy link margin (e.g., 10-20 dB) is essential to ensure the connection remains reliable even when conditions are not ideal.
\end{nontechnical}


\subsection{Overview}

A \keyterm{link budget} is a comprehensive and systematic accounting of all power gains and losses that a signal experiences as it travels from a transmitter to a receiver. It is the fundamental tool used by RF engineers to design and analyse a communication link, and its primary purpose is to calculate the final received power and compare it against the receiver's sensitivity to determine the \keyterm{link margin}.

\begin{keyconcept}
    The link budget is the culmination of all the concepts discussed in the preceding chapters. It combines transmit power, antenna gains, propagation losses (FSPL, atmospheric, weather), and receiver noise performance into a single master equation. A positive link margin indicates a viable link; a negative margin indicates a link that will fail.
\end{keyconcept}


\subsection{The Link Budget Equation}

The link budget is calculated in logarithmic units (decibels), which allows gains and losses to be simply added and subtracted. The master equation for the received power, $P_r$, is:
\begin{equation}
    P_{r, \text{dBm}} = \text{EIRP}_{\text{dBm}} - L_{\text{total, dB}} + G_{r, \text{dBi}} - L_{\text{rx, dB}}
\end{equation}
where:
\begin{description}
    \item[EIRP] is the Effective Isotropic Radiated Power ($P_t + G_t - L_{tx}$).
    \item[$L_{\text{total}}$] is the total path loss (FSPL + weather + fading margins).
    \item[$G_r$] is the receive antenna gain.
    \item[$L_{rx}$] are the receiver-side losses (e.g., cable loss).
\end{description}
This received power is then compared to the receiver's sensitivity, $P_{\text{min}}$, to find the link margin:
\begin{equation}
    \text{Link Margin}_{\text{dB}} = P_{r, \text{dBm}} - P_{\text{min, dBm}}
\end{equation}
A robust link typically requires a margin of at least 10-20 dB to account for unmodelled variations and component aging.


\subsection{Receiver Sensitivity}

The receiver's sensitivity, $P_{\text{min}}$, is the minimum received power required to achieve a target performance, such as a specific Bit Error Rate (BER). It is determined by the receiver's noise floor and the SNR required by the modulation and coding scheme.
\begin{equation}
    P_{\text{min, dBm}} = (\text{Noise Floor})_{\text{dBm}} + \text{SNR}_{\text{req, dB}}
\end{equation}
The noise floor is calculated from the thermal noise density ($-174$ dBm/Hz), the channel bandwidth ($B$), and the receiver's noise figure (NF):
\begin{equation}
    (\text{Noise Floor})_{\text{dBm}} = -174 + 10\log_{10}(B) + \text{NF}
\end{equation}


\begin{workedexample}{GEO Satellite Link Budget}
    \parhead{Problem} Perform a complete link budget for a Ku-band geostationary satellite downlink to determine the required receive antenna size for 99.9\% availability.
    
    \parhead{System Parameters}
    \begin{itemize}
        \item \textbf{Transmit:} Satellite EIRP = \qty{53}{dBW}.
        \item \textbf{Path:} Frequency = \qty{12}{GHz}, Distance = \qty{38,000}{km}.
        \item \textbf{Environment:} Temperate climate, requiring a 5 dB rain margin for 99.9\% availability. Other atmospheric losses are 1 dB.
        \item \textbf{Receiver:} System noise temperature $T_s = \qty{150}{K}$, Bandwidth $B = \qty{27}{MHz}$.
        \item \textbf{Performance:} Modulation is QPSK with rate-3/4 FEC, requiring a final $E_b/N_0$ of \qty{5.5}{dB} for a BER of $10^{-7}$.
    \end{itemize}

    \parhead{Solution}
    \begin{derivationsteps}
        \step \textbf{Calculate Total Path Loss.}
        \[ \text{FSPL} = 20\log_{10}(38000) + 20\log_{10}(12000) + 32.45 \approx \qty{205.9}{dB} \]
        \[ L_{\text{total}} = \text{FSPL} + L_{\text{atm}} + L_{\text{rain}} = 205.9 + 1.0 + 5.0 = \qty{211.9}{dB} \]

        \step \textbf{Calculate Required Received Power.} First, find the required carrier-to-noise ratio ($C/N$). The data rate is $R_b = 27 \text{ Msps} \times 2 \text{ bits/sym} \times 3/4 = 40.5$ Mbps.
        \[ (C/N)_{\text{req}} = (E_b/N_0)_{\text{req}} + 10\log_{10}(R_b/B) = 5.5 + 10\log_{10}(40.5/27) \approx \qty{7.3}{dB} \]
        The noise floor is $N = 10\log_{10}(kT_sB) = -228.6 + 10\log_{10}(150) + 10\log_{10}(27\times10^6) = \qty{-132.4}{dBW}$.
        \[ P_{\text{req}} = (C/N)_{\text{req}} + N = 7.3 - 132.4 = \qty{-125.1}{dBW} \]

        \step \textbf{Calculate Required Antenna Gain.} The received power is $P_r = \text{EIRP} - L_{\text{total}} + G_r$. We need $P_r \ge P_{\text{req}}$.
        \[ G_{r, \text{req}} = P_{\text{req}} - \text{EIRP} + L_{\text{total}} = -125.1 - 53 + 211.9 = \qty{33.8}{dBi} \]

        \step \textbf{Calculate Antenna Dish Size.} The gain of a parabolic dish is $G \approx 10\log_{10}(\eta (\pi D / \lambda)^2)$. For Ku-band ($\lambda=0.025$ m) and 60\% efficiency, a 33.8 dBi gain corresponds to a diameter of approximately:
        \[ D \approx \frac{\lambda}{\pi} \sqrt{10^{(G/10)}/\eta} \approx \frac{0.025}{\pi} \sqrt{10^{(33.8/10)}/0.6} \approx \textbf{\qty{0.5}{meters}} \]
    \end{derivationsteps}
    
    \parhead{Interpretation} To close the link with a 99.9\% availability in a temperate climate, the user requires a receiving dish of at least 50 cm in diameter. If the user was in a tropical region with a 15 dB rain margin requirement, the required gain would increase by 10 dB, necessitating a much larger dish (approx. 1.5 meters).
\end{workedexample}

\begin{table}[H]
    \centering
    \caption{Link Budget Template Summary}
    \label{tab:link-budget-template}
    \begin{tabular}{@{}ll@{}}
        \toprule
        \multicolumn{2}{c}{\tableheaderfont Link Budget Parameter} \\
        \midrule
        \textbf{Transmitter Side} & \\
        \quad Power Amplifier Output & ($P_{\text{amp}}$) \\
        \quad Transmit Losses (cables, connectors) & ($-L_{tx}$) \\
        \quad Transmit Antenna Gain & ($+G_t$) \\
        \textbf{Effective Isotropic Radiated Power} & \textbf{(= EIRP)} \\
        \addlinespace
        \textbf{Path Losses} & \\
        \quad Free-Space Path Loss & ($-L_{\text{FSPL}}$) \\
        \quad Atmospheric and Weather Losses & ($-L_{\text{atm}}, -L_{\text{rain}}$) \\
        \quad Fading Margin & ($-M_{\text{fade}}$) \\
        \addlinespace
        \textbf{Receiver Side} & \\
        \quad Receive Antenna Gain & ($+G_r$) \\
        \quad Receive Losses (cables, connectors) & ($-L_{rx}$) \\
        \textbf{Final Received Power} & \textbf{(= $P_r$)} \\
        \addlinespace
        \textbf{Performance Check} & \\
        \quad Receiver Sensitivity & ($-P_{\text{min}}$) \\
        \textbf{Link Margin} & \textbf{(= $P_r - P_{\text{min}}$)} \\
        \bottomrule
    \end{tabular}
\end{table}


\begin{importantbox}[title={Further Reading}]
    The link budget is the practical application of nearly every concept in this section.
    \begin{description}
        \item[FSPL, Atmospheric, and Weather Effects] (\Cref{ch:fspl}, \Cref{ch:atmospheric}, \Cref{ch:weather-effects}) provide the detailed models for calculating the various path loss components.
        \item[Noise Figure and Energy Ratios] (\Cref{ch:noise}, \Cref{ch:energy-ratios}) are essential for calculating the receiver's sensitivity, $P_{\text{min}}$.
        \item[Adaptive Modulation and Coding] (\Cref{ch:amc}) is the primary technique used to manage the link budget dynamically, trading data rate for link margin as channel conditions change.
    \end{description}
\end{importantbox}