% ==============================================================================
% CHAPTER 20: Quadrature Amplitude Modulation (QAM)
% ==============================================================================

\chapter{Quadrature Amplitude Modulation (QAM)}
\label{ch:qam}

\begin{nontechnical}
    \textbf{QAM is like using a grid of laser pointers to send data.} Instead of just varying the colour (phase) or brightness (amplitude), QAM varies \textbf{both simultaneously}. This creates a large grid of possible signal states, allowing a massive amount of information to be sent with each pulse.

    \parhead{The grid system}
    \begin{itemize}
        \item \textbf{16-QAM:} A 4x4 grid of 16 points, sending 4 bits at a time.
        \item \textbf{64-QAM:} An 8x8 grid of 64 points, sending 6 bits at a time.
        \item \textbf{256-QAM:} A 16x16 grid of 256 points, sending 8 bits at a time.
        \item \textbf{1024-QAM:} A 32x32 grid, sending 10 bits at a time (used in WiFi 6).
    \end{itemize}

    \parhead{The speed vs. reliability trade-off} A larger grid means a much faster data rate. 256-QAM is four times faster than QPSK. However, the points on the grid are much closer together, making them harder to distinguish in a noisy environment.

    \parhead{Real-world use: Adaptive Modulation} Your smartphone and WiFi router perform this trade-off automatically hundreds of times per second.
    \begin{itemize}
        \item When the signal is strong (high SNR, close to the router), they use a large grid like 256-QAM for blazing-fast speeds.
        \item As the signal gets weaker (low SNR, far from the router), they switch to a smaller, more robust grid like 16-QAM or QPSK to keep the connection reliable.
    \end{itemize}
    This is why your internet speed drops as you move further from your router.
\end{nontechnical}


\subsection{Overview}

\keyterm{Quadrature Amplitude Modulation (QAM)} is a highly spectrally efficient modulation scheme that encodes data by varying both the \keyterm{amplitude} and \keyterm{phase} of a carrier wave. It can be viewed as the combination of Amplitude-Shift Keying (ASK) and Phase-Shift Keying (PSK). By utilizing the full two-dimensional IQ plane, QAM can achieve a much higher data rate than ASK or PSK alone for a given bandwidth.

\begin{keyconcept}
    QAM is the workhorse of modern, high-speed digital communications. Its ability to pack many bits into a single symbol makes it the standard for \textbf{bandwidth-limited} systems where high data throughput is required, such as WiFi, 4G/5G cellular, and cable modems. The trade-off for this high efficiency is a requirement for a high signal-to-noise ratio and linear amplification.
\end{keyconcept}


\subsection{Mathematical Representation}

An M-ary QAM signal is constructed from two independent baseband signals, $I(t)$ and $Q(t)$, which modulate orthogonal carriers. Each symbol corresponds to a point $(I_k, Q_k)$ on a grid in the IQ plane. The passband signal is:
\begin{equation}
    s(t) = I_k \cos(2\pi f_c t) - Q_k \sin(2\pi f_c t)
\end{equation}
For a square M-QAM constellation, there are $L = \sqrt{M}$ distinct amplitude levels for both the I and Q components, creating a grid of $M=L^2$ points. Each symbol encodes $k = \log_2(M)$ bits.


\subsection{Performance Characteristics}

\paragraph{Bit Error Rate (BER)}
The performance of QAM is determined by the Euclidean distance between adjacent constellation points. As the order of M-QAM increases, the points become more crowded for a given average power, making the system more susceptible to noise.
\begin{equation}
    \text{BER}_{\text{M-QAM}} \approx \frac{4}{\log_2 M} Q\left(\sqrt{\frac{3\log_2 M}{M-1} \frac{E_b}{N_0}}\right)
\end{equation}

\paragraph{Power vs. Spectral Efficiency Trade-off}
QAM provides a direct trade-off between power and bandwidth. Each time the number of bits per symbol is increased by one (e.g., moving from 16-QAM to 32-QAM), the spectral efficiency increases, but the required $E_b/N_0$ to maintain the same BER also increases by approximately 3-4 dB.

\begin{table}[H]
    \centering
    \caption{Performance of Common QAM Schemes (for BER $10^{-6}$)}
    \label{tab:qam-performance}
    \begin{tabular}{@{}lccc@{}}
        \toprule
        \tableheaderfont Modulation & \tableheaderfont Bits/Symbol & \tableheaderfont Spectral Eff. ($\eta$) & \tableheaderfont Required $E_b/N_0$ \\
        \midrule
        QPSK (4-QAM) & 2 & 1.6 bps/Hz & 10.5 dB \\
        16-QAM & 4 & 3.2 bps/Hz & 14.5 dB \\
        64-QAM & 6 & 4.8 bps/Hz & 18.8 dB \\
        256-QAM & 8 & 6.4 bps/Hz & 23.4 dB \\
        1024-QAM & 10 & 8.0 bps/Hz & 28.5 dB \\
        \bottomrule
    \end{tabular}
    \par\vspace{0.5em}
    \small Note: Spectral efficiency calculated with a typical roll-off factor of $\alpha=0.25$.
\end{table}


\subsection{Implementation Challenges}

\paragraph{Power Amplifier Linearity}
Unlike constant-envelope schemes like PSK and FSK, QAM has a varying signal envelope. This results in a high \keyterm{Peak-to-Average Power Ratio (PAPR)}, especially for higher orders. A high PAPR requires the transmitter's power amplifier (PA) to be highly linear to avoid distorting the signal, which reduces the PA's power efficiency. This is a critical design trade-off in mobile devices.

\paragraph{Receiver Complexity}
QAM requires a fully coherent receiver with precise carrier and timing synchronization. It is also highly sensitive to hardware imperfections:
\begin{itemize}
    \item \textbf{Phase Noise:} Causes rotation and blurring of the constellation points.
    \item \textbf{IQ Imbalance:} Skews the constellation, reducing the distance between points.
    \item \textbf{Amplitude/Phase Ripple:} Uneven frequency response in filters can distort the positions of the constellation points.
\end{itemize}
These impairments become increasingly problematic for higher-order schemes like 256-QAM and 1024-QAM, which require an EVM of less than 3.5\% and 1.8\%, respectively.


\begin{workedexample}{Adaptive Modulation in WiFi}
    \parhead{Problem} A WiFi 6 (802.11ax) device is streaming video. Analyse its choice of modulation based on the received signal strength.
    
    \parhead{System Parameters}
    \begin{itemize}
        \item Available Modulations: QPSK, 16-QAM, 64-QAM, 256-QAM, 1024-QAM.
        \item Channel: Indoor with multipath fading.
    \end{itemize}
    
    \parhead{Analysis (Adaptive Modulation)}
    The device continuously measures the received SNR and selects the highest-order Modulation and Coding Scheme (MCS) that can be reliably decoded.
    \begin{description}
        \item[Scenario 1: Strong Signal] The user is in the same room as the router. The received SNR is excellent (>35 dB). The device selects the highest possible modulation, \textbf{1024-QAM}, achieving the maximum data rate (e.g., >1 Gbps).
        \item[Scenario 2: Medium Signal] The user moves to an adjacent room. The signal passes through a wall, reducing the SNR to around 25 dB. This is insufficient for 1024-QAM, so the device's firmware automatically steps down to \textbf{256-QAM} or \textbf{64-QAM}, maintaining a stable connection at a reduced data rate (e.g., 300-600 Mbps).
        \item[Scenario 3: Weak Signal] The user moves to the edge of the router's range. The SNR drops below 15 dB. The device falls back to the most robust modulation available, \textbf{QPSK}, to maintain the link, even if the data rate is only a few Mbps.
    \end{description}
    \parhead{Interpretation} This process of \keyterm{link adaptation} is the key to modern wireless performance. It allows the system to continuously track the Shannon Capacity of the channel, squeezing out the maximum possible data rate at any given moment while ensuring the connection remains reliable.
\end{workedexample}


\begin{importantbox}[title={Further Reading}]
    QAM is the culmination of the single-carrier modulation techniques discussed in this part of the book.
    \begin{description}
        \item[OFDM] (\Cref{ch:ofdm}) is the essential next step. It explains how thousands of parallel QAM signals are used to create the high-speed, multipath-resistant links that power modern WiFi and 5G.
        \item[Error Vector Magnitude (EVM)] (\Cref{ch:evm}) provides a deep dive into the critical metric used to quantify the quality of a QAM constellation and diagnose hardware impairments.
        \item[Channel Equalization] (\Cref{ch:equalization}) describes the DSP techniques required to undo the distortion a channel inflicts on a QAM signal, allowing the receiver to recover the original data.
    \end{description}
\end{importantbox}