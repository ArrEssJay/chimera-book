% ==============================================================================
% CHAPTER 36: Polar Codes
% ==============================================================================

\chapter{Polar Codes}
\label{ch:polar-codes}

\begin{nontechnical}
    \textbf{Polar codes are the newest champions of error correction}, the first class of codes with a rigorous mathematical proof that they can achieve the theoretical "speed limit" of a communication channel. This is why they were chosen for the critical control signalling in 5G.

    \parhead{The magic trick: Sorting the good from the bad}
    Imagine a radio channel as a set of 1024 parallel lanes for data, all of them moderately noisy. The mathematics of polar coding is a clever recursive process that "sorts" these lanes.
    \begin{itemize}
        \item After the process, some lanes become \textbf{perfectly clean, noise-free channels}.
        \item The rest become \textbf{completely useless, pure-noise channels}.
    \end{itemize}
    This effect is called \keyterm{channel polarisation}.

    \parhead{How it's used}
    The encoder now has a simple task:
    \begin{enumerate}
        \item Send the important \textbf{information bits} down the perfectly clean channels.
        \item Send simple, known patterns (usually just zeros), called \textbf{frozen bits}, down the useless channels.
    \end{enumerate}
    The receiver, knowing which channels were supposed to be frozen, can use this information to work backwards and perfectly decode the data from the good channels.

    \parhead{Why they are special} While Turbo and LDPC codes were discovered through clever engineering and experimentation to get close to the Shannon limit, Polar codes were the first to be constructed with a mathematical \emph{proof} of their optimality. This theoretical elegance, combined with their excellent performance for short messages, made them the ideal choice for the ultra-reliable, low-latency control channels of 5G.
\end{nontechnical}


\subsection{Overview}

Invented by Erdal Arıkan in 2008, \keyterm{Polar codes} represent a monumental achievement in information theory: they are the first explicitly constructed codes that are mathematically proven to achieve the Shannon capacity of a communication channel.

The core principle is \keyterm{channel polarisation}. Through a recursive transformation, a set of $N$ identical noisy channel uses are synthesised into a new set of $N$ virtual channels where, as $N \to \infty$, a fraction of these channels become perfectly noise-free, and the remainder become completely useless. By transmitting information only on the "good" channels and sending known "frozen" bits on the "bad" ones, reliable communication at rates approaching the Shannon limit is achieved.

\begin{keyconcept}
    Polar codes are defined by their provable capacity-achieving nature and their low-complexity \textbf{successive cancellation (SC)} decoding algorithm. Their excellent performance for short to medium block lengths and their low latency made them the winning candidate for the control channels in the \textbf{5G New Radio (NR)} standard.
\end{keyconcept}


\subsection{Code Construction and Encoding}

The construction of a polar code is a deterministic process.
\begin{description}
    \item[1. Channel Sorting] For a given channel quality (SNR) and block length $N$ (which must be a power of 2), a reliability sequence is calculated. This sequence ranks each of the $N$ virtual bit-channels from most reliable to least reliable.
    \item[2. Information and Frozen Sets] For a desired code rate $R=K/N$, the $K$ most reliable indices are chosen as the \keyterm{information set}, and the remaining $N-K$ indices form the \keyterm{frozen set}.
    \item[3. Encoding] The $K$ information bits are placed into the positions corresponding to the information set. The frozen set positions are filled with a known value, typically all zeros. This $N$-bit vector is then encoded by multiplying it with the $N \times N$ polar generator matrix, $G_N$. The encoding process can be implemented efficiently with a complexity of $O(N \log N)$.
\end{description}


\subsection{Decoding Algorithms}

\paragraph{Successive Cancellation (SC) Decoding}
The original decoding algorithm proposed by Arıkan. The SC decoder estimates the transmitted bits one by one, from $\hat{u}_1$ to $\hat{u}_N$. When it reaches an index in the information set, it makes a hard decision on that bit using the received signal and all previously decoded bits. When it reaches a frozen index, it already knows the bit's value (zero), and uses this known information to help decode the subsequent information bits. While it is capacity-achieving in the asymptotic limit, its performance for finite block lengths is modest.

\paragraph{Successive Cancellation List (SCL) Decoding}
A significant improvement over SC decoding. Instead of committing to a single decision at each step, the SCL decoder maintains a list of $L$ most likely candidate paths (typically $L=4$ or 8). At each stage, it expands all paths and prunes the list back down to the $L$ most probable candidates. At the end, it selects the most likely path from the final list. SCL decoding provides performance very close to that of a maximum-likelihood decoder.

\paragraph{CRC-Aided SCL Decoding (CA-SCL)}
The standard in 5G. A Cyclic Redundancy Check (CRC) is appended to the information bits before polar encoding. The SCL decoder runs as normal, and at the end, the decoder simply picks the one path from its list that satisfies the CRC check. This simple addition dramatically improves performance, making it competitive with, and often superior to, Turbo and LDPC codes for short block lengths.

\begin{table}[H]
    \centering
    \caption{Performance of Polar Code Decoders vs. Competitors}
    \label{tab:polar-performance}
    \begin{tabularx}{\textwidth}{@{}XXXX@{}}
        \toprule
        \tableheaderfont Decoder / Code & \tableheaderfont Gap to Shannon Limit & \tableheaderfont Latency & \tableheaderfont Primary Use Case \\
        \midrule
        Polar (SC) & 1.5 - 2.0 dB & Low & Theoretical baseline \\
        \textbf{Polar (CA-SCL)} & \textbf{0.8 - 1.2 dB} & \textbf{Low} & \textbf{5G Control Channels} \\
        Turbo Codes & 0.5 - 0.7 dB & High & 4G Data Channels \\
        LDPC Codes & 0.3 - 0.5 dB & Moderate & 5G Data Channels \\
        \bottomrule
    \end{tabularx}
\end{table}


\begin{workedexample}{5G NR Control Channel}
    \parhead{Problem} A 5G phone needs to send a 100-bit control message. Analyse the polar coding scheme used.
    \parhead{System Parameters}
    \begin{itemize}
        \item Information bits: 100.
        \item FEC Scheme: Polar code with an 11-bit CRC, as specified by 3GPP for this payload size.
        \item Total information block ($K$): $100 + 11 = 111$ bits.
        \item Target Code Rate ($R$): Approx. 1/3 for high reliability.
        \item Final Block Length ($N$): The smallest power of 2 that satisfies the rate, e.g., $N=256$.
    \end{itemize}
    \parhead{Process}
    \begin{derivationsteps}
        \step \textbf{Encoding.} The 100 data bits and 11 CRC bits are placed in the 111 most reliable positions of a 256-bit vector. The remaining 145 positions are "frozen" to zero. This vector is then polar encoded into a 256-bit codeword.
        \step \textbf{Transmission.} The 256 coded bits are modulated (e.g., using QPSK) and transmitted.
        \step \textbf{Decoding.} The receiver performs CA-SCL decoding. It maintains a list of the most likely decoded paths and, at the end, selects the single path that has a valid 11-bit CRC.
    \end{derivationsteps}
    \parhead{Interpretation} This CA-SCL scheme was chosen for 5G control channels because it offers excellent error-correction performance for short block lengths while maintaining the low latency required for real-time control signalling. The CRC provides both a powerful error detection mechanism and an elegant way to improve the SCL decoder's performance.
\end{workedexample}


\begin{importantbox}[title={Further Reading}]
    Polar codes are the most recent major development in a long history of channel coding.
    \begin{description}
        \item[Shannon's Channel Capacity] (\Cref{ch:shannon}) is the theoretical limit that polar codes are the first to be proven to achieve.
        \item[LDPC and Turbo Codes] (\Cref{ch:ldpc}, \Cref{ch:turbo-codes}) are the main competing capacity-approaching codes against which polar codes are benchmarked, particularly in the context of the 5G standardisation process.
        \item[Forward Error Correction] (\Cref{ch:fec}) provides the broader context for how polar codes fit into the family of FEC techniques.
    \end{description}
\end{importantbox}