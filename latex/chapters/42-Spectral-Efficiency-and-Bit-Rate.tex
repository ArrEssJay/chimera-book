% ==============================================================================
% CHAPTER 42: Spectral Efficiency
% ==============================================================================

\chapter{Spectral Efficiency}
\label{ch:spectral-efficiency}

\begin{nontechnical}
    \textbf{Spectral efficiency measures how efficiently a radio system uses its assigned slice of spectrum.} It's the ultimate metric of performance in a world where radio spectrum is a finite and incredibly expensive resource.

    \parhead{The highway analogy}
    \begin{itemize}
        \item \textbf{Bandwidth} is the width of your highway, measured in Hertz (Hz).
        \item \textbf{Bit Rate} is how much traffic (data) you are moving, measured in bits per second (bps).
        \item \textbf{Spectral Efficiency} is the amount of traffic you can fit per lane, measured in \textbf{bits per second per Hertz (bps/Hz)}.
    \end{itemize}

    \parhead{How to increase efficiency}
    To move more traffic on the same highway, you can either put more people in each car (higher-order modulation) or pack the cars closer together (more advanced coding).
    \begin{itemize}
        \item \textbf{QPSK (2 bits/symbol):} Like two motorcycles side-by-side in one lane ($\eta \approx 2$ bps/Hz).
        \item \textbf{64-QAM (6 bits/symbol):} Like a fully-loaded minivan ($\eta \approx 6$ bps/Hz).
        \item \textbf{1024-QAM (10 bits/symbol):} Like a double-decker bus ($\eta \approx 10$ bps/Hz).
    \end{itemize}

    \parhead{Why it matters}
    Wireless carriers spend billions of pounds on licenses for small slices of spectrum. Doubling their spectral efficiency means they can serve twice as many customers or offer twice the speed over the same licensed band. This is the primary economic driver behind the evolution from 3G to 4G to 5G.
\end{nontechnical}


\section{Overview and Properties}

\subsection{Overview}

\keyterm{Spectral efficiency}, $\eta$, is a measure of how much data can be transmitted over a given amount of spectrum. It is the single most important figure of merit for a bandwidth-limited communication system. It is defined as the net information rate (in bits/sec) divided by the occupied channel bandwidth (in Hz), and its unit is bits per second per Hertz (bps/Hz).

\begin{keyconcept}
    The spectral efficiency of a system is fundamentally limited by its Signal-to-Noise Ratio, as described by the \textbf{Shannon-Hartley theorem}. It is determined by three main factors:
    \begin{enumerate}
        \item The \textbf{modulation order ($M$)}, which sets the number of bits per symbol.
        \item The \textbf{FEC code rate ($R$)}, which determines the fraction of bits that are actual information.
        \item The \textbf{number of spatial streams ($N_s$)}, in a MIMO system.
    \end{enumerate}
\end{keyconcept}


\subsection{Fundamental Relationships}

The spectral efficiency of a single-carrier, single-antenna (SISO) system is given by:
\begin{equation}
    \eta = \frac{R_b}{B} = \frac{R_s \cdot \log_2(M) \cdot R}{R_s(1+\alpha)} = \frac{\log_2(M) \cdot R}{1+\alpha}
\end{equation}
where $M$ is the modulation order, $R$ is the code rate, and $\alpha$ is the filter roll-off factor. This shows that to increase spectral efficiency, one must either use a higher-order modulation or a higher code rate, both of which require a better signal-to-noise ratio.


\subsection{The Shannon Limit}

The theoretical maximum spectral efficiency for a given SNR is dictated by the Shannon capacity formula:
\begin{equation}
    \eta_{\max} = \frac{C}{B} = \log_2(1 + \text{SNR})
\end{equation}
This curve represents the absolute upper bound on performance. Modern communication systems, using capacity-approaching codes like LDPC and Turbo codes, are now able to operate within 1-2 dB of this fundamental limit.

\begin{table}[H]
    \centering
    \caption{Shannon Limit: Required SNR for a Target Spectral Efficiency}
    \label{tab:shannon-efficiency}
    \begin{tabular}{@{}cc@{}}
        \toprule
        \tableheaderfont Desired Spectral Efficiency ($\eta$) & \tableheaderfont Minimum Required SNR \\
        \midrule
        1 bps/Hz & 0.0 dB \\
        2 bps/Hz & 4.8 dB \\
        4 bps/Hz & 11.8 dB \\
        6 bps/Hz & 18.0 dB \\
        8 bps/Hz & 24.1 dB \\
        10 bps/Hz & 30.1 dB \\
        \bottomrule
    \end{tabular}
\end{table}


\subsection{Practical System Performance}

Real-world systems achieve a fraction of the Shannon limit due to practical constraints such as implementation losses, protocol overhead, and the need for a link margin.

\begin{workedexample}{WiFi 6 (802.11ax) Spectral Efficiency}
    \parhead{Problem} Calculate the maximum achievable spectral efficiency and the corresponding data rate for a single stream of WiFi 6 in an 80 MHz channel.
    \parhead{System Parameters}
    \begin{itemize}
        \item Channel Bandwidth: \qty{80}{MHz}.
        \item Highest Modulation: 1024-QAM ($M=1024 \implies 10$ bits/symbol).
        \item Highest Code Rate: $R=5/6$.
        \item OFDM Parameters: 980 data subcarriers out of 1024 total, CP overhead $\approx 1/8$.
    \end{itemize}
    \parhead{Solution}
    \begin{derivationsteps}
        \step \textbf{Calculate the raw spectral efficiency.} This ignores overhead.
        \[ \eta_{\text{raw}} = \log_2(M) \times R = 10 \times (5/6) \approx 8.33 \text{ bps/Hz} \]
        \step \textbf{Account for OFDM overhead.} Only a fraction of subcarriers carry data, and the cyclic prefix adds a time overhead.
        \[ \eta_{\text{coded}} = \eta_{\text{raw}} \times \left(\frac{N_{\text{data}}}{N_{\text{total}}}\right) \times \left(1 - \text{CP}_{\text{overhead}}\right) \]
        \[ \eta_{\text{coded}} = 8.33 \times \left(\frac{980}{1024}\right) \times \left(1 - \frac{1}{8}\right) \approx \textbf{7.0 bps/Hz} \]
        \step \textbf{Calculate the final data rate.}
        \[ R_b = \eta_{\text{coded}} \times B = 7.0 \text{ bps/Hz} \times (80 \times 10^6 \text{ Hz}) = \textbf{\qty{560}{Mbps}} \]
    \end{derivationsteps}
    \parhead{Interpretation} The maximum data rate for a single spatial stream is 560 Mbps. A 2x2 MIMO device could achieve double this rate, approx. 1.12 Gbps. To achieve this, the link requires an SNR of over 35 dB, which is only possible at very close range with no interference.
\end{workedexample}

\begin{importantbox}[title={Further Reading}]
    Spectral efficiency is the key performance indicator that drives the evolution of communication standards.
    \begin{description}
        \item[Shannon's Channel Capacity] (\Cref{ch:shannon}) provides the theoretical upper bound on what is achievable.
        \item[Modulation Schemes] (\Cref{ch:qam}) are the primary tool used to increase spectral efficiency in high-SNR environments.
        \item[MIMO Systems] (\Cref{ch:mimo}) explains how using multiple antennas can multiply the spectral efficiency by creating parallel spatial streams.
        \item[5G and WiFi Systems] (\Cref{ch:5g}, \Cref{ch:wifi}) are case studies in the relentless pursuit of higher spectral efficiency through a combination of advanced modulation, coding, and MIMO techniques.
    \end{description}
\end{importantbox}
