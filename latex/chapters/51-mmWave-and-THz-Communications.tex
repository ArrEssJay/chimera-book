% ==============================================================================
% CHAPTER 47: mmWave & THz Communications
% ==============================================================================

\chapter{mmWave \& THz Communications}
\label{ch:mmwave-thz}

\begin{nontechnical}
    \textbf{Millimetre-wave (mmWave) and Terahertz (THz) communications are the next frontier in wireless, offering fibre-optic speeds without the cables.} They use ultra-high frequencies to open up massive new "highways" for data.

    \parhead{The water pipe analogy}
    Imagine the available radio spectrum as a set of water pipes:
    \begin{itemize}
        \item \textbf{4G / traditional 5G (below 6 GHz):} A standard household pipe. Good flow, but limited capacity.
        \item \textbf{5G mmWave (24-100 GHz):} A city fire hydrant. Incredible flow rate, but over a shorter distance.
        \item \textbf{Future 6G (100-300 GHz):} A massive industrial water main. Capable of Tbps speeds, but only for very short, direct connections.
    \end{itemize}

    \parhead{The trade-off: Speed vs. range}
    The reason these ultra-high frequencies are not used for everything is their poor propagation.
    \begin{itemize}
        \item \textbf{They are Line-of-Sight:} Like a laser pointer, they cannot pass through walls, trees, or even your hand.
        \item \textbf{They have a short range:} Atmospheric absorption and rain weaken the signal significantly over distance.
    \end{itemize}

    \parhead{The enabling technology: Beamforming}
    To overcome the massive path loss, mmWave systems use large phased-array antennas to focus all their energy into a single, narrow, pencil-like beam, pointed directly at the user. This \keyterm{beamforming} is what makes the link possible, but it requires constant, precise tracking as a user moves.

    \parhead{Where you will see it}
    You will not see mmWave replacing traditional cellular everywhere. Instead, it will be used for "surgical" capacity injections in specific, high-demand locations: airports, stadiums, dense urban centres, and as a "fixed wireless" alternative to home fibre broadband.
\end{nontechnical}


\section{Overview and Properties}

\subsection{Overview}

\keyterm{Millimetre-wave (mmWave)} and \keyterm{Terahertz (THz)} communications refer to wireless systems operating in the frequency bands from approximately 24 GHz to 10 THz. The primary motivation for moving to these bands is the immense amount of available bandwidth. While the entire cellular spectrum below 6 GHz is only a few gigahertz wide, the mmWave bands offer tens of gigahertz of contiguous spectrum, and the THz bands offer terahertz of spectrum.

This vast bandwidth is the key to achieving the multi-gigabit and, eventually, terabit-per-second data rates envisioned for 5G-Advanced and 6G. However, these frequencies present extreme propagation challenges, including severe path loss, atmospheric absorption, and blockage by common objects.

\begin{keyconcept}
    The viability of mmWave and THz communication hinges entirely on the use of high-gain antennas and sophisticated \textbf{beamforming}. Large-scale phased arrays with hundreds or thousands of elements are used to generate narrow, steerable beams that can focus energy, overcome path loss, and track mobile users.
\end{keyconcept}


\subsection{Propagation Challenges}

\paragraph{Path Loss}
Free-Space Path Loss is proportional to the square of the frequency ($L \propto f^2$). A 28 GHz mmWave signal suffers over 20 dB more path loss than a 2.4 GHz WiFi signal over the same distance. This must be compensated for by antenna gain.

\paragraph{Atmospheric Absorption}
Molecular oxygen and water vapour have strong absorption lines in the mmWave and THz bands, most notably at 60 GHz (oxygen) and 183 GHz (water vapour). These absorption peaks create "windows" of usable spectrum but severely limit long-distance propagation.

\paragraph{Blockage}
At these high frequencies, wavelengths are in the millimetre or sub-millimetre range. This means that objects like walls, foliage, and even the human body are electrically large and act as complete blockers, not just attenuators. This makes a clear line-of-sight (LOS) path essential.


\subsection{Beamforming: The Enabling Technology}

\keyterm{Beamforming} is the technique of using a phased-array antenna to steer a narrow beam of radio energy in a specific direction. For mmWave and THz, this is not an optional feature; it is mandatory. The high gain of the beam (which scales with the number of antenna elements) is required to close the link budget.

\paragraph{Hybrid Beamforming}
Because implementing a full digital transceiver for each of the hundreds of antenna elements would be prohibitively expensive and power-hungry, modern systems use \keyterm{hybrid beamforming}. A small number of digital streams are first processed, and then a network of analogue phase shifters steers the beams for the large antenna array. This provides a practical balance between performance and complexity.

\paragraph{Beam Management}
With beamwidths of only a few degrees, the base station and the user device must constantly perform \keyterm{beam management} to maintain alignment. This involves:
\begin{itemize}
    \item \textbf{Beam Sweeping:} The base station periodically transmits synchronisation signals in different directions to allow users to identify the best beam.
    \item \textbf{Beam Tracking:} Once a link is established, the system must continuously refine the beam's direction to track user movement.
    \item \textbf{Beam Recovery:} A rapid procedure to find a new beam if the current one is suddenly blocked.
\end{itemize}


\subsection{5G New Radio (NR) in FR2 (mmWave)}

The 5G standard defines Frequency Range 2 (FR2) for mmWave operation, primarily in the 24, 28, and 39 GHz bands.
\begin{itemize}
    \item \textbf{Bandwidth:} Provides very wide channel bandwidths, from 50 MHz up to 400 MHz per component carrier.
    \item \textbf{Numerology:} Uses a flexible OFDM structure with a wide subcarrier spacing (e.g., 120 kHz) to create very short symbols, which makes the system more robust to the high Doppler shifts encountered at these frequencies.
    \item \textbf{Applications:} Primarily targeted at \keyterm{enhanced Mobile Broadband (eMBB)} in dense urban hotspots (stadiums, airports) and \keyterm{Fixed Wireless Access (FWA)} as a "last-mile" fibre alternative.
\end{itemize}


\begin{workedexample}{5G mmWave Link Budget}
    \parhead{Problem} Assess the feasibility of a 100-meter 5G mmWave link at 28 GHz.
    \parhead{System Parameters}
    \begin{itemize}
        \item Base Station (gNB): 256-element phased array, total transmit power of \qty{40}{dBm}.
        \item User Device (UE): 16-element phased array.
        \item Channel Bandwidth: \qty{100}{MHz}.
        \item Receiver Noise Figure: 8 dB.
    \end{itemize}
    \parhead{Analysis}
    \begin{derivationsteps}
        \step \textbf{Calculate Link Gains.} The array gain is approx. $10\log_{10}(N)$.
        \[ G_t \approx 10\log_{10}(256) = 24 \text{ dBi}; \quad G_r \approx 10\log_{10}(16) = 12 \text{ dBi} \]
        The total transmit power (EIRP) is $40 \text{ dBm} + 24 \text{ dBi} = \qty{64}{dBm}$.

        \step \textbf{Calculate Path Loss.}
        \[ \text{FSPL} = 20\log_{10}(100 \text{ m}) + 20\log_{10}(28 \text{ GHz}) + 20.4 \approx 40 + 29.0 + 20.4 = \qty{89.4}{dB} \]

        \step \textbf{Calculate Received Power.}
        \[ P_r = \text{EIRP} - L_{\text{FSPL}} + G_r = 64 - 89.4 + 12 = \qty{-13.4}{dBm} \]
        This is an extremely strong received signal. However, this is for a perfect LOS path. A more realistic scenario includes blockage. Assume a 30 dB loss from passing through a modern, energy-efficient window.
        \[ P_r (\text{NLOS}) = -13.4 - 30 = \qty{-43.4}{dBm} \]

        \step \textbf{Calculate SNR.} The noise floor is $N = -174 + 10\log_{10}(100 \times 10^6) + 8 = \qty{-86}{dBm}$.
        \[ \text{SNR} = P_r - N = -43.4 - (-86) = \textbf{\qty{42.6}{dB}} \]
    \end{derivationsteps}
    \parhead{Interpretation} Even with a significant 30 dB blockage loss, the combination of high transmit power and massive beamforming gains at both ends results in an excellent SNR of over 40 dB. This is more than sufficient to support the highest-order 256-QAM modulation, enabling multi-gigabit data rates. This demonstrates that while mmWave propagation is challenging, it is entirely feasible with advanced antenna technology.
\end{workedexample}

\begin{table}[H]
    \centering
    \caption{Comparison of Wireless Technology Generations}
    \label{tab:generation-comparison}
    \begin{tabularx}{\textwidth}{@{}lXXX@{}}
        \toprule
        \tableheaderfont Technology & \tableheaderfont Frequency Band & \tableheaderfont Key Technologies & \tableheaderfont Peak Data Rate \\
        \midrule
        4G (LTE-A) & < 6 GHz & MIMO-OFDM, Carrier Agg. & 1 Gbps \\
        5G (NR) & < 6 GHz + mmWave & Massive MIMO, Beamforming & 10-20 Gbps \\
        6G (Future) & Sub-THz & Ultra-Massive MIMO, RIS & >100 Gbps \\
        \bottomrule
    \end{tabularx}
\end{table}

\begin{importantbox}[title={Further Reading}]
    mmWave and THz are the future of high-capacity wireless, building directly on many concepts in this book.
    \begin{description}
        \item[MIMO Systems] (\Cref{ch:mimo}) provides the foundational theory for the beamforming and spatial multiplexing techniques that are mandatory for mmWave.
        \item[OFDM] (\Cref{ch:ofdm}) is the modulation scheme used, but with a flexible "numerology" to adapt to the unique challenges of the mmWave channel.
        \item[Antenna Theory] (\Cref{ch:antenna}) discusses the principles of the phased-array antennas that are the key enabling hardware.
    \end{description}
\end{importantbox}
