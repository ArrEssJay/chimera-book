% ==============================================================================
% CHAPTER 10: Bit Error Rate (BER)
% ==============================================================================

\chapter{Bit Error Rate (BER)}
\label{ch:ber}

\begin{nontechnical}
    \textbf{Bit Error Rate (BER) measures how many "typos" occur when transmitting digital data.} It is the single most important measure of the quality and reliability of a digital communication link.

    \parhead{Simple idea}
    \begin{itemize}
        \item Wireless signals are inevitably corrupted by noise (like static on an old radio).
        \item This noise can cause the receiver to make mistakes, flipping a transmitted "1" into a "0" or vice-versa.
        \item BER is simply the fraction of bits that get flipped during transmission.
    \end{itemize}

    \parhead{Real-world impact}
    \begin{itemize}
        \item \textbf{Pixelated Video or Garbled Audio:} A high BER means corrupted data, leading to noticeable artifacts. A BER of $10^{-3}$ (one error per 1,000 bits) is often the threshold for acceptable voice quality.
        \item \textbf{Slow Internet:} Your WiFi router constantly monitors the error rate. If the BER gets too high, it will automatically switch to a slower but more robust modulation scheme to maintain a stable connection.
        \item \textbf{Corrupted File Downloads:} For data files, even a single bit error can render the entire file unusable. This is why data links require a very low BER, often less than $10^{-9}$.
    \end{itemize}
\end{nontechnical}


\section{Overview and Properties}

\subsection{Overview}

The \keyterm{Bit Error Rate (BER)} is the fundamental performance metric for any digital communication system. It is defined as the number of bit errors received divided by the total number of bits transmitted. BER provides a quantitative measure of the end-to-end quality of a link, encapsulating the combined effects of modulation, channel noise, interference, and receiver performance.

\begin{keyconcept}
    The BER of a system is a direct, exponential function of the received $E_b/N_0$. A small improvement in the energy-to-noise ratio can lead to a dramatic, orders-of-magnitude improvement in the BER. This "waterfall" relationship is the central characteristic of digital communication performance curves and the primary focus of link budget design.
\end{keyconcept}


\subsection{Definition and Measurement}

BER is the ratio of the number of bit errors, $N_e$, to the total number of transmitted bits, $N_t$:
\begin{equation}
    \text{BER} = \frac{N_e}{N_t}
\end{equation}
It is a dimensionless quantity, often expressed in scientific notation (e.g., $10^{-6}$). To measure BER, a known sequence of bits, such as a Pseudo-Random Binary Sequence (PRBS), is sent through the system. The receiver compares the decoded bits to the known sequence and counts the errors.

\begin{warningbox}
    Measuring low BER values requires a statistically significant number of bits. To confidently measure a BER of $10^{-9}$, one must transmit at least $10^{11}$ bits and observe approximately 100 errors. For a 1 Gbps link, this would take over a minute and a half.
\end{warningbox}


\subsection{Theoretical BER in an AWGN Channel}

For a channel impaired only by Additive White Gaussian Noise (AWGN), the BER can be calculated theoretically. The probability of error is determined by the probability that the noise voltage exceeds the decision threshold, which is given by the Gaussian \keyterm{Q-function}.

\paragraph{BPSK and QPSK Performance}
For coherent BPSK and QPSK, which are the most power-efficient modulation schemes, the theoretical BER is given by:
\begin{equation}
    \text{BER} = Q\left(\sqrt{\frac{2E_b}{N_0}}\right)
\end{equation}
where $E_b/N_0$ is the energy per bit to noise density ratio.

\paragraph{Higher-Order Modulation}
More spectrally efficient but less power-efficient schemes like M-QAM have a higher BER for the same $E_b/N_0$ because their constellation points are packed closer together, making them more susceptible to noise.

\begin{table}[H]
    \centering
    \caption{Required $E_b/N_0$ for a BER of $10^{-5}$ (Uncoded, AWGN Channel)}
    \label{tab:ber-reqs-uncoded}
    \begin{tabular}{@{}lc@{}}
        \toprule
        \tableheaderfont Modulation Scheme & \tableheaderfont Required $E_b/N_0$ (dB) \\
        \midrule
        BPSK / QPSK & 9.6 \\
        8-PSK & 14.0 \\
        16-QAM & 14.5 \\
        64-QAM & 18.8 \\
        \bottomrule
    \end{tabular}
\end{table}


\subsection{Forward Error Correction (FEC) and Coding Gain}

Modern communication systems employ \keyterm{Forward Error Correction (FEC)} to dramatically improve BER performance. FEC adds carefully structured redundant bits to the data stream, allowing the receiver to detect and correct a certain number of errors without needing retransmission.
\begin{description}
    \item[Pre-FEC BER] The "raw" bit error rate of the channel before the FEC decoder. This reflects the actual quality of the physical link.
    \item[Post-FEC BER] The final, corrected bit error rate delivered to the user. This is the ultimate measure of system performance.
\end{description}
The improvement provided by FEC is known as \keyterm{coding gain}. It is the reduction in the required $E_b/N_0$ to achieve a given post-FEC BER, compared to an uncoded system. Modern codes like LDPC and Turbo codes can provide 8--11 dB of coding gain, allowing systems to operate reliably at much lower SNRs.

\begin{center}
    \begin{tikzpicture}[scale=1.0]
        \begin{axis}[
            width=12cm, height=8cm,
            xlabel={$E_b/N_0$ (dB)},
            ylabel={BER},
            ymode=log,
            ymin=1e-8, ymax=1e-1,
            xmin=0, xmax=12,
            grid=major,
            legend pos=south west,
            font=\sffamily,
        ]
        % Uncoded QPSK BER curve (pre-calculated data points)
        \addplot[thick, diagramprimary, smooth] coordinates {
            (0, 7.86e-2) (1, 6.21e-2) (2, 4.55e-2) (3, 3.09e-2) (4, 1.91e-2)
            (5, 1.08e-2) (6, 5.59e-3) (7, 2.60e-3) (8, 1.08e-3) (9, 3.87e-4)
            (10, 1.24e-4) (11, 3.45e-5) (12, 8.37e-6)
        };
        \addlegendentry{Uncoded QPSK}
        
        % QPSK with 8 dB coding gain (shifted left by 8 dB)
        \addplot[thick, diagramsecondary, smooth, dashed] coordinates {
            (0, 3.87e-4) (1, 1.24e-4) (2, 3.45e-5) (3, 8.37e-6) (4, 1.78e-6)
            (5, 3.29e-7) (6, 5.35e-8) (7, 7.59e-9) (8, 9.34e-10) (9, 9.99e-11)
            (10, 9.33e-12) (11, 7.48e-13) (12, 5.18e-14)
        };
        \addlegendentry{QPSK with FEC (8 dB Coding Gain)}
        
        \draw[<->, thick] (axis cs:2.5,1e-5) -- (axis cs:10.5,1e-5);
        \node[above, font=\sffamily\small] at (axis cs:6.5,1e-5) {Coding Gain $\approx$ 8 dB};
        
        \draw[dotted, thick] (axis cs:0,1e-5) -- (axis cs:10.5,1e-5);
        \node[left, font=\sffamily\small] at (axis cs:0,1e-5) {Target BER $10^{-5}$};
        \end{axis}
    \end{tikzpicture}
\end{center}


\begin{workedexample}{Satellite Link BER Analysis}
    \parhead{Problem} For the satellite link designed in the previous chapter, determine the final link margin required to achieve a post-FEC BER of $10^{-7}$.
    \parhead{System Parameters}
    \begin{itemize}
        \item Modulation: QPSK
        \item Target Post-FEC BER: $10^{-7}$
        \item FEC: A modern code providing 8 dB of coding gain.
        \item Implementation Margin: 2.5 dB
        \item Available $E_b/N_0$: 23.3 dB (from previous chapter's calculation)
    \end{itemize}
    \parhead{Solution}
    \begin{derivationsteps}
        \step Find the required $E_b/N_0$ for an \emph{uncoded} system to achieve the target BER. For QPSK, a BER of $10^{-7}$ requires a theoretical $E_b/N_0$ of approximately 11.3 dB.
        \step Apply the coding gain. The FEC allows the system to achieve this performance at a much lower input $E_b/N_0$.
        \[ \text{Required } E_b/N_0 \text{ (with FEC)} = 11.3 \text{ dB} - 8.0 \text{ dB (Coding Gain)} = 3.3 \text{ dB} \]
        \step Add the implementation margin. This accounts for real-world hardware imperfections.
        \[ \text{Total Required } E_b/N_0 = 3.3 \text{ dB} + 2.5 \text{ dB (Margin)} = 5.8 \text{ dB} \]
        \step Calculate the final link margin.
        \[ \text{Link Margin} = (\text{Available}) - (\text{Required}) = 23.3 - 5.8 = \qty{17.5}{dB} \]
    \end{derivationsteps}
    \parhead{Interpretation} The system has a very large link margin of 17.5 dB. This is more than sufficient to handle even severe rain fade and other impairments, ensuring a highly reliable link. The powerful FEC is the key enabler, reducing the required SNR by a factor of more than six.
\end{workedexample}


\begin{importantbox}[title={Further Reading}]
    BER is the ultimate measure of link performance and is central to the design of all digital systems.
    \begin{description}
        \item[Energy Ratios ($E_b/N_0$)] (\Cref{ch:energy-ratios}) is the fundamental metric that determines the theoretical BER.
        \item[Forward Error Correction] (\Cref{ch:fec}) describes the powerful coding techniques used to dramatically reduce the BER in practical systems.
        \item[Link Budget Analysis] (\Cref{ch:linkbudget}) is the end-to-end process of calculating all gains and losses in a system to ensure the final, available $E_b/N_0$ is sufficient to meet the target BER.
    \end{description}
\end{importantbox}
