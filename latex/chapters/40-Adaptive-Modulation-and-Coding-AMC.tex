% ==============================================================================
% CHAPTER 39: Adaptive Modulation and Coding (AMC)
% ==============================================================================

\chapter{Adaptive Modulation \& Coding (AMC)}
\label{ch:amc}

\begin{nontechnical}
    \textbf{Adaptive Modulation and Coding (AMC) is like an automatic gearbox for your wireless connection.} It constantly senses the "road conditions" (the signal quality) and shifts to the perfect "gear" (the modulation and coding scheme) to maximise speed without sacrificing reliability.

    \parhead{The gearbox analogy}
    \begin{itemize}
        \item \textbf{Clear, open highway (strong signal):} The system shifts into its highest gear (e.g., 256-QAM with light error correction) for maximum speed.
        \item \textbf{Winding country road (medium signal):} It shifts down to a medium gear (e.g., 16-QAM) for a balance of speed and control.
        \item \textbf{Steep, icy hill (weak signal):} It shifts into its lowest, most powerful gear (e.g., QPSK with heavy error correction) to ensure the connection makes it through, even if it's slow.
    \end{itemize}

    \parhead{How it works}
    Your phone or laptop is constantly measuring the signal quality (the SNR) and reporting it back to the base station or router as a \textbf{Channel Quality Indicator (CQI)}. The transmitter then consults a pre-defined table of \textbf{Modulation and Coding Schemes (MCS)} and picks the fastest one that will work reliably for that reported quality. This entire feedback loop happens hundreds of times per second.

    \parhead{Why it's essential}
    Without AMC, a system would have to be designed for the worst-case channel conditions, meaning it would be unnecessarily slow most of the time. AMC allows a system to opportunistically exploit good channel conditions to deliver the highest possible data rate at any given moment. It is the core technology that makes high-speed, reliable mobile broadband a reality.
\end{nontechnical}


\section{Overview and Properties}

\subsection{Overview}

In a wireless environment, the channel quality can fluctuate dramatically due to fading, mobility, and interference. A fixed modulation and coding scheme that is robust enough for the worst-case conditions will be grossly inefficient when channel conditions are good. \keyterm{Adaptive Modulation and Coding (AMC)} is the technique that solves this problem by dynamically adapting the transmission parameters to match the instantaneous channel state.

\begin{keyconcept}
    AMC is a closed-loop system that allows a transmitter to track the time-varying Shannon Capacity of a channel. By selecting the highest-order modulation and weakest error correction code that can be supported by the current Signal-to-Noise Ratio, AMC maximises data throughput while maintaining a target Bit Error Rate. It is the key enabling technology for high-performance in all modern wireless standards, including 4G LTE, 5G NR, and WiFi.
\end{keyconcept}


\subsection{The Link Adaptation Framework}

The AMC process involves a continuous feedback loop between the receiver and the transmitter.
\begin{description}
    \item[1. Channel Estimation] The receiver measures the quality of the channel, typically by analysing known pilot or reference signals embedded in the transmission. This yields an estimate of the instantaneous SNR.
    \item[2. CQI Reporting] The receiver quantises this SNR value into a \keyterm{Channel Quality Indicator (CQI)} index and sends this small piece of feedback to the transmitter over a reverse control channel.
    \item[3. MCS Selection] The transmitter receives the CQI and uses it as an index into a predefined \keyterm{Modulation and Coding Scheme (MCS)} table. This table specifies the optimal combination of modulation (e.g., QPSK, 16-QAM, 64-QAM) and FEC code rate (e.g., 1/2, 2/3, 3/4) for that level of channel quality.
    \item[4. Data Transmission] The transmitter then formats and sends the next data packet using the selected MCS.
\end{description}
This entire loop repeats continuously, allowing the system to adapt to changes in the channel on a millisecond timescale.


\subsection{Performance and Throughput}

The gain from AMC is substantial. Instead of being limited by a fixed data rate designed for poor conditions, the system can achieve a much higher \keyterm{average throughput} by taking advantage of moments when the channel is strong.

\begin{table}[H]
    \centering
    \caption{Example MCS Table for a 4G LTE System}
    \label{tab:mcs-table}
    \begin{tabularx}{\textwidth}{@{}XXcc@{}}
        \toprule
        \tableheaderfont CQI Index & \tableheaderfont Modulation & \tableheaderfont Code Rate & \tableheaderfont Approx. Required SNR (dB) \\
        \midrule
        1--6 (Low Quality) & QPSK & 1/3 -- 1/2 & < 5 dB \\
        7--9 (Medium Quality) & 16-QAM & 1/2 -- 2/3 & 5 -- 12 dB \\
        10--15 (High Quality) & 64-QAM & 2/3 -- 5/6 & > 12 dB \\
        \bottomrule
    \end{tabularx}
\end{table}

\begin{warningbox}
    \textbf{The problem of feedback delay.} In a rapidly moving environment (e.g., a user in a car), the channel can change significantly in the time it takes for the CQI to be measured, sent back to the transmitter, and acted upon. If the channel quality degrades during this delay, the chosen MCS may be too aggressive, leading to a block error. This is why a robust system combines AMC with a fast retransmission protocol like \keyterm{Hybrid ARQ (HARQ)}.
\end{warningbox}


\begin{workedexample}{Throughput Gain of AMC}
    \parhead{Problem} Compare the average throughput of a fixed-rate system versus an AMC system in a simple fading channel.
    \parhead{Channel Conditions}
    Assume the channel is in one of two states 50\% of the time:
    \begin{itemize}
        \item \textbf{Good State:} SNR = 20 dB, capable of supporting 64-QAM (6 bits/symbol).
        \item \textbf{Bad State:} SNR = 10 dB, only capable of supporting QPSK (2 bits/symbol).
    \end{itemize}
    
    \parhead{Analysis}
    \begin{derivationsteps}
        \step \textbf{Fixed-Rate System.} To guarantee reliability, the system must be designed for the worst-case "Bad State". It must therefore use QPSK all the time.
        \[ \text{Throughput}_{\text{fixed}} = 2 \text{ bits/symbol} \]
        
        \step \textbf{AMC System.} The system adapts its modulation to the conditions.
        \begin{itemize}
            \item 50\% of the time (Good State): It uses 64-QAM, achieving a rate of 6 bits/symbol.
            \item 50\% of the time (Bad State): It uses QPSK, achieving a rate of 2 bits/symbol.
        \end{itemize}
        The average throughput is the weighted average of these two rates.
        \[ \text{Throughput}_{\text{AMC}} = (0.5 \times 6) + (0.5 \times 2) = 3 + 1 = 4 \text{ bits/symbol} \]
    \end{derivationsteps}
    
    \parhead{Interpretation} The AMC system achieves an average throughput of 4 bits/symbol, which is \textbf{double the throughput} of the fixed-rate system. It has successfully converted the channel's time-varying quality into a higher average data rate.
\end{workedexample}


\begin{importantbox}[title={Further Reading}]
    AMC is a system-level technique that integrates modulation, coding, and channel estimation.
    \begin{description}
        \item[Shannon's Channel Capacity] (\Cref{ch:shannon}) provides the theoretical justification for AMC, showing that the maximum data rate is a direct function of the instantaneous SNR.
        \item[Modulation Schemes] (\Cref{ch:qam}, \Cref{ch:qpsk}) are the "gears" that the AMC system shifts between.
        \item[Forward Error Correction] (\Cref{ch:fec}) provides the different code rates that are selected in conjunction with the modulation order to form a complete MCS.
        \item[5G and WiFi Systems] (\Cref{ch:5g}, \Cref{ch:wifi}) provide detailed case studies of how AMC is implemented in the world's most advanced wireless standards.
    \end{description}
\end{importantbox}
