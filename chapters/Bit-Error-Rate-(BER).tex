\section{Bit Error Rate (BER)}\label{bit-error-rate-ber}

\subsection{\texorpdfstring{ For Non-Technical
Readers}{ For Non-Technical Readers}}\label{for-non-technical-readers}

\textbf{BER measures how many mistakes happen when transmitting digital
data-\/-\/-like counting typos in a text message.}

When you send data wirelessly, noise can flip bits
(0\$\textbackslash leftrightarrow\$1 or
1\$\textbackslash leftrightarrow\$0). BER counts how often this happens.

\textbf{Real examples}: - \textbf{Pixelated video}: High BER
\$\textbackslash rightarrow\$ corrupted data
\$\textbackslash rightarrow\$ artifacts - \textbf{Dropped calls}: BER
\textgreater{}
10\textbackslash textsuperscript\{-\}\textbackslash textsuperscript\{3\}
(1 error per 1000 bits) \$\textbackslash rightarrow\$ bad quality -
\textbf{Corrupted downloads}: Even 1 flipped bit can break a file!

\textbf{Acceptable levels}: Voice =
10\textbackslash textsuperscript\{-\}\textbackslash textsuperscript\{3\}
OK, Data = need \textless{}
10\textbackslash textsuperscript\{-\}\textbackslash textsuperscript\{6\},
Banking = \textless{}
10\textbackslash textsuperscript\{-\}\textbackslash textsuperscript\{1\}\textbackslash textsuperscript\{2\}

\textbf{Improve BER}: Move closer to WiFi, use error correction, slow
down transmission rate.

\textbf{Fun fact}: WiFi automatically adjusts speed based on
BER-\/-\/-closer = faster (low errors), farther = slower (keep errors
acceptable).

\begin{center}\rule{0.5\linewidth}{0.5pt}\end{center}

\textbf{Bit Error Rate (BER)} is the ratio of incorrectly decoded bits
to total transmitted bits.

\subsection{Definition}\label{definition}

\begin{verbatim}
BER = Number of Bit Errors / Total Number of Bits
\end{verbatim}

\subsection{BER Scale}\label{ber-scale}

BER is typically expressed as a decimal or in scientific notation:

{\def\LTcaptype{} % do not increment counter
\begin{longtable}[]{@{}lll@{}}
\toprule\noalign{}
BER Value & Meaning & Quality \\
\midrule\noalign{}
\endhead
\bottomrule\noalign{}
\endlastfoot
10\textbackslash textsuperscript\{-\}\textbackslash textsuperscript\{1\}
& 1 error per 10 bits & Terrible \\
10\textbackslash textsuperscript\{-\}\textbackslash textsuperscript\{2\}
& 1 error per 100 bits & Very Poor \\
10\textbackslash textsuperscript\{-\}\textbackslash textsuperscript\{3\}
& 1 error per 1,000 bits & Poor \\
10\textbackslash textsuperscript\{-\}\textbackslash textsuperscript\{4\}
& 1 error per 10,000 bits & Marginal \\
10\textbackslash textsuperscript\{-\}\textbackslash textsuperscript\{6\}
& 1 error per 1,000,000 bits & Good \\
10\textbackslash textsuperscript\{-\}\textbackslash textsuperscript\{9\}
& 1 error per 1 billion bits & Excellent \\
10\textbackslash textsuperscript\{-\}\textbackslash textsuperscript\{1\}\textbackslash textsuperscript\{2\}
& 1 error per 1 trillion bits & Exceptional \\
\end{longtable}
}

\subsection{Pre-FEC vs Post-FEC BER}\label{pre-fec-vs-post-fec-ber}

\subsubsection{Pre-FEC BER}\label{pre-fec-ber}

Error rate \textbf{before} error correction - Directly reflects channel
quality - Higher at low SNR - Called ``raw BER'' or ``channel BER''

\subsubsection{Post-FEC BER}\label{post-fec-ber}

Error rate \textbf{after} error correction (LDPC decoding) - Shows
effectiveness of error correction - Should be much lower than Pre-FEC -
The ``residual errors'' that couldn\textquotesingle t be corrected

\begin{verbatim}
Example:
Pre-FEC BER:  10² (1 error per 100 bits)
              
        [LDPC Decoder]
              
Post-FEC BER: 10 (1 error per million bits)

Coding Gain: 40 dB improvement! 
\end{verbatim}

\subsection{BER vs SNR Curves}\label{ber-vs-snr-curves}

A BER vs SNR curve shows system performance:

\begin{verbatim}
BER
 
 |         
10|        Unusable
   |        
10³|        Poor
   |      
10|        ___ Good (threshold)
   |            +___
10|                +___ Excellent
   |
   +--------------------- SNR (dB)
    -5  0   5  10  15  20
\end{verbatim}

\subsubsection{Key Features}\label{key-features}

\begin{itemize}
\tightlist
\item
  \textbf{Waterfall region}: Steep decrease in BER as SNR increases
\item
  \textbf{Threshold}: SNR where BER becomes acceptable (often
  10\textbackslash textsuperscript\{-\}\textbackslash textsuperscript\{6\})
\item
  \textbf{Error floor}: Minimum achievable BER (implementation limits)
\end{itemize}

\subsection{Theoretical vs Measured
BER}\label{theoretical-vs-measured-ber}

\subsubsection{Theoretical BER for QPSK}\label{theoretical-ber-for-qpsk}

\begin{verbatim}
BER_QPSK  (1/2) · erfc((Eb/N0))
\end{verbatim}

\subsubsection{In Chimera}\label{in-chimera}

\begin{itemize}
\tightlist
\item
  \textbf{Theoretical}: Based on the formula above
\item
  \textbf{Measured}: Actual errors observed in simulation
\item
  \textbf{Difference}: Processing gain, implementation effects, finite
  sample size
\end{itemize}

\subsection{Factors Affecting BER}\label{factors-affecting-ber}

\begin{enumerate}
\def\labelenumi{\arabic{enumi}.}
\tightlist
\item
  \textbf{{[}{[}Signal-to-Noise-Ratio-(SNR){]}{]}}: Primary factor

  \begin{itemize}
  \tightlist
  \item
    Higher SNR \$\textbackslash rightarrow\$ Lower BER
  \end{itemize}
\item
  \textbf{Modulation Scheme}:

  \begin{itemize}
  \tightlist
  \item
    QPSK more robust than 16QAM
  \item
    Lower order = better BER at same SNR
  \end{itemize}
\item
  \textbf{{[}{[}Forward-Error-Correction-(FEC){]}{]}}:

  \begin{itemize}
  \tightlist
  \item
    Can reduce BER by orders of magnitude
  \item
    LDPC codes provide near-optimal performance
  \end{itemize}
\item
  \textbf{Channel Impairments}:

  \begin{itemize}
  \tightlist
  \item
    Phase noise, frequency offset
  \item
    Timing errors, multipath
  \end{itemize}
\item
  \textbf{Implementation}:

  \begin{itemize}
  \tightlist
  \item
    Quantization effects
  \item
    Synchronization accuracy
  \end{itemize}
\end{enumerate}

\subsection{BER in Chimera}\label{ber-in-chimera}

Chimera displays multiple BER metrics:

\subsubsection{Pre-FEC Metrics}\label{pre-fec-metrics}

\begin{itemize}
\tightlist
\item
  \textbf{Symbol Errors}: Count of incorrect symbol decisions
\item
  \textbf{Bit Errors (Pre-FEC)}: Bit errors before LDPC decoding
\item
  \textbf{Pre-FEC BER}: Bit error rate at demodulator output
\end{itemize}

\subsubsection{Post-FEC Metrics}\label{post-fec-metrics}

\begin{itemize}
\tightlist
\item
  \textbf{Bit Errors (Post-FEC)}: Residual errors after LDPC
\item
  \textbf{Post-FEC BER}: Final bit error rate
\item
  \textbf{Frame Error Rate (FER)}: Percentage of frames with
  uncorrectable errors
\end{itemize}

\subsubsection{Example Output}\label{example-output}

\begin{verbatim}
Pre-FEC BER:  2.3 × 10² (2.3% bit errors)
Post-FEC BER: 0 (all errors corrected!) 
FER:          0% (no frame errors)
\end{verbatim}

\subsection{Acceptable BER Thresholds}\label{acceptable-ber-thresholds}

Different applications have different requirements:

{\def\LTcaptype{} % do not increment counter
\begin{longtable}[]{@{}lll@{}}
\toprule\noalign{}
Application & Required BER & Rationale \\
\midrule\noalign{}
\endhead
\bottomrule\noalign{}
\endlastfoot
Voice (analog) &
10\textbackslash textsuperscript\{-\}\textbackslash textsuperscript\{3\}
& Some crackling acceptable \\
Data (with retransmission) &
10\textbackslash textsuperscript\{-\}\textbackslash textsuperscript\{4\}
-
10\textbackslash textsuperscript\{-\}\textbackslash textsuperscript\{6\}
& Retries handle errors \\
Streaming video &
10\textbackslash textsuperscript\{-\}\textbackslash textsuperscript\{6\}
& Occasional glitch OK \\
File transfer &
10\textbackslash textsuperscript\{-\}\textbackslash textsuperscript\{9\}
& Data integrity critical \\
Financial transactions &
10\textbackslash textsuperscript\{-\}\textbackslash textsuperscript\{1\}\textbackslash textsuperscript\{2\}
& Zero tolerance \\
\end{longtable}
}

\subsection{See Also}\label{see-also}

\begin{itemize}
\tightlist
\item
  {[}{[}Signal-to-Noise-Ratio-(SNR){]}{]} - Primary BER determinant
\item
  {[}{[}Forward-Error-Correction-(FEC){]}{]} - BER improvement technique
\item
  {[}{[}Energy-Ratios-(Es-N0-and-Eb-N0){]}{]} - Used in BER formulas
\item
  {[}{[}Understanding BER Curves{]}{]} - Interpreting performance plots
\end{itemize}
