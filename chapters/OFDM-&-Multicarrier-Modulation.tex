\section{OFDM \& Multicarrier
Modulation}\label{ofdm-multicarrier-modulation}

\subsection{\texorpdfstring{ For Non-Technical
Readers}{ For Non-Technical Readers}}\label{for-non-technical-readers}

\textbf{OFDM is like splitting a highway into many lanes-\/-\/-if one
lane has an accident (interference), the other lanes keep traffic
flowing!}

\textbf{The problem OFDM solves}: - Sending data fast on one channel =
short pulses = easily disrupted by echoes/reflections -
It\textquotesingle s like trying to drive 100 mph on a narrow
road-\/-\/-one pothole ruins everything!

\textbf{The OFDM solution}: - Split data across \textbf{hundreds or
thousands} of narrow ``lanes'' (subcarriers) - Each lane moves slowly
(easier to handle) - If one lane fades or gets interference, you still
have 999 others working!

\textbf{Real-world example - WiFi}: - \textbf{WiFi 5 (802.11ac)}: Uses
52-468 subcarriers - \textbf{WiFi 6 (802.11ax)}: Uses up to 1960
subcarriers! - Each subcarrier is only 78 kHz wide (vs 20-160 MHz total
channel) - It\textquotesingle s like delivering packages: 1 huge truck
(risky!) vs 100 small vans (resilient!)

\textbf{Why it\textquotesingle s everywhere}: - \textbf{WiFi}: All
modern WiFi uses OFDM - \textbf{4G/5G}: LTE and 5G NR are OFDM-based -
\textbf{Digital TV}: DVB-T uses OFDM for broadcast - \textbf{DSL}: Even
wired broadband uses OFDM (DMT variant)!

\textbf{How you experience it}: WiFi works in your house even with
walls/furniture blocking some frequencies-\/-\/-OFDM automatically uses
the clear subcarriers and avoids the blocked ones.

\textbf{Fun fact}: OFDM uses a clever math trick (FFT) to pack
subcarriers so tightly they overlap without
interfering-\/-\/-it\textquotesingle s called ``orthogonality'' (like
fingers interlaced).

\begin{center}\rule{0.5\linewidth}{0.5pt}\end{center}

\textbf{Orthogonal Frequency-Division Multiplexing (OFDM)} is a
multicarrier modulation technique that divides a wideband channel into
many narrow, orthogonal subcarriers. It has become the foundation of
modern wireless standards including WiFi (802.11a/g/n/ac/ax), LTE, 5G
NR, and DVB-T.

\begin{center}\rule{0.5\linewidth}{0.5pt}\end{center}

\subsection{\texorpdfstring{ The Core
Concept}{ The Core Concept}}\label{the-core-concept}

\textbf{Single-carrier problem}: High-speed data
\$\textbackslash rightarrow\$ short symbol duration
\$\textbackslash rightarrow\$ susceptible to multipath fading and
intersymbol interference (ISI).

\textbf{OFDM solution}: Divide spectrum into N narrow subcarriers
\$\textbackslash rightarrow\$ each carries low-rate data
\$\textbackslash rightarrow\$ longer symbol duration
\$\textbackslash rightarrow\$ robust against multipath.

\begin{verbatim}
Single Carrier (100 Mbps):
||  Wide, fast, ISI-prone
      OFDM Transformation 
Multi-carrier (100 Mbps total):
|||||||||||||||||  N narrow, slow, ISI-resistant
 1 2 3 4 ... N subcarriers
\end{verbatim}

\begin{center}\rule{0.5\linewidth}{0.5pt}\end{center}

\subsection{\texorpdfstring{ Mathematical
Foundation}{ Mathematical Foundation}}\label{mathematical-foundation}

\subsubsection{Orthogonality Condition}\label{orthogonality-condition}

Subcarriers are \textbf{orthogonal} when their frequencies are spaced by
1/T:

\begin{verbatim}
f = f + k·f

where:
- f = center frequency
- k = subcarrier index (0, 1, 2, ..., N-1)
- f = subcarrier spacing = 1/T_symbol
- T_symbol = OFDM symbol duration
\end{verbatim}

\textbf{Orthogonality integral}:

\begin{verbatim}
^{T} exp(j2·f·t) · exp(-j2·f·t) dt = { T  if k = m
                                         { 0  if k  m
\end{verbatim}

This ensures subcarriers don\textquotesingle t interfere despite
\textbf{spectral overlap}.

\begin{center}\rule{0.5\linewidth}{0.5pt}\end{center}

\subsubsection{IFFT/FFT Implementation}\label{ifftfft-implementation}

\textbf{Key insight}: OFDM modulation/demodulation is mathematically
equivalent to Inverse Fast Fourier Transform (IFFT) and FFT.

\textbf{Transmitter (IFFT)}:

\begin{verbatim}
x[n] = (1/N) · ^(N-1) X · exp(j2kn/N)

where:
- X = complex data symbol on subcarrier k (from QAM/PSK constellation)
- x[n] = time-domain OFDM sample
- N = number of subcarriers (typically 64, 128, 256, 512, 1024, 2048)
\end{verbatim}

\textbf{Receiver (FFT)}:

\begin{verbatim}
Y = (1/N) · ^(N-1) y[n] · exp(-j2kn/N)

where:
- y[n] = received time-domain samples
- Y = recovered symbol on subcarrier k
\end{verbatim}

\textbf{Computational advantage}: FFT reduces complexity from
O(N\textbackslash textsuperscript\{2\}) to O(N log N).

\begin{center}\rule{0.5\linewidth}{0.5pt}\end{center}

\subsection{\texorpdfstring{ OFDM System
Architecture}{ OFDM System Architecture}}\label{ofdm-system-architecture}

\subsubsection{Transmitter Block
Diagram}\label{transmitter-block-diagram}

\begin{verbatim}
Data bits
   
Serial-to-Parallel Converter (splits into N streams)
   
QAM/PSK Mapper (maps each stream to constellation point)
   
Pilot Insertion & Subcarrier Mapping
   
IFFT (N-point)
   
Add Cyclic Prefix (CP)
   
Parallel-to-Serial Converter
   
D/A Converter & RF Upconversion
   
Antenna
\end{verbatim}

\subsubsection{Receiver Block Diagram}\label{receiver-block-diagram}

\begin{verbatim}
Antenna
   
RF Downconversion & A/D Converter
   
Serial-to-Parallel Converter
   
Remove Cyclic Prefix
   
FFT (N-point)
   
Channel Estimation & Equalization (per-subcarrier)
   
QAM/PSK Demapper
   
Parallel-to-Serial Converter
   
Data bits
\end{verbatim}

\begin{center}\rule{0.5\linewidth}{0.5pt}\end{center}

\subsection{\texorpdfstring{ Cyclic Prefix
(CP)}{ Cyclic Prefix (CP)}}\label{cyclic-prefix-cp}

The \textbf{cyclic prefix} is OFDM\textquotesingle s defense against
multipath-induced ISI.

\subsubsection{What Is It?}\label{what-is-it}

Copy the \textbf{last L samples} of the OFDM symbol and prepend them:

\begin{verbatim}
Original OFDM symbol (N samples):
[s s s ... s_(N-2) s_(N-1)]

With CP (N+L samples):
[s_(N-L) ... s_(N-1) | s s s ... s_(N-2) s_(N-1)]
 +--- CP (L) ----+     +---- Original Symbol (N) ----+
\end{verbatim}

\subsubsection{Why Does It Work?}\label{why-does-it-work}

\textbf{Problem}: Multipath creates delayed copies of the signal
\$\textbackslash rightarrow\$ samples from adjacent symbols overlap
(ISI).

\textbf{Solution}: CP acts as a \textbf{guard interval}: - If delay
spread \textless{} CP duration, ISI from previous symbol falls entirely
within the CP - Receiver discards CP \$\textbackslash rightarrow\$ only
clean samples remain - CP makes \textbf{linear convolution appear as
circular convolution} \$\textbackslash rightarrow\$ simple
per-subcarrier equalization

\subsubsection{CP Overhead}\label{cp-overhead}

\begin{verbatim}
Overhead = L / (N + L)

Example (WiFi 802.11a):
- N = 64 subcarriers
- L = 16 samples (CP)
- Overhead = 16/80 = 20% (loss in spectral efficiency)

Tradeoff:
- Longer CP  more robust to delay spread
- Longer CP  higher overhead (lower data rate)
\end{verbatim}

\begin{center}\rule{0.5\linewidth}{0.5pt}\end{center}

\subsection{\texorpdfstring{ OFDM
Parameters}{ OFDM Parameters}}\label{ofdm-parameters}

\subsubsection{Key Design Choices}\label{key-design-choices}

{\def\LTcaptype{} % do not increment counter
\begin{longtable}[]{@{}
  >{\raggedright\arraybackslash}p{(\linewidth - 6\tabcolsep) * \real{0.2558}}
  >{\raggedright\arraybackslash}p{(\linewidth - 6\tabcolsep) * \real{0.1860}}
  >{\raggedright\arraybackslash}p{(\linewidth - 6\tabcolsep) * \real{0.3721}}
  >{\raggedright\arraybackslash}p{(\linewidth - 6\tabcolsep) * \real{0.1860}}@{}}
\toprule\noalign{}
\begin{minipage}[b]{\linewidth}\raggedright
Parameter
\end{minipage} & \begin{minipage}[b]{\linewidth}\raggedright
Symbol
\end{minipage} & \begin{minipage}[b]{\linewidth}\raggedright
Typical Values
\end{minipage} & \begin{minipage}[b]{\linewidth}\raggedright
Impact
\end{minipage} \\
\midrule\noalign{}
\endhead
\bottomrule\noalign{}
\endlastfoot
FFT Size & N & 64-2048 & Granularity, latency \\
Subcarrier Spacing & \(\Delta f\) & 15 kHz (LTE), 312.5 kHz (WiFi) &
Doppler tolerance \\
Symbol Duration & \(T_{\text{symbol}}\) & \(1/\Delta f\) & ISI
resistance \\
CP Length & L & N/4, N/8, N/16 & Delay spread tolerance \\
Bandwidth & BW & \(N \cdot \Delta f\) & Throughput \\
\end{longtable}
}

\subsubsection{Example: LTE}\label{example-lte}

\begin{verbatim}
Configuration:
- FFT Size: 1024 (20 MHz BW) or 512 (10 MHz)
- Subcarrier Spacing: 15 kHz
- Symbol Duration: 66.67 s
- CP (normal): 4.69 s (first symbol), 5.21 s (others)
- 12 subcarriers per Resource Block (180 kHz)
- 7 OFDM symbols per slot (0.5 ms)
\end{verbatim}

\begin{center}\rule{0.5\linewidth}{0.5pt}\end{center}

\subsection{\texorpdfstring{ Pilot Subcarriers \& Channel
Estimation}{ Pilot Subcarriers \& Channel Estimation}}\label{pilot-subcarriers-channel-estimation}

Not all subcarriers carry data-\/-\/-some are \textbf{pilots} for
channel estimation.

\subsubsection{Pilot Types}\label{pilot-types}

\textbf{1. Scattered Pilots} (time + frequency diversity):

\begin{verbatim}
Subcarrier
    
    | D D D P D D D P D D D P     Symbol 4
    | D D P D D D P D D D P D     Symbol 3
    | D P D D D P D D D P D D     Symbol 2
    | P D D D P D D D P D D D     Symbol 1
    +--------------------------> Time
     (P = Pilot, D = Data)
\end{verbatim}

\textbf{2. Continual Pilots} (phase/frequency tracking):

\begin{verbatim}
Fixed subcarriers (e.g., k = -21, -7, 7, 21 in WiFi) always carry pilots.
\end{verbatim}

\textbf{3. Preamble/Training Symbols}:

\begin{verbatim}
First OFDM symbol(s) in a frame are all pilots for initial synchronization.
\end{verbatim}

\subsubsection{Channel Estimation}\label{channel-estimation}

\textbf{Per-subcarrier channel model}:

\begin{verbatim}
Y = H · X + N

where:
- H = complex channel gain on subcarrier k
- X = transmitted symbol
- Y = received symbol
- N = noise
\end{verbatim}

\textbf{Estimation process}: 1. Transmitter sends known pilots
P\textbackslash textsubscript\{k\} 2. Receiver measures
Y\textbackslash textsubscript\{k\} =
H\textbackslash textsubscript\{k\}\$\textbackslash cdot\$P\textbackslash textsubscript\{k\}
+ N\textbackslash textsubscript\{k\} 3. Estimate:
\textbackslash\^{}\{H\}\textbackslash textsubscript\{k\} =
Y\textbackslash textsubscript\{k\} / P\textbackslash textsubscript\{k\}
4. Interpolate \textbackslash\^{}\{H\}\textbackslash textsubscript\{k\}
across data subcarriers (2D interpolation) 5. Equalize data:
X\textbackslash textsubscript\{k\} = Y\textbackslash textsubscript\{k\}
/ \textbackslash\^{}\{H\}\textbackslash textsubscript\{k\}

\begin{center}\rule{0.5\linewidth}{0.5pt}\end{center}

\subsection{\texorpdfstring{ Multipath \& Frequency-Selective
Fading}{ Multipath \& Frequency-Selective Fading}}\label{multipath-frequency-selective-fading}

\subsubsection{Why OFDM Excels in
Multipath}\label{why-ofdm-excels-in-multipath}

\textbf{Single-carrier}: Entire bandwidth experiences
\textbf{frequency-selective fading} \$\textbackslash rightarrow\$ deep
nulls can wipe out signal.

\textbf{OFDM}: Channel appears \textbf{flat} within each narrow
subcarrier \$\textbackslash rightarrow\$ only some subcarriers fade
deeply, others remain strong.

\begin{verbatim}
Frequency Response (Multipath Channel):
Magnitude
    
    |  ___       ___
    | |   |     |   |___       Single-carrier spans entire BW
    | |   |_____|       |        (suffers deep null)
    |_|___|_____________|___ Frequency
                   
       |   |   |   |   |
     OFDM subcarriers (most unaffected, a few degraded)
\end{verbatim}

\textbf{Per-subcarrier equalization}:

\begin{verbatim}
X = Y / H  (simple division per subcarrier)
\end{verbatim}

Much simpler than \textbf{time-domain equalization} (which requires
complex filters).

\begin{center}\rule{0.5\linewidth}{0.5pt}\end{center}

\subsection{\texorpdfstring{ Peak-to-Average Power Ratio
(PAPR)}{ Peak-to-Average Power Ratio (PAPR)}}\label{peak-to-average-power-ratio-papr}

\subsubsection{The OFDM Challenge}\label{the-ofdm-challenge}

\textbf{Problem}: When N subcarriers add constructively, instantaneous
power spikes far above average.

\begin{verbatim}
PAPR = Peak Power / Average Power

Theoretical worst case: PAPR = N (e.g., 20 dB for N=100)
Typical OFDM: PAPR  10-13 dB (3-5 dB worse than single-carrier)
\end{verbatim}

\subsubsection{Why It Matters}\label{why-it-matters}

\begin{itemize}
\tightlist
\item
  \textbf{Power Amplifier (PA) must operate in linear region}
  \$\textbackslash rightarrow\$ inefficient (backed-off from saturation)
\item
  High PAPR \$\textbackslash rightarrow\$ PA must handle peaks
  \$\textbackslash rightarrow\$ more expensive, power-hungry RF hardware
\item
  Non-linear PA \$\textbackslash rightarrow\$ intermodulation
  distortion, spectral regrowth
\end{itemize}

\subsubsection{PAPR Reduction
Techniques}\label{papr-reduction-techniques}

\textbf{1. Clipping \& Filtering}:

\begin{verbatim}
Clip peaks at threshold  filter out-of-band distortion  slight BER degradation
\end{verbatim}

\textbf{2. Tone Reservation}:

\begin{verbatim}
Reserve some subcarriers to generate "anti-peaks" that cancel large peaks.
\end{verbatim}

\textbf{3. Selective Mapping (SLM)}:

\begin{verbatim}
Generate multiple OFDM symbols with different phase rotations  choose one with lowest PAPR.
\end{verbatim}

\textbf{4. Partial Transmit Sequence (PTS)}:

\begin{verbatim}
Divide subcarriers into blocks  optimize phase per block to minimize PAPR.
\end{verbatim}

\textbf{Tradeoff}: PAPR reduction adds complexity, may reduce spectral
efficiency or increase BER.

\begin{center}\rule{0.5\linewidth}{0.5pt}\end{center}

\subsection{\texorpdfstring{ Synchronization
Challenges}{ Synchronization Challenges}}\label{synchronization-challenges}

OFDM is \textbf{sensitive} to timing and frequency offsets.

\subsubsection{Timing Offset}\label{timing-offset}

\textbf{Consequence}: If FFT window is misaligned: - Within CP: No ISI,
but phase rotation per subcarrier - Beyond CP: ISI from adjacent symbols

\textbf{Solution}: Preamble correlation, CP-based timing metrics.

\subsubsection{Carrier Frequency Offset
(CFO)}\label{carrier-frequency-offset-cfo}

\textbf{Consequence}: Subcarriers lose orthogonality
\$\textbackslash rightarrow\$ Inter-Carrier Interference (ICI).

\begin{verbatim}
CFO = f / f_subcarrier

Example:
- 1 kHz offset on 15 kHz subcarrier spacing  CFO = 0.067
- Causes ~0.2 dB SNR loss
\end{verbatim}

\textbf{Solution}: 1. \textbf{Coarse CFO estimation}: Preamble
autocorrelation (range:
\$\textbackslash pm\$\$\textbackslash Delta\$f\_subcarrier/2) 2.
\textbf{Fine CFO tracking}: Continual pilots

\subsubsection{Sampling Clock Offset
(SCO)}\label{sampling-clock-offset-sco}

\textbf{Consequence}: Slow drift in FFT window position
\$\textbackslash rightarrow\$ phase rotation accumulates over time.

\textbf{Solution}: Track phase of continual pilots
\$\textbackslash rightarrow\$ adjust sampling clock or compensate
digitally.

\begin{center}\rule{0.5\linewidth}{0.5pt}\end{center}

\subsection{\texorpdfstring{ OFDM in Real-World
Standards}{ OFDM in Real-World Standards}}\label{ofdm-in-real-world-standards}

\subsubsection{WiFi 802.11a/g/n/ac/ax}\label{wifi-802.11agnacax}

\textbf{802.11a/g (54 Mbps)}:

\begin{verbatim}
- FFT Size: 64
- Used Subcarriers: 52 (48 data + 4 pilots)
- Subcarrier Spacing: 312.5 kHz
- Bandwidth: 20 MHz
- Modulation: BPSK, QPSK, 16-QAM, 64-QAM
\end{verbatim}

\textbf{802.11n (600 Mbps)}:

\begin{verbatim}
- Up to 4×4 MIMO-OFDM
- 40 MHz channels (108 data subcarriers)
- Short Guard Interval: 400 ns (vs. 800 ns)
\end{verbatim}

\textbf{802.11ax (WiFi 6, 9.6 Gbps)}:

\begin{verbatim}
- OFDMA (multi-user OFDM): allocate subcarriers to different users
- 1024-QAM, 160 MHz channels
- MU-MIMO (8×8)
\end{verbatim}

\begin{center}\rule{0.5\linewidth}{0.5pt}\end{center}

\subsubsection{LTE \& 5G NR}\label{lte-5g-nr}

\textbf{LTE Downlink}:

\begin{verbatim}
- SC-FDMA uplink (low PAPR variant)
- 15 kHz subcarrier spacing
- 1.4, 3, 5, 10, 15, 20 MHz bandwidths
- CP-OFDM with MIMO (up to 8×8)
\end{verbatim}

\textbf{5G NR}:

\begin{verbatim}
- Scalable numerology: f = 15, 30, 60, 120, 240 kHz
  (higher spacing for mmWave  shorter symbols  Doppler tolerance)
- Massive MIMO (64+ antennas)
- Flexible frame structure (dynamic TDD)
\end{verbatim}

\begin{center}\rule{0.5\linewidth}{0.5pt}\end{center}

\subsubsection{DVB-T/T2 (Digital Video Broadcasting -
Terrestrial)}\label{dvb-tt2-digital-video-broadcasting---terrestrial}

\textbf{DVB-T}:

\begin{verbatim}
- FFT: 2048 or 8192
- Guard intervals: 1/4, 1/8, 1/16, 1/32
- Optimized for high-mobility (trains, cars)
- COFDM (Coded OFDM with interleaving)
\end{verbatim}

\textbf{DVB-T2} (next-gen):

\begin{verbatim}
- Up to 256-QAM
- LDPC + BCH FEC
- Rotated constellations (diversity against deep fades)
\end{verbatim}

\begin{center}\rule{0.5\linewidth}{0.5pt}\end{center}

\subsection{\texorpdfstring{ Spectral Efficiency
Analysis}{ Spectral Efficiency Analysis}}\label{spectral-efficiency-analysis}

\subsubsection{Calculation}\label{calculation}

\begin{verbatim}
Spectral Efficiency (SE) = R / BW  bits/s/Hz

where:
R = N_data · log(M) · (1 - CP_overhead) / T_symbol

Example (LTE 20 MHz):
- N_data = 1200 subcarriers (100 RBs × 12)
- M = 64 (64-QAM  6 bits/symbol)
- CP overhead = 7%
- T_symbol = 66.67 s

SE = 1200 · 6 · 0.93 / (66.67×10 · 20×10)
   = 6696 / 1.33 = 5.0 bits/s/Hz

(Theoretical peak with MIMO: 30 bits/s/Hz for 4×4 spatial streams)
\end{verbatim}

\begin{center}\rule{0.5\linewidth}{0.5pt}\end{center}

\subsection{\texorpdfstring{ OFDM
vs.~Single-Carrier}{ OFDM vs.~Single-Carrier}}\label{ofdm-vs.-single-carrier}

{\def\LTcaptype{} % do not increment counter
\begin{longtable}[]{@{}
  >{\raggedright\arraybackslash}p{(\linewidth - 4\tabcolsep) * \real{0.2667}}
  >{\raggedright\arraybackslash}p{(\linewidth - 4\tabcolsep) * \real{0.2000}}
  >{\raggedright\arraybackslash}p{(\linewidth - 4\tabcolsep) * \real{0.5333}}@{}}
\toprule\noalign{}
\begin{minipage}[b]{\linewidth}\raggedright
Aspect
\end{minipage} & \begin{minipage}[b]{\linewidth}\raggedright
OFDM
\end{minipage} & \begin{minipage}[b]{\linewidth}\raggedright
Single-Carrier
\end{minipage} \\
\midrule\noalign{}
\endhead
\bottomrule\noalign{}
\endlastfoot
\textbf{ISI Robustness} & Excellent (CP + long symbols) & Requires
complex equalizer \\
\textbf{Frequency-Selective Fading} & Simple per-subcarrier EQ &
Time-domain EQ (adaptive filter) \\
\textbf{PAPR} & High (\textasciitilde10-13 dB) & Low (\textasciitilde3-5
dB) \\
\textbf{Spectral Efficiency} & Moderate (CP overhead) & Higher (no
CP) \\
\textbf{Implementation} & FFT/IFFT (efficient) & FIR filters
(complex) \\
\textbf{Doppler Sensitivity} & Moderate (ICI from CFO) & Lower \\
\textbf{Best For} & Wideband, fixed/low-mobility & Narrowband,
high-mobility \\
\end{longtable}
}

\begin{center}\rule{0.5\linewidth}{0.5pt}\end{center}

\subsection{\texorpdfstring{ Advanced OFDM
Variants}{ Advanced OFDM Variants}}\label{advanced-ofdm-variants}

\subsubsection{OFDMA (Orthogonal Frequency-Division Multiple
Access)}\label{ofdma-orthogonal-frequency-division-multiple-access}

\textbf{Concept}: Assign different subcarriers to different users.

\begin{verbatim}
User 1: Subcarriers 0-15
User 2: Subcarriers 16-31
User 3: Subcarriers 32-47
...

Advantages:
- Multi-user diversity
- Flexible resource allocation
- Uplink/downlink efficiency
\end{verbatim}

\textbf{Used in}: LTE, 5G NR, WiFi 6 (802.11ax).

\begin{center}\rule{0.5\linewidth}{0.5pt}\end{center}

\subsubsection{SC-FDMA (Single-Carrier
FDMA)}\label{sc-fdma-single-carrier-fdma}

\textbf{Motivation}: Lower PAPR for mobile devices (saves battery).

\textbf{Method}: DFT-spread OFDM:

\begin{verbatim}
Data  DFT  Subcarrier Mapping  IFFT  CP
\end{verbatim}

\textbf{Effect}: Maintains OFDM benefits but with \textbf{3-5 dB lower
PAPR}.

\textbf{Used in}: LTE uplink, 5G NR uplink option.

\begin{center}\rule{0.5\linewidth}{0.5pt}\end{center}

\subsubsection{Filter-Bank Multicarrier
(FBMC)}\label{filter-bank-multicarrier-fbmc}

\textbf{Improvement}: Replace rectangular pulse (sinc spectrum) with
well-designed filters \$\textbackslash rightarrow\$ reduced out-of-band
emissions.

\textbf{Advantage}: No CP needed \$\textbackslash rightarrow\$ higher
spectral efficiency.

\textbf{Disadvantage}: More complex, incompatible with MIMO (without
workarounds).

\textbf{Status}: Considered for 5G but not adopted (OFDM with windowing
chosen instead).

\begin{center}\rule{0.5\linewidth}{0.5pt}\end{center}

\subsection{\texorpdfstring{ Python Implementation
Example}{ Python Implementation Example}}\label{python-implementation-example}

\subsubsection{Basic OFDM Transmitter}\label{basic-ofdm-transmitter}

\begin{Shaded}
\begin{Highlighting}[]
\ImportTok{import}\NormalTok{ numpy }\ImportTok{as}\NormalTok{ np}

\KeywordTok{def}\NormalTok{ ofdm\_modulate(data\_symbols, N}\OperatorTok{=}\DecValTok{64}\NormalTok{, L\_cp}\OperatorTok{=}\DecValTok{16}\NormalTok{):}
    \CommentTok{"""}
\CommentTok{    OFDM modulation via IFFT.}
\CommentTok{    }
\CommentTok{    Args:}
\CommentTok{        data\_symbols: Array of QAM/PSK symbols (length N)}
\CommentTok{        N: FFT size}
\CommentTok{        L\_cp: Cyclic prefix length}
\CommentTok{    }
\CommentTok{    Returns:}
\CommentTok{        OFDM time{-}domain signal (length N + L\_cp)}
\CommentTok{    """}
    \CommentTok{\# IFFT (convert frequency domain to time domain)}
\NormalTok{    time\_domain }\OperatorTok{=}\NormalTok{ np.fft.ifft(data\_symbols, N)}
    
    \CommentTok{\# Add cyclic prefix}
\NormalTok{    cp }\OperatorTok{=}\NormalTok{ time\_domain[}\OperatorTok{{-}}\NormalTok{L\_cp:]}
\NormalTok{    ofdm\_symbol }\OperatorTok{=}\NormalTok{ np.concatenate([cp, time\_domain])}
    
    \ControlFlowTok{return}\NormalTok{ ofdm\_symbol}

\CommentTok{\# Example usage}
\NormalTok{N }\OperatorTok{=} \DecValTok{64}
\NormalTok{L\_cp }\OperatorTok{=} \DecValTok{16}

\CommentTok{\# Generate random QPSK symbols}
\NormalTok{data\_symbols }\OperatorTok{=}\NormalTok{ (}\DecValTok{2} \OperatorTok{*}\NormalTok{ np.random.randint(}\DecValTok{0}\NormalTok{, }\DecValTok{2}\NormalTok{, N) }\OperatorTok{{-}} \DecValTok{1}\NormalTok{) }\OperatorTok{+} \OperatorTok{\textbackslash{}}
               \OtherTok{1j} \OperatorTok{*}\NormalTok{ (}\DecValTok{2} \OperatorTok{*}\NormalTok{ np.random.randint(}\DecValTok{0}\NormalTok{, }\DecValTok{2}\NormalTok{, N) }\OperatorTok{{-}} \DecValTok{1}\NormalTok{)}
\NormalTok{data\_symbols }\OperatorTok{/=}\NormalTok{ np.sqrt(}\DecValTok{2}\NormalTok{)  }\CommentTok{\# Normalize}

\CommentTok{\# Modulate}
\NormalTok{tx\_signal }\OperatorTok{=}\NormalTok{ ofdm\_modulate(data\_symbols, N, L\_cp)}

\BuiltInTok{print}\NormalTok{(}\SpecialStringTok{f"Input symbols: }\SpecialCharTok{\{}\BuiltInTok{len}\NormalTok{(data\_symbols)}\SpecialCharTok{\}}\SpecialStringTok{"}\NormalTok{)}
\BuiltInTok{print}\NormalTok{(}\SpecialStringTok{f"OFDM signal: }\SpecialCharTok{\{}\BuiltInTok{len}\NormalTok{(tx\_signal)}\SpecialCharTok{\}}\SpecialStringTok{ samples (N=}\SpecialCharTok{\{}\NormalTok{N}\SpecialCharTok{\}}\SpecialStringTok{ + CP=}\SpecialCharTok{\{}\NormalTok{L\_cp}\SpecialCharTok{\}}\SpecialStringTok{)"}\NormalTok{)}
\BuiltInTok{print}\NormalTok{(}\SpecialStringTok{f"PAPR: }\SpecialCharTok{\{}\DecValTok{10} \OperatorTok{*}\NormalTok{ np}\SpecialCharTok{.}\NormalTok{log10(np.}\BuiltInTok{max}\NormalTok{(np.}\BuiltInTok{abs}\NormalTok{(tx\_signal)}\OperatorTok{**}\DecValTok{2}\NormalTok{) }\OperatorTok{/}\NormalTok{ np.mean(np.}\BuiltInTok{abs}\NormalTok{(tx\_signal)}\OperatorTok{**}\DecValTok{2}\NormalTok{))}\SpecialCharTok{:.2f\}}\SpecialStringTok{ dB"}\NormalTok{)}
\end{Highlighting}
\end{Shaded}

\subsubsection{Basic OFDM Receiver}\label{basic-ofdm-receiver}

\begin{Shaded}
\begin{Highlighting}[]
\KeywordTok{def}\NormalTok{ ofdm\_demodulate(rx\_signal, N}\OperatorTok{=}\DecValTok{64}\NormalTok{, L\_cp}\OperatorTok{=}\DecValTok{16}\NormalTok{):}
    \CommentTok{"""}
\CommentTok{    OFDM demodulation via FFT.}
\CommentTok{    }
\CommentTok{    Args:}
\CommentTok{        rx\_signal: Received time{-}domain signal}
\CommentTok{        N: FFT size}
\CommentTok{        L\_cp: Cyclic prefix length}
\CommentTok{    }
\CommentTok{    Returns:}
\CommentTok{        Recovered frequency{-}domain symbols}
\CommentTok{    """}
    \CommentTok{\# Remove cyclic prefix}
\NormalTok{    rx\_no\_cp }\OperatorTok{=}\NormalTok{ rx\_signal[L\_cp:]}
    
    \CommentTok{\# FFT (convert time domain to frequency domain)}
\NormalTok{    recovered\_symbols }\OperatorTok{=}\NormalTok{ np.fft.fft(rx\_no\_cp, N)}
    
    \ControlFlowTok{return}\NormalTok{ recovered\_symbols}

\CommentTok{\# Demodulate}
\NormalTok{rx\_symbols }\OperatorTok{=}\NormalTok{ ofdm\_demodulate(tx\_signal, N, L\_cp)}

\CommentTok{\# Compare (should be identical in ideal channel)}
\NormalTok{error }\OperatorTok{=}\NormalTok{ np.}\BuiltInTok{max}\NormalTok{(np.}\BuiltInTok{abs}\NormalTok{(data\_symbols }\OperatorTok{{-}}\NormalTok{ rx\_symbols))}
\BuiltInTok{print}\NormalTok{(}\SpecialStringTok{f"Reconstruction error: }\SpecialCharTok{\{}\NormalTok{error}\SpecialCharTok{:.2e\}}\SpecialStringTok{"}\NormalTok{)}
\end{Highlighting}
\end{Shaded}

\begin{center}\rule{0.5\linewidth}{0.5pt}\end{center}

\subsection{\texorpdfstring{ Performance
Analysis}{ Performance Analysis}}\label{performance-analysis}

\subsubsection{BER in AWGN Channel}\label{ber-in-awgn-channel}

For OFDM with M-QAM modulation on each subcarrier:

\begin{verbatim}
BER  (4/log(M)) · (1 - 1/M) · Q((3·log(M)·SNR / (M-1)))

where Q(x) = Gaussian Q-function

Example (16-QAM OFDM at SNR = 20 dB):
BER  10 (without coding)
BER  10 (with rate-1/2 LDPC)
\end{verbatim}

\subsubsection{Frequency-Selective
Channel}\label{frequency-selective-channel}

\begin{Shaded}
\begin{Highlighting}[]
\CommentTok{\# Generate multipath channel}
\KeywordTok{def}\NormalTok{ multipath\_channel(ofdm\_signal, delays, gains):}
    \CommentTok{"""}
\CommentTok{    Apply multipath fading.}
\CommentTok{    }
\CommentTok{    Args:}
\CommentTok{        delays: Array of tap delays (in samples)}
\CommentTok{        gains: Array of tap gains (complex)}
\CommentTok{    """}
\NormalTok{    output }\OperatorTok{=}\NormalTok{ np.zeros(}\BuiltInTok{len}\NormalTok{(ofdm\_signal) }\OperatorTok{+} \BuiltInTok{max}\NormalTok{(delays), dtype}\OperatorTok{=}\BuiltInTok{complex}\NormalTok{)}
    
    \ControlFlowTok{for}\NormalTok{ delay, gain }\KeywordTok{in} \BuiltInTok{zip}\NormalTok{(delays, gains):}
\NormalTok{        output[delay:delay}\OperatorTok{+}\BuiltInTok{len}\NormalTok{(ofdm\_signal)] }\OperatorTok{+=}\NormalTok{ gain }\OperatorTok{*}\NormalTok{ ofdm\_signal}
    
    \ControlFlowTok{return}\NormalTok{ output[:}\BuiltInTok{len}\NormalTok{(ofdm\_signal)]}

\CommentTok{\# Example: 2{-}tap channel}
\NormalTok{delays }\OperatorTok{=}\NormalTok{ [}\DecValTok{0}\NormalTok{, }\DecValTok{8}\NormalTok{]  }\CommentTok{\# Direct path + 8{-}sample delayed path}
\NormalTok{gains }\OperatorTok{=}\NormalTok{ [}\FloatTok{1.0}\NormalTok{, }\FloatTok{0.5}\OperatorTok{*}\NormalTok{np.exp(}\OtherTok{1j}\OperatorTok{*}\NormalTok{np.pi}\OperatorTok{/}\DecValTok{4}\NormalTok{)]  }\CommentTok{\# 6 dB echo with phase}

\NormalTok{rx\_signal }\OperatorTok{=}\NormalTok{ multipath\_channel(tx\_signal, delays, gains)}
\NormalTok{rx\_signal }\OperatorTok{+=} \FloatTok{0.01} \OperatorTok{*}\NormalTok{ (np.random.randn(}\BuiltInTok{len}\NormalTok{(rx\_signal)) }\OperatorTok{+} 
                     \OtherTok{1j} \OperatorTok{*}\NormalTok{ np.random.randn(}\BuiltInTok{len}\NormalTok{(rx\_signal)))  }\CommentTok{\# Add noise}

\CommentTok{\# Demodulate}
\NormalTok{rx\_symbols }\OperatorTok{=}\NormalTok{ ofdm\_demodulate(rx\_signal, N, L\_cp)}

\CommentTok{\# Per{-}subcarrier channel estimation (if pilots known)}
\NormalTok{H\_estimated }\OperatorTok{=}\NormalTok{ rx\_symbols }\OperatorTok{/}\NormalTok{ data\_symbols  }\CommentTok{\# Assumes data\_symbols are pilots}
\end{Highlighting}
\end{Shaded}

\begin{center}\rule{0.5\linewidth}{0.5pt}\end{center}

\subsection{\texorpdfstring{ When to Use
OFDM}{ When to Use OFDM}}\label{when-to-use-ofdm}

\subsubsection{OFDM is Ideal For:}\label{ofdm-is-ideal-for}

\textbf{Wideband channels} (\textgreater{} 1 MHz) with
frequency-selective fading\\
\textbf{Multipath-rich environments} (urban, indoor)\\
\textbf{Fixed or low-mobility users} (\textless{} 120 km/h)\\
\textbf{Multiple users} needing flexible resource allocation (OFDMA)\\
\textbf{High spectral efficiency} requirements

\subsubsection{Avoid OFDM For:}\label{avoid-ofdm-for}

\textbf{Power-constrained devices} (high PAPR
\$\textbackslash rightarrow\$ inefficient PA)\\
\textbf{High-mobility} (Doppler \$\textbackslash rightarrow\$ severe
ICI)\\
\textbf{Narrowband channels} (CP overhead too high)\\
\textbf{Non-linear channels} (PAPR sensitive to distortion)

\begin{center}\rule{0.5\linewidth}{0.5pt}\end{center}

\subsection{\texorpdfstring{ Further
Reading}{ Further Reading}}\label{further-reading}

\subsubsection{Textbooks}\label{textbooks}

\begin{itemize}
\tightlist
\item
  \textbf{Prasad}, \emph{OFDM for Wireless Communications Systems} -
  Comprehensive treatment
\item
  \textbf{Cho et al.}, \emph{MIMO-OFDM Wireless Communications with
  MATLAB} - Practical implementation
\item
  \textbf{Goldsmith}, \emph{Wireless Communications} (Chapter 13) -
  Theoretical foundation
\end{itemize}

\subsubsection{Standards Documents}\label{standards-documents}

\begin{itemize}
\tightlist
\item
  \textbf{IEEE 802.11-2020}: WiFi OFDM/OFDMA specifications
\item
  \textbf{3GPP TS 36.211}: LTE Physical Layer (OFDM parameters)
\item
  \textbf{3GPP TS 38.211}: 5G NR Physical Layer (scalable OFDM)
\end{itemize}

\subsubsection{Related Topics}\label{related-topics}

\begin{itemize}
\tightlist
\item
  {[}{[}MIMO-\&-Spatial-Multiplexing{]}{]} - Combining OFDM with
  multiple antennas
\item
  {[}{[}Channel-Equalization{]}{]} - Frequency-domain equalization in
  OFDM
\item
  {[}{[}Adaptive-Modulation-\&-Coding-(AMC){]}{]} - Per-subcarrier link
  adaptation
\item
  {[}{[}Synchronization-(Carrier,-Timing,-Frame){]}{]} - OFDM sync
  techniques
\item
  {[}{[}Real-World-System-Examples{]}{]} - LTE, 5G, WiFi implementations
\end{itemize}

\begin{center}\rule{0.5\linewidth}{0.5pt}\end{center}

\textbf{Summary}: OFDM transforms wideband frequency-selective channels
into many narrowband flat-fading channels, enabling simple equalization
and high spectral efficiency. The FFT/IFFT makes it computationally
efficient, while the cyclic prefix provides ISI immunity. Despite high
PAPR and synchronization sensitivity, OFDM dominates modern wireless due
to its robustness in multipath environments and natural fit for MIMO and
multi-user scenarios.
