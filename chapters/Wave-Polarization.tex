\section{Wave Polarization}\label{wave-polarization}

{[}{[}Home{]}{]} \textbar{} \textbf{EM Fundamentals} \textbar{}
{[}{[}Maxwell\textquotesingle s-Equations-\&-Wave-Propagation{]}{]}
\textbar{} {[}{[}Electromagnetic-Spectrum{]}{]}

\begin{center}\rule{0.5\linewidth}{0.5pt}\end{center}

\subsection{\texorpdfstring{ For Non-Technical
Readers}{ For Non-Technical Readers}}\label{for-non-technical-readers}

\textbf{Wave polarization is like the orientation of a jump
rope-\/-\/-you can shake it up/down (vertical), side-to-side
(horizontal), or in circles (circular). Antennas must match this
orientation to catch the signal!}

\textbf{The idea}: - Radio waves are oscillating electric/magnetic
fields - The \textbf{electric field} can point in different directions -
\textbf{Polarization} = which direction the field oscillates

\textbf{Three main types}:

\textbf{1. Linear Polarization} (most common): - Field oscillates in one
fixed direction - \textbf{Vertical}: Field points up/down
(\$\textbackslash updownarrow\$) - \textbf{Horizontal}: Field points
left/right (\$\textbackslash leftrightarrow\$) -
\textbf{45\$\^{}\textbackslash circ\$}: Somewhere in between
(\$\textbackslash nearrow\$)

\textbf{2. Circular Polarization}: - Field rotates in a circle as wave
travels - \textbf{Right-hand circular} (RHCP): Rotates clockwise -
\textbf{Left-hand circular} (LHCP): Rotates counter-clockwise

\textbf{3. Elliptical Polarization}: - Field traces an ellipse
(in-between linear and circular) - Most real-world signals (not
perfectly linear/circular)

\textbf{Real-world examples}:

\textbf{FM Radio}: - \textbf{Vertical polarization} - Your car antenna:
Vertical rod - Must be vertical to match transmitter!

\textbf{TV Broadcasting}: - \textbf{Horizontal polarization} (old analog
TV) - Roof antennas: Horizontal elements - Must be horizontal to receive
signal

\textbf{WiFi}: - \textbf{Usually vertical} (your
router\textquotesingle s antennas) - Laptop: Internal antenna usually
vertical - This is why tilting laptop changes signal strength!

\textbf{Satellite}: - \textbf{Circular polarization} (GPS, satellite TV)
- Why circular? Survives Faraday rotation in ionosphere - Your satellite
dish: Works at any angle!

\textbf{Why polarization matters}:

\textbf{Antenna alignment}: - \textbf{Matched polarization}: Maximum
signal (0 dB loss) - \textbf{Cross-polarization}
(90\$\^{}\textbackslash circ\$ off): -20 to -30 dB loss! - This is why
rotating your phone can improve reception

\textbf{Example}: - Cell tower: Vertical polarization - Your phone held
horizontally: Antenna now horizontal - Signal loss: 10-20 dB! - Result:
Dropped call

\textbf{Frequency reuse}: - Send two signals at same frequency,
different polarization - Vertical signal + Horizontal signal = no
interference! - \textbf{Satellite TV}: Uses both RHCP and LHCP to double
capacity

\textbf{Faraday rotation}: - Ionosphere rotates polarization (like
twisting jump rope) - Linear polarization \$\textbackslash rightarrow\$
gets rotated \$\textbackslash rightarrow\$ antenna mismatch! -
\textbf{Solution}: Use circular (rotation doesn\textquotesingle t
matter) - This is why GPS uses circular!

\textbf{Your experience}:

\textbf{Old TV ``rabbit ears''}: - Had to rotate/tilt for best picture -
You were matching antenna polarization! - Horizontal = horizontal
polarization - V-shape = trying to catch both!

\textbf{Cell phone}: - Hold normally: Antenna vertical (good) - Hold
horizontally (watching video): Antenna horizontal (bad!) - This is
``death grip'' effect (partly)

\textbf{WiFi router antennas}: - Multiple antennas at different angles -
Catches signals from devices in any orientation - Some routers: Mix
vertical/horizontal for diversity

\textbf{Satellite dish}: - Circular polarization
\$\textbackslash rightarrow\$ dish angle doesn\textquotesingle t matter
for polarization - Only matters for pointing at satellite!

\textbf{Fun fact}: GPS satellites transmit right-hand circular
polarization (RHCP). If you flip your GPS receiver upside-down, it
receives left-hand circular polarization (LHCP)-\/-\/-and the signal is
20-30 dB weaker, basically unusable. This is why your
phone\textquotesingle s GPS doesn\textquotesingle t work well face-down
on a table!

\begin{center}\rule{0.5\linewidth}{0.5pt}\end{center}

\subsection{Overview}\label{overview}

\textbf{Polarization} describes the \textbf{orientation of the electric
field vector} as an electromagnetic wave propagates through space.

\textbf{Key insight}: While the wave travels in one direction (e.g.,
+z), the electric field \textbf{oscillates in a plane perpendicular} to
propagation. The pattern traced by the E-field tip defines polarization.

\textbf{Why it matters}: - \textbf{Antenna alignment}: RX antenna must
match TX polarization for maximum signal capture - \textbf{Propagation
effects}: Ionosphere rotates polarization (Faraday rotation) -
\textbf{Interference mitigation}: Orthogonal polarizations enable
frequency reuse - \textbf{Satellite communications}: Circular
polarization combats ionospheric effects

\begin{center}\rule{0.5\linewidth}{0.5pt}\end{center}

\subsection{Mathematical Foundation}\label{mathematical-foundation}

\subsubsection{Plane Wave
Representation}\label{plane-wave-representation}

\textbf{General electric field} (propagating in +z direction):

\[
\vec{E}(z,t) = E_x \cos(\omega t - kz + \phi_x)\hat{x} + E_y \cos(\omega t - kz + \phi_y)\hat{y}
\]

Where: - \(E_x\), \(E_y\) = Amplitudes in x and y directions -
\(\phi_x\), \(\phi_y\) = Phase offsets -
\(\Delta\phi = \phi_y - \phi_x\) = \textbf{Relative phase} (determines
polarization type)

\textbf{At fixed observation point} (z=0):

\[
\vec{E}(t) = E_x \cos(\omega t + \phi_x)\hat{x} + E_y \cos(\omega t + \phi_y)\hat{y}
\]

\begin{center}\rule{0.5\linewidth}{0.5pt}\end{center}

\subsection{Polarization Types}\label{polarization-types}

\subsubsection{1. Linear Polarization}\label{linear-polarization}

\textbf{Condition}: \(\Delta\phi = 0°\) or \(180°\) (in-phase or
anti-phase)

\textbf{Result}: E-field oscillates along a \textbf{fixed line}

\paragraph{Vertical Polarization}\label{vertical-polarization}

\[
\vec{E}(t) = E_0 \cos(\omega t)\hat{y}
\]

\textbf{E-field aligned with y-axis} (vertical if antenna vertical)

\textbf{Applications}: - AM/FM broadcast (vertical monopoles) - HF
vertical antennas (ground wave propagation) - Mobile handsets (typically
held vertically)

\begin{center}\rule{0.5\linewidth}{0.5pt}\end{center}

\paragraph{Horizontal Polarization}\label{horizontal-polarization}

\[
\vec{E}(t) = E_0 \cos(\omega t)\hat{x}
\]

\textbf{E-field aligned with x-axis} (horizontal)

\textbf{Applications}: - TV broadcast (horizontal dipoles) - WiFi (many
routers use horizontal dipoles) - Yagi antennas (horizontal for TV
reception)

\begin{center}\rule{0.5\linewidth}{0.5pt}\end{center}

\paragraph{Slant Polarization}\label{slant-polarization}

\[
\vec{E}(t) = E_0 \cos(\omega t)(\cos\theta\hat{x} + \sin\theta\hat{y})
\]

\textbf{E-field at angle} \(\theta\) from horizontal

\textbf{Example}: 45\$\^{}\textbackslash circ\$ slant
(\$\textbackslash pm\$45\$\^{}\textbackslash circ\$):

\[
\vec{E}(t) = \frac{E_0}{\sqrt{2}} \cos(\omega t)(\hat{x} + \hat{y})
\]

\textbf{Applications}: - Satellite polarization diversity
(\$\textbackslash pm\$45\$\^{}\textbackslash circ\$ orthogonal channels)
- Reduce building penetration loss (less reflection)

\begin{center}\rule{0.5\linewidth}{0.5pt}\end{center}

\subsubsection{2. Circular Polarization}\label{circular-polarization}

\textbf{Condition}: \(E_x = E_y\) and \(\Delta\phi = \pm 90°\)

\textbf{Result}: E-field tip traces a \textbf{circle}, rotating as wave
propagates

\paragraph{Right-Hand Circular Polarization
(RHCP)}\label{right-hand-circular-polarization-rhcp}

\[
\vec{E}(t) = E_0[\cos(\omega t)\hat{x} - \sin(\omega t)\hat{y}]
\]

\textbf{Viewed from receiver} (wave approaching): E-field rotates
\textbf{clockwise}

\textbf{Phase}: \(\Delta\phi = -90°\) (y lags x by
90\$\^{}\textbackslash circ\$)

\begin{center}\rule{0.5\linewidth}{0.5pt}\end{center}

\paragraph{Left-Hand Circular Polarization
(LHCP)}\label{left-hand-circular-polarization-lhcp}

\[
\vec{E}(t) = E_0[\cos(\omega t)\hat{x} + \sin(\omega t)\hat{y}]
\]

\textbf{Viewed from receiver}: E-field rotates \textbf{counterclockwise}

\textbf{Phase}: \(\Delta\phi = +90°\) (y leads x by
90\$\^{}\textbackslash circ\$)

\begin{center}\rule{0.5\linewidth}{0.5pt}\end{center}

\paragraph{Properties}\label{properties}

\textbf{Axial ratio}: AR = 1 (perfect circle)

\textbf{Isolation between RHCP/LHCP}: Theoretically infinite
(orthogonal)

\textbf{Practical isolation}: 20-30 dB (antenna imperfections)

\textbf{Applications}: - \textbf{GPS satellites} (RHCP) - Mitigates
Faraday rotation, multipath - \textbf{Satellite communications} (RHCP or
LHCP) - Reduces rain depolarization - \textbf{RFID tags} -
Orientation-insensitive - \textbf{Radar} (circular) - Target
discrimination via polarization

\begin{center}\rule{0.5\linewidth}{0.5pt}\end{center}

\subsubsection{3. Elliptical
Polarization}\label{elliptical-polarization}

\textbf{Condition}: General case where \(E_x \neq E_y\) and/or
\(\Delta\phi \neq 0°, 90°, 180°\)

\textbf{Result}: E-field tip traces an \textbf{ellipse}

\[
\frac{E_x^2(t)}{A^2} + \frac{E_y^2(t)}{B^2} = 1
\]

Where A, B are semi-major/minor axes

\begin{center}\rule{0.5\linewidth}{0.5pt}\end{center}

\paragraph{Axial Ratio (AR)}\label{axial-ratio-ar}

\textbf{Measure of ellipse eccentricity}:

\[
\text{AR} = \frac{\text{Major axis}}{\text{Minor axis}} = \frac{A}{B}
\]

\textbf{In dB}:

\[
\text{AR}_{\text{dB}} = 20\log_{10}\left(\frac{A}{B}\right)
\]

\textbf{Special cases}: - \textbf{AR = 1 (0 dB)}: Circular polarization
- \textbf{AR \$\textbackslash rightarrow\$ \$\textbackslash infty\$}:
Linear polarization

\textbf{Typical spec}: AR \textless{} 3 dB for ``circular'' antennas
(ellipticity acceptable)

\begin{center}\rule{0.5\linewidth}{0.5pt}\end{center}

\paragraph{Sense of Rotation}\label{sense-of-rotation}

\textbf{Right-hand elliptical}: Rotates clockwise (RHEP)

\textbf{Left-hand elliptical}: Rotates counterclockwise (LHEP)

\textbf{Example}: \(E_x = 2E_y\), \(\Delta\phi = 90°\) - Elliptical (not
circular due to unequal amplitudes) - Right-hand sense
(90\$\^{}\textbackslash circ\$ phase like RHCP) - AR = 2 (6 dB)

\begin{center}\rule{0.5\linewidth}{0.5pt}\end{center}

\subsection{Polarization Loss Factor
(PLF)}\label{polarization-loss-factor-plf}

\textbf{Mismatch between TX and RX polarizations causes loss}:

\[
\text{PLF} = |\hat{e}_{\text{TX}} \cdot \hat{e}_{\text{RX}}^*|^2
\]

Where \(\hat{e}\) = Normalized polarization vectors (complex)

\begin{center}\rule{0.5\linewidth}{0.5pt}\end{center}

\subsubsection{Linear Polarizations}\label{linear-polarizations}

\textbf{Angle mismatch} \(\theta\) between TX and RX:

\[
\text{PLF} = \cos^2\theta
\]

\textbf{In dB}:

\[
L_{\text{pol}} = -10\log_{10}(\cos^2\theta) = -20\log_{10}(\cos\theta)
\]

\textbf{Examples}: - \textbf{0\$\^{}\textbackslash circ\$}: 0 dB loss
(perfect match) - \textbf{30\$\^{}\textbackslash circ\$}: 1.2 dB loss -
\textbf{45\$\^{}\textbackslash circ\$}: 3 dB loss (half power) -
\textbf{90\$\^{}\textbackslash circ\$}: \$\textbackslash infty\$ dB loss
(complete null - orthogonal)

\begin{center}\rule{0.5\linewidth}{0.5pt}\end{center}

\subsubsection{Circular Polarizations}\label{circular-polarizations}

{\def\LTcaptype{} % do not increment counter
\begin{longtable}[]{@{}llll@{}}
\toprule\noalign{}
TX & RX & PLF & Loss \\
\midrule\noalign{}
\endhead
\bottomrule\noalign{}
\endlastfoot
RHCP & RHCP & 1 & 0 dB (match) \\
LHCP & LHCP & 1 & 0 dB (match) \\
RHCP & LHCP & 0 & \$\textbackslash infty\$ dB (null) \\
LHCP & RHCP & 0 & \$\textbackslash infty\$ dB (null) \\
\end{longtable}
}

\textbf{Co-pol vs cross-pol}: - \textbf{Co-pol}: Same sense (RHCP-RHCP
or LHCP-LHCP) - \textbf{Cross-pol}: Opposite sense (RHCP-LHCP or
LHCP-RHCP)

\begin{center}\rule{0.5\linewidth}{0.5pt}\end{center}

\subsubsection{Linear to Circular}\label{linear-to-circular}

\textbf{Linear antenna receiving circular wave} (or vice versa):

\[
\text{PLF} = 0.5 \quad (-3\ \text{dB loss})
\]

\textbf{Explanation}: Linear antenna captures only one component of
circular wave (e.g., vertical dipole receives only vertical component of
RHCP)

\textbf{Example}: GPS receiver with linear patch antenna - GPS
satellites transmit RHCP - Linear patch: 3 dB polarization loss - Need
higher gain to compensate

\begin{center}\rule{0.5\linewidth}{0.5pt}\end{center}

\subsection{Polarization Generation}\label{polarization-generation}

\subsubsection{Linear Polarization}\label{linear-polarization-1}

\textbf{Simple dipole or monopole}: - Current flows in one direction
\$\textbackslash rightarrow\$ E-field perpendicular to current -
Vertical monopole \$\textbackslash rightarrow\$ Vertical polarization -
Horizontal dipole \$\textbackslash rightarrow\$ Horizontal polarization

\begin{center}\rule{0.5\linewidth}{0.5pt}\end{center}

\subsubsection{Circular Polarization}\label{circular-polarization-1}

\paragraph{Crossed Dipoles with 90\$\^{}\textbackslash circ\$ Phase
Shift}\label{crossed-dipoles-with-90-phase-shift}

\textbf{Two perpendicular dipoles}, fed with: - Equal amplitude -
90\$\^{}\textbackslash circ\$ phase difference (quadrature)

\textbf{Geometry}:

\begin{verbatim}
      y (vertical dipole)
      |
      |
      +------ x (horizontal dipole)
\end{verbatim}

\textbf{Feed}: - Horizontal: \(I_x = I_0 \cos(\omega t)\) - Vertical:
\(I_y = I_0 \cos(\omega t - 90°) = I_0 \sin(\omega t)\)

\textbf{Result}: RHCP (assuming correct phase)

\textbf{Implementation}: 90\$\^{}\textbackslash circ\$ hybrid coupler
(branch-line, Lange coupler)

\begin{center}\rule{0.5\linewidth}{0.5pt}\end{center}

\paragraph{Helical Antenna}\label{helical-antenna}

\textbf{Helix wound around cylinder} (axial mode):

\textbf{Geometry}: - Diameter: \(D \approx \lambda/\pi\) (circumference
\$\textbackslash approx\$ \$\textbackslash lambda\$) - Pitch angle:
12-15\$\^{}\textbackslash circ\$ - Turns: 5-10 for good AR

\textbf{Result}: Circular polarization (sense depends on helix
direction) - Right-hand helix \$\textbackslash rightarrow\$ RHCP -
Left-hand helix \$\textbackslash rightarrow\$ LHCP

\textbf{Applications}: GPS antennas, satellite ground stations

\begin{center}\rule{0.5\linewidth}{0.5pt}\end{center}

\paragraph{Patch Antenna with Corners
Truncated}\label{patch-antenna-with-corners-truncated}

\textbf{Circular or square patch} with: - Two opposite corners cut
(truncated) - Single feed point

\textbf{Mechanism}: Truncation creates two orthogonal modes with
\textasciitilde90\$\^{}\textbackslash circ\$ phase difference

\textbf{Result}: Circular polarization (RHCP or LHCP depending on which
corners cut)

\textbf{Applications}: GPS receivers, compact GNSS antennas

\begin{center}\rule{0.5\linewidth}{0.5pt}\end{center}

\subsection{Propagation Effects on
Polarization}\label{propagation-effects-on-polarization}

\subsubsection{Faraday Rotation}\label{faraday-rotation}

\textbf{Ionosphere causes polarization rotation} (linear
\$\textbackslash rightarrow\$ rotated linear):

\[
\Omega = 2.36 \times 10^4 \frac{B_\parallel \cdot \text{TEC}}{f^2} \quad (\text{radians})
\]

Where: - \(B_\parallel\) = Earth\textquotesingle s magnetic field
component along path (Tesla) - TEC = Total Electron Content
(electrons/m\textbackslash textsuperscript\{2\}) - \(f\) = Frequency
(Hz)

\textbf{Effect scales as} \(1/f^2\) (severe at HF, negligible at SHF)

\textbf{Example}: HF @ 10 MHz, TEC =
10\textbackslash textsuperscript\{1\}\textbackslash textsuperscript\{8\}
el/m\textbackslash textsuperscript\{2\} - Rotation:
\textasciitilde500\$\^{}\textbackslash circ\$ (multiple full rotations!)
- Linear polarization unusable (unpredictable rotation)

\textbf{Mitigation}: Use \textbf{circular polarization} (immune to
Faraday rotation)

\begin{center}\rule{0.5\linewidth}{0.5pt}\end{center}

\subsubsection{Differential Propagation (Rain
Depolarization)}\label{differential-propagation-rain-depolarization}

\textbf{Rain causes differential attenuation} between H and V
components:

\textbf{Horizontal attenuated more} than vertical (raindrops are oblate)

\textbf{Effect}: Linear \$\textbackslash rightarrow\$ Elliptical,
Circular \$\textbackslash rightarrow\$ Elliptical

\textbf{Cross-Polarization Discrimination (XPD)}:

\[
\text{XPD} = \frac{\text{Co-pol power}}{\text{Cross-pol power}} \quad (\text{dB})
\]

\textbf{Typical}: 30 dB in clear air, degrades to 15-20 dB in heavy rain

\textbf{Example}: Satellite Ku-band,
\$\textbackslash pm\$45\$\^{}\textbackslash circ\$ linear polarization -
Clear air: 30 dB isolation between channels - Heavy rain: 20 dB
isolation (increased interference)

\textbf{Mitigation}: Adaptive coding/modulation, switch to single
polarization in heavy rain

\begin{center}\rule{0.5\linewidth}{0.5pt}\end{center}

\subsubsection{Reflection}\label{reflection}

\textbf{Polarization changes upon reflection}:

\paragraph{Perpendicular Incidence
(Normal)}\label{perpendicular-incidence-normal}

\textbf{Horizontal and vertical polarizations reflect with} \(180°\)
\textbf{phase shift} (for good conductors)

\begin{center}\rule{0.5\linewidth}{0.5pt}\end{center}

\paragraph{Oblique Incidence}\label{oblique-incidence}

\textbf{Brewster angle} (\(\theta_B\)):

\[
\theta_B = \arctan\left(\frac{n_2}{n_1}\right)
\]

\textbf{At Brewster angle}: Parallel (horizontal) polarization
\textbf{not reflected} (complete transmission)

\textbf{Example}: Air-to-glass (\(n_1=1\), \(n_2=1.5\)):

\[
\theta_B = \arctan(1.5) \approx 56°
\]

\textbf{Application}: Polarizing filters, anti-reflection coatings

\begin{center}\rule{0.5\linewidth}{0.5pt}\end{center}

\subsection{Applications}\label{applications}

\subsubsection{1. Satellite
Communications}\label{satellite-communications}

\textbf{Frequency reuse via polarization diversity}:

\textbf{Traditional}: H and V (or
\$\textbackslash pm\$45\$\^{}\textbackslash circ\$ linear) - Two
independent channels on same frequency - Isolation: 25-30 dB (limited by
cross-pol)

\textbf{Modern}: RHCP and LHCP - Better rain performance (less
depolarization) - Isolation: 20-30 dB

\textbf{Example}: Ku-band DBS (Direct Broadcast Satellite) - 12 GHz
downlink - Odd transponders: RHCP - Even transponders: LHCP - Doubles
capacity

\begin{center}\rule{0.5\linewidth}{0.5pt}\end{center}

\subsubsection{2. GPS and GNSS}\label{gps-and-gnss}

\textbf{All GPS satellites transmit RHCP}:

\textbf{Reasons}: 1. \textbf{Faraday rotation immunity}: Circular
unaffected by ionosphere rotation 2. \textbf{Multipath rejection}:
Ground reflection flips RHCP \$\textbackslash rightarrow\$ LHCP
(cross-pol rejected) 3. \textbf{Orientation insensitive}: Works at any
receiver angle (within hemisphere)

\textbf{Receiver antenna}: RHCP patch or helix

\begin{center}\rule{0.5\linewidth}{0.5pt}\end{center}

\subsubsection{3. Radar}\label{radar}

\textbf{Polarimetric radar} uses multiple polarizations:

\textbf{Modes}: - HH: Transmit H, receive H - VV: Transmit V, receive V
- HV: Transmit H, receive V (cross-pol) - VH: Transmit V, receive H
(cross-pol)

\textbf{Applications}: - \textbf{Weather radar}: Distinguish rain, hail,
snow (different depolarization) - \textbf{SAR imaging}: Surface type
classification (vegetation vs metal vs water) - \textbf{Target
identification}: Military (tanks vs trees)

\begin{center}\rule{0.5\linewidth}{0.5pt}\end{center}

\subsubsection{4. WiFi and Cellular}\label{wifi-and-cellular}

\textbf{Diversity antennas} use orthogonal polarizations:

\textbf{MIMO systems}: 2\$\textbackslash times\$2,
4\$\textbackslash times\$4 with
\$\textbackslash pm\$45\$\^{}\textbackslash circ\$ slant polarization -
Reduce correlation between antenna elements - Improve capacity in rich
scattering environments

\textbf{Example}: WiFi 802.11n/ac router - Antenna 1:
+45\$\^{}\textbackslash circ\$ slant - Antenna 2:
-45\$\^{}\textbackslash circ\$ slant - Independent fading
\$\textbackslash rightarrow\$ Diversity gain

\begin{center}\rule{0.5\linewidth}{0.5pt}\end{center}

\subsubsection{5. EMI/EMC Testing}\label{emiemc-testing}

\textbf{Measure emissions in both H and V polarizations}:

\textbf{Standards} (FCC, CISPR): Require testing at both polarizations
to find worst-case emissions

\begin{center}\rule{0.5\linewidth}{0.5pt}\end{center}

\subsection{Stokes Parameters}\label{stokes-parameters}

\textbf{Complete polarization description} (intensity + polarization
state):

\[
\begin{aligned}
S_0 &= E_x^2 + E_y^2 \quad (\text{Total intensity}) \\
S_1 &= E_x^2 - E_y^2 \quad (\text{H vs V preference}) \\
S_2 &= 2E_xE_y\cos\Delta\phi \quad (\text{±45° preference}) \\
S_3 &= 2E_xE_y\sin\Delta\phi \quad (\text{Circular preference})
\end{aligned}
\]

\textbf{Interpretation}: - \(S_3 > 0\): RHCP dominant - \(S_3 < 0\):
LHCP dominant - \(S_3 = 0\): Linear polarization

\textbf{Degree of polarization}:

\[
\text{DOP} = \frac{\sqrt{S_1^2 + S_2^2 + S_3^2}}{S_0}
\]

\textbf{Range}: 0 (unpolarized) to 1 (fully polarized)

\begin{center}\rule{0.5\linewidth}{0.5pt}\end{center}

\subsection{Poincaré Sphere}\label{poincaruxe9-sphere}

\textbf{Graphical representation} of polarization states:

\textbf{3D sphere} where: - \textbf{North pole}: LHCP - \textbf{South
pole}: RHCP - \textbf{Equator}: All linear polarizations (H, V,
\$\textbackslash pm\$45\$\^{}\textbackslash circ\$)

\textbf{Coordinates}: \((S_1/S_0, S_2/S_0, S_3/S_0)\)

\textbf{Application}: Visualize polarization transformations (e.g.,
Faraday rotation = rotation around S\textbackslash textsubscript\{3\}
axis)

\begin{center}\rule{0.5\linewidth}{0.5pt}\end{center}

\subsection{Polarization Measurement}\label{polarization-measurement}

\subsubsection{Antenna Pattern Testing}\label{antenna-pattern-testing}

\textbf{Measure co-pol and cross-pol patterns}:

\textbf{Setup}: 1. Rotate RX antenna 90\$\^{}\textbackslash circ\$ (H
\$\textbackslash leftrightarrow\$ V) 2. Measure received power vs angle
3. Plot co-pol and cross-pol gains

\textbf{Cross-pol level}: Typically 20-30 dB below co-pol for good
antenna

\begin{center}\rule{0.5\linewidth}{0.5pt}\end{center}

\subsubsection{Polarization Ratio}\label{polarization-ratio}

\textbf{For linear polarization}:

\[
\text{PR} = \frac{P_{\text{co}}}{P_{\text{cross}}} \quad (\text{dB})
\]

\textbf{Typical}: \textgreater{} 20 dB for well-designed antenna

\begin{center}\rule{0.5\linewidth}{0.5pt}\end{center}

\subsubsection{Axial Ratio Measurement}\label{axial-ratio-measurement}

\textbf{For circular polarization}:

\textbf{Method 1}: Spinning linear dipole - Rotate RX linear antenna
360\$\^{}\textbackslash circ\$ - Measure \(P_{\text{max}}\) and
\(P_{\text{min}}\)

\[
\text{AR} = \frac{P_{\text{max}}}{P_{\text{min}}} \quad (\text{linear}), \quad \text{AR}_{\text{dB}} = 10\log_{10}\left(\frac{P_{\text{max}}}{P_{\text{min}}}\right)
\]

\textbf{Method 2}: Dual-pol receiver - Measure \(E_x\) and \(E_y\)
amplitudes and phases - Calculate AR from ellipse parameters

\textbf{Good circular antenna}: AR \textless{} 3 dB

\begin{center}\rule{0.5\linewidth}{0.5pt}\end{center}

\subsection{Practical Considerations}\label{practical-considerations}

\subsubsection{Antenna Orientation}\label{antenna-orientation}

\textbf{Must match TX/RX polarizations}:

\textbf{Example}: Vertical monopole on car - Works well with vertical
base station antenna - 90\$\^{}\textbackslash circ\$ mismatch if base
station is horizontal (\$\textbackslash infty\$ dB loss)

\textbf{WiFi routers}: Often mixed polarizations (multiple antennas at
different angles) for robustness

\begin{center}\rule{0.5\linewidth}{0.5pt}\end{center}

\subsubsection{Polarization vs
Bandwidth}\label{polarization-vs-bandwidth}

\textbf{Circular polarization antennas are narrowband}:

\textbf{Reason}: 90\$\^{}\textbackslash circ\$ phase shift only accurate
over limited bandwidth

\textbf{Typical}: 2-5\% bandwidth for AR \textless{} 3 dB

\textbf{Wideband circular polarization}: Difficult (requires complex
feed networks)

\begin{center}\rule{0.5\linewidth}{0.5pt}\end{center}

\subsubsection{Cost/Complexity}\label{costcomplexity}

{\def\LTcaptype{} % do not increment counter
\begin{longtable}[]{@{}llll@{}}
\toprule\noalign{}
Polarization & Complexity & Cost & Applications \\
\midrule\noalign{}
\endhead
\bottomrule\noalign{}
\endlastfoot
\textbf{Linear} & Simple & Low & Most terrestrial (AM/FM, WiFi,
cellular) \\
\textbf{Circular} & Moderate & Medium & GPS, satellite, RFID \\
\textbf{Dual-pol} & High & High & Radar, satellite (frequency reuse) \\
\end{longtable}
}

\begin{center}\rule{0.5\linewidth}{0.5pt}\end{center}

\subsection{Summary Table}\label{summary-table}

{\def\LTcaptype{} % do not increment counter
\begin{longtable}[]{@{}
  >{\raggedright\arraybackslash}p{(\linewidth - 8\tabcolsep) * \real{0.2059}}
  >{\raggedright\arraybackslash}p{(\linewidth - 8\tabcolsep) * \real{0.2059}}
  >{\raggedright\arraybackslash}p{(\linewidth - 8\tabcolsep) * \real{0.1912}}
  >{\raggedright\arraybackslash}p{(\linewidth - 8\tabcolsep) * \real{0.1912}}
  >{\raggedright\arraybackslash}p{(\linewidth - 8\tabcolsep) * \real{0.2059}}@{}}
\toprule\noalign{}
\begin{minipage}[b]{\linewidth}\raggedright
Polarization
\end{minipage} & \begin{minipage}[b]{\linewidth}\raggedright
\(\Delta\phi\)
\end{minipage} & \begin{minipage}[b]{\linewidth}\raggedright
\(E_x : E_y\)
\end{minipage} & \begin{minipage}[b]{\linewidth}\raggedright
Axial Ratio
\end{minipage} & \begin{minipage}[b]{\linewidth}\raggedright
Applications
\end{minipage} \\
\midrule\noalign{}
\endhead
\bottomrule\noalign{}
\endlastfoot
\textbf{Horizontal} & 0\$\^{}\textbackslash circ\$ &
\$\textbackslash infty\$ : 1 & \$\textbackslash infty\$ & TV broadcast,
WiFi \\
\textbf{Vertical} & 0\$\^{}\textbackslash circ\$ & 1 :
\$\textbackslash infty\$ & \$\textbackslash infty\$ & AM/FM, mobile \\
\textbf{\$\textbackslash pm\$45\$\^{}\textbackslash circ\$ Slant} &
0\$\^{}\textbackslash circ\$ & 1 : 1 & \$\textbackslash infty\$ &
Satellite downlink \\
\textbf{RHCP} & -90\$\^{}\textbackslash circ\$ & 1 : 1 & 1 & GPS,
satellite \\
\textbf{LHCP} & +90\$\^{}\textbackslash circ\$ & 1 : 1 & 1 & Satellite,
RFID \\
\textbf{Elliptical} & Arbitrary & Arbitrary & 1-\$\textbackslash infty\$
& Imperfect circular \\
\end{longtable}
}

\begin{center}\rule{0.5\linewidth}{0.5pt}\end{center}

\subsection{Related Topics}\label{related-topics}

\begin{itemize}
\tightlist
\item
  \textbf{{[}{[}Maxwell\textquotesingle s-Equations-\&-Wave-Propagation{]}{]}}:
  E and H fields in EM waves
\item
  \textbf{{[}{[}Antenna-Theory-Basics{]}{]}}: Polarization matching for
  maximum gain
\item
  \textbf{{[}{[}Atmospheric-Effects-(Ionospheric,-Tropospheric){]}{]}}:
  Faraday rotation
\item
  \textbf{{[}{[}Multipath-Propagation-\&-Fading-(Rayleigh,-Rician){]}{]}}:
  Depolarization in multipath
\item
  \textbf{{[}{[}Free-Space-Path-Loss-(FSPL){]}{]}}: Friis equation
  assumes matched polarization
\end{itemize}

\begin{center}\rule{0.5\linewidth}{0.5pt}\end{center}

\textbf{Key takeaway}: \textbf{Polarization is the ``orientation'' of
the electric field}. Linear (H, V, slant) is simplest and most common.
Circular (RHCP, LHCP) is robust to ionospheric effects and multipath,
used in GPS and satellites. Mismatch causes 3 dB to
\$\textbackslash infty\$ dB loss depending on angle. Propagation effects
(Faraday rotation, rain depolarization) degrade polarization purity.
Match TX and RX polarizations for optimal link performance.

\begin{center}\rule{0.5\linewidth}{0.5pt}\end{center}

\emph{This wiki is part of the {[}{[}Home\textbar Chimera Project{]}{]}
documentation.}
