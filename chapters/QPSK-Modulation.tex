\section{QPSK Modulation}\label{qpsk-modulation}

\subsection{\texorpdfstring{ For Non-Technical
Readers}{ For Non-Technical Readers}}\label{for-non-technical-readers}

\textbf{QPSK is like using 4 different hand signals instead of
2-\/-\/-you can send messages twice as fast!}

\textbf{The idea}: Instead of just ``wave up'' or ``wave down'' (BPSK),
QPSK uses \textbf{4 directions}: - Up-right \$\textbackslash nearrow\$ =
00 - Up-left \$\textbackslash nwarrow\$ = 01\\
- Down-left \$\textbackslash swarrow\$ = 10 - Down-right
\$\textbackslash searrow\$ = 11

\textbf{Real-world use}: - \textbf{Satellite TV} (DVB-S): Uses QPSK for
reliable transmission from space - \textbf{4G LTE}: Uses QPSK when
signal is weak (more reliable than faster modes) - \textbf{GPS}: Newer
signals use QPSK for twice the data rate

\textbf{Why 4 directions?} - Sends \textbf{2 bits per symbol} (vs 1 bit
for BPSK) = twice as fast! - Still pretty reliable (the 4 directions are
well-separated) - Sweet spot between speed and robustness

\textbf{When you see it}: Your phone uses QPSK when cell signal is
weak-\/-\/-slower than 16-QAM or 64-QAM, but way more reliable.

\begin{center}\rule{0.5\linewidth}{0.5pt}\end{center}

\textbf{QPSK} stands for \textbf{Quadrature Phase-Shift Keying}.
It\textquotesingle s a modulation technique that encodes 2 bits per
symbol by varying the phase of the carrier wave.

\subsection{The Four QPSK States}\label{the-four-qpsk-states}

QPSK uses four distinct phase states, each representing a unique 2-bit
pattern:

\begin{verbatim}
        Q (Quadrature)
              
              |
    00       |       01
              |
   -----------+-----------> I (In-phase)
              |
    11       |       10
              |
\end{verbatim}

\subsection{Bit-to-Phase Mapping in
Chimera}\label{bit-to-phase-mapping-in-chimera}

\begin{itemize}
\tightlist
\item
  \texttt{00} \$\textbackslash rightarrow\$ Phase:
  135\$\^{}\textbackslash circ\$ (upper-left quadrant)
\item
  \texttt{01} \$\textbackslash rightarrow\$ Phase:
  45\$\^{}\textbackslash circ\$ (upper-right quadrant)
\item
  \texttt{11} \$\textbackslash rightarrow\$ Phase:
  225\$\^{}\textbackslash circ\$ (lower-left quadrant)
\item
  \texttt{10} \$\textbackslash rightarrow\$ Phase:
  315\$\^{}\textbackslash circ\$ (lower-right quadrant)
\end{itemize}

\subsection{Mathematical
Representation}\label{mathematical-representation}

For normalized QPSK (unit energy), the four symbols are:

\begin{verbatim}
Symbol 00: I = -1/2,  Q = +1/2
Symbol 01: I = +1/2,  Q = +1/2
Symbol 11: I = -1/2,  Q = -1/2
Symbol 10: I = +1/2,  Q = -1/2
\end{verbatim}

\subsection{Why QPSK?}\label{why-qpsk}

\begin{itemize}
\tightlist
\item
  \textbf{Spectral Efficiency}: Transmits 2 bits per symbol
\item
  \textbf{Robustness}: The large phase separation
  (90\$\^{}\textbackslash circ\$) makes it resilient to noise
\item
  \textbf{Simplicity}: Relatively simple to implement and demodulate
\item
  \textbf{Widespread Use}: Used in many real-world systems (satellite,
  WiFi, LTE)
\end{itemize}

\subsection{QPSK in Chimera}\label{qpsk-in-chimera}

Chimera\textquotesingle s implementation: - \textbf{Symbol Rate}:
Configurable (typically 16-1000 symbols/second depending on preset) -
\textbf{Carrier Frequency}: 12.0 kHz (audio frequency for demonstration)
- \textbf{Frame Structure}: Organized into sync, command, data, and ECC
sections

\subsection{See Also}\label{see-also}

\begin{itemize}
\tightlist
\item
  {[}{[}What-Are-Symbols{]}{]} - Understanding the fundamental unit
\item
  {[}{[}IQ-Representation{]}{]} - In-phase and Quadrature components
\item
  {[}{[}Constellation-Diagrams{]}{]} - Visualizing QPSK
\item
  {[}{[}Real-World-System-Examples{]}{]} - QPSK in GPS, DVB-S2, and LTE
\end{itemize}

\subsection{External Resources}\label{external-resources}

\textbf{Interactive Demonstrations}: -
\href{https://wiki.gnuradio.org/index.php/Guided_Tutorial_PSK_Demodulation}{GNURadio
QPSK Tutorial} - Hands-on QPSK implementation -
\href{https://www.dsprelated.com/showarticle/153.php}{DSP Related: QPSK
Articles} - Technical deep dive

\textbf{Signal Examples}: -
\href{https://www.sigidwiki.com/wiki/Category:QPSK}{sigidwiki QPSK
Signals} - Real-world QPSK signal recordings -
\href{https://www.etsi.org/deliver/etsi_en/302300_302399/30230701/}{DVB-S2
QPSK Usage} - Satellite TV standard (ETSI EN 302 307-1)

\textbf{Textbook References}: - Proakis \& Salehi, \emph{Digital
Communications} (5th ed.), Chapter 4.2 - QPSK theory - See
{[}{[}Bibliography{]}{]} for complete reference list
