\section{Baseband vs Passband
Signals}\label{baseband-vs-passband-signals}

{[}{[}Home{]}{]} \textbar{} \textbf{Digital Modulation} \textbar{}
{[}{[}QPSK-Modulation{]}{]} \textbar{} {[}{[}IQ-Representation{]}{]}

\begin{center}\rule{0.5\linewidth}{0.5pt}\end{center}

\subsection{\texorpdfstring{ For Non-Technical
Readers}{ For Non-Technical Readers}}\label{for-non-technical-readers}

\textbf{Baseband vs passband is like the difference between sheet music
(the notes you play) and the actual sound coming out of a trumpet
(shifted to a specific pitch range).}

\textbf{Baseband = The raw information}: - Your data, voice, video in
its original form - Frequency near 0 Hz (DC) - Like: Microphone output
(20 Hz - 20 kHz) - Example: MP3 file on your computer

\textbf{Passband = Information shifted to radio frequency}: - Same
information, but ``moved'' to carrier frequency - Frequency at
\textasciitilde MHz/GHz (radio waves) - Like: FM radio station at 101.5
MHz - Example: WiFi signal at 2.4 GHz carrying your data

\textbf{Musical analogy}: - \textbf{Baseband}: Musical melody (the
pattern of notes) - \textbf{Passband}: Same melody played on a flute
(high pitch) vs tuba (low pitch) - The melody (information) is
identical, just at different frequency ranges!

\textbf{Why we need BOTH}:

\textbf{Baseband is better for}: - Processing (computers work in
baseband) - Storage (files are baseband) - Display (audio speakers
output baseband) - Development (easier to analyze/test)

\textbf{Passband is better for}: - Radio transmission (antennas need
high frequency) - Multiple channels (FM 88.1, 88.3, 88.5
don\textquotesingle t interfere) - Long distance (higher frequency =
better propagation) - Regulation (FCC assigns frequency bands)

\textbf{Real examples}:

\textbf{Your phone call journey}: 1. \textbf{Your voice}: Baseband (20
Hz - 3.4 kHz) 2. \textbf{Cell phone transmitter}: Shifts to passband
(e.g., 1.9 GHz) 3. \textbf{Over the air}: Passband signal travels to
tower 4. \textbf{Tower receiver}: Shifts back down to baseband 5.
\textbf{Phone network}: Processes in baseband 6.
\textbf{Recipient\textquotesingle s phone}: Shifts to passband again
(transmit) 7. \textbf{Recipient\textquotesingle s speaker}: Back to
baseband (audio)

\textbf{WiFi example}: - \textbf{Your laptop}: Creates baseband IQ data
(MHz range) - \textbf{WiFi chip}: Shifts baseband up to 2.4 GHz or 5 GHz
(passband) - \textbf{Transmit antenna}: Radiates passband signal -
\textbf{Router antenna}: Receives passband signal\\
- \textbf{Router WiFi chip}: Shifts back down to baseband -
\textbf{Router processes}: Baseband Ethernet data

\textbf{The frequency shift process = ``modulation''}: -
\textbf{Upconversion}: Baseband \$\textbackslash rightarrow\$ Passband
(multiply by carrier) - \textbf{Downconversion}: Passband
\$\textbackslash rightarrow\$ Baseband (multiply by carrier again) -
Same information, just at different frequencies!

\textbf{Why antennas need passband}: - Efficient antenna size
\$\textbackslash approx\$ \$\textbackslash lambda\$/2 (half wavelength)
- Audio (baseband): 20 Hz \$\textbackslash rightarrow\$
\$\textbackslash lambda\$ = 15,000 km \$\textbackslash rightarrow\$
antenna = 7,500 km! - WiFi (passband): 2.4 GHz
\$\textbackslash rightarrow\$ \$\textbackslash lambda\$ = 12.5 cm
\$\textbackslash rightarrow\$ antenna = 6 cm

\textbf{Fun fact}: Software Defined Radio (SDR) works by keeping signals
in baseband as long as possible-\/-\/-only converting to passband at the
last moment. This is why your phone\textquotesingle s ``radio'' is
mostly software running on baseband signals!

\begin{center}\rule{0.5\linewidth}{0.5pt}\end{center}

\subsection{Overview}\label{overview}

\textbf{Baseband signal}: Information signal at \textbf{original
frequency range} (near DC, \textasciitilde0 Hz)

\textbf{Passband signal}: Information signal \textbf{shifted to carrier
frequency} \(f_c\) (RF)

\textbf{Why we need both}: - \textbf{Baseband}: Digital signal
processing, modulation/demodulation, algorithm development -
\textbf{Passband}: Radio transmission (antennas need RF, spectrum
allocation, propagation)

\textbf{Key operation}: \textbf{Upconversion} (baseband
\$\textbackslash rightarrow\$ passband) and \textbf{downconversion}
(passband \$\textbackslash rightarrow\$ baseband)

\begin{center}\rule{0.5\linewidth}{0.5pt}\end{center}

\subsection{Baseband Signal}\label{baseband-signal}

\textbf{Definition}: Signal with frequency content centered
\textbf{around DC (0 Hz)}

\textbf{Spectrum}: Extends from \textasciitilde0 Hz to \(B\) Hz
(bandwidth)

\subsubsection{Examples}\label{examples}

\textbf{Digital baseband}: - NRZ (Non-Return-to-Zero): Rectangular
pulses, \$\textbackslash pm\$1 - Manchester encoding: Phase transitions
- Pulse-shaped symbols: Raised cosine, RRC

\textbf{Analog baseband}: - Voice: 300-3400 Hz - Audio: 20 Hz - 20 kHz -
Video: DC - 6 MHz (NTSC)

\begin{center}\rule{0.5\linewidth}{0.5pt}\end{center}

\subsubsection{Complex Baseband
Representation}\label{complex-baseband-representation}

\textbf{For bandpass systems}, represent signal as \textbf{complex
envelope}:

\[
s(t) = s_I(t) + j s_Q(t)
\]

Where: - \(s_I(t)\) = In-phase component - \(s_Q(t)\) = Quadrature
component

\textbf{Advantages}: - Simplifies DSP (single complex signal vs two real
signals) - Natural representation for IQ modulation - Halves sampling
rate requirement (no negative frequencies)

\textbf{See}: {[}{[}IQ-Representation{]}{]}

\begin{center}\rule{0.5\linewidth}{0.5pt}\end{center}

\subsubsection{Baseband Bandwidth}\label{baseband-bandwidth}

\textbf{Occupied bandwidth} depends on symbol rate \(R_s\) and pulse
shaping:

\textbf{Ideal rectangular pulses}:

\[
B = R_s \quad (\text{Hz})
\]

\textbf{Raised cosine pulse shaping} (roll-off \(\alpha\)):

\[
B = R_s (1 + \alpha) \quad (\text{Hz})
\]

\textbf{Example}: QPSK @ 1 Msps, \$\textbackslash alpha\$ = 0.35 -
Bandwidth: 1 \$\textbackslash times\$ (1 + 0.35) = 1.35 MHz (baseband)

\begin{center}\rule{0.5\linewidth}{0.5pt}\end{center}

\subsection{Passband Signal}\label{passband-signal}

\textbf{Definition}: Signal with frequency content centered
\textbf{around carrier \(f_c\)}

\textbf{Spectrum}: Extends from \(f_c - B/2\) to \(f_c + B/2\)

\subsubsection{Why Passband?}\label{why-passband}

\begin{enumerate}
\def\labelenumi{\arabic{enumi}.}
\item
  \textbf{Antenna efficiency}: Antenna size \textasciitilde{}
  \$\textbackslash lambda\$/4, need high frequency for practical size

  \begin{itemize}
  \tightlist
  \item
    100 Hz baseband: \$\textbackslash lambda\$ = 3000 km
    \$\textbackslash rightarrow\$ 750 km antenna (infeasible!)
  \item
    2.4 GHz RF: \$\textbackslash lambda\$ = 12.5 cm
    \$\textbackslash rightarrow\$ 3 cm antenna (WiFi)
  \end{itemize}
\item
  \textbf{Spectrum allocation}: Different services assigned different
  frequency bands (AM 540-1600 kHz, FM 88-108 MHz, WiFi 2.4/5 GHz)
\item
  \textbf{Propagation characteristics}: HF skips ionosphere, VHF
  line-of-sight, UHF penetrates buildings
\item
  \textbf{Multiplexing}: Multiple baseband signals upconverted to
  different carriers (FDM)
\end{enumerate}

\begin{center}\rule{0.5\linewidth}{0.5pt}\end{center}

\subsubsection{Passband Representation}\label{passband-representation}

\textbf{Real passband signal} from complex baseband:

\[
s_{\text{RF}}(t) = \text{Re}\{s(t) e^{j2\pi f_c t}\}
\]

\[
= s_I(t) \cos(2\pi f_c t) - s_Q(t) \sin(2\pi f_c t)
\]

\textbf{Interpretation}: \textbf{IQ modulation} - I channel modulates
cosine (0\$\^{}\textbackslash circ\$ phase) - Q channel modulates sine
(90\$\^{}\textbackslash circ\$ phase)

\textbf{Example}: QPSK - \(s(t) = A e^{j\phi}\) where
\(\phi \in \{45°, 135°, 225°, 315°\}\) - \(s_I(t) = A\cos\phi\),
\(s_Q(t) = A\sin\phi\) -
\(s_{\text{RF}}(t) = A\cos\phi \cos(2\pi f_c t) - A\sin\phi \sin(2\pi f_c t) = A\cos(2\pi f_c t + \phi)\)

\begin{center}\rule{0.5\linewidth}{0.5pt}\end{center}

\subsection{Upconversion (Modulation)}\label{upconversion-modulation}

\textbf{Process}: Shift baseband signal to carrier frequency

\subsubsection{IQ Modulator (Quadrature
Modulator)}\label{iq-modulator-quadrature-modulator}

\textbf{Block diagram}:

\begin{verbatim}
           cos(2f_c t)
                |
    s_I(t) --> [×] ----\
                        [+] --> s_RF(t)
    s_Q(t) --> [×] ----/
                |
          -sin(2f_c t)
\end{verbatim}

\textbf{Output}:

\[
s_{\text{RF}}(t) = s_I(t) \cos(2\pi f_c t) - s_Q(t) \sin(2\pi f_c t)
\]

\begin{center}\rule{0.5\linewidth}{0.5pt}\end{center}

\subsubsection{Single-Sideband (SSB)
Upconversion}\label{single-sideband-ssb-upconversion}

\textbf{Complex multiplication}:

\[
s_{\text{RF}}(t) = \text{Re}\{s(t) e^{j2\pi f_c t}\}
\]

\textbf{In frequency domain}:

\[
S_{\text{RF}}(f) = \frac{1}{2}[S(f - f_c) + S^*(-f - f_c)]
\]

\textbf{Result}: Positive frequencies shifted to \(f_c\), negative
frequencies to \(-f_c\) (conjugate)

\textbf{Since \(s(t)\) real RF signal}: Spectrum symmetric around 0, so
both sidebands present

\begin{center}\rule{0.5\linewidth}{0.5pt}\end{center}

\subsubsection{Image Rejection}\label{image-rejection}

\textbf{Problem}: Real mixer produces both \(f_c + f_{\text{BB}}\) and
\(f_c - f_{\text{BB}}\) (USB and LSB)

\textbf{IQ modulator advantage}: Can select \textbf{one sideband} by
controlling I/Q phase - USB only: I/Q phase =
+90\$\^{}\textbackslash circ\$ - LSB only: I/Q phase =
-90\$\^{}\textbackslash circ\$ - DSB: I only (Q = 0)

\begin{center}\rule{0.5\linewidth}{0.5pt}\end{center}

\subsubsection{Example: WiFi 2.4 GHz}\label{example-wifi-2.4-ghz}

\textbf{Baseband}: - Symbol rate: 20 Msps (20 MHz OFDM) - Complex
baseband: -10 MHz to +10 MHz

\textbf{Upconversion}: - Carrier: 2.412 GHz (channel 1) - RF spectrum:
2.402-2.422 GHz (20 MHz)

\textbf{Transmit chain}: 1. Generate OFDM baseband (I/Q symbols) 2. DAC
@ 40 Msps (2\$\textbackslash times\$ oversampling) 3. IQ modulator @
2.412 GHz 4. PA \$\textbackslash rightarrow\$ antenna

\begin{center}\rule{0.5\linewidth}{0.5pt}\end{center}

\subsection{Downconversion
(Demodulation)}\label{downconversion-demodulation}

\textbf{Process}: Shift RF signal back to baseband

\subsubsection{IQ Demodulator (Quadrature
Demodulator)}\label{iq-demodulator-quadrature-demodulator}

\textbf{Block diagram}:

\begin{verbatim}
               cos(2f_c t)
                    |
s_RF(t) --> [×] --> [LPF] --> s_I(t)
         |
         |  -sin(2f_c t)
         |      |
         +--> [×] --> [LPF] --> s_Q(t)
\end{verbatim}

\textbf{I channel}:

\[
s_I(t) = \text{LPF}\{s_{\text{RF}}(t) \cos(2\pi f_c t)\}
\]

\textbf{Q channel}:

\[
s_Q(t) = \text{LPF}\{s_{\text{RF}}(t) \cdot [-\sin(2\pi f_c t)]\}
\]

\begin{center}\rule{0.5\linewidth}{0.5pt}\end{center}

\subsubsection{Mathematical Derivation}\label{mathematical-derivation}

\textbf{Input}:

\[
s_{\text{RF}}(t) = s_I^{\text{TX}}(t) \cos(2\pi f_c t) - s_Q^{\text{TX}}(t) \sin(2\pi f_c t)
\]

\textbf{I channel after mixing}:

\[
s_I^{\text{mix}}(t) = [s_I^{\text{TX}} \cos(2\pi f_c t) - s_Q^{\text{TX}} \sin(2\pi f_c t)] \cos(2\pi f_c t)
\]

\[
= s_I^{\text{TX}} \cos^2(2\pi f_c t) - s_Q^{\text{TX}} \sin(2\pi f_c t)\cos(2\pi f_c t)
\]

\textbf{Using trig identities}: -
\(\cos^2\theta = \frac{1 + \cos(2\theta)}{2}\) -
\(\sin\theta\cos\theta = \frac{\sin(2\theta)}{2}\)

\[
s_I^{\text{mix}}(t) = s_I^{\text{TX}} \frac{1 + \cos(4\pi f_c t)}{2} - s_Q^{\text{TX}} \frac{\sin(4\pi f_c t)}{2}
\]

\textbf{After LPF} (removes \(2f_c\) terms):

\[
s_I(t) = \frac{1}{2} s_I^{\text{TX}}(t)
\]

\textbf{Similarly for Q channel}:

\[
s_Q(t) = \frac{1}{2} s_Q^{\text{TX}}(t)
\]

\textbf{Recovered baseband} (with 1/2 amplitude, easily corrected):

\[
s(t) = s_I(t) + j s_Q(t) = \frac{1}{2}[s_I^{\text{TX}}(t) + j s_Q^{\text{TX}}(t)]
\]

\begin{center}\rule{0.5\linewidth}{0.5pt}\end{center}

\subsubsection{Image Frequency}\label{image-frequency}

\textbf{Problem}: Mixer sensitive to both \(f_c + f\) and \(f_c - f\)

\textbf{Image frequency}:
\(f_{\text{image}} = 2f_c - f_{\text{desired}}\)

\textbf{Example}: Desired signal @ 2.45 GHz, LO @ 2.4 GHz -
Downconverted to: 2.45 - 2.4 = 50 MHz - Image @ 2.4 - 0.05 = 2.35 GHz
also downconverts to 50 MHz!

\textbf{Mitigation}: - \textbf{Image-reject filter} before mixer (RF
bandpass filter) - \textbf{IQ demodulator} (natural image rejection if
I/Q balanced) - \textbf{Superheterodyne} (multiple conversion stages
with filtering)

\begin{center}\rule{0.5\linewidth}{0.5pt}\end{center}

\subsection{Superheterodyne Receiver}\label{superheterodyne-receiver}

\textbf{Classic architecture}: RF \$\textbackslash rightarrow\$ IF
\$\textbackslash rightarrow\$ Baseband

\textbf{Stages}: 1. \textbf{RF stage}: LNA, RF bandpass filter 2.
\textbf{First mixer}: RF \$\textbackslash rightarrow\$ IF (intermediate
frequency, e.g., 10.7 MHz for FM radio) 3. \textbf{IF stage}: IF filter
(high selectivity), IF amplifier 4. \textbf{Second mixer}: IF
\$\textbackslash rightarrow\$ Baseband (or direct demodulation at IF)

\textbf{Advantages}: - \textbf{Image rejection}: IF filter very
selective - \textbf{Fixed IF}: Optimized filters regardless of RF tuning
- \textbf{Gain distribution}: Spread gain across stages (avoid
instability)

\textbf{Example}: FM radio receiver - RF: 88-108 MHz (tunable) - LO:
98.7-118.7 MHz (tracks RF + 10.7 MHz) - IF: 10.7 MHz (fixed) - Crystal
filter @ IF: 150 kHz bandwidth (adjacent channel rejection)

\begin{center}\rule{0.5\linewidth}{0.5pt}\end{center}

\subsection{Zero-IF (Direct Conversion)
Receiver}\label{zero-if-direct-conversion-receiver}

\textbf{Modern SDR approach}: RF \$\textbackslash rightarrow\$ Baseband
(no IF)

\textbf{Advantages}: - Fewer components (no IF filters, single LO) -
Compact, low power (mobile devices) - Flexible (software-defined
bandwidth)

\textbf{Challenges}: - \textbf{DC offset}: LO leakage self-mixes to DC
(corrupts baseband) - \textbf{Flicker noise}: 1/f noise near DC -
\textbf{I/Q imbalance}: Gain/phase mismatch between I/Q paths

\textbf{Mitigation}: - AC coupling (removes DC) - High-pass filtering
(kills flicker noise) - Digital calibration (I/Q imbalance correction)

\begin{center}\rule{0.5\linewidth}{0.5pt}\end{center}

\subsection{Sampling Considerations}\label{sampling-considerations}

\subsubsection{Nyquist for Passband
Signals}\label{nyquist-for-passband-signals}

\textbf{Real passband signal} \(s_{\text{RF}}(t)\) centered at \(f_c\),
bandwidth \(B\):

\textbf{Bandpass sampling theorem}: Can sample at \(f_s < 2f_c\) if:

\[
f_s \geq 2B
\]

\textbf{Condition}: \(f_c = n \frac{f_s}{4}\) (integer \(n\)) for easy
downconversion

\textbf{Example}: WiFi @ 2.4 GHz, 20 MHz BW - Minimum
\(f_s = 2 \times 20 = 40\) MHz (bandpass sampling) - Typical \(f_s\) =
80-100 MHz (allows filtering roll-off)

\begin{center}\rule{0.5\linewidth}{0.5pt}\end{center}

\subsubsection{Complex Baseband
Sampling}\label{complex-baseband-sampling}

\textbf{Complex baseband} \(s(t) = s_I(t) + j s_Q(t)\):

\textbf{Sampling rate}:

\[
f_s \geq B \quad (\text{Hz})
\]

\textbf{Why lower?} Negative frequencies meaningful (complex signal
asymmetric)

\textbf{Example}: QPSK @ 1 MHz baseband bandwidth - Real passband @ 2.4
GHz: Need \(f_s \geq 2\) MHz (bandpass sampling) - Complex baseband:
Need \(f_s \geq 1\) MHz (but typically 2\$\textbackslash times\$ for
pulse shaping)

\begin{center}\rule{0.5\linewidth}{0.5pt}\end{center}

\subsection{Practical Impairments}\label{practical-impairments}

\subsubsection{1. Carrier Frequency Offset
(CFO)}\label{carrier-frequency-offset-cfo}

\textbf{TX and RX oscillators not perfectly matched}:

\[
\Delta f = f_{\text{TX}} - f_{\text{RX}}
\]

\textbf{Effect on baseband}:

\[
s_{\text{RX}}(t) = s(t) e^{j2\pi \Delta f t}
\]

\textbf{Consequence}: Constellation rotates over time

\textbf{Typical}: \$\textbackslash pm\$10 ppm (parts per million) - @
2.4 GHz: \$\textbackslash pm\$24 kHz offset - @ 28 GHz (5G mmWave):
\$\textbackslash pm\$280 kHz offset

\textbf{Mitigation}: Frequency synchronization (see
{[}{[}Synchronization-(Carrier,-Timing,-Frame){]}{]})

\begin{center}\rule{0.5\linewidth}{0.5pt}\end{center}

\subsubsection{2. Phase Noise}\label{phase-noise}

\textbf{Oscillator jitter} causes random phase variations:

\[
s_{\text{RF}}(t) = s_I(t) \cos(2\pi f_c t + \phi_n(t))
\]

Where \(\phi_n(t)\) = Random phase noise process

\textbf{Effect}: Constellation spreading, ICI (inter-carrier
interference in OFDM)

\textbf{Spec}: \(\mathcal{L}(f_m)\) (phase noise PSD at offset \(f_m\)
from carrier, dBc/Hz)

\textbf{Example}: Good TCXO @ 10 kHz offset - Phase noise: -110 dBc/Hz -
Integrated phase error: \textasciitilde1\$\^{}\textbackslash circ\$ RMS
(acceptable for QPSK)

\begin{center}\rule{0.5\linewidth}{0.5pt}\end{center}

\subsubsection{3. I/Q Imbalance}\label{iq-imbalance}

\textbf{Gain mismatch}: \(G_I \neq G_Q\)

\textbf{Phase mismatch}: 90\$\^{}\textbackslash circ\$ shifter imperfect
(e.g., 88\$\^{}\textbackslash circ\$ or 92\$\^{}\textbackslash circ\$)

\textbf{Effect}: Image sideband leakage, constellation distortion

\textbf{Model}:

\[
s_{\text{imb}}(t) = G_I s_I(t) + G_Q e^{j(\pi/2 + \epsilon)} s_Q(t)
\]

\textbf{Typical}: \$\textbackslash pm\$0.5 dB gain,
\$\textbackslash pm\$2\$\^{}\textbackslash circ\$ phase (good hardware)

\textbf{Mitigation}: Digital pre-distortion, calibration using known
pilots

\begin{center}\rule{0.5\linewidth}{0.5pt}\end{center}

\subsubsection{4. LO Leakage (DC Offset)}\label{lo-leakage-dc-offset}

\textbf{TX LO leaks} into RF path \$\textbackslash rightarrow\$
Self-mixing at RX \$\textbackslash rightarrow\$ DC component

\textbf{Effect}: DC spike in baseband spectrum

\textbf{Mitigation}: - AC coupling (blocks DC) - Blank center subcarrier
(OFDM) - Digital DC offset estimation/cancellation

\begin{center}\rule{0.5\linewidth}{0.5pt}\end{center}

\subsection{Spectral Efficiency
Comparison}\label{spectral-efficiency-comparison}

{\def\LTcaptype{} % do not increment counter
\begin{longtable}[]{@{}llll@{}}
\toprule\noalign{}
Architecture & Bandwidth Used & Spectral Efficiency & Example \\
\midrule\noalign{}
\endhead
\bottomrule\noalign{}
\endlastfoot
\textbf{Baseband (DSB)} & \(2B\) (USB + LSB) & N/A (not RF) &
Ethernet \\
\textbf{SSB (analog)} & \(B\) & 1\$\textbackslash times\$ & HAM radio \\
\textbf{DSB-SC} & \(2B\) & 0.5\$\textbackslash times\$ & AM radio
(suppressed carrier) \\
\textbf{VSB} & \(1.25B\) & 0.8\$\textbackslash times\$ & Analog TV \\
\textbf{IQ modulation} & \(B\) & 1\$\textbackslash times\$ & QPSK, QAM
(most digital) \\
\end{longtable}
}

\begin{center}\rule{0.5\linewidth}{0.5pt}\end{center}

\subsection{Summary Table}\label{summary-table}

{\def\LTcaptype{} % do not increment counter
\begin{longtable}[]{@{}
  >{\raggedright\arraybackslash}p{(\linewidth - 4\tabcolsep) * \real{0.2857}}
  >{\raggedright\arraybackslash}p{(\linewidth - 4\tabcolsep) * \real{0.3571}}
  >{\raggedright\arraybackslash}p{(\linewidth - 4\tabcolsep) * \real{0.3571}}@{}}
\toprule\noalign{}
\begin{minipage}[b]{\linewidth}\raggedright
Aspect
\end{minipage} & \begin{minipage}[b]{\linewidth}\raggedright
Baseband
\end{minipage} & \begin{minipage}[b]{\linewidth}\raggedright
Passband
\end{minipage} \\
\midrule\noalign{}
\endhead
\bottomrule\noalign{}
\endlastfoot
\textbf{Frequency range} & \textasciitilde0 to \(B\) Hz & \(f_c - B/2\)
to \(f_c + B/2\) \\
\textbf{Signal type} & Complex or real & Real only \\
\textbf{Sampling rate} & \(\geq B\) (complex) or \(\geq 2B\) (real) &
\(\geq 2B\) (bandpass) \\
\textbf{Processing} & Digital (DSP) & Analog (RF) or digital (SDR) \\
\textbf{Transmission} & Wired (Ethernet) & Wireless (antenna) \\
\textbf{Representation} & \(s(t) = s_I + js_Q\) &
\(s_{\text{RF}} = s_I\cos\omega t - s_Q\sin\omega t\) \\
\end{longtable}
}

\begin{center}\rule{0.5\linewidth}{0.5pt}\end{center}

\subsection{Related Topics}\label{related-topics}

\begin{itemize}
\tightlist
\item
  \textbf{{[}{[}IQ-Representation{]}{]}}: Complex baseband I/Q signals
\item
  \textbf{{[}{[}QPSK-Modulation{]}{]}}: Example of IQ modulation
\item
  \textbf{{[}{[}Constellation-Diagrams{]}{]}}: Visualizing baseband IQ
  symbols
\item
  \textbf{{[}{[}Synchronization-(Carrier,-Timing,-Frame){]}{]}}: Carrier
  frequency/phase recovery
\item
  \textbf{{[}{[}OFDM-\&-Multicarrier-Modulation{]}{]}}: Uses IQ
  modulation per subcarrier
\item
  \textbf{{[}{[}Free-Space-Path-Loss-(FSPL){]}{]}}: Why we need RF
  (antenna efficiency)
\end{itemize}

\begin{center}\rule{0.5\linewidth}{0.5pt}\end{center}

\textbf{Key takeaway}: \textbf{Baseband = information at low frequency,
passband = shifted to RF carrier.} IQ modulation (quadrature
upconversion) shifts complex baseband to RF without image.
Downconversion reverses process. Complex baseband simplifies DSP, halves
sample rate. Passband required for wireless (antenna, propagation,
spectrum). Practical impairments: CFO, phase noise, I/Q imbalance, LO
leakage. Superheterodyne =
RF\$\textbackslash rightarrow\$IF\$\textbackslash rightarrow\$BB
(classic), zero-IF = RF\$\textbackslash rightarrow\$BB (modern SDR).

\begin{center}\rule{0.5\linewidth}{0.5pt}\end{center}

\emph{This wiki is part of the {[}{[}Home\textbar Chimera Project{]}{]}
documentation.}
