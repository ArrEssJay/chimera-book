\section{8PSK \& Higher-Order PSK}\label{psk-higher-order-psk}

{[}{[}Home{]}{]} \textbar{} \textbf{Digital Modulation} \textbar{}
{[}{[}QPSK-Modulation{]}{]} \textbar{}
{[}{[}Binary-Phase-Shift-Keying-(BPSK){]}{]}

\begin{center}\rule{0.5\linewidth}{0.5pt}\end{center}

\subsection{\texorpdfstring{ For Non-Technical
Readers}{ For Non-Technical Readers}}\label{for-non-technical-readers}

\textbf{8PSK is like using 8 different hand gestures instead of
4-\/-\/-you can send 50\% more data, but the gestures are closer
together, so easier to confuse!}

\textbf{The progression}: - \textbf{BPSK}: 2 positions (up/down) = 1
bit/symbol - \textbf{QPSK}: 4 positions (4 corners) = 2 bits/symbol -
\textbf{8PSK}: 8 positions (8 directions) = 3 bits/symbol We are here -
\textbf{16PSK}: 16 positions = 4 bits/symbol - \textbf{32PSK}: 32
positions = 5 bits/symbol

\textbf{Visual - 8PSK positions (like a compass)}:

\begin{verbatim}
        N (000)
   NW   |   NE
  (001) | (011)
W ----+---- E
  (010) | (110)
   SW   |   SE
      S (100)
\end{verbatim}

\textbf{The trade-off}: - \textbf{More positions} = faster data rate! -
QPSK: 2 bits/symbol - 8PSK: 3 bits/symbol =
\textbf{1.5\$\textbackslash times\$ faster}! - \textbf{BUT} positions
are closer together - Easier to mistake ``North'' for ``Northeast'' when
noisy - Needs stronger signal (higher SNR) to work reliably

\textbf{Real-world use - Satellite TV}: - \textbf{DVB-S2} (digital
satellite): Uses 8PSK for HD channels - Why? Satellite bandwidth is
expensive! - 50\% more data in same bandwidth = 50\% more channels -
Trade-off: Need bigger dish (better SNR) for 8PSK vs QPSK

\textbf{Higher-order PSK (16PSK, 32PSK)}: - \textbf{16PSK}: 16 positions
= 4 bits/symbol - \textbf{32PSK}: 32 positions = 5 bits/symbol -
Problem: Positions so close together, barely used in practice! -
\textbf{Solution}: Switch to QAM (varies amplitude too) for better
performance

\textbf{Why not go higher?}: - Beyond 8PSK, positions are TOO close -
Even tiny noise causes errors - QAM (varying amplitude + phase) is more
efficient - This is why WiFi uses QAM, not 16PSK/32PSK!

\textbf{When you encounter it}: - \textbf{Satellite TV}: 8PSK for HD
channels - \textbf{Military communications}: 8PSK for satellite links -
\textbf{Deep space}: NASA sometimes uses 8PSK for high-rate data -
\textbf{Microwave backhaul}: Point-to-point links between cell towers

\textbf{The math}: - QPSK: 45\$\^{}\textbackslash circ\$ between
positions (lots of margin) - 8PSK: 22.5\$\^{}\textbackslash circ\$
between positions (tight!) - 16PSK: 11.25\$\^{}\textbackslash circ\$
between positions (very tight!) - Smaller angles = easier to confuse =
needs cleaner signal

\textbf{Fun fact}: The Hubble Space Telescope originally used QPSK, but
was upgraded to 8PSK to send more science data per day-\/-\/-saving
millions in ground station time!

\begin{center}\rule{0.5\linewidth}{0.5pt}\end{center}

\subsection{Overview}\label{overview}

\textbf{8PSK (8-ary Phase-Shift Keying)} encodes data using \textbf{8
phase states}, transmitting \textbf{3 bits per symbol}.

\textbf{Higher-order PSK} (M-PSK): M phase states, \(\log_2(M)\) bits
per symbol

\textbf{Trade-off}: Higher spectral efficiency but increased SNR
requirement

\textbf{Applications}: Satellite (DVB-S2), military (MILSTAR), microwave
backhaul

\begin{center}\rule{0.5\linewidth}{0.5pt}\end{center}

\subsection{8PSK Modulation}\label{psk-modulation}

\subsubsection{Constellation}\label{constellation}

\textbf{8 equally-spaced phases} around unit circle:

\[
\phi_m = \frac{2\pi m}{8} = \frac{\pi m}{4}, \quad m = 0, 1, \ldots, 7
\]

\textbf{Symbol} \(m\):

\[
s_m(t) = A\cos(2\pi f_c t + \phi_m)
\]

\textbf{Complex baseband}:

\[
s_m = A e^{j\phi_m} = A e^{j\pi m/4}
\]

\begin{center}\rule{0.5\linewidth}{0.5pt}\end{center}

\subsubsection{Constellation Diagram}\label{constellation-diagram}

\begin{verbatim}
          Q
          
     010  |  011
         |  
         \|/
  001 ---+--- 100
         /|\
         |  
     000  |  111
          
          I
\end{verbatim}

\textbf{Phases}: 0\$\^{}\textbackslash circ\$,
45\$\^{}\textbackslash circ\$, 90\$\^{}\textbackslash circ\$,
135\$\^{}\textbackslash circ\$, 180\$\^{}\textbackslash circ\$,
225\$\^{}\textbackslash circ\$, 270\$\^{}\textbackslash circ\$,
315\$\^{}\textbackslash circ\$

\textbf{Gray coding} (adjacent symbols differ by 1 bit):

{\def\LTcaptype{} % do not increment counter
\begin{longtable}[]{@{}lllll@{}}
\toprule\noalign{}
Symbol & Bits & Phase (\$\^{}\textbackslash circ\$) & I & Q \\
\midrule\noalign{}
\endhead
\bottomrule\noalign{}
\endlastfoot
0 & 000 & 0 & 1 & 0 \\
1 & 001 & 45 & 0.707 & 0.707 \\
2 & 010 & 90 & 0 & 1 \\
3 & 011 & 135 & -0.707 & 0.707 \\
4 & 100 & 180 & -1 & 0 \\
5 & 101 & 225 & -0.707 & -0.707 \\
6 & 110 & 270 & 0 & -1 \\
7 & 111 & 315 & 0.707 & -0.707 \\
\end{longtable}
}

\begin{center}\rule{0.5\linewidth}{0.5pt}\end{center}

\subsection{Signal Characteristics}\label{signal-characteristics}

\subsubsection{Constant Envelope}\label{constant-envelope}

\textbf{All symbols same amplitude} A:

\[
|s_m| = A, \quad \forall m
\]

\textbf{Advantage}: Power amplifier can operate at saturation (maximum
efficiency)

\textbf{PAPR} (Peak-to-Average Power Ratio): 0 dB (constant)

\begin{center}\rule{0.5\linewidth}{0.5pt}\end{center}

\subsubsection{Symbol Energy}\label{symbol-energy}

\[
E_s = \int_0^{T_s} |s_m(t)|^2 dt = A^2 T_s = A^2
\]

\textbf{Energy per bit}:

\[
E_b = \frac{E_s}{\log_2(8)} = \frac{E_s}{3}
\]

\begin{center}\rule{0.5\linewidth}{0.5pt}\end{center}

\subsubsection{Minimum Distance}\label{minimum-distance}

\textbf{Euclidean distance} between adjacent symbols:

\[
d_{\min} = 2A\sin\left(\frac{\pi}{8}\right) = 2A \times 0.383 = 0.765A
\]

\textbf{Normalized} (A=1): \(d_{\min} = 0.765\)

\textbf{Comparison}: - \textbf{QPSK}: \(d_{\min} = \sqrt{2}A = 1.414A\)
(same energy) - \textbf{8PSK}: \(d_{\min} = 0.765A\) - \textbf{Ratio}:
8PSK is 1.85\$\textbackslash times\$ worse (5.3 dB)

\begin{center}\rule{0.5\linewidth}{0.5pt}\end{center}

\subsection{Modulation \& Demodulation}\label{modulation-demodulation}

\subsubsection{IQ Modulator}\label{iq-modulator}

\textbf{Baseband I/Q} for symbol \(m\):

\[
I_m = A\cos(\phi_m), \quad Q_m = A\sin(\phi_m)
\]

\textbf{Modulated signal}:

\[
s_{\text{RF}}(t) = I_m \cos(2\pi f_c t) - Q_m \sin(2\pi f_c t)
\]

\textbf{Implementation}: Standard IQ modulator (same as QPSK)

\begin{center}\rule{0.5\linewidth}{0.5pt}\end{center}

\subsubsection{Coherent Demodulation}\label{coherent-demodulation}

\textbf{Receiver}: 1. \textbf{IQ demodulation}: Recover I and Q
components 2. \textbf{Phase calculation}: \(\hat{\phi} = \arctan(Q/I)\)
3. \textbf{Decision}: Find closest constellation point

\textbf{Decision regions}: 8 pie-slice wedges, each
45\$\^{}\textbackslash circ\$ wide

\textbf{Hard decision}:

\[
\hat{m} = \left\lfloor \frac{\hat{\phi} + \pi/8}{2\pi/8} \right\rfloor \mod 8
\]

\begin{center}\rule{0.5\linewidth}{0.5pt}\end{center}

\subsubsection{Differential 8PSK (D8PSK)}\label{differential-8psk-d8psk}

\textbf{Differential encoding} avoids phase ambiguity:

\textbf{Transmitted phase}:

\[
\phi_k = \phi_{k-1} + \Delta\phi_k \mod 2\pi
\]

Where \(\Delta\phi_k\) encodes 3 bits

\textbf{Demodulation}: Compute phase difference between consecutive
symbols

\[
\Delta\hat{\phi}_k = \hat{\phi}_k - \hat{\phi}_{k-1}
\]

\textbf{Advantage}: No carrier phase recovery needed (only frequency
sync)

\textbf{Disadvantage}: \textasciitilde3 dB penalty vs coherent

\begin{center}\rule{0.5\linewidth}{0.5pt}\end{center}

\subsection{Performance Analysis}\label{performance-analysis}

\subsubsection{Symbol Error Rate (SER)}\label{symbol-error-rate-ser}

\textbf{8PSK in AWGN} (approximate, high SNR):

\[
P_s \approx 2Q\left(2\sin\left(\frac{\pi}{8}\right)\sqrt{\frac{E_s}{N_0}}\right) = 2Q\left(0.765\sqrt{\frac{E_s}{N_0}}\right)
\]

\textbf{Where}:
\(Q(x) = \frac{1}{\sqrt{2\pi}} \int_x^\infty e^{-t^2/2} dt\)

\begin{center}\rule{0.5\linewidth}{0.5pt}\end{center}

\subsubsection{Bit Error Rate (BER)}\label{bit-error-rate-ber}

\textbf{With Gray coding}:

\[
\text{BER} \approx \frac{P_s}{\log_2(8)} = \frac{P_s}{3}
\]

\textbf{In terms of Eb/N0}:

\[
\text{BER} \approx \frac{2}{3}Q\left(0.765\sqrt{\frac{3E_b}{N_0}}\right) = \frac{2}{3}Q\left(1.325\sqrt{\frac{E_b}{N_0}}\right)
\]

\begin{center}\rule{0.5\linewidth}{0.5pt}\end{center}

\subsubsection{Required Eb/N0 for BER =
10\textbackslash textsuperscript\{-\}\textbackslash textsuperscript\{6\}}\label{required-ebn0-for-ber-10ux2076}

\textbf{8PSK}: 14 dB (coherent detection)

\textbf{Comparison}: - \textbf{BPSK}: 10.5 dB - \textbf{QPSK}: 10.5 dB
(same as BPSK) - \textbf{8PSK}: 14 dB (+3.5 dB penalty vs QPSK) -
\textbf{16-PSK}: 18 dB (+7.5 dB penalty vs QPSK)

\textbf{Pattern}: Each doubling of M adds \textasciitilde3.5-4 dB
penalty

\begin{center}\rule{0.5\linewidth}{0.5pt}\end{center}

\subsubsection{BER vs SNR Curves}\label{ber-vs-snr-curves}

{\def\LTcaptype{} % do not increment counter
\begin{longtable}[]{@{}lllll@{}}
\toprule\noalign{}
Eb/N0 (dB) & BPSK & QPSK & 8PSK & 16-PSK \\
\midrule\noalign{}
\endhead
\bottomrule\noalign{}
\endlastfoot
6 &
1.9\$\textbackslash times\$10\textbackslash textsuperscript\{-\}\textbackslash textsuperscript\{3\}
&
1.9\$\textbackslash times\$10\textbackslash textsuperscript\{-\}\textbackslash textsuperscript\{3\}
& 0.04 & 0.15 \\
8 &
5.6\$\textbackslash times\$10\textbackslash textsuperscript\{-\}\textbackslash textsuperscript\{5\}
&
5.6\$\textbackslash times\$10\textbackslash textsuperscript\{-\}\textbackslash textsuperscript\{5\}
&
8\$\textbackslash times\$10\textbackslash textsuperscript\{-\}\textbackslash textsuperscript\{3\}
& 0.08 \\
10 &
3.9\$\textbackslash times\$10\textbackslash textsuperscript\{-\}\textbackslash textsuperscript\{6\}
&
3.9\$\textbackslash times\$10\textbackslash textsuperscript\{-\}\textbackslash textsuperscript\{6\}
&
7\$\textbackslash times\$10\textbackslash textsuperscript\{-\}\textbackslash textsuperscript\{4\}
& 0.03 \\
12 &
7.8\$\textbackslash times\$10\textbackslash textsuperscript\{-\}\textbackslash textsuperscript\{8\}
&
7.8\$\textbackslash times\$10\textbackslash textsuperscript\{-\}\textbackslash textsuperscript\{8\}
&
4\$\textbackslash times\$10\textbackslash textsuperscript\{-\}\textbackslash textsuperscript\{5\}
&
8\$\textbackslash times\$10\textbackslash textsuperscript\{-\}\textbackslash textsuperscript\{3\} \\
14 &
7.7\$\textbackslash times\$10\textbackslash textsuperscript\{-\}\textbackslash textsuperscript\{1\}\textbackslash textsuperscript\{0\}
&
7.7\$\textbackslash times\$10\textbackslash textsuperscript\{-\}\textbackslash textsuperscript\{1\}\textbackslash textsuperscript\{0\}
&
1\$\textbackslash times\$10\textbackslash textsuperscript\{-\}\textbackslash textsuperscript\{6\}
&
7\$\textbackslash times\$10\textbackslash textsuperscript\{-\}\textbackslash textsuperscript\{4\} \\
\end{longtable}
}

\textbf{Observation}: Higher-order PSK needs significantly more SNR for
same BER

\begin{center}\rule{0.5\linewidth}{0.5pt}\end{center}

\subsection{Bandwidth Efficiency}\label{bandwidth-efficiency}

\textbf{Symbol rate} \(R_s\) (symbols/sec):

\[
R_s = \frac{R_b}{\log_2(M)}
\]

\textbf{Occupied bandwidth} (with pulse shaping):

\[
B = (1 + \alpha) R_s = (1 + \alpha) \frac{R_b}{\log_2(M)} \quad (\text{Hz})
\]

\textbf{Spectral efficiency}:

\[
\eta = \frac{R_b}{B} = \frac{\log_2(M)}{1 + \alpha} \quad (\text{bits/sec/Hz})
\]

\begin{center}\rule{0.5\linewidth}{0.5pt}\end{center}

\subsubsection{Comparison (\$\textbackslash alpha\$ =
0.35)}\label{comparison-ux3b1-0.35}

{\def\LTcaptype{} % do not increment counter
\begin{longtable}[]{@{}
  >{\raggedright\arraybackslash}p{(\linewidth - 6\tabcolsep) * \real{0.1791}}
  >{\raggedright\arraybackslash}p{(\linewidth - 6\tabcolsep) * \real{0.1940}}
  >{\raggedright\arraybackslash}p{(\linewidth - 6\tabcolsep) * \real{0.3134}}
  >{\raggedright\arraybackslash}p{(\linewidth - 6\tabcolsep) * \real{0.3134}}@{}}
\toprule\noalign{}
\begin{minipage}[b]{\linewidth}\raggedright
Modulation
\end{minipage} & \begin{minipage}[b]{\linewidth}\raggedright
Bits/symbol
\end{minipage} & \begin{minipage}[b]{\linewidth}\raggedright
Spectral Efficiency
\end{minipage} & \begin{minipage}[b]{\linewidth}\raggedright
Required Eb/N0, BER 10\^{}-6
\end{minipage} \\
\midrule\noalign{}
\endhead
\bottomrule\noalign{}
\endlastfoot
\textbf{BPSK} & 1 & 0.74 & 10.5 dB \\
\end{longtable}
}

\textbf{QPSK} \textbar{} 2 \textbar{} 1.48 \textbar{} 10.5 dB
\textbar{}\\
\textbf{8PSK} \textbar{} 3 \textbar{} 2.22 \textbar{} 14 dB \textbar{}\\
\textbf{16-PSK} \textbar{} 4 \textbar{} 2.96 \textbar{} 18 dB
\textbar{}\\
\textbf{32-PSK} \textbar{} 5 \textbar{} 3.70 \textbar{} 22 dB \textbar{}

\textbf{Trade-off}: Higher spectral efficiency requires higher SNR

\begin{center}\rule{0.5\linewidth}{0.5pt}\end{center}

\subsection{Higher-Order PSK}\label{higher-order-psk}

\subsubsection{16-PSK}\label{psk}

\textbf{16 phase states}: 22.5\$\^{}\textbackslash circ\$ spacing

\textbf{Bits per symbol}: 4

\textbf{Minimum distance}: \(d_{\min} = 2A\sin(\pi/16) = 0.39A\)

\textbf{Performance}: \textasciitilde4 dB worse than 8PSK (at same BER)

\textbf{Problem}: Very sensitive to phase noise (small angular
separation)

\begin{center}\rule{0.5\linewidth}{0.5pt}\end{center}

\subsubsection{32-PSK and Beyond}\label{psk-and-beyond}

\textbf{32-PSK}: 11.25\$\^{}\textbackslash circ\$ spacing, 5 bits/symbol

\textbf{64-PSK}: 5.625\$\^{}\textbackslash circ\$ spacing, 6 bits/symbol

\textbf{Practical limit}: M \textgreater{} 16 rarely used - Phase noise
becomes limiting factor - QAM more efficient for M \textgreater{} 8

\begin{center}\rule{0.5\linewidth}{0.5pt}\end{center}

\subsection{8PSK vs Other Modulations}\label{psk-vs-other-modulations}

\subsubsection{8PSK vs 16-QAM}\label{psk-vs-16-qam}

\textbf{Same spectral efficiency} (\$\textbackslash approx\$2.2
bits/sec/Hz with \$\textbackslash alpha\$=0.35): - \textbf{8PSK}: 3
bits/symbol - \textbf{16-QAM}: 4 bits/symbol @
1.33\$\textbackslash times\$ symbol rate

\textbf{BER comparison} @ BER =
10\textbackslash textsuperscript\{-\}\textbackslash textsuperscript\{6\}:
- \textbf{8PSK}: 14 dB Eb/N0 - \textbf{16-QAM}: 14.5 dB Eb/N0

\textbf{Advantage 8PSK}: Constant envelope (PA efficiency)

\textbf{Advantage 16-QAM}: Slightly better BER, more flexible coding
rates

\begin{center}\rule{0.5\linewidth}{0.5pt}\end{center}

\subsubsection{8PSK vs OFDM with QPSK}\label{psk-vs-ofdm-with-qpsk}

\textbf{Wideband system} (20 MHz):

\textbf{8PSK single carrier}: - 6.67 Msps, 20 Mbps - Requires
equalization (frequency-selective fading) - Constant envelope

\textbf{OFDM with QPSK} (64 subcarriers): - 312.5 kHz per subcarrier
(flat fading) - 20 Mbps total - Varying envelope (PAPR \textasciitilde10
dB)

\textbf{Trade-off}: OFDM handles multipath better, 8PSK more
PA-efficient

\begin{center}\rule{0.5\linewidth}{0.5pt}\end{center}

\subsection{Phase Noise Sensitivity}\label{phase-noise-sensitivity}

\textbf{Oscillator phase noise} \(\phi_n(t)\) rotates constellation:

\[
r_m(t) = A e^{j(\phi_m + \phi_n(t))} + n(t)
\]

\textbf{Phase error} \(\phi_n\) causes: - \textbf{Rotation}: All symbols
rotate equally - \textbf{Spreading}: Random jitter
\$\textbackslash rightarrow\$ Constellation blur

\textbf{Sensitivity} (angular spacing): - \textbf{QPSK}:
90\$\^{}\textbackslash circ\$ spacing (robust) - \textbf{8PSK}:
45\$\^{}\textbackslash circ\$ spacing (moderate) - \textbf{16-PSK}:
22.5\$\^{}\textbackslash circ\$ spacing (sensitive) - \textbf{32-PSK}:
11.25\$\^{}\textbackslash circ\$ spacing (very sensitive)

\textbf{Rule of thumb}: Phase noise RMS should be \textless{} 1/10 of
angular spacing

\textbf{Example}: 8PSK with 45\$\^{}\textbackslash circ\$ spacing -
Tolerable phase noise: \textasciitilde4.5\$\^{}\textbackslash circ\$ RMS
- Equivalent phase noise: \textasciitilde-25 dBc integrated (tight
spec!)

\begin{center}\rule{0.5\linewidth}{0.5pt}\end{center}

\subsection{Practical Implementations}\label{practical-implementations}

\subsubsection{1. DVB-S2 (Satellite TV)}\label{dvb-s2-satellite-tv}

\textbf{8PSK} used for high data rates: - \textbf{QPSK}: Low C/N (rain
fade conditions) - \textbf{8PSK}: Clear sky, high throughput -
\textbf{Adaptive Coding \& Modulation (ACM)}: Switch based on link
quality

\textbf{Example}: - QPSK 1/2: 1 bit/symbol effective
\$\textbackslash rightarrow\$ 0.74 bits/sec/Hz - 8PSK 3/4: 2.25
bits/symbol effective \$\textbackslash rightarrow\$ 1.67 bits/sec/Hz -
\textbf{2.25\$\textbackslash times\$ throughput} when SNR permits

\begin{center}\rule{0.5\linewidth}{0.5pt}\end{center}

\subsubsection{2. Military SATCOM
(MILSTAR)}\label{military-satcom-milstar}

\textbf{Differential 8PSK}: - Robust against jamming -
Low-probability-of-intercept (LPI) - Spread spectrum combined with D8PSK

\begin{center}\rule{0.5\linewidth}{0.5pt}\end{center}

\subsubsection{3. Microwave Backhaul}\label{microwave-backhaul}

\textbf{Point-to-point links} (cellular backhaul): - \textbf{Clear
weather}: 256-QAM (8 bits/symbol) - \textbf{Rain fade}: Adaptive down to
8PSK or QPSK - \textbf{Example}: 6-11 GHz bands, 28/56 MHz channels

\begin{center}\rule{0.5\linewidth}{0.5pt}\end{center}

\subsubsection{4. Deep Space
Communications}\label{deep-space-communications}

\textbf{NASA/ESA}: Primarily BPSK/QPSK (maximize link margin)

\textbf{Emerging}: 8PSK for high-rate science data return - \textbf{Mars
orbiters}: 8PSK @ Ka-band (32 GHz) - \textbf{Trade-off}:
3\$\textbackslash times\$ data rate vs 3.5 dB link margin

\begin{center}\rule{0.5\linewidth}{0.5pt}\end{center}

\subsection{Implementation Challenges}\label{implementation-challenges}

\subsubsection{1. Carrier Phase Recovery}\label{carrier-phase-recovery}

\textbf{8PSK phase ambiguity}: 8-fold (every
45\$\^{}\textbackslash circ\$)

\textbf{Pilot-aided sync}: - Insert known pilot symbols - Estimate phase
offset - Correct data symbols

\textbf{Blind sync}: - 8th-power loop (remove modulation) - Costas loop
(feedback) - Decision-directed (after initial acquisition)

\textbf{See}: {[}{[}Synchronization-(Carrier,-Timing,-Frame){]}{]}

\begin{center}\rule{0.5\linewidth}{0.5pt}\end{center}

\subsubsection{2. Timing Recovery}\label{timing-recovery}

\textbf{Symbol clock} must be accurate:

\textbf{Timing jitter} causes: - Sampling offset
\$\textbackslash rightarrow\$ ISI - Increased BER

\textbf{Early-late gate} detector: - Sample early, on-time, late -
Adjust clock based on correlation

\begin{center}\rule{0.5\linewidth}{0.5pt}\end{center}

\subsubsection{3. Nonlinear PA
Distortion}\label{nonlinear-pa-distortion}

\textbf{8PSK constant envelope}: Tolerates PA saturation

\textbf{BUT}: Pulse shaping filter creates envelope variations - Raised
cosine filter \$\textbackslash rightarrow\$ 3-4 dB PAPR - PA must back
off \$\textbackslash rightarrow\$ Reduced efficiency

\textbf{Mitigation}: - \textbf{Constant envelope pulse shaping}: MSK,
GMSK (no overshoot) - \textbf{Predistortion}: Digital or analog
linearization

\begin{center}\rule{0.5\linewidth}{0.5pt}\end{center}

\subsubsection{4. Frequency Offset}\label{frequency-offset}

\textbf{Carrier frequency offset} \(\Delta f\) rotates constellation:

\[
r(t) = s(t) e^{j2\pi \Delta f t}
\]

\textbf{Tolerable offset} (rule of thumb):
\(|\Delta f| < 0.01 \times R_s\)

\textbf{Example}: 8PSK @ 1 Msps - Tolerable offset: \textless{} 10 kHz -
Oscillator spec: \textless{} 10 ppm @ 1 GHz carrier (= 10 kHz)

\begin{center}\rule{0.5\linewidth}{0.5pt}\end{center}

\subsection{Adaptive Modulation \& Coding
(AMC)}\label{adaptive-modulation-coding-amc}

\textbf{Dynamically select modulation} based on channel quality:

\textbf{Link adaptation table}:

{\def\LTcaptype{} % do not increment counter
\begin{longtable}[]{@{}lllll@{}}
\toprule\noalign{}
C/N (dB) & Modulation & Code Rate & Spectral Eff. & Target BER \\
\midrule\noalign{}
\endhead
\bottomrule\noalign{}
\endlastfoot
2-5 & QPSK & 1/4 & 0.5 &
10\textbackslash textsuperscript\{-\}\textbackslash textsuperscript\{7\} \\
5-7 & QPSK & 1/2 & 1.0 &
10\textbackslash textsuperscript\{-\}\textbackslash textsuperscript\{7\} \\
7-9 & QPSK & 3/4 & 1.5 &
10\textbackslash textsuperscript\{-\}\textbackslash textsuperscript\{7\} \\
9-11 & 8PSK & 2/3 & 2.0 &
10\textbackslash textsuperscript\{-\}\textbackslash textsuperscript\{7\} \\
11-13 & 8PSK & 3/4 & 2.25 &
10\textbackslash textsuperscript\{-\}\textbackslash textsuperscript\{7\} \\
13-15 & 16-QAM & 2/3 & 2.67 &
10\textbackslash textsuperscript\{-\}\textbackslash textsuperscript\{7\} \\
\end{longtable}
}

\textbf{Benefit}: Maximize throughput while maintaining target BER

\begin{center}\rule{0.5\linewidth}{0.5pt}\end{center}

\subsection{Gray Coding}\label{gray-coding}

\textbf{Gray code}: Adjacent symbols differ by \textbf{1 bit}

\textbf{Benefit}: Symbol error \$\textbackslash rightarrow\$ Likely
1-bit error (not 2 or 3)

\textbf{8PSK Gray mapping}:

{\def\LTcaptype{} % do not increment counter
\begin{longtable}[]{@{}llll@{}}
\toprule\noalign{}
Symbol & Binary & Phase (\$\^{}\textbackslash circ\$) & Gray Code \\
\midrule\noalign{}
\endhead
\bottomrule\noalign{}
\endlastfoot
0 & 000 & 0 & 000 \\
1 & 001 & 45 & 001 \\
2 & 010 & 90 & 011 \\
3 & 011 & 135 & 010 \\
4 & 100 & 180 & 110 \\
5 & 101 & 225 & 111 \\
6 & 110 & 270 & 101 \\
7 & 111 & 315 & 100 \\
\end{longtable}
}

\textbf{Natural binary}: Symbol error \$\textbackslash rightarrow\$ Up
to 3-bit error

\textbf{Gray coding}: Symbol error \$\textbackslash rightarrow\$
Typically 1-bit error (maybe 2)

\textbf{BER improvement}: \textasciitilde2\$\textbackslash times\$
better with Gray coding

\begin{center}\rule{0.5\linewidth}{0.5pt}\end{center}

\subsection{Pulse Shaping}\label{pulse-shaping}

\textbf{Rectangular pulses}: Infinite bandwidth (sinc spectrum)

\textbf{Raised cosine} (RC):

\[
P(f) = \begin{cases}
T_s & |f| \leq \frac{1-\alpha}{2T_s} \\
\frac{T_s}{2}\left[1 + \cos\left(\frac{\pi T_s}{\alpha}\left[|f| - \frac{1-\alpha}{2T_s}\right]\right)\right] & \frac{1-\alpha}{2T_s} < |f| \leq \frac{1+\alpha}{2T_s} \\
0 & |f| > \frac{1+\alpha}{2T_s}
\end{cases}
\]

\textbf{Roll-off factor} \$\textbackslash alpha\$: -
\textbf{\$\textbackslash alpha\$ = 0}: Brick-wall (infinite time,
impractical) - \textbf{\$\textbackslash alpha\$ = 0.35}: Common (35\%
excess BW, moderate time decay) - \textbf{\$\textbackslash alpha\$ = 1}:
Gentle roll-off (100\% excess BW, fast time decay)

\textbf{Root raised cosine} (RRC): Split between TX and RX (matched
filter)

\begin{center}\rule{0.5\linewidth}{0.5pt}\end{center}

\subsection{Summary Table}\label{summary-table}

{\def\LTcaptype{} % do not increment counter
\begin{longtable}[]{@{}
  >{\raggedright\arraybackslash}p{(\linewidth - 12\tabcolsep) * \real{0.1395}}
  >{\raggedright\arraybackslash}p{(\linewidth - 12\tabcolsep) * \real{0.1163}}
  >{\raggedright\arraybackslash}p{(\linewidth - 12\tabcolsep) * \real{0.1628}}
  >{\raggedright\arraybackslash}p{(\linewidth - 12\tabcolsep) * \real{0.1628}}
  >{\raggedright\arraybackslash}p{(\linewidth - 12\tabcolsep) * \real{0.1744}}
  >{\raggedright\arraybackslash}p{(\linewidth - 12\tabcolsep) * \real{0.0698}}
  >{\raggedright\arraybackslash}p{(\linewidth - 12\tabcolsep) * \real{0.1744}}@{}}
\toprule\noalign{}
\begin{minipage}[b]{\linewidth}\raggedright
Modulation
\end{minipage} & \begin{minipage}[b]{\linewidth}\raggedright
Bits/sym
\end{minipage} & \begin{minipage}[b]{\linewidth}\raggedright
Min Distance
\end{minipage} & \begin{minipage}[b]{\linewidth}\raggedright
Eb/N0 (\(10^{-6}\))
\end{minipage} & \begin{minipage}[b]{\linewidth}\raggedright
Spectral Eff.
\end{minipage} & \begin{minipage}[b]{\linewidth}\raggedright
PAPR
\end{minipage} & \begin{minipage}[b]{\linewidth}\raggedright
Best Use Case
\end{minipage} \\
\midrule\noalign{}
\endhead
\bottomrule\noalign{}
\endlastfoot
\textbf{BPSK} & 1 & 2A & 10.5 dB & 0.74 & 0 dB & Deep space, long
range \\
\textbf{QPSK} & 2 & \$\textbackslash sqrt\{\}\$2 A & 10.5 dB & 1.48 & 0
dB & Balanced (most common) \\
\textbf{8PSK} & 3 & 0.765A & 14 dB & 2.22 & 0 dB & High throughput, PA
efficiency \\
\textbf{16-PSK} & 4 & 0.39A & 18 dB & 2.96 & 0 dB & Rarely (QAM
better) \\
\textbf{16-QAM} & 4 & 0.63A & 14.5 dB & 2.96 & 2.6 dB & High throughput
(non-const env) \\
\end{longtable}
}

\begin{center}\rule{0.5\linewidth}{0.5pt}\end{center}

\subsection{Related Topics}\label{related-topics}

\begin{itemize}
\tightlist
\item
  \textbf{{[}{[}QPSK-Modulation{]}{]}}: Lower-order PSK (2 bits/symbol)
\item
  \textbf{{[}{[}Binary-Phase-Shift-Keying-(BPSK){]}{]}}: Simplest PSK
\item
  \textbf{{[}{[}Constellation-Diagrams{]}{]}}: Visualizing PSK
\item
  \textbf{{[}{[}Bit-Error-Rate-(BER){]}{]}}: Performance metric
\item
  \textbf{{[}{[}Synchronization-(Carrier,-Timing,-Frame){]}{]}}: Carrier
  recovery for coherent detection
\item
  \textbf{{[}{[}OFDM-\&-Multicarrier-Modulation{]}{]}}: Uses QPSK/8PSK
  per subcarrier
\end{itemize}

\begin{center}\rule{0.5\linewidth}{0.5pt}\end{center}

\textbf{Key takeaway}: \textbf{8PSK transmits 3 bits/symbol using 8
phase states.} Constant envelope = PA efficient. 50\% more spectral
efficiency than QPSK but needs +3.5 dB SNR. Used in satellite (DVB-S2)
and backhaul. Higher-order PSK (16, 32, 64) rarely used due to phase
noise sensitivity-\/-\/-QAM preferred for M \textgreater{} 8. Gray
coding reduces BER by limiting bit errors per symbol error. Adaptive
modulation switches between QPSK/8PSK/16-QAM based on link quality.

\begin{center}\rule{0.5\linewidth}{0.5pt}\end{center}

\emph{This wiki is part of the {[}{[}Home\textbar Chimera Project{]}{]}
documentation.}
