\section{Frey Microwave Auditory
Effect}\label{frey-microwave-auditory-effect}

{[}{[}Home{]}{]} \textbar{}
{[}{[}Non-Linear-Biological-Demodulation{]}{]} \textbar{}
{[}{[}Acoustic-Heterodyning{]}{]} \textbar{}
{[}{[}Intermodulation-Distortion-in-Biology{]}{]}

\begin{center}\rule{0.5\linewidth}{0.5pt}\end{center}

\subsection{Overview}\label{overview}

The \textbf{Frey microwave auditory effect} (also called
\textbf{microwave hearing} or \textbf{RF hearing}) is the perception of
auditory sensations (clicks, buzzes, or tones) when exposed to
\textbf{pulsed microwave radiation} (typically 1-10 GHz). The effect is
well-documented and occurs \textbf{without external sound}-\/-\/-the
perception arises from \textbf{thermoelastic expansion} in the cochlea.

\textbf{Key features} : - Requires \textbf{pulsed} microwaves (CW
ineffective) - Perceived sound frequency \textasciitilde pulse
repetition rate (not microwave carrier frequency) - Threshold:
\textasciitilde1-10
\$\textbackslash mu\$J/cm\textbackslash textsuperscript\{2\} per pulse
(very low energy) - Mechanism: Rapid heating
\$\textbackslash rightarrow\$ acoustic pressure wave
\$\textbackslash rightarrow\$ cochlear stimulation

\textbf{Applications} (potential ): - Non-lethal weapons (``active
denial'' communication) - Assistive hearing devices (cochlear implant
alternative?) - Covert communication

\begin{center}\rule{0.5\linewidth}{0.5pt}\end{center}

\subsection{\texorpdfstring{Simple Explanation (For Non-Technical
Readers)
}{Simple Explanation (For Non-Technical Readers) }}\label{simple-explanation-for-non-technical-readers}

\subsubsection{What Is It?}\label{what-is-it}

Imagine hearing sounds-\/-\/-clicks, buzzes, or even tones-\/-\/-but
there\textquotesingle s no speaker, no headphones, and no actual sound
waves in the air. That\textquotesingle s the Frey microwave auditory
effect.

When certain types of microwave signals (like what\textquotesingle s in
a radar) are pulsed rapidly on and off, people near them sometimes hear
mysterious noises. It\textquotesingle s not science
fiction-\/-\/-it\textquotesingle s a real, well-studied phenomenon
discovered in the 1960s.

\subsubsection{How Does It Work? (The Simple
Version)}\label{how-does-it-work-the-simple-version}

Think of it like this:

\begin{enumerate}
\def\labelenumi{\arabic{enumi}.}
\tightlist
\item
  \textbf{Microwave pulses hit your head} (don\textquotesingle t
  worry-\/-\/-very tiny amounts of energy, much less than a microwave
  oven)
\item
  \textbf{They make tissue heat up just a tiny, tiny bit}
  (we\textquotesingle re talking millionths of a degree-\/-\/-you
  can\textquotesingle t feel it)
\item
  \textbf{That tiny heating happens so fast it makes the tissue expand
  suddenly} (like how metal expands when heated, but much faster)
\item
  \textbf{The expansion creates a pressure wave} (basically a tiny
  ``pop'' inside your head)
\item
  \textbf{That pressure wave reaches your inner ear} (the cochlea)
\item
  \textbf{Your ear detects it as sound} (your brain thinks ``I heard a
  click!'')
\end{enumerate}

It\textquotesingle s like tapping on a microphone to test
it-\/-\/-except the ``tap'' comes from inside your head, caused by
invisible microwaves.

\subsubsection{Key Points to Remember}\label{key-points-to-remember}

\textbf{It\textquotesingle s safe (at normal levels)} - The energy
levels that cause the effect are far below what would harm you -
It\textquotesingle s like hearing a distant whisper-\/-\/-noticeable but
not dangerous - No tissue damage occurs at the levels needed to hear the
sound

\textbf{It\textquotesingle s not mind control} - Despite what conspiracy
theories say, this effect only creates sounds - It can\textquotesingle t
implant thoughts or control your actions - It\textquotesingle s no
different from hearing any other sound with your ears

\textbf{Your cell phone can\textquotesingle t do this} - Cell phones use
continuous signals, not rapid pulses - They don\textquotesingle t have
enough power (need kilowatts, not milliwatts) - The frequency
isn\textquotesingle t quite right for the effect

\textbf{Why you might care:} - It\textquotesingle s a fascinating
example of how physics and biology interact - It shows our bodies can be
``antennas'' for certain signals - It has potential uses (and misuses)
in technology and defense

\subsubsection{The ``Wow'' Factor}\label{the-wow-factor}

The coolest part? \textbf{The sound isn\textquotesingle t ``out
there''-\/-\/-it\textquotesingle s created inside your head.} Someone
standing right next to you won\textquotesingle t hear it. But you will.
It\textquotesingle s your own personal acoustic experience, generated by
electromagnetic waves.

Scientists have even used this to transmit simple speech
patterns-\/-\/-imagine hearing words that no one spoke, with no device
in your ear. That\textquotesingle s the Frey effect in action.

\begin{center}\rule{0.5\linewidth}{0.5pt}\end{center}

\subsection{1. Discovery and Historical
Background}\label{discovery-and-historical-background}

\subsubsection{1.1 Allan Frey\textquotesingle s Experiments
(1962)}\label{allan-freys-experiments-1962}

\textbf{Original observation}: Frey reported that humans near radar
installations heard ``clicking'' or ``buzzing'' sounds synchronized with
radar pulses.

\textbf{Controlled experiment}: - Subjects exposed to pulsed microwaves
(1.3 GHz, \textasciitilde10 \$\textbackslash mu\$s pulses, 100-1000 pps)
- Auditory perception reported even in \textbf{deaf subjects}
(conductive hearing loss; sensorineural deaf individuals did not
perceive) - Sound localized to head, not external space

\textbf{Frey\textquotesingle s conclusion}: Microwaves directly
stimulate auditory system, bypassing external ear.

\textbf{Controversy}: Initial skepticism; effect dismissed as equipment
artifact (electromagnetic interference with auditory nerves). Later
confirmed by multiple independent labs.

\subsubsection{1.2 Subsequent Research
(1970s-1990s)}\label{subsequent-research-1970s-1990s}

\textbf{U.S. military studies} (classified then declassified): -
Confirmed Frey effect in animals and humans - Explored for communication
(``voice-to-skull'') and non-lethal weapons

\textbf{Key findings}: - Effect requires intact cochlea (direct neural
stimulation ruled out) - Perceived frequency matches pulse repetition
rate (10 pps \$\textbackslash rightarrow\$ 10 Hz perceived tone) - Peak
sensitivity \textasciitilde2.45 GHz (ISM band)

\begin{center}\rule{0.5\linewidth}{0.5pt}\end{center}

\subsection{2. Mechanism: Thermoelastic
Expansion}\label{mechanism-thermoelastic-expansion}

\subsubsection{2.1 Physical Process}\label{physical-process}

\textbf{Step 1: Microwave absorption} - Pulsed microwave energy absorbed
by tissue (primarily water) - Absorption depth (1/e): \textasciitilde1-3
cm at 1-10 GHz

\textbf{Step 2: Rapid heating} - Pulse duration: \textasciitilde1-10
\$\textbackslash mu\$s (shorter than thermal diffusion time
\textasciitilde1 ms) - Temperature rise: \(\Delta T \approx 10^{-6}\) to
\(10^{-5}\) \$\^{}\textbackslash circ\$C per pulse (tiny!)

\textbf{Step 3: Thermoelastic expansion} - Heated tissue expands:
\(\Delta V/V = 3\alpha \Delta T\) (where
\(\alpha \approx 3 \times 10^{-4}\)
K\textbackslash textsuperscript\{-\}\textbackslash textsuperscript\{1\}
is thermal expansion coefficient) - Expansion occurs on timescale of
pulse (\textasciitilde\$\textbackslash mu\$s)
\$\textbackslash rightarrow\$ \textbf{launches acoustic wave}

\textbf{Step 4: Acoustic propagation} - Pressure wave propagates through
head tissue to cochlea - Inner ear hair cells (stereocilia) detect
pressure \$\textbackslash rightarrow\$ neural signal

\textbf{Step 5: Perception} - Auditory cortex processes signal
\$\textbackslash rightarrow\$ perceived as sound

\subsubsection{2.2 Quantitative Model}\label{quantitative-model}

\textbf{Absorbed energy per pulse}:
\[E = \text{SAR} \times \tau \times m\] where: - SAR: Specific
absorption rate (W/kg) - \(\tau\): Pulse duration (s) - \(m\): Mass of
absorbing tissue (kg)

\textbf{Temperature rise}:
\[\Delta T = \frac{E}{c_p m} = \frac{\text{SAR} \times \tau}{c_p}\]
where \(c_p \approx 3600\) J/kg/K (specific heat capacity).

\textbf{For SAR = 1 W/kg, \(\tau = 1\) \$\textbackslash mu\$s}:
\[\Delta T = \frac{1 \times 10^{-6}}{3600} \approx 3 \times 10^{-10} \text{ K} \quad (\text{negligible heating!})\]

\textbf{Pressure amplitude} (Lin \& Wang model):
\[p = \frac{\beta}{\rho_0 c_p} \cdot \text{SAR} \cdot \tau \cdot f_c\]
where: - \(\beta\): Thermal expansion coefficient
(\textasciitilde{}\(10^{-4}\)
K\textbackslash textsuperscript\{-\}\textbackslash textsuperscript\{1\})
- \(\rho_0\): Density (\textasciitilde1000
kg/m\textbackslash textsuperscript\{3\}) - \(f_c\): Microwave frequency
(Hz)

\textbf{Threshold pressure} for hearing: \textasciitilde20
\$\textbackslash mu\$Pa (0 dB SPL)

\textbf{Implication}: Very low energy pulses sufficient to exceed
hearing threshold.

\subsubsection{2.3 Why Pulsed, Not CW?}\label{why-pulsed-not-cw}

\textbf{CW microwaves}: Steady heating \$\textbackslash rightarrow\$ no
rapid expansion \$\textbackslash rightarrow\$ no acoustic wave

\textbf{Pulsed microwaves}: Rapid on-off \$\textbackslash rightarrow\$
expansion-contraction cycles \$\textbackslash rightarrow\$ acoustic
transient

\textbf{Pulse duration}: Must be shorter than thermal diffusion time
(\textasciitilde1 ms) and comparable to acoustic period
(\textasciitilde10 \$\textbackslash mu\$s for 100 kHz).

\begin{center}\rule{0.5\linewidth}{0.5pt}\end{center}

\subsection{3. Experimental Evidence}\label{experimental-evidence}

\subsubsection{3.1 Human Psychophysics}\label{human-psychophysics}

\textbf{Threshold measurements} (Guy et al., 1975): - Frequency range:
200 MHz - 10 GHz - Peak sensitivity: \textbf{2.45 GHz} (coincides with
peak brain absorption) - Threshold: \textasciitilde1-10
\$\textbackslash mu\$J/cm\textbackslash textsuperscript\{2\} per pulse
(0.1-1 mW/cm\textbackslash textsuperscript\{2\} average for 1\% duty
cycle)

\textbf{Perceived sound characteristics}: - \textbf{Click}: Single pulse
- \textbf{Buzz}: Pulse train (10-100 pps) - \textbf{Tone}: High pulse
rate (\textgreater1000 pps), perceived pitch = PRF - \textbf{No sound}:
CW exposure (even at high power)

\textbf{Deaf subjects}: Conductively deaf individuals (middle ear
damage) perceive effect; sensorineural deaf (cochlear damage) do not
\$\textbackslash rightarrow\$ confirms cochlear origin.

\subsubsection{3.2 Animal Studies}\label{animal-studies}

\textbf{Cochlear microphonics} (Elder \& Chou, 2003): - Microelectrode
in guinea pig cochlea - Pulsed microwaves \$\textbackslash rightarrow\$
electrical signal matching pulse rate - Signal abolished by cochlear
destruction \$\textbackslash rightarrow\$ direct evidence for cochlear
transduction

\textbf{Auditory brainstem response} (ABR): - EEG-like measurement of
auditory pathway activity - Pulsed microwaves evoke ABR similar to
acoustic clicks

\subsubsection{3.3 Simulations and
Modeling}\label{simulations-and-modeling}

\textbf{Lin (1978)}: Developed thermoelastic theory; predicted threshold
within factor of 2-3 of measured values.

\textbf{Foster \& Finch (1974)}: Showed calculated pressure waves
consistent with psychophysical thresholds.

\textbf{Consensus}: Thermoelastic mechanism \textbf{firmly established}
.

\begin{center}\rule{0.5\linewidth}{0.5pt}\end{center}

\subsection{4. Frequency and Pulse Parameter
Dependence}\label{frequency-and-pulse-parameter-dependence}

\subsubsection{4.1 Carrier Frequency}\label{carrier-frequency}

\textbf{Optimal frequency}: 1-10 GHz - \textbf{Lower (\textless100
MHz)}: Penetrates too deeply, low absorption in head
\$\textbackslash rightarrow\$ weak effect - \textbf{Higher
(\textgreater30 GHz)}: Absorbed at skin surface, doesn\textquotesingle t
reach cochlea

\textbf{Peak sensitivity} \textasciitilde2.45 GHz: Balance between
penetration and absorption.

\subsubsection{4.2 Pulse Duration}\label{pulse-duration}

\textbf{Optimal range}: 1-100 \$\textbackslash mu\$s - \textbf{Shorter
(\textless1 \$\textbackslash mu\$s)}: Lower total energy, weaker
acoustic wave - \textbf{Longer (\textgreater1 ms)}: Heat diffuses before
expansion \$\textbackslash rightarrow\$ less efficient pressure
generation

\subsubsection{4.3 Pulse Repetition Frequency
(PRF)}\label{pulse-repetition-frequency-prf}

\textbf{PRF determines perceived pitch}: - 10 Hz
\$\textbackslash rightarrow\$ low hum - 100 Hz
\$\textbackslash rightarrow\$ buzz - 1 kHz \$\textbackslash rightarrow\$
audible tone - 10 kHz \$\textbackslash rightarrow\$ high-pitched whistle

\textbf{Audible range}: 20 Hz - 20 kHz (same as acoustic hearing)

\subsubsection{4.4 Peak Power vs.~Average
Power}\label{peak-power-vs.-average-power}

\textbf{Key insight}: Effect depends on \textbf{peak power per pulse},
not average power.

\textbf{Example}: - Pulse: 1 kW peak, 1 \$\textbackslash mu\$s duration,
100 pps - Average power: \(1000 \times 10^{-6} \times 100 = 0.1\) W
(weak!) - But peak intensity high enough to trigger effect

\textbf{Safety implication}: Average power density can be below safety
limits while still causing perception.

\begin{center}\rule{0.5\linewidth}{0.5pt}\end{center}

\subsection{5. Safety Considerations}\label{safety-considerations}

\subsubsection{5.1 Exposure Limits}\label{exposure-limits}

\textbf{IEEE/ICNIRP guidelines}: Based on thermal effects (tissue
heating) - \textbf{Occupational}: \textasciitilde10
mW/cm\textbackslash textsuperscript\{2\} (averaged over 6 minutes) -
\textbf{General public}: \textasciitilde2
mW/cm\textbackslash textsuperscript\{2\}

\textbf{Frey effect threshold}: \textasciitilde1
\$\textbackslash mu\$J/cm\textbackslash textsuperscript\{2\} per pulse -
For 1 \$\textbackslash mu\$s pulse at 100 pps (0.01\% duty cycle):
Average = \(1 \times 10^{-6} \times 100 = 10^{-4}\)
J/cm\textbackslash textsuperscript\{2\}/s = \textbf{0.01
mW/cm\textbackslash textsuperscript\{2\}} - \textbf{Well below safety
limits}

\textbf{Conclusion}: Frey effect can occur at exposures considered safe
for thermal damage.

\subsubsection{5.2 Health Effects}\label{health-effects}

\textbf{Acute}: - Auditory perception (transient, reversible) -
Annoyance, distraction - No tissue damage at threshold levels

\textbf{Chronic}: - No known long-term effects from brief exposures -
High-intensity repeated exposure could cause cochlear damage (acoustic
trauma-like)

\textbf{Comparison to acoustic hearing}: Frey effect pressure waves
\textasciitilde60-80 dB SPL equivalent (moderate loudness, not
hazardous).

\begin{center}\rule{0.5\linewidth}{0.5pt}\end{center}

\subsection{6. Applications (Potential )}\label{applications-potential}

\subsubsection{6.1 Non-Lethal Weapons /
Deterrents}\label{non-lethal-weapons-deterrents}

\textbf{Concept}: Direct pulsed microwaves at target
\$\textbackslash rightarrow\$ induce disorienting sounds (``voice in
head'')

\textbf{Advantages}: - No physical projectile - Reversible effect - Can
encode information (modulate PRF to transmit speech)

\textbf{Challenges}: - Requires high peak power (kW)
\$\textbackslash rightarrow\$ bulky equipment - Line-of-sight only
(microwaves don\textquotesingle t penetrate walls at GHz) - Ethical
concerns (psychological effects of ``voices'')

\textbf{Status}: Prototypes exist (U.S. military ``MEDUSA'' system);
deployment unclear.

\subsubsection{6.2 Assistive Hearing
Devices}\label{assistive-hearing-devices}

\textbf{Concept}: For sensorineural deaf (damaged hair cells), bypass
cochlea with direct microwave stimulation of auditory nerve.

\textbf{Problem}: Cochlear damage also eliminates microwave effect
(relies on intact cochlea).

\textbf{Alternative}: Cochlear implants (electrical stimulation) are
more effective.

\subsubsection{6.3 Covert Communication}\label{covert-communication}

\textbf{Concept}: Transmit speech via modulated microwave pulses
\$\textbackslash rightarrow\$ target hears without nearby listeners.

\textbf{Challenge}: Requires target to be stationary (beam focusing);
speech intelligibility limited by PRF bandwidth (\textasciitilde10 kHz
max).

\subsubsection{6.4 Scientific Tool}\label{scientific-tool}

\textbf{Brain imaging}: Could microwave pulses selectively activate
auditory cortex for fMRI mapping?

\textbf{Status}: Not pursued (ethical/safety barriers).

\begin{center}\rule{0.5\linewidth}{0.5pt}\end{center}

\subsection{7. Comparison to Other
Phenomena}\label{comparison-to-other-phenomena}

\subsubsection{7.1 Acoustic Heterodyning}\label{acoustic-heterodyning}

\textbf{Different}: Heterodyning mixes two acoustic waves; Frey effect
is \textbf{EM-to-acoustic transduction}.

\textbf{Similarity}: Both create sound ``from nothing'' (no external
source).

\textbf{See}: {[}{[}Acoustic-Heterodyning{]}{]}

\subsubsection{7.2 THz Bioeffects}\label{thz-bioeffects}

\textbf{THz frequencies} (0.1-10 THz) are
\textasciitilde100-1000\$\textbackslash times\$ higher than Frey effect
microwaves (GHz).

\textbf{Could THz cause similar effect?} - \textbf{No}: THz absorbed at
skin (\textless1 mm penetration), never reaches cochlea. - Frey effect
requires \textbf{volumetric heating in brain tissue} near cochlea.

\textbf{See}: {[}{[}THz-Bioeffects-Thermal-and-Non-Thermal{]}{]}

\begin{center}\rule{0.5\linewidth}{0.5pt}\end{center}

\subsection{8. Controversies and
Misconceptions}\label{controversies-and-misconceptions}

\subsubsection{8.1 ``Mind Control'' and Conspiracy
Theories}\label{mind-control-and-conspiracy-theories}

\textbf{Misconception}: Frey effect can implant thoughts or control
behavior.

\textbf{Reality}: Effect only creates auditory perception; cannot write
information directly to brain. No different from hearing a sound via
ears.

\subsubsection{8.2 ``Havana Syndrome''}\label{havana-syndrome}

\textbf{Speculation}: Unexplained health incidents (2016-present)
involving U.S. diplomats attributed to ``sonic attacks'' or directed
energy weapons.

\textbf{Possible explanations}: - Pulsed microwaves (Frey effect) -
Ultrasound - Mass psychogenic illness

\textbf{Scientific consensus}: Mechanism unproven; microwave explanation
plausible but not confirmed.

\subsubsection{8.3 5G and Cell Phones}\label{g-and-cell-phones}

\textbf{Question}: Can 5G towers or cell phones cause Frey effect?

\textbf{Answer}: \textbf{No} - Cell signals are CW or quasi-CW (not
short pulses) - Power too low (milliwatts vs.~kilowatts needed) -
Frequency wrong (5G uses 3-30 GHz; sub-optimal for deep penetration)

\begin{center}\rule{0.5\linewidth}{0.5pt}\end{center}

\subsection{9. Connections to Other Wiki
Pages}\label{connections-to-other-wiki-pages}

\begin{itemize}
\tightlist
\item
  {[}{[}Non-Linear-Biological-Demodulation{]}{]} -\/-\/- Overview of
  nonlinear EM-biology interactions
\item
  {[}{[}Acoustic-Heterodyning{]}{]} -\/-\/- Parametric acoustic arrays
  (different mechanism)
\item
  {[}{[}Intermodulation-Distortion-in-Biology{]}{]} -\/-\/- Nonlinear
  mixing (Frey is not IMD, but related)
\item
  {[}{[}THz-Bioeffects-Thermal-and-Non-Thermal{]}{]} -\/-\/- Comparison
  to THz interactions
\item
  {[}{[}mmWave-\&-THz-Communications{]}{]} -\/-\/- Frequency context
\end{itemize}

\begin{center}\rule{0.5\linewidth}{0.5pt}\end{center}

\subsection{10. References}\label{references}

\subsubsection{Original Discovery}\label{original-discovery}

\begin{enumerate}
\def\labelenumi{\arabic{enumi}.}
\tightlist
\item
  \textbf{Frey, \emph{J. Appl. Physiol.} 17, 689 (1962)} -\/-\/- First
  report of microwave hearing
\end{enumerate}

\subsubsection{Mechanism}\label{mechanism}

\begin{enumerate}
\def\labelenumi{\arabic{enumi}.}
\setcounter{enumi}{1}
\tightlist
\item
  \textbf{Lin, \emph{Proc. IEEE} 68, 67 (1980)} -\/-\/- Thermoelastic
  theory (definitive review)
\item
  \textbf{Foster \& Finch, \emph{Science} 185, 256 (1974)} -\/-\/-
  Pressure wave calculations
\end{enumerate}

\subsubsection{Experimental
Confirmation}\label{experimental-confirmation}

\begin{enumerate}
\def\labelenumi{\arabic{enumi}.}
\setcounter{enumi}{3}
\tightlist
\item
  \textbf{Guy et al., \emph{Radio Sci.} 10, 109 (1975)} -\/-\/- Human
  psychophysical thresholds
\item
  \textbf{Elder \& Chou, \emph{Bioelectromagnetics} 24, 568 (2003)}
  -\/-\/- Cochlear microphonics in animals
\end{enumerate}

\subsubsection{Reviews and Safety}\label{reviews-and-safety}

\begin{enumerate}
\def\labelenumi{\arabic{enumi}.}
\setcounter{enumi}{5}
\tightlist
\item
  \textbf{Lin \& Gandhi, \emph{IEEE Trans. Antennas Propag.} 44, 1413
  (1996)} -\/-\/- Safety assessment
\item
  \textbf{Elder, \emph{Health Phys.} 83, 580 (2002)} -\/-\/-
  Comprehensive review
\end{enumerate}

\subsubsection{Applications
(Speculative)}\label{applications-speculative}

\begin{enumerate}
\def\labelenumi{\arabic{enumi}.}
\setcounter{enumi}{7}
\tightlist
\item
  \textbf{U.S. Army MEDUSA project} (DARPA, 2008) -\/-\/- Non-lethal
  weapon prototype
\end{enumerate}

\begin{center}\rule{0.5\linewidth}{0.5pt}\end{center}

\textbf{Last updated}: October 2025

\begin{center}\rule{0.5\linewidth}{0.5pt}\end{center}

\subsection{Planned Sections}\label{planned-sections}

\begin{itemize}
\tightlist
\item
  Discovery and history
\item
  Physical mechanism
\item
  Experimental evidence
\item
  Safety considerations
\item
  References
\end{itemize}
