\section{Channel Equalization}\label{channel-equalization}

{[}{[}Home{]}{]} \textbar{} \textbf{System Implementation} \textbar{}
{[}{[}Synchronization-(Carrier,-Timing,-Frame){]}{]} \textbar{}
{[}{[}Multipath-Propagation-\&-Fading-(Rayleigh,-Rician){]}{]}

\begin{center}\rule{0.5\linewidth}{0.5pt}\end{center}

\subsection{\texorpdfstring{ For Non-Technical
Readers}{ For Non-Technical Readers}}\label{for-non-technical-readers}

\textbf{Channel equalization is like using an audio equalizer to undo
the distortion from a bad microphone-\/-\/-it reverses the damage the
radio channel causes to your signal!}

\textbf{The problem}: - Radio signals bounce off buildings/walls
(multipath) - Echoes arrive at different times and smear together
\$\textbackslash rightarrow\$ \textbf{inter-symbol interference (ISI)} -
It\textquotesingle s like someone talking in a cave-\/-\/-echoes make
words blur together

\textbf{The solution - Equalization}: 1. Receiver analyzes how the
channel distorts known training signals 2. Calculates inverse filter:
``Channel made signal quieter at 5 kHz? Let\textquotesingle s boost 5
kHz!'' 3. Applies correction to all received data 4. Result: Clean,
sharp signal restored!

\textbf{Real-world analogy - Audio equalizer}: - Cheap headphones boost
bass (distortion) - Audio app detects this and reduces bass to
compensate - Result: Flat, accurate sound - Channel equalizer does the
same for radio signals!

\textbf{Types you encounter}: - \textbf{Linear equalizers} (simple,
fast): WiFi, basic cellular - \textbf{Decision feedback equalizers}
(smarter): High-speed data links - \textbf{Adaptive equalizers} (learns
channel in real-time): Your phone constantly adjusts as you move!

\textbf{When you see it}: - \textbf{4G/5G handoff}: Brief pause = phone
learning new tower\textquotesingle s channel - \textbf{Fast internet
over long phone lines}: DSL equalizers undo cable distortion -
\textbf{Underwater communications}: Extreme multipath requires heavy
equalization

\textbf{Fun fact}: Modern equalizers update hundreds of times per second
as you walk-\/-\/-they track the changing radio environment in
real-time!

\begin{center}\rule{0.5\linewidth}{0.5pt}\end{center}

\subsection{Overview}\label{overview}

\textbf{Channel equalization} compensates for \textbf{Inter-Symbol
Interference (ISI)} caused by multipath propagation.

\textbf{Problem}: Delayed signal copies overlap with current symbol
\$\textbackslash rightarrow\$ ISI

\textbf{Solution}: Apply inverse channel filter to restore original
signal

\textbf{Types}: 1. \textbf{Linear equalizers}: ZF, MMSE 2.
\textbf{Nonlinear equalizers}: DFE (Decision Feedback) 3.
\textbf{Adaptive equalizers}: LMS, RLS 4. \textbf{Frequency-domain}:
OFDM per-subcarrier

\begin{center}\rule{0.5\linewidth}{0.5pt}\end{center}

\subsection{Inter-Symbol Interference
(ISI)}\label{inter-symbol-interference-isi}

\subsubsection{Cause}\label{cause}

\textbf{Multipath channel}:

\[
h(t) = \sum_{l=0}^{L-1} h_l \delta(t - \tau_l)
\]

\textbf{Received signal}:

\[
r(t) = \sum_{l=0}^{L-1} h_l \cdot s(t - \tau_l) + n(t)
\]

\textbf{Effect}: Current symbol affected by \(L-1\) previous symbols

\begin{center}\rule{0.5\linewidth}{0.5pt}\end{center}

\subsubsection{ISI Illustration}\label{isi-illustration}

\textbf{Transmit}: 1 0 1 1

\textbf{Channel}: 2-tap (\(h_0 = 1\), \(h_1 = 0.5\), delay = 1 symbol)

\textbf{Received}: - Symbol 0: \(1 \cdot h_0 = 1.0\) - Symbol 1:
\(0 \cdot h_0 + 1 \cdot h_1 = 0.5\) (ISI from symbol 0) - Symbol 2:
\(1 \cdot h_0 + 0 \cdot h_1 = 1.0\) - Symbol 3:
\(1 \cdot h_0 + 1 \cdot h_1 = 1.5\) (ISI from symbol 2)

\textbf{Equalizer goal}: Remove \(h_1\) contribution

\begin{center}\rule{0.5\linewidth}{0.5pt}\end{center}

\subsubsection{Delay Spread}\label{delay-spread}

\textbf{RMS delay spread} \(\tau_{\text{RMS}}\): Channel memory duration

\textbf{Coherence bandwidth}:

\[
B_c \approx \frac{1}{5 \tau_{\text{RMS}}}
\]

\textbf{Flat fading}: \(B_{\text{signal}} < B_c\) (no ISI)

\textbf{Frequency-selective fading}: \(B_{\text{signal}} > B_c\) (ISI
present, equalization needed)

\begin{center}\rule{0.5\linewidth}{0.5pt}\end{center}

\subsection{Zero-Forcing (ZF)
Equalizer}\label{zero-forcing-zf-equalizer}

\textbf{Idea}: Perfect inversion of channel (force ISI to zero)

\textbf{Frequency domain}:

\[
W(f) = \frac{1}{H(f)}
\]

\textbf{Time domain} (FIR filter, \(N\) taps):

\[
y[n] = \sum_{k=0}^{N-1} w_k \cdot r[n-k]
\]

\textbf{Optimal taps}:
\(\mathbf{w} = (\mathbf{H}^H \mathbf{H})^{-1} \mathbf{H}^H \mathbf{e}_0\)

Where \(\mathbf{e}_0 = [1, 0, \ldots, 0]^T\) (force zero ISI)

\begin{center}\rule{0.5\linewidth}{0.5pt}\end{center}

\subsubsection{ZF Performance}\label{zf-performance}

\textbf{Advantage}: \textbf{Perfect ISI cancellation} (if channel known)

\textbf{Disadvantage}: \textbf{Noise enhancement} at frequency nulls

\textbf{Example}: If \(H(f_0) \approx 0\) (deep fade),
\(W(f_0) \to \infty\) \$\textbackslash rightarrow\$ Amplifies noise

\textbf{Result}: ZF poor at low SNR

\begin{center}\rule{0.5\linewidth}{0.5pt}\end{center}

\subsection{Minimum Mean Square Error (MMSE)
Equalizer}\label{minimum-mean-square-error-mmse-equalizer}

\textbf{Idea}: Minimize \textbf{combined} ISI + noise (trade-off)

\textbf{Cost function}:

\[
\text{MSE} = E[|s[n] - y[n]|^2]
\]

\textbf{Optimal taps} (Wiener solution):

\[
\mathbf{w}_{\text{MMSE}} = (\mathbf{H}^H \mathbf{H} + \sigma^2 \mathbf{I})^{-1} \mathbf{H}^H \mathbf{e}_0
\]

Where \(\sigma^2 = N_0/E_s\) (normalized noise)

\begin{center}\rule{0.5\linewidth}{0.5pt}\end{center}

\subsubsection{MMSE vs ZF}\label{mmse-vs-zf}

\textbf{Frequency domain}:

\[
W_{\text{MMSE}}(f) = \frac{H^*(f)}{|H(f)|^2 + \sigma^2}
\]

\textbf{At deep fade} (\(|H(f)| \ll 1\)): \(W \approx H^*/\sigma^2\)
(doesn\textquotesingle t blow up)

\textbf{At high SNR} (\(\sigma^2 \to 0\)): \(W \to H^*/|H|^2 = 1/H\)
(converges to ZF)

\textbf{Result}: MMSE better than ZF at low-moderate SNR

\begin{center}\rule{0.5\linewidth}{0.5pt}\end{center}

\subsubsection{Performance Comparison}\label{performance-comparison}

{\def\LTcaptype{} % do not increment counter
\begin{longtable}[]{@{}llll@{}}
\toprule\noalign{}
SNR (dB) & ZF BER & MMSE BER & Improvement \\
\midrule\noalign{}
\endhead
\bottomrule\noalign{}
\endlastfoot
5 & 0.05 & 0.02 & 2.5\$\textbackslash times\$ better \\
10 & 0.01 & 0.005 & 2\$\textbackslash times\$ better \\
20 &
10\textbackslash textsuperscript\{-\}\textbackslash textsuperscript\{4\}
&
10\textbackslash textsuperscript\{-\}\textbackslash textsuperscript\{4\}
& Same \\
30 &
10\textbackslash textsuperscript\{-\}\textbackslash textsuperscript\{6\}
&
10\textbackslash textsuperscript\{-\}\textbackslash textsuperscript\{6\}
& Same \\
\end{longtable}
}

\textbf{Pattern}: MMSE wins at low SNR, converge at high SNR

\begin{center}\rule{0.5\linewidth}{0.5pt}\end{center}

\subsection{Decision Feedback Equalizer
(DFE)}\label{decision-feedback-equalizer-dfe}

\textbf{Idea}: Use \textbf{past decisions} to cancel ISI from previous
symbols

\textbf{Structure}:

\begin{verbatim}
           [Feedforward Filter]
                    |
Input ---------> [] -----> [Slicer] --> Output
                             |
                 |            v
           [Feedback Filter] <-
\end{verbatim}

\textbf{Feedforward} (FF): Linear filter (like MMSE)

\textbf{Feedback} (FB): Use previous decisions to cancel post-cursor ISI

\begin{center}\rule{0.5\linewidth}{0.5pt}\end{center}

\subsubsection{DFE Equations}\label{dfe-equations}

\textbf{Feedforward}:

\[
z[n] = \sum_{k=0}^{N_f-1} w_k \cdot r[n-k]
\]

\textbf{Feedback}:

\[
y[n] = z[n] - \sum_{k=1}^{N_b} b_k \cdot \hat{s}[n-k]
\]

\textbf{Decision}: \(\hat{s}[n] = \text{slicer}(y[n])\) (nearest
constellation point)

\begin{center}\rule{0.5\linewidth}{0.5pt}\end{center}

\subsubsection{DFE Advantages}\label{dfe-advantages}

\begin{enumerate}
\def\labelenumi{\arabic{enumi}.}
\tightlist
\item
  \textbf{No noise enhancement}: Feedback uses clean decisions (no noise
  amplification from channel inversion)
\item
  \textbf{Better than linear}: Handles severe ISI
\item
  \textbf{Practical}: Moderate complexity
\end{enumerate}

\begin{center}\rule{0.5\linewidth}{0.5pt}\end{center}

\subsubsection{DFE Disadvantages}\label{dfe-disadvantages}

\begin{enumerate}
\def\labelenumi{\arabic{enumi}.}
\tightlist
\item
  \textbf{Error propagation}: Wrong decision
  \$\textbackslash rightarrow\$ Future decisions corrupted
\item
  \textbf{Training needed}: Requires channel estimate
\item
  \textbf{Latency}: Sequential decisions (can\textquotesingle t
  parallelize easily)
\end{enumerate}

\begin{center}\rule{0.5\linewidth}{0.5pt}\end{center}

\subsubsection{Error Propagation}\label{error-propagation}

\textbf{Example}: 2-tap feedback, BER =
10\textbackslash textsuperscript\{-\}\textbackslash textsuperscript\{3\}

\textbf{Error probability} (1 wrong decision in past 2):

\[
P_{\text{error}} \approx 2 \times 10^{-3} = 2 \times 10^{-3}
\]

\textbf{If decision wrong}: Feedback adds wrong ISI
\$\textbackslash rightarrow\$ Higher BER

\textbf{Mitigation}: Use coding (corrects burst errors from propagation)

\begin{center}\rule{0.5\linewidth}{0.5pt}\end{center}

\subsection{Adaptive Equalization}\label{adaptive-equalization}

\textbf{Problem}: Channel unknown or time-varying (mobile, fading)

\textbf{Solution}: \textbf{Adaptive algorithms} adjust equalizer taps in
real-time

\begin{center}\rule{0.5\linewidth}{0.5pt}\end{center}

\subsubsection{Least Mean Squares (LMS)}\label{least-mean-squares-lms}

\textbf{Stochastic gradient descent}:

\[
\mathbf{w}[n+1] = \mathbf{w}[n] + \mu \cdot e^*[n] \cdot \mathbf{r}[n]
\]

Where: - \(e[n] = d[n] - y[n]\) (error) - \(d[n]\) = Desired output
(training symbol or decision) - \(\mu\) = Step size (0.01-0.1)

\textbf{Advantages}: - Simple (\textasciitilde{}\(2N\) operations) - Low
memory - Stable

\textbf{Disadvantages}: - Slow convergence (\textasciitilde1000+
symbols) - Step size trade-off (fast vs stable)

\begin{center}\rule{0.5\linewidth}{0.5pt}\end{center}

\subsubsection{Recursive Least Squares
(RLS)}\label{recursive-least-squares-rls}

\textbf{Minimize weighted sum} of all past errors:

\[
\min_{\mathbf{w}} \sum_{i=1}^{n} \lambda^{n-i} |d[i] - \mathbf{w}^H \mathbf{r}[i]|^2
\]

\textbf{Update} (Kalman gain):

\[
\mathbf{w}[n] = \mathbf{w}[n-1] + \mathbf{k}[n] \cdot e^*[n]
\]

\textbf{Advantages}: - Fast convergence (\textasciitilde{}\(2N\)
symbols) - Better tracking

\textbf{Disadvantages}: - High complexity (\(O(N^2)\)) - Numerical
instability

\begin{center}\rule{0.5\linewidth}{0.5pt}\end{center}

\subsubsection{LMS vs RLS}\label{lms-vs-rls}

{\def\LTcaptype{} % do not increment counter
\begin{longtable}[]{@{}lll@{}}
\toprule\noalign{}
Aspect & LMS & RLS \\
\midrule\noalign{}
\endhead
\bottomrule\noalign{}
\endlastfoot
\textbf{Complexity} & \(O(N)\) & \(O(N^2)\) \\
\textbf{Convergence} & Slow (1000+) & Fast (10-100) \\
\textbf{Tracking} & Poor & Excellent \\
\textbf{Stability} & Robust & Can diverge \\
\textbf{Use case} & Slow channels & Fast fading \\
\end{longtable}
}

\begin{center}\rule{0.5\linewidth}{0.5pt}\end{center}

\subsection{Training vs Blind
Equalization}\label{training-vs-blind-equalization}

\subsubsection{Training Mode}\label{training-mode}

\textbf{Transmit known symbols} (preamble, midamble)

\textbf{Receiver}: Compare \(y[n]\) to \(d[n]\), adjust taps

\textbf{Duration}: 50-500 symbols (depends on channel)

\textbf{Example}: WiFi long preamble (64 OFDM symbols for channel
estimation)

\begin{center}\rule{0.5\linewidth}{0.5pt}\end{center}

\subsubsection{Decision-Directed Mode}\label{decision-directed-mode}

\textbf{After training}, use \textbf{decisions} as reference:

\[
d[n] = \hat{s}[n] \quad (\text{slicer output})
\]

\textbf{Works if}: BER low enough
(\textasciitilde10\textbackslash textsuperscript\{-\}\textbackslash textsuperscript\{2\}
after training)

\textbf{Tracks slowly varying channel}

\begin{center}\rule{0.5\linewidth}{0.5pt}\end{center}

\subsubsection{Blind Equalization}\label{blind-equalization}

\textbf{No training sequence} (constant modulus, higher-order
statistics)

\textbf{Constant Modulus Algorithm (CMA)}:

\[
e[n] = |y[n]|^2 - R_2
\]

Where \(R_2 = E[|s|^4] / E[|s|^2]\) (modulus)

\textbf{For QPSK}: \(R_2 = 1\) (all symbols same magnitude)

\textbf{Update}: Same as LMS with \(e[n]\) above

\textbf{Advantage}: No preamble overhead

\textbf{Disadvantage}: Slower convergence, phase ambiguity

\begin{center}\rule{0.5\linewidth}{0.5pt}\end{center}

\subsection{Fractionally-Spaced Equalizer
(FSE)}\label{fractionally-spaced-equalizer-fse}

\textbf{Problem}: Symbol-rate sampling misses information
(timing-dependent)

\textbf{Solution}: Sample at \textbf{T/2} (twice symbol rate) or faster

\textbf{Structure}: \(2N\) taps at T/2 spacing

\textbf{Advantages}: 1. \textbf{Timing-independent}: Works at any
sampling phase 2. \textbf{Better performance}: Exploits oversampled
signal 3. \textbf{Joint timing + equalization}

\textbf{Complexity}: 2\$\textbackslash times\$ taps, but worth it

\begin{center}\rule{0.5\linewidth}{0.5pt}\end{center}

\subsection{Frequency-Domain
Equalization}\label{frequency-domain-equalization}

\textbf{For OFDM}: Equalize each subcarrier independently

\textbf{Per-subcarrier}:

\[
\hat{S}_k = \frac{R_k}{H_k}
\]

\textbf{Where}: - \(R_k\) = Received symbol on subcarrier \(k\) -
\(H_k\) = Channel frequency response at subcarrier \(k\) - \(\hat{S}_k\)
= Equalized symbol

\textbf{Equivalent to}: ZF equalizer per tone

\textbf{MMSE variant}:

\[
\hat{S}_k = \frac{H_k^*}{|H_k|^2 + \sigma^2} R_k
\]

\begin{center}\rule{0.5\linewidth}{0.5pt}\end{center}

\subsubsection{OFDM Advantage}\label{ofdm-advantage}

\textbf{Flat fading per subcarrier}: - Wideband channel
\$\textbackslash rightarrow\$ Frequency-selective - Each subcarrier
\$\textbackslash rightarrow\$ Narrow (\textless{} \(B_c\))
\$\textbackslash rightarrow\$ Flat

\textbf{Simple equalization}: Single complex multiply per subcarrier

\textbf{Example}: WiFi 802.11a - 64 subcarriers (52 used) - 20 MHz
channel (312.5 kHz per subcarrier) - Delay spread \textasciitilde200 ns
\$\textbackslash rightarrow\$ \(B_c \approx 1\) MHz - Each subcarrier
flat \$\textbackslash rightarrow\$ 1-tap equalizer

\begin{center}\rule{0.5\linewidth}{0.5pt}\end{center}

\subsection{Channel Estimation}\label{channel-estimation}

\textbf{Equalizer needs} \(H[k]\) or \(\mathbf{h}\)

\begin{center}\rule{0.5\linewidth}{0.5pt}\end{center}

\subsubsection{Pilot-Based Estimation}\label{pilot-based-estimation}

\textbf{Known symbols} (pilots) at indices \(\mathcal{P}\):

\[
\hat{H}_k = \frac{R_k}{S_k}, \quad k \in \mathcal{P}
\]

\textbf{Interpolation} (for data subcarriers):

\[
\hat{H}_k = \sum_{p \in \mathcal{P}} H_p \cdot \text{sinc}(k - p), \quad k \notin \mathcal{P}
\]

\textbf{Or}: Wiener interpolation (MMSE), spline

\textbf{Example}: LTE - 4 pilots per 12 subcarriers (every 3rd
subcarrier) - Linear interpolation (frequency) - Averaging (time,
multiple OFDM symbols)

\begin{center}\rule{0.5\linewidth}{0.5pt}\end{center}

\subsubsection{Least-Squares (LS)
Estimation}\label{least-squares-ls-estimation}

\textbf{Training sequence} \(\mathbf{S}\) (length \(N\)):

\[
\hat{\mathbf{h}} = (\mathbf{S}^H \mathbf{S})^{-1} \mathbf{S}^H \mathbf{r}
\]

\textbf{For pilots}: \(\hat{H}_k = R_k / S_k\) (same as above)

\textbf{Noise}: Not suppressed (LS unbiased but noisy)

\begin{center}\rule{0.5\linewidth}{0.5pt}\end{center}

\subsubsection{MMSE Channel Estimation}\label{mmse-channel-estimation}

\textbf{Incorporate statistics}:

\[
\hat{\mathbf{h}} = \mathbf{R}_{hh} \mathbf{S}^H (\mathbf{S} \mathbf{R}_{hh} \mathbf{S}^H + \sigma^2 \mathbf{I})^{-1} \mathbf{r}
\]

\textbf{Requires}: Channel correlation \(\mathbf{R}_{hh}\) (from delay
profile)

\textbf{Advantage}: Noise suppression (\textasciitilde3 dB gain over LS)

\textbf{Disadvantage}: Complexity, needs statistics

\begin{center}\rule{0.5\linewidth}{0.5pt}\end{center}

\subsection{Practical Examples}\label{practical-examples}

\subsubsection{1. WiFi 802.11n (MIMO)}\label{wifi-802.11n-mimo}

\textbf{Channel estimation}: Long preamble (HT-LTF) - 2 OFDM symbols per
spatial stream - LS estimation - Linear interpolation (frequency)

\textbf{Equalization}: MMSE per subcarrier -
\(\hat{\mathbf{S}} = (\mathbf{H}^H \mathbf{H} + \sigma^2 \mathbf{I})^{-1} \mathbf{H}^H \mathbf{R}\)
- Per-subcarrier 2\$\textbackslash times\$2 or
4\$\textbackslash times\$4 matrix inversion

\textbf{Tracking}: Pilot tones (4 per 56 subcarriers)

\begin{center}\rule{0.5\linewidth}{0.5pt}\end{center}

\subsubsection{2. LTE Downlink}\label{lte-downlink}

\textbf{Channel estimation}: Cell-Specific Reference Signals (CRS) - 4
pilots per 12 subcarriers per OFDM symbol - MMSE estimation (Wiener
filtering)

\textbf{Equalization}: MMSE (frequency domain) - Per-subcarrier,
per-antenna

\textbf{Interference}: MRC (Maximum Ratio Combining) across antennas

\textbf{Result}: Supports 300 km/h (high Doppler)

\begin{center}\rule{0.5\linewidth}{0.5pt}\end{center}

\subsubsection{3. DVB-T (Terrestrial TV)}\label{dvb-t-terrestrial-tv}

\textbf{Channel estimation}: Scattered pilots (8\%) - Wiener
interpolation (time + frequency) - Handles long delay spread (SFN
networks, 200 \$\textbackslash mu\$s)

\textbf{Equalization}: Per-subcarrier ZF or MMSE

\textbf{Guard interval}: 1/4, 1/8, 1/16, 1/32 of symbol
(user-selectable)

\begin{center}\rule{0.5\linewidth}{0.5pt}\end{center}

\subsubsection{4. GSM (Legacy Cellular)}\label{gsm-legacy-cellular}

\textbf{Training sequence}: 26-bit midamble

\textbf{Equalization}: Viterbi (MLSE, Maximum Likelihood Sequence
Estimation) - 5-tap channel \$\textbackslash rightarrow\$ 16 states -
Optimal for short bursts

\textbf{ISI}: \textasciitilde5-15 symbols (urban, hilly)

\textbf{Result}: Works up to 10 \$\textbackslash mu\$s delay spread

\begin{center}\rule{0.5\linewidth}{0.5pt}\end{center}

\subsection{Advanced Techniques}\label{advanced-techniques}

\subsubsection{1. Turbo Equalization}\label{turbo-equalization}

\textbf{Iterative}: Equalizer \$\textbackslash leftrightarrow\$ Decoder
exchange soft information

\textbf{Structure}:

\begin{verbatim}
Received --> [SISO      <---> [Deinterleaver] <---> [SISO
              Equalizer]                              Decoder]
                                                       
           (extrinsic LLRs)                        (decoded bits)
\end{verbatim}

\textbf{Iterations}: 3-5

\textbf{Gain}: \textasciitilde2-3 dB over separate equalization +
decoding

\textbf{Used in}: Deep space, underwater acoustics

\begin{center}\rule{0.5\linewidth}{0.5pt}\end{center}

\subsubsection{2. Precoding (Transmit
Equalization)}\label{precoding-transmit-equalization}

\textbf{Pre-invert channel at transmitter} (if channel known via
feedback):

\textbf{Transmit}: \(\mathbf{x} = \mathbf{W} \mathbf{s}\)

\textbf{Where}: \(\mathbf{W} = \mathbf{H}^{-1}\) or MMSE variant

\textbf{Advantage}: Simple receiver (no equalization)

\textbf{Disadvantage}: Requires CSI at TX (feedback latency)

\textbf{Used in}: TDD systems (reciprocity), MU-MIMO downlink

\begin{center}\rule{0.5\linewidth}{0.5pt}\end{center}

\subsubsection{3. Dirty Paper Coding
(DPC)}\label{dirty-paper-coding-dpc}

\textbf{Theoretical}: Pre-cancel interference without power penalty

\textbf{Practical approximation}: Tomlinson-Harashima Precoding (THP)

\textbf{Gain}: Approaches capacity (multi-user downlink)

\textbf{Complexity}: High (not widely deployed)

\begin{center}\rule{0.5\linewidth}{0.5pt}\end{center}

\subsection{Equalization Complexity}\label{equalization-complexity}

{\def\LTcaptype{} % do not increment counter
\begin{longtable}[]{@{}
  >{\raggedright\arraybackslash}p{(\linewidth - 4\tabcolsep) * \real{0.2000}}
  >{\raggedright\arraybackslash}p{(\linewidth - 4\tabcolsep) * \real{0.6250}}
  >{\raggedright\arraybackslash}p{(\linewidth - 4\tabcolsep) * \real{0.1750}}@{}}
\toprule\noalign{}
\begin{minipage}[b]{\linewidth}\raggedright
Method
\end{minipage} & \begin{minipage}[b]{\linewidth}\raggedright
Complexity (per symbol)
\end{minipage} & \begin{minipage}[b]{\linewidth}\raggedright
Notes
\end{minipage} \\
\midrule\noalign{}
\endhead
\bottomrule\noalign{}
\endlastfoot
\textbf{ZF (freq domain)} & \(O(\log N)\) & FFT + per-tone multiply \\
\textbf{MMSE (freq domain)} & \(O(\log N)\) & FFT + per-tone multiply \\
\textbf{Linear (time domain)} & \(O(N_{\text{taps}})\) & FIR filter \\
\textbf{DFE} & \(O(N_f + N_b)\) & FF + FB filters \\
\textbf{LMS} & \(O(N)\) & Simple update \\
\textbf{RLS} & \(O(N^2)\) & Matrix operations \\
\textbf{MLSE (Viterbi)} & \(O(M^L)\) & \(M\) = constellation, \(L\) =
ISI length \\
\end{longtable}
}

\begin{center}\rule{0.5\linewidth}{0.5pt}\end{center}

\subsection{Design Guidelines}\label{design-guidelines}

\subsubsection{1. Choose Equalizer Type}\label{choose-equalizer-type}

\textbf{Flat fading} (delay spread \textless{} 0.1 symbol period): - No
equalizer needed (or 1-tap phase correction)

\textbf{Mild ISI} (delay spread 0.1-1 symbol period): - Linear MMSE
(5-15 taps) - Fractionally-spaced

\textbf{Severe ISI} (delay spread \textgreater{} 1 symbol period): - DFE
(15+ feedforward, 5-10 feedback) - Or OFDM (avoid time-domain
equalization)

\textbf{Very severe ISI} (delay spread \textgreater{} 5 symbols): - OFDM
with guard interval - Or MLSE (if short burst)

\begin{center}\rule{0.5\linewidth}{0.5pt}\end{center}

\subsubsection{2. Select Adaptation
Algorithm}\label{select-adaptation-algorithm}

\textbf{Slow channel} (\textless{} 1 Hz Doppler): - LMS (\(\mu = 0.01\))
- Low complexity

\textbf{Moderate channel} (1-100 Hz): - LMS (\(\mu = 0.05\)) or RLS -
Update every symbol

\textbf{Fast channel} (100+ Hz): - RLS or decision-directed -
Pilot-aided tracking

\begin{center}\rule{0.5\linewidth}{0.5pt}\end{center}

\subsubsection{3. Training Overhead}\label{training-overhead}

\textbf{Packet systems} (WiFi, 5G): - Training per packet (10-20\%
overhead) - Decision-directed within packet

\textbf{Continuous} (TV, broadcast): - Sparse pilots (1-5\% overhead) -
Continuous tracking

\textbf{Burst} (GSM, satellite TDMA): - Midamble (10-15\% overhead) -
Per-burst estimation

\begin{center}\rule{0.5\linewidth}{0.5pt}\end{center}

\subsection{Equalization vs Coding}\label{equalization-vs-coding}

\textbf{Equalization}: Removes ISI (deterministic distortion)

\textbf{Coding}: Corrects random errors (noise)

\textbf{Combined}: Achieves near-capacity - Coding gain: 5-10 dB -
Equalization: Enables coding to work (removes ISI)

\textbf{Without equalization}: Coding fails (BER floor from ISI)

\begin{center}\rule{0.5\linewidth}{0.5pt}\end{center}

\subsection{Related Topics}\label{related-topics}

\begin{itemize}
\tightlist
\item
  \textbf{{[}{[}Multipath-Propagation-\&-Fading-(Rayleigh,-Rician){]}{]}}:
  Cause of ISI
\item
  \textbf{{[}{[}Channel-Models-(Rayleigh-\&-Rician){]}{]}}: Simulation
  models
\item
  \textbf{{[}{[}OFDM-\&-Multicarrier-Modulation{]}{]}}: Frequency-domain
  approach
\item
  \textbf{{[}{[}Synchronization-(Carrier,-Timing,-Frame){]}{]}}:
  Complements equalization
\item
  \textbf{{[}{[}MIMO-\&-Spatial-Multiplexing{]}{]}}: Multi-antenna
  equalization
\end{itemize}

\begin{center}\rule{0.5\linewidth}{0.5pt}\end{center}

\textbf{Key takeaway}: \textbf{Channel equalization removes ISI from
multipath.} ZF inverts channel perfectly but amplifies noise (poor low
SNR). MMSE trades ISI vs noise (\(W = H^*/(|H|^2 + \sigma^2)\)), optimal
at moderate SNR. DFE uses past decisions (feedback) to avoid noise
enhancement-\/-\/-better than linear but error propagation risk.
Adaptive: LMS simple (\(O(N)\), slow), RLS fast (\(O(N^2)\), complex).
Fractionally-spaced (T/2 sampling) is timing-independent. OFDM:
Per-subcarrier 1-tap equalizer (frequency-domain, flat fading per tone).
Channel estimation: Pilots (LS noisy, MMSE better). WiFi: Long preamble
+ pilots. LTE: CRS pilots, Wiener filtering. Turbo equalization:
Iterative with decoder (+2-3 dB). Delay spread \textgreater{} 0.1T needs
equalization. Severe ISI (\textgreater1T) \$\textbackslash rightarrow\$
Use OFDM or DFE. Coding + equalization = near-capacity.

\begin{center}\rule{0.5\linewidth}{0.5pt}\end{center}

\emph{This wiki is part of the {[}{[}Home\textbar Chimera Project{]}{]}
documentation.}
