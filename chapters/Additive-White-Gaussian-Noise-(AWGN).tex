\section{Additive White Gaussian Noise
(AWGN)}\label{additive-white-gaussian-noise-awgn}

\subsection{\texorpdfstring{ For Non-Technical
Readers}{ For Non-Technical Readers}}\label{for-non-technical-readers}

\textbf{AWGN is the ``static'' or ``hiss'' you hear on old
radios-\/-\/-random interference that corrupts signals.}

\textbf{Real-world analogies}: - \textbf{Old TV static}: Random noise
when you lose the signal - \textbf{Grainy photos in low light}: Camera
sensors produce AWGN when there\textquotesingle s not enough light -
\textbf{Bluetooth stuttering}: Weak signals get corrupted by AWGN

\textbf{Where it comes from}: Thermal noise (electrons jiggling from
heat), electronic components, even faint cosmic microwave background
radiation.

\textbf{Why it matters}: AWGN is the \textbf{fundamental limit} on
communication. You can\textquotesingle t eliminate it, but you can
overcome it with more power, error correction, or better antennas.

\begin{center}\rule{0.5\linewidth}{0.5pt}\end{center}

\textbf{AWGN} is a basic noise model used in communication systems.

\subsection{What is AWGN?}\label{what-is-awgn}

\begin{itemize}
\tightlist
\item
  \textbf{Additive}: Noise is added to the signal
\item
  \textbf{White}: Uniform power across all frequencies
\item
  \textbf{Gaussian}: Follows a normal (Gaussian) probability
  distribution
\end{itemize}

\subsection{Visualizing AWGN}\label{visualizing-awgn}

\begin{verbatim}
Clean Signal:     ----------
                         
AWGN:            
                         
Noisy Signal:    -----
                (Clean + AWGN)
\end{verbatim}

\subsection{AWGN Channel Model}\label{awgn-channel-model}

In the I/Q plane, AWGN adds independent Gaussian noise to both
components:

\begin{verbatim}
Received Symbol = Transmitted Symbol + Noise

I_received = I_transmitted + N_I
Q_received = Q_transmitted + N_Q

where N_I and N_Q are independent Gaussian random variables
\end{verbatim}

\subsection{Mathematical Properties}\label{mathematical-properties}

The noise samples have: - \textbf{Mean}: 0 (centered around zero) -
\textbf{Variance}:
\$\textbackslash sigma\$\textbackslash textsuperscript\{2\} (determined
by noise power) - \textbf{Probability Distribution}: Gaussian (bell
curve)

\begin{verbatim}
Probability Density:
  
  |    
  |   +  +
  |  +    +
  | +      +___
  |+            +___
  +----------------- Amplitude
       0
\end{verbatim}

\subsection{Why AWGN is Used}\label{why-awgn-is-used}

\begin{enumerate}
\def\labelenumi{\arabic{enumi}.}
\tightlist
\item
  \textbf{Simplicity}: Mathematical tractability for analysis
\item
  \textbf{Fundamental Model}: Many real noise sources approximate
  Gaussian statistics
\item
  \textbf{Worst Case}: Often represents a pessimistic but realistic
  scenario
\item
  \textbf{Standard Benchmark}: Industry-standard for comparing systems
\end{enumerate}

\subsection{Sources of Noise in Real
Systems}\label{sources-of-noise-in-real-systems}

\begin{itemize}
\tightlist
\item
  \textbf{Thermal Noise}: Random motion of electrons (kTB)
\item
  \textbf{Amplifier Noise}: Electronic component noise
\item
  \textbf{Cosmic Noise}: Background radiation
\item
  \textbf{Interference}: Other signals (approximates Gaussian when many
  sources)
\end{itemize}

\subsection{AWGN in Chimera}\label{awgn-in-chimera}

Chimera\textquotesingle s simulation applies AWGN to model the
communication channel: - Noise is added separately to I and Q components
- Noise power is controlled by the
{[}{[}Signal-to-Noise-Ratio-(SNR){]}{]} setting - Higher SNR = less
noise variance = tighter constellation clusters

\subsubsection{Noise Power Calculation}\label{noise-power-calculation}

\begin{verbatim}
Noise Variance (²) = Signal Power / SNR_linear
                    = Signal Power / 10^(SNR_dB/10)
\end{verbatim}

For unit signal power and SNR = 10 dB:

\begin{verbatim}
² = 1 / 10^(10/10) = 1/10 = 0.1
 = 0.1  0.316
\end{verbatim}

\subsection{Impact on Constellation}\label{impact-on-constellation}

\begin{verbatim}
High Noise ( = 0.5):          Low Noise ( = 0.1):
  
  Q                               Q
                                 
  |                            |  
  |                          | 
  |                            |  
--+---- I                    ----+-- I

Large scatter = high errors    Tight cluster = low errors
\end{verbatim}

\subsection{See Also}\label{see-also}

\begin{itemize}
\tightlist
\item
  {[}{[}Signal-to-Noise-Ratio-(SNR){]}{]} - Controls noise power
\item
  {[}{[}Link-Loss-vs-Noise{]}{]} - Distinction between attenuation and
  noise
\item
  {[}{[}Constellation-Diagrams{]}{]} - Visualizing noise effects
\item
  {[}{[}IQ-Representation{]}{]} - How noise affects I/Q components
\end{itemize}
