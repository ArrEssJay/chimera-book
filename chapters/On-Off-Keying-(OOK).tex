\section{On-Off Keying (OOK)}\label{on-off-keying-ook}

\subsection{\texorpdfstring{ For Non-Technical
Readers}{ For Non-Technical Readers}}\label{for-non-technical-readers}

\textbf{OOK is literally just turning a signal ON and
OFF-\/-\/-it\textquotesingle s the simplest possible way to send data,
like morse code with a flashlight!}

\textbf{The idea}: - \textbf{ON} (signal present) = binary \textbf{1} -
\textbf{OFF} (no signal) = binary \textbf{0} - That\textquotesingle s
it! Simplest modulation possible.

\textbf{Flashlight analogy}: - Shine flashlight = 1 - Turn off
flashlight = 0 - Sequence: ON-ON-OFF-ON = ``1101'' - Morse code uses the
same principle!

\textbf{Why it\textquotesingle s everywhere (despite being old)}: -
\textbf{Dead simple}: Easiest to transmit and receive - \textbf{Lowest
power}: No signal = no power consumption for 0s! - \textbf{Cheap
hardware}: Basic transistor switch = complete transmitter - \textbf{Good
enough}: For short-range, low-speed, it just works

\textbf{Where you see OOK every day}: - \textbf{Car key fobs}: Unlock
button uses OOK! - \textbf{Garage door openers}: Yep, OOK -
\textbf{Wireless doorbells}: OOK at \textasciitilde315/433 MHz -
\textbf{Cheap weather sensors}: Temperature transmitter
\$\textbackslash rightarrow\$ receiver - \textbf{RC toys}: Simple remote
controls - \textbf{Old telegraph}: On/off keying of electrical circuit!

\textbf{Why it\textquotesingle s not used for high-speed}: -
\textbf{Bandwidth inefficient}: Need wide frequency band for sharp
on/off transitions - \textbf{Noise sensitive}: Hard to tell weak signal
from noise - \textbf{No error detection}: Unlike PSK/QAM,
can\textquotesingle t detect phase errors - \textbf{Synchronization
issues}: Receiver must guess when bits start/end

\textbf{Modern variant - ASK}: - OOK is binary ASK (Amplitude-Shift
Keying) - Instead of on/off, use multiple power levels - Still simple,
slightly more efficient

\textbf{The ultimate simplicity}: - \textbf{Transmitter}:
Microcontroller + transistor + antenna - \textbf{Receiver}: Antenna +
diode + microcontroller - Total cost: \textless\$2 for both sides! -
This is why every wireless doorbell uses OOK

\textbf{Fun fact}: The first wireless telegraph (Marconi, 1895) used
OOK-\/-\/-literally just turning a spark-gap transmitter on and off to
send morse code. 130 years later, your car keys still use the same basic
principle!

\begin{center}\rule{0.5\linewidth}{0.5pt}\end{center}

\textbf{On-Off Keying (OOK)} is the simplest form of digital modulation,
where the presence or absence of a carrier wave represents binary data.

\begin{center}\rule{0.5\linewidth}{0.5pt}\end{center}

\subsection{\texorpdfstring{ Basic
Principle}{ Basic Principle}}\label{basic-principle}

\begin{verbatim}
Bit "1": Carrier ON   s(t) = A·cos(2f_c·t)
Bit "0": Carrier OFF  s(t) = 0

where:
- A = carrier amplitude
- f_c = carrier frequency
- T_b = bit duration
\end{verbatim}

\textbf{Time-domain representation}:

\begin{verbatim}
       ___     ___         ___
      |   |   |   |       |   |
  ____|   |___|   |_______|   |___
      1   0   1       0       1     (data bits)
\end{verbatim}

\begin{center}\rule{0.5\linewidth}{0.5pt}\end{center}

\subsection{\texorpdfstring{ Mathematical
Description}{ Mathematical Description}}\label{mathematical-description}

\textbf{Transmitted signal}:

\begin{verbatim}
s(t) =  b_k · A·cos(2f_c·t)     for kT_b  t < (k+1)T_b
       k

where b_k  {0, 1}
\end{verbatim}

\textbf{Modulation index}: m = 1 (100\% modulation depth)

\begin{center}\rule{0.5\linewidth}{0.5pt}\end{center}

\subsection{\texorpdfstring{ Spectral
Characteristics}{ Spectral Characteristics}}\label{spectral-characteristics}

\textbf{Bandwidth} (null-to-null):

\begin{verbatim}
B = 2/T_b = 2R_b

where R_b = bit rate (bps)
\end{verbatim}

\textbf{Power spectral density}: Sinc\textbackslash textsuperscript\{2\}
function centered at f\_c

\textbf{Example}: 1 kbps data rate \$\textbackslash rightarrow\$ 2 kHz
bandwidth

\begin{center}\rule{0.5\linewidth}{0.5pt}\end{center}

\subsection{\texorpdfstring{
Demodulation}{ Demodulation}}\label{demodulation}

\subsubsection{Non-Coherent Detection (Envelope
Detector)}\label{non-coherent-detection-envelope-detector}

\textbf{Simplest receiver} - no carrier phase recovery needed!

\begin{verbatim}
Received signal:
r(t) = s(t) + n(t)

Envelope detector:
e(t) = |r(t)| = [I²(t) + Q²(t)]

Decision:
If e(t) > threshold: bit = 1
If e(t) < threshold: bit = 0
\end{verbatim}

\textbf{Advantage}: Very simple hardware (diode + capacitor)
\textbf{Disadvantage}: 3 dB worse performance than coherent detection

\begin{center}\rule{0.5\linewidth}{0.5pt}\end{center}

\subsubsection{Coherent Detection
(Correlation)}\label{coherent-detection-correlation}

\textbf{Better performance} but requires carrier synchronization:

\begin{verbatim}
Correlator output:
z = ^Tb r(t)·cos(2f_c·t) dt

Decision:
If z > 0: bit = 1
If z < 0: bit = 0
\end{verbatim}

\begin{center}\rule{0.5\linewidth}{0.5pt}\end{center}

\subsection{\texorpdfstring{ Performance
Analysis}{ Performance Analysis}}\label{performance-analysis}

\subsubsection{Bit Error Rate (BER)}\label{bit-error-rate-ber}

\textbf{With coherent detection} (AWGN channel):

\begin{verbatim}
BER = Q((E_b/N))

where:
- E_b = bit energy = (A²T_b)/2
- N = noise power spectral density
- Q(x) = (1/2) _x^ e^(-t²/2) dt  (tail probability of Gaussian)
\end{verbatim}

\textbf{With non-coherent detection}:

\begin{verbatim}
BER = (1/2)exp(-E_b/2N)    (3 dB worse!)
\end{verbatim}

\textbf{Example}: For BER =
10\textbackslash textsuperscript\{-\}\textbackslash textsuperscript\{6\}
- Coherent OOK: E\_b/N\textbackslash textsubscript\{0\}
\$\textbackslash approx\$ 13.5 dB - Non-coherent OOK:
E\_b/N\textbackslash textsubscript\{0\} \$\textbackslash approx\$ 16.5
dB - {[}{[}QPSK-Modulation\textbar QPSK{]}{]}:
E\_b/N\textbackslash textsubscript\{0\} \$\textbackslash approx\$ 10.5
dB (better!)

\begin{center}\rule{0.5\linewidth}{0.5pt}\end{center}

\subsection{\texorpdfstring{ Advantages \&
Disadvantages}{ Advantages \& Disadvantages}}\label{advantages-disadvantages}

\subsubsection{Advantages}\label{advantages}

\textbf{Simplest modulation} - minimal transmitter complexity \textbf{No
phase synchronization} (non-coherent detection) \textbf{Power efficient
when off} - ideal for low duty cycle \textbf{Easy to implement} -
analog/digital

\subsubsection{Disadvantages}\label{disadvantages}

\textbf{Poor spectral efficiency} - 0.5 bits/s/Hz (twice bandwidth of
BPSK) \textbf{Poor power efficiency} - needs 3 dB more power than BPSK
for same BER \textbf{Susceptible to fading} - deep fades completely
eliminate signal \textbf{No use of ``0'' transmission} - wastes half the
signal space

\begin{center}\rule{0.5\linewidth}{0.5pt}\end{center}

\subsection{\texorpdfstring{
Applications}{ Applications}}\label{applications}

\subsubsection{Historical}\label{historical}

\begin{itemize}
\tightlist
\item
  \textbf{Morse code} (telegraphy, 1840s)
\item
  \textbf{Early radio} (spark-gap transmitters)
\item
  \textbf{Infrared remote controls} (TV remotes, 1980s)
\end{itemize}

\subsubsection{Modern}\label{modern}

\begin{itemize}
\tightlist
\item
  \textbf{Optical fiber} (on-off of laser)
\item
  \textbf{RFID tags} (passive, backscatter modulation)
\item
  \textbf{Low-power IoT} (e.g., LoRa preamble)
\item
  \textbf{Visible light communication} (LED on-off)
\end{itemize}

\textbf{Why still used?} Simplicity trumps efficiency for low-cost,
low-power devices.

\begin{center}\rule{0.5\linewidth}{0.5pt}\end{center}

\subsection{\texorpdfstring{ Variants}{ Variants}}\label{variants}

\subsubsection{Amplitude-Shift Keying
(ASK)}\label{amplitude-shift-keying-ask}

\textbf{Generalization of OOK} with non-zero ``off'' level:

\begin{verbatim}
Bit "1": s(t) = A·cos(2f_c·t)
Bit "0": s(t) = A·cos(2f_c·t)    (A > 0)
\end{verbatim}

\textbf{OOK is special case}: A\textbackslash textsubscript\{0\} = 0

\begin{center}\rule{0.5\linewidth}{0.5pt}\end{center}

\subsubsection{Pulse-Position Modulation
(PPM)}\label{pulse-position-modulation-ppm}

\textbf{Used in optical communications}:

\begin{verbatim}
Bit "1": Pulse at t = 0
Bit "0": Pulse at t = T_b/2
\end{verbatim}

\textbf{More power-efficient} than OOK for optical systems.

\begin{center}\rule{0.5\linewidth}{0.5pt}\end{center}

\subsection{\texorpdfstring{ Constellation
Diagram}{ Constellation Diagram}}\label{constellation-diagram}

\begin{verbatim}
      Q
      
      |
  0   |    1   Only two points!
      |   (A, 0)
------+------ I
      |
\end{verbatim}

\textbf{Single dimension} (amplitude only, no phase modulation).

\textbf{Distance between points}: d = A

\textbf{Compare to {[}{[}QPSK-Modulation\textbar QPSK{]}{]}}: Four
points, better use of signal space.

\begin{center}\rule{0.5\linewidth}{0.5pt}\end{center}

\subsection{\texorpdfstring{ Comparison to Other
Modulations}{ Comparison to Other Modulations}}\label{comparison-to-other-modulations}

{\def\LTcaptype{} % do not increment counter
\begin{longtable}[]{@{}
  >{\raggedright\arraybackslash}p{(\linewidth - 8\tabcolsep) * \real{0.1714}}
  >{\raggedright\arraybackslash}p{(\linewidth - 8\tabcolsep) * \real{0.1857}}
  >{\raggedright\arraybackslash}p{(\linewidth - 8\tabcolsep) * \real{0.1571}}
  >{\raggedright\arraybackslash}p{(\linewidth - 8\tabcolsep) * \real{0.3143}}
  >{\raggedright\arraybackslash}p{(\linewidth - 8\tabcolsep) * \real{0.1714}}@{}}
\toprule\noalign{}
\begin{minipage}[b]{\linewidth}\raggedright
Modulation
\end{minipage} & \begin{minipage}[b]{\linewidth}\raggedright
Bits/Symbol
\end{minipage} & \begin{minipage}[b]{\linewidth}\raggedright
Bandwidth
\end{minipage} & \begin{minipage}[b]{\linewidth}\raggedright
Power (for BER 10\^{}-6)
\end{minipage} & \begin{minipage}[b]{\linewidth}\raggedright
Complexity
\end{minipage} \\
\midrule\noalign{}
\endhead
\bottomrule\noalign{}
\endlastfoot
\textbf{OOK} & 1 & 2R\_b & 16.5 dB (non-coh) & Lowest \\
{[}{[}BPSK{]}{]} & 1 & R\_b & 10.5 dB & Low \\
{[}{[}QPSK-Modulation & QPSK{]}{]} & 2 & R\_b & 10.5 dB \\
{[}{[}16-QAM{]}{]} & 4 & R\_b & 18.5 dB & High \\
\end{longtable}
}

\textbf{Key insight}: OOK is simple but inefficient. {[}{[}BPSK{]}{]} is
better in almost every way (except hardware complexity).

\begin{center}\rule{0.5\linewidth}{0.5pt}\end{center}

\subsection{\texorpdfstring{ Key
Takeaways}{ Key Takeaways}}\label{key-takeaways}

\begin{enumerate}
\def\labelenumi{\arabic{enumi}.}
\tightlist
\item
  \textbf{Simplest modulation}: Just turn carrier on/off
\item
  \textbf{Non-coherent detection possible}: No carrier recovery needed
\item
  \textbf{Poor efficiency}: Both spectral and power
\item
  \textbf{Historical importance}: First digital modulation
\item
  \textbf{Still used}: Low-cost, low-power applications (optical, RFID)
\item
  \textbf{Gateway to understanding}: Good starting point before
  {[}{[}FSK{]}{]}, {[}{[}BPSK{]}{]}
\end{enumerate}

\begin{center}\rule{0.5\linewidth}{0.5pt}\end{center}

\subsection{\texorpdfstring{ See Also}{ See Also}}\label{see-also}

\begin{itemize}
\tightlist
\item
  {[}{[}Amplitude-Shift-Keying-(ASK){]}{]} - Generalization of OOK
  \emph{(coming soon)}
\item
  {[}{[}Frequency-Shift-Keying-(FSK){]}{]} - Next step in modulation
  complexity
\item
  {[}{[}Binary-Phase-Shift-Keying-(BPSK){]}{]} - Better alternative
  (same complexity, better performance)
\item
  {[}{[}QPSK-Modulation{]}{]} - Even more efficient
\item
  {[}{[}Constellation-Diagrams{]}{]} - Visual representation of
  modulations
\end{itemize}

\begin{center}\rule{0.5\linewidth}{0.5pt}\end{center}

\subsection{\texorpdfstring{ References}{ References}}\label{references}

\begin{enumerate}
\def\labelenumi{\arabic{enumi}.}
\tightlist
\item
  \textbf{Morse, S.} (1840) - First practical OOK system (telegraph)
\item
  \textbf{Proakis, J.G. \& Salehi, M.} (2008) \emph{Digital
  Communications} 5th ed.~(McGraw-Hill)
\item
  \textbf{Sklar, B.} (2001) \emph{Digital Communications: Fundamentals
  and Applications} 2nd ed.~(Prentice Hall)
\end{enumerate}
