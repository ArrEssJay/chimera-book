\section{Free-Space Path Loss (FSPL)}\label{free-space-path-loss-fspl}

\subsection{\texorpdfstring{ For Non-Technical
Readers}{ For Non-Technical Readers}}\label{for-non-technical-readers}

\textbf{Like shouting across a field-\/-\/-the farther away, the
quieter. Radio waves spread out and weaken with distance.}

\textbf{Key insights}: - \textbf{Double the distance} = signal becomes
\textbf{4\$\textbackslash times\$ weaker} - \textbf{Higher frequency}
(5G) = weaker than lower frequency (4G) over same distance -
\textbf{Satellites 36,000 km away}: Signal weakens by 10 trillion
trillion times! (That\textquotesingle s why dishes are big)

\textbf{Real examples}: WiFi weakens 10,000\$\textbackslash times\$ over
50 meters. Cell towers need to be closer for 5G than 4G.

\begin{center}\rule{0.5\linewidth}{0.5pt}\end{center}

\textbf{Free-Space Path Loss (FSPL)} quantifies how much signal power is
lost as an electromagnetic wave propagates through free space.

\begin{center}\rule{0.5\linewidth}{0.5pt}\end{center}

\subsection{\texorpdfstring{ The Friis Transmission
Equation}{ The Friis Transmission Equation}}\label{the-friis-transmission-equation}

\textbf{Fundamental equation} linking transmitter and receiver:

\begin{verbatim}
P_R = P_T · G_T · G_R · (/4d)²

where:
- P_R = received power (W)
- P_T = transmitted power (W)
- G_T = transmit antenna gain (linear, dimensionless)
- G_R = receive antenna gain (linear, dimensionless)
-  = wavelength (m)
- d = distance between antennas (m)
\end{verbatim}

\textbf{Assumptions}: - Free space (no obstacles, no atmosphere) -
Far-field (d \textgreater\textgreater{} antenna dimensions) -
Polarization matched - Antennas aligned

\begin{center}\rule{0.5\linewidth}{0.5pt}\end{center}

\subsection{\texorpdfstring{ Path Loss
Definition}{ Path Loss Definition}}\label{path-loss-definition}

\textbf{Path Loss (L)} is the ratio of transmitted to received power:

\begin{verbatim}
L = P_T/P_R

In dB:
L_dB = 10 log(L) = 10 log(P_T/P_R)
\end{verbatim}

From Friis equation (assuming isotropic antennas, G\_T = G\_R = 1):

\begin{verbatim}
L = (4d/)²

In dB:
FSPL_dB = 20 log(4d/)
        = 20 log(d) + 20 log(f) + 20 log(4/c)
        = 20 log(d) + 20 log(f) - 147.55
\end{verbatim}

\textbf{More practical form} (f in Hz, d in m):

\begin{verbatim}
FSPL_dB = 20 log(d) + 20 log(f) + 92.45
\end{verbatim}

\textbf{Or} (f in MHz, d in km):

\begin{verbatim}
FSPL_dB = 20 log(d_km) + 20 log(f_MHz) + 32.45
\end{verbatim}

\begin{center}\rule{0.5\linewidth}{0.5pt}\end{center}

\subsection{\texorpdfstring{ Example
Calculations}{ Example Calculations}}\label{example-calculations}

\subsubsection{Example 1: WiFi (2.4 GHz, 10
m)}\label{example-1-wifi-2.4-ghz-10-m}

\begin{verbatim}
f = 2.4 GHz = 2.4×10 Hz
d = 10 m

FSPL = 20 log(10) + 20 log(2.4×10) + 92.45
     = 20 + 187.6 + 92.45
     = 100 dB
\end{verbatim}

\textbf{Interpretation}: Signal power drops by factor of
10\textbackslash textsuperscript\{1\}\textbackslash textsuperscript\{0\}
(10 billion) over 10 m!

\begin{center}\rule{0.5\linewidth}{0.5pt}\end{center}

\subsubsection{Example 2: Cell Phone (900 MHz, 1
km)}\label{example-2-cell-phone-900-mhz-1-km}

\begin{verbatim}
f = 900 MHz
d = 1 km = 1000 m

FSPL = 20 log(1000) + 20 log(900×10) + 92.45
     = 60 + 179 + 92.45
     = 131.5 dB
\end{verbatim}

\begin{center}\rule{0.5\linewidth}{0.5pt}\end{center}

\subsubsection{Example 3: Satellite (12 GHz, 36,000 km -
GEO)}\label{example-3-satellite-12-ghz-36000-km---geo}

\begin{verbatim}
f = 12 GHz
d = 36,000 km = 3.6×10 m

FSPL = 20 log(3.6×10) + 20 log(12×10) + 92.45
     = 151 + 201.6 + 92.45
     = 205 dB
\end{verbatim}

\textbf{Massive loss!} Requires high TX power + high-gain antennas.

\begin{center}\rule{0.5\linewidth}{0.5pt}\end{center}

\subsubsection{Example 4: THz Link (1 THz, 10 m) - For
{[}{[}AID-Protocol-Case-Study{]}{]}}\label{example-4-thz-link-1-thz-10-m---for-aid-protocol-case-study}

\begin{verbatim}
f = 1 THz = 1×10¹² Hz
d = 10 m

FSPL = 20 log(10) + 20 log(1×10¹²) + 92.45
     = 20 + 240 + 92.45
     = 352.5 dB
\end{verbatim}

\textbf{Extreme loss!} This is why
{[}{[}Terahertz-(THz)-Technology\textbar THz communications{]}{]} are
short-range only.

\begin{center}\rule{0.5\linewidth}{0.5pt}\end{center}

\subsection{\texorpdfstring{ Scaling
Laws}{ Scaling Laws}}\label{scaling-laws}

\subsubsection{Distance Dependence}\label{distance-dependence}

\begin{verbatim}
FSPL  d²  (power law)

In dB: FSPL_dB increases by 20 dB per decade of distance

Examples:
- 1 m  10 m: +20 dB loss
- 10 m  100 m: +20 dB loss
- 100 m  1 km: +20 dB loss
\end{verbatim}

\textbf{Doubling distance}: +6 dB loss (power drops to 1/4)

\begin{center}\rule{0.5\linewidth}{0.5pt}\end{center}

\subsubsection{Frequency Dependence}\label{frequency-dependence}

\begin{verbatim}
FSPL  f²  (power law)

In dB: FSPL_dB increases by 20 dB per decade of frequency

Examples:
- 100 MHz  1 GHz: +20 dB loss
- 1 GHz  10 GHz: +20 dB loss
- 10 GHz  100 GHz: +20 dB loss
\end{verbatim}

\textbf{Doubling frequency}: +6 dB loss (higher frequencies lose more
power!)

\textbf{Why?} Effective aperture of receiving antenna
\$\textbackslash propto\$
\$\textbackslash lambda\$\textbackslash textsuperscript\{2\} (smaller at
higher f)

\begin{center}\rule{0.5\linewidth}{0.5pt}\end{center}

\subsection{\texorpdfstring{ Physical
Interpretation}{ Physical Interpretation}}\label{physical-interpretation}

\subsubsection{Not True ``Loss''}\label{not-true-loss}

FSPL is \textbf{NOT} energy dissipation (free space is lossless!).
It\textquotesingle s \textbf{geometric spreading}:

\begin{verbatim}
Transmit antenna radiates P_T into sphere
Surface area: A = 4d²
Power density at distance d:

S = P_T/(4d²)  (W/m²)

Received power:
P_R = S · A_eff

where A_eff = G_R ²/4 (effective area of RX antenna)

Result:
P_R = P_T · G_T · G_R · (/4d)²  (Friis equation!)
\end{verbatim}

\textbf{Analogy}: Flashlight beam spreads out
\$\textbackslash rightarrow\$ same total power, but lower intensity at
greater distance.

\begin{center}\rule{0.5\linewidth}{0.5pt}\end{center}

\subsection{\texorpdfstring{ Link Budget
Analysis}{ Link Budget Analysis}}\label{link-budget-analysis}

\textbf{Link Budget} accounts for all gains and losses:

\begin{verbatim}
P_R [dBm] = P_T [dBm] + G_T [dBi] + G_R [dBi] - FSPL [dB] - L_other [dB]

where:
- P_T = transmit power (dBm, referenced to 1 mW)
- G_T, G_R = antenna gains (dBi, referenced to isotropic)
- FSPL = free-space path loss (dB)
- L_other = other losses (cables, connectors, atmosphere, etc.)
\end{verbatim}

\textbf{Goal}: Ensure P\_R \textgreater\textgreater{} P\_noise (receiver
noise floor) for reliable communication.

\begin{center}\rule{0.5\linewidth}{0.5pt}\end{center}

\subsubsection{Example: WiFi Link Budget (2.4 GHz, 50
m)}\label{example-wifi-link-budget-2.4-ghz-50-m}

\begin{verbatim}
Transmitter:
- TX power: +20 dBm (100 mW, typical WiFi)
- TX antenna gain: +2 dBi (dipole)
- EIRP: 22 dBm

Channel:
- Distance: 50 m
- FSPL: 20log(50) + 20log(2400) + 32.45 = 34 + 67.6 + 32.45 = 134 dB
- Indoor losses (walls, furniture): ~15 dB
- Total loss: 149 dB

Receiver:
- RX antenna gain: +2 dBi
- Cable loss: -1 dB
- Net RX gain: +1 dB

Received power:
P_R = 22 + 1 - 149 = -126 dBm

Noise floor (10 MHz bandwidth, 300K):
N = -174 + 10log(10^7) = -174 + 70 = -104 dBm

SNR = P_R - N = -126 - (-104) = -22 dB
\end{verbatim}

\textbf{Too low!} WiFi needs \textasciitilde-65 dBm minimum. This link
would fail.

\textbf{Solution}: Reduce distance, add amplifiers, or use directional
antennas.

\begin{center}\rule{0.5\linewidth}{0.5pt}\end{center}

\subsection{\texorpdfstring{ Real-World
Deviations}{ Real-World Deviations}}\label{real-world-deviations}

\subsubsection{FSPL Assumes Free Space}\label{fspl-assumes-free-space}

\textbf{Reality}: - Atmosphere absorbs (especially water vapor at
mmWave/THz) - Obstacles block (buildings, trees, terrain) - Ground
reflections create multipath - Weather attenuates (rain, fog)

\textbf{Actual path loss} \textgreater{} FSPL

\begin{center}\rule{0.5\linewidth}{0.5pt}\end{center}

\subsubsection{Frequency-Specific
Effects}\label{frequency-specific-effects}

\textbf{Low Frequencies (\textless{} 30 MHz)}: - Ground wave propagation
- Ionospheric reflection - \textbf{Can exceed FSPL predictions} (longer
range!)

\textbf{Mid Frequencies (30 MHz - 3 GHz)}: - Mostly line-of-sight (LOS)
- FSPL + diffraction - Close to FSPL predictions

\textbf{High Frequencies (\textgreater{} 3 GHz)}: - Atmospheric
absorption becomes significant - Rain fade (especially \textgreater{} 10
GHz) - \textbf{Path loss \textgreater{} FSPL}

\textbf{THz (\textgreater{} 300 GHz)}: - Extreme atmospheric absorption
- Water vapor resonances - \textbf{Path loss \textgreater\textgreater{}
FSPL} (can be +100 dB extra!)

\begin{center}\rule{0.5\linewidth}{0.5pt}\end{center}

\subsection{\texorpdfstring{ Measurement
vs.~Prediction}{ Measurement vs.~Prediction}}\label{measurement-vs.-prediction}

\subsubsection{Received Power
Measurement}\label{received-power-measurement}

\begin{verbatim}
Measured:  P_R,meas
Predicted: P_R,pred (from Friis equation)

Path loss exponent n:
P_R  d^(-n)

Free space: n = 2
Urban: n = 3-4
Indoor: n = 4-6
\end{verbatim}

\textbf{Empirical models} (e.g., Okumura-Hata, COST 231) fit measured
data to more complex formulas.

\begin{center}\rule{0.5\linewidth}{0.5pt}\end{center}

\subsection{\texorpdfstring{ Key
Takeaways}{ Key Takeaways}}\label{key-takeaways}

\begin{enumerate}
\def\labelenumi{\arabic{enumi}.}
\tightlist
\item
  \textbf{FSPL \$\textbackslash propto\$
  d\textbackslash textsuperscript\{2\} \$\textbackslash cdot\$
  f\textbackslash textsuperscript\{2\}}: Geometric spreading, worse at
  high frequencies
\item
  \textbf{20 dB per decade}: Doubling d or f adds 6 dB loss
\item
  \textbf{Not energy loss}: Power spreads out, doesn\textquotesingle t
  vanish
\item
  \textbf{Baseline for link budgets}: Real losses are usually higher
\item
  \textbf{Frequency trade-off}: Higher f \$\textbackslash rightarrow\$
  more bandwidth but more path loss
\item
  \textbf{THz communications}: FSPL alone is \textasciitilde350 dB at 10
  m, 1 THz!
\end{enumerate}

\begin{center}\rule{0.5\linewidth}{0.5pt}\end{center}

\subsection{\texorpdfstring{ See Also}{ See Also}}\label{see-also}

\begin{itemize}
\tightlist
\item
  {[}{[}Maxwell\textquotesingle s-Equations-\&-Wave-Propagation{]}{]} -
  Theoretical foundation
\item
  {[}{[}Antenna-Theory-Basics{]}{]} - Antenna gain (G\_T, G\_R)
\item
  {[}{[}Link-Loss-vs-Noise{]}{]} - FSPL vs additive noise
\item
  {[}{[}Atmospheric Effects{]}{]} - Additional losses beyond FSPL
  \emph{(coming soon)}
\item
  {[}{[}Multipath-Propagation-\&-Fading-(Rayleigh,-Rician){]}{]} -
  Deviations from FSPL \emph{(coming soon)}
\item
  {[}{[}Terahertz-(THz)-Technology{]}{]} - Extreme FSPL regime
\end{itemize}

\begin{center}\rule{0.5\linewidth}{0.5pt}\end{center}

\subsection{\texorpdfstring{ References}{ References}}\label{references}

\begin{enumerate}
\def\labelenumi{\arabic{enumi}.}
\tightlist
\item
  \textbf{Friis, H.T.} (1946) ``A note on a simple transmission
  formula'' \emph{Proc. IRE} 34, 254-256
\item
  \textbf{Rappaport, T.S.} (2002) \emph{Wireless Communications:
  Principles and Practice} 2nd ed.~(Prentice Hall)
\item
  \textbf{Goldsmith, A.} (2005) \emph{Wireless Communications}
  (Cambridge UP)
\item
  \textbf{ITU-R P.525} (2019) ``Calculation of free-space attenuation''
\end{enumerate}
