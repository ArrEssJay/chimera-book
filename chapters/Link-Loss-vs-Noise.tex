\section{Link Loss vs Noise}\label{link-loss-vs-noise}

\subsection{\texorpdfstring{ For Non-Technical
Readers}{ For Non-Technical Readers}}\label{for-non-technical-readers}

\textbf{Link loss vs noise is like the difference between someone
whispering (weak signal) vs shouting in a loud room (noise
interference)-\/-\/-two different problems!}

\textbf{Link Loss} - Signal gets weaker: - \textbf{What it is}: Your
WiFi router is far away, so signal is weak by the time it reaches you -
\textbf{Analogy}: Shouting across a football field-\/-\/-your voice
spreads out and gets quieter - \textbf{Predictable}: Same distance =
same loss every time - \textbf{Solution}: Move closer, use bigger
antenna, increase transmit power

\textbf{Examples of link loss}: - \textbf{WiFi}: 50 meters away =
10,000\$\textbackslash times\$ weaker signal - \textbf{Cell phone}: Far
from tower = fewer bars - \textbf{Satellite}: Space is far! Signal
arrives incredibly weak

\textbf{Noise} - Random interference added: - \textbf{What it is}:
Random electrical static from electronics, thermal energy, cosmic rays -
\textbf{Analogy}: Trying to hear someone whisper in a noisy
restaurant-\/-\/-extra sound covers the signal - \textbf{Random}:
Unpredictable, changes moment-to-moment - \textbf{Solution}:
Can\textquotesingle t remove it! Must send stronger signal or use error
correction

\textbf{Examples of noise}: - \textbf{Bluetooth stuttering near
microwave}: Microwave adds noise - \textbf{AM radio crackle}:
Thunderstorms add noise - \textbf{TV static}: No signal?
You\textquotesingle re seeing pure noise!

\textbf{Key difference}: - \textbf{Link loss}: Makes signal weaker
(deterministic, predictable) - \textbf{Noise}: Adds random garbage on
top (random, unpredictable) - \textbf{Both hurt you}: Weak signal (loss)
covered by noise = errors!

\textbf{The engineering ratio: SNR (Signal-to-Noise Ratio)} - Strong
signal + low noise = high SNR = perfect communication - Weak signal
(loss) + high noise = low SNR = errors everywhere

\textbf{When you see it}: Your phone shows ``5 bars'' (link loss is low)
but internet is slow (noise is high from interference).

\begin{center}\rule{0.5\linewidth}{0.5pt}\end{center}

In a real communication system, the received signal is degraded by
\textbf{two distinct mechanisms}: link loss and additive noise.

\subsection{Link Loss (Path Loss)}\label{link-loss-path-loss}

\textbf{Link Loss} represents the reduction in signal power as it
travels from transmitter to receiver.

\subsubsection{Characteristics}\label{characteristics}

\begin{itemize}
\tightlist
\item
  \textbf{Deterministic}: Same loss every time (for a given scenario)
\item
  \textbf{Multiplicative}: Scales the entire signal uniformly
\item
  \textbf{Predictable}: Can be calculated from path distance, frequency,
  antenna gains
\end{itemize}

\subsubsection{Sources}\label{sources}

\begin{itemize}
\tightlist
\item
  Free-space path loss
\item
  Antenna gains (TX and RX)
\item
  Cable losses
\item
  Atmospheric absorption
\item
  Rain attenuation
\end{itemize}

\subsubsection{Mathematical Model}\label{mathematical-model}

\begin{verbatim}
P_received = P_transmitted / Loss_Factor

In dB: P_received (dBm) = P_transmitted (dBm) - Link_Loss (dB)
\end{verbatim}

\subsubsection{Example Link Budget}\label{example-link-budget}

\begin{verbatim}
Transmit Power:        +30 dBm
Antenna Gain (TX):     +10 dB
Free-Space Loss:       -120 dB
Antenna Gain (RX):     +5 dB
Cable/Implementation:  -5 dB
-----------------------------
Received Signal Power: -80 dBm
Total Link Loss:       100 dB
\end{verbatim}

\subsection{Additive Noise}\label{additive-noise}

\textbf{Noise} adds random fluctuations on top of the received signal.

\subsubsection{Characteristics}\label{characteristics-1}

\begin{itemize}
\tightlist
\item
  \textbf{Random}: Different every time, unpredictable
\item
  \textbf{Additive}: Added to the signal (not multiplicative)
\item
  \textbf{Stochastic}: Described by statistical properties (power,
  distribution)
\end{itemize}

\subsubsection{Sources}\label{sources-1}

\begin{itemize}
\tightlist
\item
  Thermal noise (kTB)
\item
  Amplifier noise
\item
  Cosmic background
\item
  Interference from other signals
\end{itemize}

\subsubsection{Mathematical Model}\label{mathematical-model-1}

\begin{verbatim}
Received Signal = (Transmitted Signal / Link_Loss) + Noise

where Noise has power determined by SNR or N
\end{verbatim}

\subsection{Combined Channel Model}\label{combined-channel-model}

In Chimera\textquotesingle s simulation, both effects are applied:

\begin{verbatim}
1. Transmit Signal (Power = P_tx)
        
2. Apply Link Loss (Power reduced to P_tx / 10^(Loss_dB/10))
        
3. Add AWGN (Noise power = Attenuated_Signal_Power / SNR)
        
4. Received Signal (Attenuated + Noisy)
\end{verbatim}

\subsection{Why Both Matter}\label{why-both-matter}

\subsubsection{Link Loss Affects Signal
Power}\label{link-loss-affects-signal-power}

\begin{itemize}
\tightlist
\item
  High link loss (100+ dB) is typical in many systems
\item
  Reduces signal level but doesn\textquotesingle t add randomness
\item
  Can be compensated with amplification (but amplifies noise too!)
\end{itemize}

\subsubsection{Noise Affects Signal Quality
(SNR)}\label{noise-affects-signal-quality-snr}

\begin{itemize}
\tightlist
\item
  Adds random errors that can\textquotesingle t be predicted
\item
  Sets the fundamental limit on achievable BER
\item
  Can be improved with processing gain, error correction
\end{itemize}

\subsection{Link Budget and SNR}\label{link-budget-and-snr}

The combination determines receiver performance:

\begin{verbatim}
Received Signal Power = P_tx - Link_Loss_dB
Noise Power = N × Bandwidth

SNR (dB) = Received_Signal_Power (dBm) - Noise_Power (dBm)
\end{verbatim}

\subsubsection{Example 1: Good Link}\label{example-1-good-link}

\begin{itemize}
\tightlist
\item
  Transmit power: +30 dBm
\item
  Link loss: 100 dB
\item
  Received signal: \textbf{-70 dBm}
\item
  Noise floor: -90 dBm
\item
  \textbf{Resulting SNR: 20 dB} Good!
\end{itemize}

\subsubsection{Example 2: Challenging
Link}\label{example-2-challenging-link}

\begin{itemize}
\tightlist
\item
  Transmit power: +30 dBm
\item
  Link loss: 120 dB
\item
  Received signal: \textbf{-90 dBm}
\item
  Noise floor: -90 dBm
\item
  \textbf{Resulting SNR: 0 dB} Very challenging!
\end{itemize}

\subsection{Link Loss in Chimera}\label{link-loss-in-chimera}

Chimera allows you to model link loss separately from SNR: -
\textbf{Link Loss}: Simulates the signal power reduction (path loss,
antenna gains, etc.) - \textbf{SNR}: Controls the additive noise level -
Both combine to determine the received signal quality - This separation
helps understand link budget analysis

With \textbf{0 dB link loss}, the SNR setting directly determines signal
quality.

With \textbf{100 dB link loss}, the signal is attenuated by
10\textbackslash textsuperscript\{1\}\textbackslash textsuperscript\{0\},
but the SNR still controls the noise-to-signal ratio at the receiver
input.

\subsection{Key Insight}\label{key-insight}

\begin{verbatim}
More TX Power  Overcomes Link Loss  Higher RX Power  Better SNR
                                                              
                                                      Lower BER 
\end{verbatim}

But there are practical limits: - Transmitter power constraints
(battery, regulations) - Receiver sensitivity (noise floor) - Cost and
complexity

\subsection{See Also}\label{see-also}

\begin{itemize}
\tightlist
\item
  {[}{[}Signal-to-Noise-Ratio-(SNR){]}{]} - The quality metric
\item
  {[}{[}Additive-White-Gaussian-Noise-(AWGN){]}{]} - The noise model
\item
  {[}{[}Complete-Link-Budget-Analysis{]}{]} - Calculating system
  performance
\item
  {[}{[}Energy-Ratios-(Es-N0-and-Eb-N0){]}{]} - Energy-based metrics
\end{itemize}
