\section{Maxwell\textquotesingle s Equations \& Wave
Propagation}\label{maxwells-equations-wave-propagation}

\textbf{Maxwell\textquotesingle s Equations} are the fundamental laws of
electromagnetism, describing how electric and magnetic fields interact
and propagate through space.

\begin{center}\rule{0.5\linewidth}{0.5pt}\end{center}

\subsection{\texorpdfstring{ For Non-Technical
Readers}{ For Non-Technical Readers}}\label{for-non-technical-readers}

\textbf{What are Maxwell\textquotesingle s Equations?}

Imagine you\textquotesingle re trying to understand how your phone
communicates with a cell tower, how light travels from the sun to Earth,
or how a radio picks up music from thin air. All of these phenomena are
explained by four elegant mathematical rules discovered by James Clerk
Maxwell in the 1860s.

\textbf{The Big Picture (in plain English):}

\begin{enumerate}
\def\labelenumi{\arabic{enumi}.}
\tightlist
\item
  \textbf{Electric charges create invisible ``force fields''} around
  them

  \begin{itemize}
  \tightlist
  \item
    Think of static electricity making your hair stand up
  \item
    Positive and negative charges attract or repel each other through
    these fields
  \end{itemize}
\item
  \textbf{Magnetic fields always come in pairs} (north and south poles
  together)

  \begin{itemize}
  \tightlist
  \item
    You can\textquotesingle t have a magnet with just a north pole or
    just a south pole
  \item
    If you break a magnet in half, you get two smaller magnets, each
    with both poles
  \end{itemize}
\item
  \textbf{Changing magnetic fields create electric fields}

  \begin{itemize}
  \tightlist
  \item
    This is how generators work: spin a magnet near a wire, and
    electricity flows
  \item
    It\textquotesingle s also why transformers can change voltage levels
  \end{itemize}
\item
  \textbf{Moving electric charges (currents) and changing electric
  fields create magnetic fields}

  \begin{itemize}
  \tightlist
  \item
    This is how electromagnets work
  \item
    It\textquotesingle s also how antennas transmit radio waves
  \end{itemize}
\end{enumerate}

\textbf{Why does this matter?}

Maxwell discovered something profound: when you combine these four
rules, they predict that electromagnetic ``waves'' can travel through
empty space at a specific speed. When he calculated that speed, it
turned out to be exactly the speed of light!

This meant \textbf{light itself is an electromagnetic wave} - the same
type of wave as radio, WiFi, X-rays, and microwaves, just at different
frequencies.

\textbf{Real-world impact:} - Every wireless device (phone, WiFi,
Bluetooth, GPS) - All lighting and solar panels - Radio, TV, and
satellite communication - Medical imaging (MRI, X-rays) - Why the sky is
blue and sunsets are red - How your eyes see color

Without Maxwell\textquotesingle s Equations, the modern wireless world
wouldn\textquotesingle t exist!

\textbf{What you\textquotesingle ll find below:}

The rest of this page dives into the mathematical details.
Don\textquotesingle t worry if the equations look intimidating - the key
concepts above are what matter for understanding how electromagnetic
waves work in practice.

\begin{center}\rule{0.5\linewidth}{0.5pt}\end{center}

\subsection{\texorpdfstring{ The Four Maxwell\textquotesingle s
Equations}{ The Four Maxwell\textquotesingle s Equations}}\label{the-four-maxwells-equations}

\subsubsection{In Differential Form
(Local)}\label{in-differential-form-local}

\textbf{1. Gauss\textquotesingle s Law} (Electric charge creates
electric field)

\begin{verbatim}
·E = /

where:
- E = electric field vector (V/m)
-  = charge density (C/m³)
-  = permittivity of free space (8.854×10¹² F/m)
\end{verbatim}

\textbf{Physical meaning}: Electric field lines originate from positive
charges and terminate on negative charges.

\begin{center}\rule{0.5\linewidth}{0.5pt}\end{center}

\textbf{2. Gauss\textquotesingle s Law for Magnetism} (No magnetic
monopoles)

\begin{verbatim}
·B = 0

where:
- B = magnetic field vector (Tesla)
\end{verbatim}

\textbf{Physical meaning}: Magnetic field lines always form closed loops
(no isolated north/south poles).

\begin{center}\rule{0.5\linewidth}{0.5pt}\end{center}

\textbf{3. Faraday\textquotesingle s Law} (Changing magnetic field
creates electric field)

\begin{verbatim}
×E = -B/t

where:
- × = curl operator (measures rotation)
- B/t = time rate of change of B
\end{verbatim}

\textbf{Physical meaning}: A time-varying magnetic field induces a
circulating electric field (basis of generators, transformers).

\begin{center}\rule{0.5\linewidth}{0.5pt}\end{center}

\textbf{4. Ampère-Maxwell Law} (Current + changing electric field
creates magnetic field)

\begin{verbatim}
×B = J +  E/t

where:
-  = permeability of free space (4×10 H/m)
- J = current density (A/m²)
- E/t = displacement current (Maxwell's addition!)
\end{verbatim}

\textbf{Physical meaning}: Moving charges (current) AND time-varying
electric fields create circulating magnetic fields.

\textbf{Maxwell\textquotesingle s insight}: The
\$\textbackslash partial\$E/\$\textbackslash partial\$t term was missing
from Ampère\textquotesingle s original law. Adding it made
electromagnetic waves possible!

\begin{center}\rule{0.5\linewidth}{0.5pt}\end{center}

\subsection{\texorpdfstring{ The Wave
Equation}{ The Wave Equation}}\label{the-wave-equation}

\subsubsection{Derivation}\label{derivation}

Taking curl of Faraday\textquotesingle s law:

\begin{verbatim}
×(×E) = -×(B/t) = -(×B)/t
\end{verbatim}

Substitute Ampère-Maxwell law (in vacuum, J=0):

\begin{verbatim}
×(×E) = - ²E/t²
\end{verbatim}

Use vector identity:
\$\textbackslash nabla\$\$\textbackslash times\$(\$\textbackslash nabla\$\$\textbackslash times\$E)
=
\$\textbackslash nabla\$(\$\textbackslash nabla\$\$\textbackslash cdot\$E)
- \$\textbackslash nabla\$\textbackslash textsuperscript\{2\}E

In vacuum (\$\textbackslash rho\$=0), Gauss\textquotesingle s law gives
\$\textbackslash nabla\$\$\textbackslash cdot\$E = 0, so:

\begin{verbatim}
²E =  ²E/t²
\end{verbatim}

\textbf{This is the wave equation!}

Similar derivation for B gives:

\begin{verbatim}
²B =  ²E/t²
\end{verbatim}

\begin{center}\rule{0.5\linewidth}{0.5pt}\end{center}

\subsubsection{Wave Speed}\label{wave-speed}

The standard wave equation is:

\begin{verbatim}
²f = (1/v²) ²f/t²
\end{verbatim}

Comparing to electromagnetic wave equation:

\begin{verbatim}
v² = 1/()

v = 1/() 
  = 1/[(4×10)(8.854×10¹²)]
  = 2.998×10 m/s
  = c (speed of light!)
\end{verbatim}

\textbf{Maxwell\textquotesingle s triumph}: Light is an electromagnetic
wave!

\begin{center}\rule{0.5\linewidth}{0.5pt}\end{center}

\subsection{\texorpdfstring{ Plane Wave
Solutions}{ Plane Wave Solutions}}\label{plane-wave-solutions}

\subsubsection{General Solution}\label{general-solution}

For propagation in +z direction:

\begin{verbatim}
E(z,t) = E cos(kz - t + ) x
B(z,t) = B cos(kz - t + ) 

where:
- k = 2/ = wave number (rad/m)
-  = 2f = angular frequency (rad/s)
-  = wavelength (m)
- f = frequency (Hz)
-  = phase constant
\end{verbatim}

\textbf{Relationship between E and B}:

\begin{verbatim}
B = E/c

B = (1/c) k × E

where k is propagation direction
\end{verbatim}

\textbf{Key insight}: E and B are perpendicular to each other AND to
propagation direction (transverse wave).

\begin{center}\rule{0.5\linewidth}{0.5pt}\end{center}

\subsubsection{Dispersion Relation}\label{dispersion-relation}

From wave equation:

\begin{verbatim}
 = ck  (in vacuum)

or:  v = f  (wave speed = frequency × wavelength)
\end{verbatim}

\textbf{In vacuum}: All frequencies travel at same speed c
(non-dispersive)

\textbf{In matter}: v = c/n (where n = refractive index, depends on
frequency \$\textbackslash rightarrow\$ dispersion)

\begin{center}\rule{0.5\linewidth}{0.5pt}\end{center}

\subsection{\texorpdfstring{ Energy and
Power}{ Energy and Power}}\label{energy-and-power}

\subsubsection{Energy Density}\label{energy-density}

\textbf{Electric field energy density}:

\begin{verbatim}
u_E = (1/2)E²  (J/m³)
\end{verbatim}

\textbf{Magnetic field energy density}:

\begin{verbatim}
u_B = (1/2)B²  (J/m³)
\end{verbatim}

\textbf{Total electromagnetic energy density}:

\begin{verbatim}
u = u_E + u_B = E²  (since B = E/c and c² = 1/)
\end{verbatim}

\begin{center}\rule{0.5\linewidth}{0.5pt}\end{center}

\subsubsection{Poynting Vector (Power
Flow)}\label{poynting-vector-power-flow}

\textbf{Poynting vector} S points in direction of energy flow:

\begin{verbatim}
S = (1/) E × B  (W/m²)

Magnitude: |S| = (1/c) E²  (for plane wave)
\end{verbatim}

\textbf{Physical meaning}: Energy flux (power per unit area) carried by
EM wave.

\textbf{Power through area A}:

\begin{verbatim}
P =  S·dA  (Watts)
\end{verbatim}

\begin{center}\rule{0.5\linewidth}{0.5pt}\end{center}

\subsubsection{Intensity}\label{intensity}

For time-harmonic wave, \textbf{intensity} (time-averaged power
density):

\begin{verbatim}
I = <|S|> = (1/2c) E² = (c/2) E²

or in terms of B:
I = (c/2) B²
\end{verbatim}

\textbf{Units}: W/m\textbackslash textsuperscript\{2\} (same as
irradiance, power density)

\begin{center}\rule{0.5\linewidth}{0.5pt}\end{center}

\subsection{\texorpdfstring{ Radiation from
Sources}{ Radiation from Sources}}\label{radiation-from-sources}

\subsubsection{Dipole Radiation}\label{dipole-radiation}

\textbf{Oscillating electric dipole} (simplest antenna):

\begin{verbatim}
Radiated power:
P = (/12c)  p²

where:
-  = oscillation frequency
- p = dipole moment amplitude
\end{verbatim}

\textbf{Key insight}: Radiated power \$\textbackslash propto\$
\$\textbackslash omega\$\textbackslash textsuperscript\{4\} (higher
frequencies radiate much more efficiently!)

\textbf{Radiation pattern}: Doughnut shape (maximum perpendicular to
dipole, zero along dipole axis)

\begin{center}\rule{0.5\linewidth}{0.5pt}\end{center}

\subsubsection{Accelerating Charges}\label{accelerating-charges}

\textbf{Larmor formula} (non-relativistic):

\begin{verbatim}
P = (q²a²)/(6c)

where:
- q = charge
- a = acceleration
\end{verbatim}

\textbf{Physical meaning}: Any accelerating charge radiates EM waves.
This is basis of: - Antennas (oscillating current = accelerating
charges) - Synchrotron radiation (electrons in magnetic fields) -
Bremsstrahlung (decelerating electrons)

\begin{center}\rule{0.5\linewidth}{0.5pt}\end{center}

\subsection{\texorpdfstring{ Propagation in
Media}{ Propagation in Media}}\label{propagation-in-media}

\subsubsection{Material Properties}\label{material-properties}

\textbf{Permittivity} \$\textbackslash epsilon\$: How much material
opposes electric field - Vacuum:
\$\textbackslash epsilon\$\textbackslash textsubscript\{0\} - Material:
\$\textbackslash epsilon\$ = \$\textbackslash epsilon\$\_r
\$\textbackslash epsilon\$\textbackslash textsubscript\{0\} (where
\$\textbackslash epsilon\$\_r = relative permittivity)

\textbf{Permeability} \$\textbackslash mu\$: How much material opposes
magnetic field - Vacuum:
\$\textbackslash mu\$\textbackslash textsubscript\{0\} - Material:
\$\textbackslash mu\$ = \$\textbackslash mu\$\_r
\$\textbackslash mu\$\textbackslash textsubscript\{0\} (where
\$\textbackslash mu\$\_r = relative permeability)

\textbf{Conductivity} \$\textbackslash sigma\$: How well material
conducts current - Insulator: \$\textbackslash sigma\$
\$\textbackslash approx\$ 0 - Conductor: \$\textbackslash sigma\$
\$\textbackslash rightarrow\$ \$\textbackslash infty\$ (ideally)

\begin{center}\rule{0.5\linewidth}{0.5pt}\end{center}

\subsubsection{Wave Speed in Media}\label{wave-speed-in-media}

\begin{verbatim}
v = 1/() = c/(_r _r) = c/n

where n = (_r _r) is refractive index
\end{verbatim}

\textbf{Examples}: - Air: n \$\textbackslash approx\$ 1.0003 (v
\$\textbackslash approx\$ c) - Water: n \$\textbackslash approx\$ 1.33
(v \$\textbackslash approx\$ 0.75c) - Glass: n \$\textbackslash approx\$
1.5 (v \$\textbackslash approx\$ 0.67c)

\begin{center}\rule{0.5\linewidth}{0.5pt}\end{center}

\subsubsection{Attenuation in Lossy
Media}\label{attenuation-in-lossy-media}

In conductive medium, wave amplitude decays:

\begin{verbatim}
E(z) = E e^(-z) cos(kz - t)

where  = skin depth parameter:
 = (f)  (for good conductors)

Skin depth:  = 1/ (depth where amplitude drops to 1/e)
\end{verbatim}

\textbf{Examples} (at 1 GHz): - Copper: \$\textbackslash delta\$
\$\textbackslash approx\$ 2 \$\textbackslash mu\$m (EM waves
don\textquotesingle t penetrate conductors!) - Seawater:
\$\textbackslash delta\$ \$\textbackslash approx\$ 0.2 m (poor
penetration) - Air: \$\textbackslash delta\$
\$\textbackslash rightarrow\$ \$\textbackslash infty\$ (negligible loss)

\begin{center}\rule{0.5\linewidth}{0.5pt}\end{center}

\subsection{\texorpdfstring{ Frequency
Spectrum}{ Frequency Spectrum}}\label{frequency-spectrum}

Maxwell\textquotesingle s equations apply to \textbf{all frequencies}:

{\def\LTcaptype{} % do not increment counter
\begin{longtable}[]{@{}llll@{}}
\toprule\noalign{}
Band & Frequency & Wavelength & Applications \\
\midrule\noalign{}
\endhead
\bottomrule\noalign{}
\endlastfoot
\textbf{ELF} & 3-30 Hz & 10,000-100,000 km & Submarine communication \\
\textbf{VLF} & 3-30 kHz & 10-100 km & Navigation \\
\textbf{LF} & 30-300 kHz & 1-10 km & AM radio \\
\textbf{MF} & 300 kHz-3 MHz & 100-1000 m & AM broadcast \\
\textbf{HF} & 3-30 MHz & 10-100 m & Shortwave \\
\textbf{VHF} & 30-300 MHz & 1-10 m & FM radio, TV \\
\textbf{UHF} & 300 MHz-3 GHz & 10 cm-1 m & Cell phones, WiFi \\
\textbf{SHF} & 3-30 GHz & 1-10 cm & Radar, satellite \\
\textbf{EHF} & 30-300 GHz & 1-10 mm & mmWave, 5G \\
\textbf{THz} & 0.3-3 THz & 0.1-1 mm & Imaging, spectroscopy \\
\textbf{IR} & 300 THz-430 THz & 700 nm-1 mm & Thermal imaging \\
\textbf{Visible} & 430-750 THz & 400-700 nm & Human vision \\
\textbf{UV} & 750 THz-30 PHz & 10-400 nm & Sterilization \\
\textbf{X-ray} & 30 PHz-30 EHz & 0.01-10 nm & Medical imaging \\
\textbf{Gamma} & \textgreater{} 30 EHz & \textless{} 0.01 nm & Nuclear
medicine \\
\end{longtable}
}

\textbf{All obey Maxwell\textquotesingle s equations!} (though quantum
effects important at high frequencies)

\begin{center}\rule{0.5\linewidth}{0.5pt}\end{center}

\subsection{\texorpdfstring{ Key
Insights}{ Key Insights}}\label{key-insights}

\begin{enumerate}
\def\labelenumi{\arabic{enumi}.}
\tightlist
\item
  \textbf{Unification}: Electricity, magnetism, and light are different
  manifestations of the same phenomenon
\item
  \textbf{Self-propagation}: EM waves don\textquotesingle t need a
  medium (unlike sound)
\item
  \textbf{Speed limit}: c is the maximum speed in universe (relativity!)
\item
  \textbf{Transverse}: E, B, and propagation direction are mutually
  perpendicular
\item
  \textbf{Duality}: E and B are inseparable (changing one creates the
  other)
\item
  \textbf{Scale invariance}: Same equations for radio
  \$\textbackslash rightarrow\$ gamma rays (though quantum effects
  matter at high f)
\end{enumerate}

\begin{center}\rule{0.5\linewidth}{0.5pt}\end{center}

\subsection{\texorpdfstring{ See Also}{ See Also}}\label{see-also}

\begin{itemize}
\tightlist
\item
  {[}{[}Electromagnetic-Spectrum{]}{]} - Detailed frequency breakdown
\item
  {[}{[}Antenna-Theory-Basics{]}{]} - How to radiate/receive EM waves
\item
  {[}{[}Wave-Polarization{]}{]} - E field orientation
\item
  {[}{[}Free-Space-Path-Loss-(FSPL){]}{]} - How waves weaken with
  distance
\item
  {[}{[}Terahertz-(THz)-Technology{]}{]} - Specific THz band
  applications
\end{itemize}

\begin{center}\rule{0.5\linewidth}{0.5pt}\end{center}

\subsection{\texorpdfstring{ References}{ References}}\label{references}

\begin{enumerate}
\def\labelenumi{\arabic{enumi}.}
\tightlist
\item
  \textbf{Maxwell, J.C.} (1865) ``A Dynamical Theory of the
  Electromagnetic Field'' \emph{Phil. Trans. R. Soc.} 155, 459-512
\item
  \textbf{Jackson, J.D.} (1999) \emph{Classical Electrodynamics} 3rd
  ed.~(Wiley)
\item
  \textbf{Griffiths, D.J.} (2017) \emph{Introduction to Electrodynamics}
  4th ed.~(Cambridge UP)
\item
  \textbf{Feynman, R.P., Leighton, R.B., Sands, M.} (1964) \emph{The
  Feynman Lectures on Physics} Vol. 2 (Addison-Wesley)
\end{enumerate}
