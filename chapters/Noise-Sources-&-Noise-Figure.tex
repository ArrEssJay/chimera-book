\section{Noise Sources \& Noise
Figure}\label{noise-sources-noise-figure}

{[}{[}Home{]}{]} \textbar{} \textbf{Link Budget \& System Performance}
\textbar{} {[}{[}Signal-to-Noise-Ratio-(SNR){]}{]} \textbar{}
{[}{[}Complete-Link-Budget-Analysis{]}{]}

\begin{center}\rule{0.5\linewidth}{0.5pt}\end{center}

\subsection{\texorpdfstring{ For Non-Technical
Readers}{ For Non-Technical Readers}}\label{for-non-technical-readers}

\textbf{Think of radio communication like having a conversation in a
crowded restaurant:}

\begin{itemize}
\tightlist
\item
  \textbf{Signal} = Your friend\textquotesingle s voice trying to reach
  you
\item
  \textbf{Noise} = All the background chatter, kitchen sounds, and air
  conditioning
\item
  \textbf{Noise Figure} = How much your hearing aids (or bad acoustics)
  make it harder to understand
\end{itemize}

\textbf{Why noise matters:} If the background noise is too loud, you
can\textquotesingle t hear your friend-\/-\/-even if
they\textquotesingle re shouting. Same with radio: if noise is too high,
the receiver can\textquotesingle t ``hear'' the signal, no matter how
powerful the transmitter.

\textbf{Key insights in plain English:}

\begin{enumerate}
\def\labelenumi{\arabic{enumi}.}
\item
  \textbf{Thermal noise is everywhere}: Just like atoms vibrating
  creates heat, electrons vibrating creates random electrical ``static''
  in every wire, antenna, and amplifier. This sets a fundamental
  limit-\/-\/-you can\textquotesingle t eliminate it, only work around
  it.
\item
  \textbf{The -174 dBm magic number}: This is the ``noise floor'' at
  room temperature for a 1 Hz bandwidth. Think of it as the quietest
  possible ``background hum'' in radio. Everything adds noise on top of
  this baseline.
\item
  \textbf{Amplifiers make noise worse}: Every amplifier adds its own
  noise (like a hearing aid with poor quality that adds hiss). The
  \textbf{noise figure} tells you how much worse the amplifier makes the
  signal-to-noise ratio.
\item
  \textbf{First stage is critical}: Just like you want your hearing aid
  right at your ear (not connected by a long cable), you want the first
  amplifier (Low-Noise Amplifier, or LNA) as close to the antenna as
  possible. Once noise is added early, you can\textquotesingle t remove
  it later.
\item
  \textbf{Wider bandwidth = more noise}: Like opening more windows lets
  in more outside noise, using a wider radio bandwidth lets in more
  thermal noise. This is why high-speed data links (wide bandwidth) need
  stronger signals than voice links (narrow bandwidth).
\end{enumerate}

\textbf{Real-world impact:} - \textbf{Satellite TV}: Premium receivers
have better (lower) noise figures, letting them work with smaller dishes
- \textbf{GPS}: Your phone\textquotesingle s GPS can detect signals
1,000\$\textbackslash times\$ weaker than WiFi because it fights noise
cleverly (spread spectrum) - \textbf{Deep space missions}: NASA uses
cryogenically-cooled amplifiers (like refrigerating your hearing aid!)
to reduce noise and hear probes billions of miles away

\textbf{Bottom line}: If you want to receive weak signals (long range,
small antenna, low power), you must minimize noise.
That\textquotesingle s why the first few inches of cable and the first
amplifier matter more than anything else in the receiver chain.

\begin{center}\rule{0.5\linewidth}{0.5pt}\end{center}

\subsection{Overview}\label{overview}

\textbf{Noise} is unwanted random signal that degrades communication
system performance.

\textbf{Key metrics}: - \textbf{Noise power} (N): Total noise at
receiver input (dBm, watts) - \textbf{Noise figure} (NF): How much a
component degrades SNR (dB) - \textbf{Noise temperature} (T\_e):
Equivalent thermal noise (Kelvin)

\textbf{Why it matters}: - Determines \textbf{receiver sensitivity}
(minimum detectable signal) - Sets \textbf{SNR} at demodulator input -
Dominates link performance in low-signal scenarios (satellite, deep
space)

\textbf{Bottom line}: Lower noise = Better sensitivity = Longer range

\begin{center}\rule{0.5\linewidth}{0.5pt}\end{center}

\subsection{Thermal Noise}\label{thermal-noise}

\textbf{Fundamental noise source}: Random motion of charge carriers due
to thermal agitation

\subsubsection{Johnson-Nyquist Noise}\label{johnson-nyquist-noise}

\textbf{Noise power} in resistor at temperature T:

\[
N = k T B \quad (\text{watts})
\]

Where: - \(k = 1.38 \times 10^{-23}\) J/K (Boltzmann constant) - \(T\) =
Absolute temperature (Kelvin) - \(B\) = Bandwidth (Hz)

\textbf{Standard reference}: \(T_0 = 290\) K (room temperature,
\textasciitilde17\$\^{}\textbackslash circ\$C)

\begin{center}\rule{0.5\linewidth}{0.5pt}\end{center}

\subsubsection{Noise Power Spectral
Density}\label{noise-power-spectral-density}

\textbf{Noise power per Hz}:

\[
N_0 = k T \quad (\text{W/Hz})
\]

\textbf{At 290 K}:

\[
N_0 = 1.38 \times 10^{-23} \times 290 = 4 \times 10^{-21}\ \text{W/Hz}
\]

\textbf{In dBm/Hz}:

\[
N_0 = 10\log_{10}\left(\frac{4 \times 10^{-21}}{10^{-3}}\right) = -174\ \text{dBm/Hz}
\]

\textbf{This is the famous -174 dBm/Hz thermal noise floor!}

\begin{center}\rule{0.5\linewidth}{0.5pt}\end{center}

\subsubsection{Noise Power in Bandwidth
B}\label{noise-power-in-bandwidth-b}

\[
N = N_0 \times B = kTB \quad (\text{watts})
\]

\textbf{In dB}:

\[
N_{\text{dBm}} = -174 + 10\log_{10}(B) \quad (\text{dBm})
\]

\textbf{Example}: 1 MHz bandwidth @ 290 K

\[
N = -174 + 10\log_{10}(10^6) = -174 + 60 = -114\ \text{dBm}
\]

\begin{center}\rule{0.5\linewidth}{0.5pt}\end{center}

\subsubsection{Typical Bandwidths and Noise
Power}\label{typical-bandwidths-and-noise-power}

{\def\LTcaptype{} % do not increment counter
\begin{longtable}[]{@{}lll@{}}
\toprule\noalign{}
System & Bandwidth & Noise Power @ 290 K \\
\midrule\noalign{}
\endhead
\bottomrule\noalign{}
\endlastfoot
\textbf{GPS C/A code} & 2 MHz & -111 dBm \\
\textbf{WiFi 20 MHz} & 20 MHz & -101 dBm \\
\textbf{LTE 10 MHz} & 10 MHz & -104 dBm \\
\textbf{DVB-S2 36 MHz} & 36 MHz & -98.4 dBm \\
\textbf{Radar (1 GHz pulse)} & 1 GHz & -84 dBm \\
\end{longtable}
}

\textbf{Key insight}: Wider bandwidth = More noise power

\begin{center}\rule{0.5\linewidth}{0.5pt}\end{center}

\subsection{Noise Figure (NF)}\label{noise-figure-nf}

\textbf{Definition}: \textbf{Degradation of SNR} through a component or
system

\[
\text{NF} = \frac{\text{SNR}_{\text{in}}}{\text{SNR}_{\text{out}}} \quad (\text{linear ratio})
\]

\textbf{In dB}:

\[
\text{NF}_{\text{dB}} = 10\log_{10}(\text{NF}) = \text{SNR}_{\text{in,dB}} - \text{SNR}_{\text{out,dB}}
\]

\textbf{Interpretation}: - \textbf{NF = 1 (0 dB)}: Ideal (no noise
added) - \textbf{NF = 2 (3 dB)}: SNR halved (doubles noise power) -
\textbf{NF = 10 (10 dB)}: SNR reduced by 10\$\textbackslash times\$
(10\$\textbackslash times\$ noise power)

\begin{center}\rule{0.5\linewidth}{0.5pt}\end{center}

\subsubsection{Noise Figure vs Noise
Factor}\label{noise-figure-vs-noise-factor}

\textbf{Noise factor} (F): Linear ratio

\textbf{Noise figure} (NF): Logarithmic (dB)

\[
\text{NF}_{\text{dB}} = 10\log_{10}(F)
\]

\textbf{Example}: F = 2 \$\textbackslash rightarrow\$ NF = 3 dB

\begin{center}\rule{0.5\linewidth}{0.5pt}\end{center}

\subsubsection{Typical Noise Figures}\label{typical-noise-figures}

{\def\LTcaptype{} % do not increment counter
\begin{longtable}[]{@{}lll@{}}
\toprule\noalign{}
Component & Noise Figure (dB) & Notes \\
\midrule\noalign{}
\endhead
\bottomrule\noalign{}
\endlastfoot
\textbf{Passive cable} & Loss in dB & Lossy line: NF = loss \\
\textbf{Ideal amplifier} & 0 & Theoretical only \\
\textbf{Cryogenic LNA} & 0.3-0.8 & Cooled to 20-80 K \\
\textbf{Premium LNA} & 0.8-1.5 & GaAs HEMT, room temp \\
\textbf{Good LNA} & 1.5-3 & Typical satellite ground \\
\textbf{WiFi/cellular front-end} & 5-9 & Consumer devices \\
\textbf{Mixer (passive)} & 6-10 & Diode mixer \\
\textbf{Mixer (active)} & 10-15 & Gilbert cell \\
\end{longtable}
}

\begin{center}\rule{0.5\linewidth}{0.5pt}\end{center}

\subsection{Noise Temperature}\label{noise-temperature}

\textbf{Alternative to noise figure}: Equivalent input noise temperature

\[
T_e = T_0 (F - 1) \quad (\text{K})
\]

Where \(T_0 = 290\) K (reference)

\textbf{Relationship}:

\[
F = 1 + \frac{T_e}{T_0}
\]

\[
\text{NF}_{\text{dB}} = 10\log_{10}\left(1 + \frac{T_e}{290}\right)
\]

\begin{center}\rule{0.5\linewidth}{0.5pt}\end{center}

\subsubsection{Noise Figure \$\textbackslash leftrightarrow\$ Noise
Temperature}\label{noise-figure-noise-temperature}

{\def\LTcaptype{} % do not increment counter
\begin{longtable}[]{@{}lll@{}}
\toprule\noalign{}
NF (dB) & Noise Factor F & \(T_e\) (K) \\
\midrule\noalign{}
\endhead
\bottomrule\noalign{}
\endlastfoot
0 & 1 & 0 \\
0.5 & 1.12 & 35 \\
1 & 1.26 & 75 \\
2 & 1.58 & 169 \\
3 & 2 & 290 \\
6 & 4 & 870 \\
10 & 10 & 2610 \\
\end{longtable}
}

\textbf{Usage}: Satellite/radio astronomy communities prefer \(T_e\), RF
engineers prefer NF

\begin{center}\rule{0.5\linewidth}{0.5pt}\end{center}

\subsection{Cascaded Noise Figure (Friis
Formula)}\label{cascaded-noise-figure-friis-formula}

\textbf{Multi-stage system}: Amplifiers, mixers, filters in series

\textbf{Total noise factor}:

\[
F_{\text{total}} = F_1 + \frac{F_2 - 1}{G_1} + \frac{F_3 - 1}{G_1 G_2} + \frac{F_4 - 1}{G_1 G_2 G_3} + \ldots
\]

Where: - \(F_i\) = Noise factor of stage \(i\) (linear) - \(G_i\) = Gain
of stage \(i\) (linear)

\textbf{In dB}:

\[
\text{NF}_{\text{total}} = 10\log_{10}(F_{\text{total}})
\]

\begin{center}\rule{0.5\linewidth}{0.5pt}\end{center}

\subsubsection{Key Insights from Friis
Formula}\label{key-insights-from-friis-formula}

\begin{enumerate}
\def\labelenumi{\arabic{enumi}.}
\tightlist
\item
  \textbf{First stage dominates}: \(F_1\) appears without division
  \$\textbackslash rightarrow\$ \textbf{LNA critical!}
\item
  \textbf{High gain helps}: Later stages divided by \(G_1 G_2 \ldots\)
  \$\textbackslash rightarrow\$ Less impact
\item
  \textbf{Avoid loss before LNA}: Cable loss before LNA directly adds to
  NF
\end{enumerate}

\begin{center}\rule{0.5\linewidth}{0.5pt}\end{center}

\subsubsection{Example 1: Simple Receiver
Chain}\label{example-1-simple-receiver-chain}

\textbf{Components}: 1. \textbf{Cable}: Loss 2 dB (NF = 2 dB, F = 1.58,
G = 0.63 = -2 dB) 2. \textbf{LNA}: NF = 1 dB, F = 1.26, G = 20 dB
(100\$\textbackslash times\$) 3. \textbf{Mixer}: NF = 10 dB, F = 10, G =
-6 dB (0.25\$\textbackslash times\$)

\textbf{Total NF}:

\[
F_{\text{total}} = 1.58 + \frac{1.26 - 1}{0.63} + \frac{10 - 1}{0.63 \times 100}
\]

\[
= 1.58 + 0.41 + 0.14 = 2.13
\]

\[
\text{NF}_{\text{total}} = 10\log_{10}(2.13) = 3.3\ \text{dB}
\]

\textbf{Dominated by cable loss!}

\begin{center}\rule{0.5\linewidth}{0.5pt}\end{center}

\subsubsection{Example 2: Cable After LNA (Best
Practice)}\label{example-2-cable-after-lna-best-practice}

\textbf{Components}: 1. \textbf{LNA}: NF = 1 dB, F = 1.26, G = 20 dB
(100\$\textbackslash times\$) 2. \textbf{Cable}: Loss 2 dB, F = 1.58, G
= 0.63 3. \textbf{Mixer}: NF = 10 dB, F = 10, G = -6 dB
(0.25\$\textbackslash times\$)

\textbf{Total NF}:

\[
F_{\text{total}} = 1.26 + \frac{1.58 - 1}{100} + \frac{10 - 1}{100 \times 0.63}
\]

\[
= 1.26 + 0.0058 + 0.14 = 1.41
\]

\[
\text{NF}_{\text{total}} = 10\log_{10}(1.41) = 1.5\ \text{dB}
\]

\textbf{Much better! LNA at antenna isolates from cable loss.}

\begin{center}\rule{0.5\linewidth}{0.5pt}\end{center}

\subsubsection{Example 3: Two-Stage LNA}\label{example-3-two-stage-lna}

\textbf{Components}: 1. \textbf{LNA1}: NF = 0.8 dB, F = 1.2, G = 15 dB
(31.6\$\textbackslash times\$) 2. \textbf{LNA2}: NF = 1.5 dB, F = 1.41,
G = 20 dB (100\$\textbackslash times\$) 3. \textbf{Mixer}: NF = 10 dB, F
= 10, G = -6 dB (0.25\$\textbackslash times\$)

\textbf{Total NF}:

\[
F_{\text{total}} = 1.2 + \frac{1.41 - 1}{31.6} + \frac{10 - 1}{31.6 \times 100}
\]

\[
= 1.2 + 0.013 + 0.0028 = 1.216
\]

\[
\text{NF}_{\text{total}} = 10\log_{10}(1.216) = 0.85\ \text{dB}
\]

\textbf{Excellent! High gain LNA1 suppresses later stages.}

\begin{center}\rule{0.5\linewidth}{0.5pt}\end{center}

\subsection{System Noise Temperature}\label{system-noise-temperature}

\textbf{Total noise temperature} of cascaded system:

\[
T_{\text{sys}} = T_{\text{ant}} + T_e
\]

Where: - \(T_{\text{ant}}\) = Antenna noise temperature (K) - \(T_e\) =
Receiver equivalent noise temperature (K)

\begin{center}\rule{0.5\linewidth}{0.5pt}\end{center}

\subsubsection{Antenna Noise
Temperature}\label{antenna-noise-temperature}

\textbf{Antenna picks up thermal radiation} from environment:

\textbf{Sources}: - \textbf{Sky}: 3-300 K (depends on frequency,
elevation) - \textbf{Ground}: 290 K (room temperature) - \textbf{Sun}:
\textasciitilde10,000 K (if pointed directly) - \textbf{Cosmic
background}: 2.7 K (everywhere)

\textbf{Typical values}:

{\def\LTcaptype{} % do not increment counter
\begin{longtable}[]{@{}llll@{}}
\toprule\noalign{}
Scenario & Frequency & Elevation & \(T_{\text{ant}}\) (K) \\
\midrule\noalign{}
\endhead
\bottomrule\noalign{}
\endlastfoot
\textbf{Deep space} & Any & - & 3-5 \\
\textbf{Satellite (clear sky)} & 1-10 GHz &
30\$\^{}\textbackslash circ\$ & 20-50 \\
\textbf{Satellite (rain)} & 12 GHz & 30\$\^{}\textbackslash circ\$ &
100-200 \\
\textbf{Terrestrial} & Any & Horizon & 290 \\
\end{longtable}
}

\begin{center}\rule{0.5\linewidth}{0.5pt}\end{center}

\subsubsection{G/T Ratio (Figure of
Merit)}\label{gt-ratio-figure-of-merit}

\textbf{System performance metric} for satellite ground stations:

\[
\frac{G}{T} = G_r - 10\log_{10}(T_{\text{sys}}) \quad (\text{dB/K})
\]

Where: - \(G_r\) = RX antenna gain (dBi) - \(T_{\text{sys}}\) = System
noise temperature (K)

\textbf{Interpretation}: Higher G/T = Better sensitivity

\textbf{Example}: 3 m Ku-band dish, LNA at feed - Antenna gain: 48 dBi -
Antenna temp: 30 K (clear sky, 30\$\^{}\textbackslash circ\$ elevation)
- LNA NF: 0.8 dB \$\textbackslash rightarrow\$ \(T_e = 55\) K -
\(T_{\text{sys}} = 30 + 55 = 85\) K

\[
\frac{G}{T} = 48 - 10\log_{10}(85) = 48 - 19.3 = 28.7\ \text{dB/K}
\]

\textbf{Typical G/T}: - \textbf{VSAT terminals} (0.6-1.2 m): 10-20 dB/K
- \textbf{Professional earth stations} (3-9 m): 25-35 dB/K -
\textbf{Large observatories} (25+ m): 40-60 dB/K

\begin{center}\rule{0.5\linewidth}{0.5pt}\end{center}

\subsection{Other Noise Sources}\label{other-noise-sources}

\subsubsection{1. Shot Noise}\label{shot-noise}

\textbf{Due to discrete nature of charge carriers}:

\[
i_n^2 = 2 q I_{\text{DC}} B
\]

Where: - \(q = 1.6 \times 10^{-19}\) C (electron charge) -
\(I_{\text{DC}}\) = DC current (A) - \(B\) = Bandwidth (Hz)

\textbf{Significant in}: Photodetectors, diodes, low-current circuits

\textbf{Example}: Photodiode @ 1 mA DC, 1 MHz BW

\[
i_n = \sqrt{2 \times 1.6 \times 10^{-19} \times 10^{-3} \times 10^6} = 5.7 \times 10^{-10}\ \text{A}_{\text{rms}}
\]

\begin{center}\rule{0.5\linewidth}{0.5pt}\end{center}

\subsubsection{2. Flicker Noise (1/f
Noise)}\label{flicker-noise-1f-noise}

\textbf{Low-frequency noise}, power inversely proportional to frequency:

\[
S(f) \propto \frac{1}{f}
\]

\textbf{Significant}: \textless{} 1 kHz (audio, low-IF systems)

\textbf{Mitigation}: Use higher IF, differential circuits, chopper
stabilization

\begin{center}\rule{0.5\linewidth}{0.5pt}\end{center}

\subsubsection{3. Phase Noise}\label{phase-noise}

\textbf{Oscillator noise} causes frequency jitter:

\textbf{Specified as} \(\mathcal{L}(f_m)\) (dBc/Hz at offset \(f_m\)
from carrier)

\textbf{Example}: Satellite LO @ 10 GHz - Phase noise: -90 dBc/Hz @ 10
kHz offset - Degrades SNR in adjacent channels

\textbf{See}: {[}{[}Synchronization-(Carrier,-Timing,-Frame){]}{]} for
impact on coherent demodulation

\begin{center}\rule{0.5\linewidth}{0.5pt}\end{center}

\subsubsection{4. Quantization Noise}\label{quantization-noise}

\textbf{Analog-to-digital conversion} introduces rounding error:

\[
\text{SNR}_{\text{quant}} = 6.02n + 1.76 \quad (\text{dB})
\]

Where \(n\) = Number of bits

\textbf{Example}: 12-bit ADC - SNR = 6.02 \$\textbackslash times\$ 12 +
1.76 = 74 dB

\textbf{Implication}: Need enough ADC bits to avoid degrading RF SNR

\begin{center}\rule{0.5\linewidth}{0.5pt}\end{center}

\subsubsection{5. Intermodulation Distortion
(IMD)}\label{intermodulation-distortion-imd}

\textbf{Non-linear components} create spurious products:

\textbf{Two-tone test}: Inputs at \(f_1\) and \(f_2\)
\$\textbackslash rightarrow\$ Products at \(2f_1 - f_2\), \(2f_2 - f_1\)
(3rd order)

\textbf{IP3} (Third-order intercept point):

\[
\text{IMD3}_{\text{dBc}} = 2(P_{\text{IP3}} - P_{\text{in}})
\]

\textbf{Example}: Mixer with IP3 = +10 dBm, input = -20 dBm - IMD3 =
2(10 - (-20)) = 60 dBc below carrier

\textbf{Implication}: Strong interferers create in-band noise

\begin{center}\rule{0.5\linewidth}{0.5pt}\end{center}

\subsubsection{6. Atmospheric Noise}\label{atmospheric-noise}

\textbf{Natural sources}: - \textbf{Lightning}: Dominates \textless{} 30
MHz (HF, VHF) - \textbf{Cosmic noise}: Galactic background (0.1-3 GHz) -
\textbf{Solar noise}: Sun radiation (all frequencies)

\textbf{External noise temperature} \(T_{\text{ext}}\):

{\def\LTcaptype{} % do not increment counter
\begin{longtable}[]{@{}lll@{}}
\toprule\noalign{}
Frequency & \(T_{\text{ext}}\) (K) & Dominant Source \\
\midrule\noalign{}
\endhead
\bottomrule\noalign{}
\endlastfoot
10 MHz & 10,000-100,000 & Lightning (HF) \\
100 MHz & 1,000-10,000 & Galactic noise \\
1 GHz & 10-100 & Cosmic background \\
10 GHz & 3-30 & Sky temp (clear) \\
60 GHz & 100-300 & Atmospheric O\textbackslash textsubscript\{2\} \\
\end{longtable}
}

\textbf{Antenna noise temp}:

\[
T_{\text{ant}} = T_{\text{ext}} \eta + T_0 (1 - \eta)
\]

Where \(\eta\) = Antenna efficiency

\begin{center}\rule{0.5\linewidth}{0.5pt}\end{center}

\subsubsection{7. Man-Made Noise}\label{man-made-noise}

\textbf{Interference from}: Power lines, electric motors, computers,
switching power supplies

\textbf{Impulsive noise}: Short bursts (microseconds)
\$\textbackslash rightarrow\$ High peak power

\textbf{Mitigation}: Filtering, shielding, time diversity
(retransmission)

\begin{center}\rule{0.5\linewidth}{0.5pt}\end{center}

\subsection{Receiver Sensitivity
Calculation}\label{receiver-sensitivity-calculation}

\textbf{Minimum detectable signal} for target SNR:

\[
P_{\text{min}} = N + \text{NF} + \text{SNR}_{\text{req}} + L_{\text{impl}}
\]

Where (all in dB): - \(N = -174 + 10\log_{10}(B)\) (thermal noise in
bandwidth B) - NF = Receiver noise figure (dB) - SNR\_req = Required SNR
for demodulation (dB) - \(L_{\text{impl}}\) = Implementation loss (1-3
dB typical)

\begin{center}\rule{0.5\linewidth}{0.5pt}\end{center}

\subsubsection{Example: GPS Receiver}\label{example-gps-receiver}

\textbf{Specs}: - Bandwidth: 2 MHz (C/A code) - NF: 3 dB (typical GPS
front-end) - SNR\_req: -20 dB (spread spectrum processing gain 43 dB,
need C/N\textbackslash textsubscript\{0\} = 33 dB-Hz) - Impl loss: 2 dB

\textbf{Thermal noise}:

\[
N = -174 + 10\log_{10}(2 \times 10^6) = -174 + 63 = -111\ \text{dBm}
\]

\textbf{Sensitivity}:

\[
P_{\text{min}} = -111 + 3 + (-20) + 2 = -126\ \text{dBm}
\]

\textbf{But wait!} GPS uses C/N\textbackslash textsubscript\{0\} metric
(per Hz):

\textbf{C/N\textbackslash textsubscript\{0\}} requirement: 33 dB-Hz
(acquisition), 28 dB-Hz (tracking)

\textbf{Sensitivity} (alternate method):

\[
P_{\text{min}} = -174 + 33 + 3 + 2 = -136\ \text{dBm}
\]

\textbf{Typical GPS signal}: -130 dBm (open sky)
\$\textbackslash rightarrow\$ 6 dB margin

\begin{center}\rule{0.5\linewidth}{0.5pt}\end{center}

\subsubsection{Example: Satellite DVB-S2
Receiver}\label{example-satellite-dvb-s2-receiver}

\textbf{Specs}: - Bandwidth: 36 MHz - NF: 1.5 dB (LNA at feed) -
Modulation: QPSK 3/4 (LDPC) - SNR\_req: 6.5 dB (for BER \textless{}
10\textbackslash textsuperscript\{-\}\textbackslash textsuperscript\{7\}
post-FEC) - Impl loss: 1.5 dB

\textbf{Thermal noise}:

\[
N = -174 + 10\log_{10}(36 \times 10^6) = -174 + 75.6 = -98.4\ \text{dBm}
\]

\textbf{Sensitivity}:

\[
P_{\text{min}} = -98.4 + 1.5 + 6.5 + 1.5 = -88.9\ \text{dBm}
\]

\textbf{Link budget must deliver} \textgreater{} -88.9 dBm at LNB output
for proper operation

\begin{center}\rule{0.5\linewidth}{0.5pt}\end{center}

\subsection{Noise Figure Measurement}\label{noise-figure-measurement}

\subsubsection{Y-Factor Method}\label{y-factor-method}

\textbf{Standard technique} using hot/cold loads:

\begin{enumerate}
\def\labelenumi{\arabic{enumi}.}
\tightlist
\item
  Measure noise power with \textbf{hot load} (\(T_h = 290\) K): \(P_h\)
\item
  Measure noise power with \textbf{cold load} (\(T_c = 77\) K, liquid
  N\textbackslash textsubscript\{2\}): \(P_c\)
\item
  Calculate \textbf{Y-factor}:
\end{enumerate}

\[
Y = \frac{P_h}{P_c}
\]

\begin{enumerate}
\def\labelenumi{\arabic{enumi}.}
\setcounter{enumi}{3}
\tightlist
\item
  \textbf{Noise figure}:
\end{enumerate}

\[
\text{NF} = 10\log_{10}\left(\frac{T_h - YT_c}{290(Y-1)}\right)
\]

\textbf{Example}: \(P_h = 100\) units, \(P_c = 80\) units - Y = 100/80 =
1.25 - NF = \(10\log_{10}[(290 - 1.25 \times 77)/(290 \times 0.25)]\) =
1.8 dB

\begin{center}\rule{0.5\linewidth}{0.5pt}\end{center}

\subsubsection{Noise Source Method}\label{noise-source-method}

\textbf{Use calibrated noise source} (ENR = Excess Noise Ratio in dB):

\[
\text{NF} = \text{ENR} - 10\log_{10}(Y - 1)
\]

Where Y = ratio of power with noise source ON/OFF

\textbf{Example}: ENR = 15 dB noise source, Y = 10 - NF = 15 - 10log(9)
= 15 - 9.54 = 5.46 dB

\begin{center}\rule{0.5\linewidth}{0.5pt}\end{center}

\subsection{Design Guidelines}\label{design-guidelines}

\subsubsection{Optimize Noise Figure}\label{optimize-noise-figure}

\begin{enumerate}
\def\labelenumi{\arabic{enumi}.}
\tightlist
\item
  \textbf{LNA at antenna}: Minimize cable loss before LNA
\item
  \textbf{High LNA gain}: 15-20 dB isolates from later stages
\item
  \textbf{Low-loss transmission}: Use low-loss cable (LMR-400, hardline)
\item
  \textbf{Cool LNA}: Cryogenic cooling for satellite ground stations
\item
  \textbf{Avoid passive loss}: No attenuators, splitters before LNA
\end{enumerate}

\begin{center}\rule{0.5\linewidth}{0.5pt}\end{center}

\subsubsection{Trade-offs}\label{trade-offs}

\textbf{Lower NF \$\textbackslash rightarrow\$ Higher cost}: - Premium
LNA: 0.8 dB NF = \$500+ - Standard LNA: 2 dB NF = \$50 - Difference: 1.2
dB sensitivity = 1.3\$\textbackslash times\$ range improvement

\textbf{Cryogenic cooling}: - Cooled LNA: 0.3 dB NF @ 20 K - Room temp
LNA: 1.5 dB NF @ 290 K - Difference: 1.2 dB (worth it for deep space,
not for WiFi!)

\begin{center}\rule{0.5\linewidth}{0.5pt}\end{center}

\subsection{Summary Table}\label{summary-table}

{\def\LTcaptype{} % do not increment counter
\begin{longtable}[]{@{}
  >{\raggedright\arraybackslash}p{(\linewidth - 6\tabcolsep) * \real{0.2258}}
  >{\raggedright\arraybackslash}p{(\linewidth - 6\tabcolsep) * \real{0.2903}}
  >{\raggedright\arraybackslash}p{(\linewidth - 6\tabcolsep) * \real{0.2903}}
  >{\raggedright\arraybackslash}p{(\linewidth - 6\tabcolsep) * \real{0.1935}}@{}}
\toprule\noalign{}
\begin{minipage}[b]{\linewidth}\raggedright
Noise Source
\end{minipage} & \begin{minipage}[b]{\linewidth}\raggedright
Spectral Density
\end{minipage} & \begin{minipage}[b]{\linewidth}\raggedright
When Significant
\end{minipage} & \begin{minipage}[b]{\linewidth}\raggedright
Mitigation
\end{minipage} \\
\midrule\noalign{}
\endhead
\bottomrule\noalign{}
\endlastfoot
\textbf{Thermal} & \(kT\) = -174 dBm/Hz & Always (fundamental) & Low NF,
high gain \\
\textbf{Shot} & \(\sqrt{2qI_{\text{DC}}B}\) & Low-light photodetectors &
Increase optical power \\
\textbf{Flicker (1/f)} & \(\propto 1/f\) & \textless{} 1 kHz & Higher
IF, differential \\
\textbf{Phase} & \(\mathcal{L}(f_m)\) & Near carrier & Better
oscillator, PLL \\
\textbf{Quantization} & \(-6n\) dB & Low SNR, few ADC bits & More bits,
higher SNR \\
\textbf{IMD} & Nonlinear products & Strong interferers & Higher IP3,
filtering \\
\textbf{Atmospheric} & Varies (10-100,000 K) & HF, low VHF & Directional
antenna \\
\textbf{Man-made} & Impulsive/broadband & Urban, near power lines &
Shielding, filtering \\
\end{longtable}
}

\begin{center}\rule{0.5\linewidth}{0.5pt}\end{center}

\subsection{Related Topics}\label{related-topics}

\begin{itemize}
\tightlist
\item
  \textbf{{[}{[}Signal-to-Noise-Ratio-(SNR){]}{]}}: Determines BER
  performance
\item
  \textbf{{[}{[}Complete-Link-Budget-Analysis{]}{]}}: Uses NF for
  sensitivity
\item
  \textbf{{[}{[}Bit-Error-Rate-(BER){]}{]}}: Degrades with noise
\item
  \textbf{{[}{[}Energy-Ratios-(Es-N0-and-Eb-N0){]}{]}}: Normalized SNR
  metrics
\item
  \textbf{{[}{[}Antenna-Theory-Basics{]}{]}}: Antenna noise temperature
\item
  \textbf{{[}{[}Free-Space-Path-Loss-(FSPL){]}{]}}: Path loss + noise
  \$\textbackslash rightarrow\$ Link budget
\end{itemize}

\begin{center}\rule{0.5\linewidth}{0.5pt}\end{center}

\textbf{Key takeaway}: \textbf{Noise limits receiver sensitivity.}
Thermal noise floor = -174 dBm/Hz @ 290 K. Noise figure (NF) quantifies
SNR degradation through receiver. Friis formula shows first stage (LNA)
dominates total NF. Low-NF LNA at antenna, high gain, and minimal
pre-LNA loss are critical. Sensitivity = Noise floor + NF + Required
SNR. Lower noise = longer range, higher data rate, better reliability.

\begin{center}\rule{0.5\linewidth}{0.5pt}\end{center}

\emph{This wiki is part of the {[}{[}Home\textbar Chimera Project{]}{]}
documentation.}
