\section{IQ Representation}\label{iq-representation}

\subsection{\texorpdfstring{ For Non-Technical
Readers}{ For Non-Technical Readers}}\label{for-non-technical-readers}

\textbf{IQ representation is like describing a location on a map using X
and Y coordinates-\/-\/-it lets you pinpoint any radio signal position
in 2D space!}

\textbf{What is IQ?} - \textbf{I (In-phase)}: Horizontal axis, like
``East-West'' on a map - \textbf{Q (Quadrature)}: Vertical axis, like
``North-South'' on a map - Together: Any point in 2D = any signal you
can send!

\textbf{Why two dimensions?} - Radio waves have \textbf{amplitude}
(strength) AND \textbf{phase} (timing) - Phase is like ``what part of
the wave cycle are you at?'' - Two dimensions let you control BOTH
simultaneously

\textbf{Real-world analogy - Clock hands}: - \textbf{12
o\textquotesingle clock position}: I = max, Q = 0 - \textbf{3
o\textquotesingle clock position}: I = 0, Q = max\\
- \textbf{6 o\textquotesingle clock position}: I = -max, Q = 0 -
\textbf{9 o\textquotesingle clock position}: I = 0, Q = -max - Any angle
= unique IQ coordinate!

\textbf{Why it\textquotesingle s everywhere}: - \textbf{Software Defined
Radio (SDR)}: All processing uses IQ data - \textbf{Digital audio}:
Left/Right channels \$\textbackslash rightarrow\$ I/Q channels -
\textbf{Your phone}: Baseband chip outputs IQ samples, radio transmits
them - \textbf{WiFi chips}: Process IQ data to decode constellations

\textbf{The magic trick}: - One wire carries I signal, another carries Q
signal - At transmitter: Combine using 90\$\^{}\textbackslash circ\$
phase-shifted carriers - At receiver: Split using
90\$\^{}\textbackslash circ\$ phase-shifted carriers - Result: Two
independent data channels on same frequency!

\textbf{Fun fact}: IQ representation is why ``quadrature'' modulation
(QPSK, QAM) is so efficient-\/-\/-you\textquotesingle re using two
perpendicular dimensions, doubling capacity compared to just varying
amplitude!

\begin{center}\rule{0.5\linewidth}{0.5pt}\end{center}

Each QPSK symbol is represented as a point in 2D space with two
components:

\begin{itemize}
\tightlist
\item
  \textbf{I (In-phase)}: The horizontal component (real part)
\item
  \textbf{Q (Quadrature)}: The vertical component (imaginary part)
\end{itemize}

\subsection{What is I/Q?}\label{what-is-iq}

\textbf{I} and \textbf{Q} are the two orthogonal (perpendicular)
components of a modulated signal. They\textquotesingle re called: -
\textbf{In-phase} (I): Aligned with the carrier wave -
\textbf{Quadrature} (Q): 90\$\^{}\textbackslash circ\$ out of phase with
the carrier wave

\subsection{Mathematical
Representation}\label{mathematical-representation}

Any modulated signal can be expressed as:

\begin{verbatim}
Signal(t) = I(t)·cos(2ft) - Q(t)·sin(2ft)
\end{verbatim}

Where: - \texttt{f} is the carrier frequency - \texttt{I(t)} is the
in-phase amplitude - \texttt{Q(t)} is the quadrature amplitude

\subsection{Complex Number Notation}\label{complex-number-notation}

In DSP, we often use complex number notation:

\begin{verbatim}
Symbol = I + jQ
\end{verbatim}

Where \texttt{j} is the imaginary unit (\$\textbackslash sqrt\{\}\$-1)

\subsubsection{QPSK Example}\label{qpsk-example}

For normalized QPSK symbols:

{\def\LTcaptype{} % do not increment counter
\begin{longtable}[]{@{}lllll@{}}
\toprule\noalign{}
Bits & I & Q & Complex & Phase \\
\midrule\noalign{}
\endhead
\bottomrule\noalign{}
\endlastfoot
00 & -0.707 & +0.707 & -0.707+j0.707 & 135\$\^{}\textbackslash circ\$ \\
01 & +0.707 & +0.707 & +0.707+j0.707 & 45\$\^{}\textbackslash circ\$ \\
11 & -0.707 & -0.707 & -0.707-j0.707 & 225\$\^{}\textbackslash circ\$ \\
10 & +0.707 & -0.707 & +0.707-j0.707 & 315\$\^{}\textbackslash circ\$ \\
\end{longtable}
}

\subsection{Why Use I/Q?}\label{why-use-iq}

\begin{enumerate}
\def\labelenumi{\arabic{enumi}.}
\tightlist
\item
  \textbf{Efficient Processing}: Easy to implement in digital
  hardware/software
\item
  \textbf{Phase and Amplitude}: Naturally represents both
  characteristics
\item
  \textbf{Orthogonality}: I and Q don\textquotesingle t interfere with
  each other
\item
  \textbf{Standard Format}: Universal in modern communications
\end{enumerate}

\subsection{I/Q in Chimera}\label{iq-in-chimera}

Chimera\textquotesingle s constellation diagrams plot: - \textbf{X-axis
(horizontal)}: I component - \textbf{Y-axis (vertical)}: Q component

When you see a dot at position (I=0.707, Q=0.707), that represents the
QPSK symbol for bits \texttt{01}.

\subsection{Adding Noise}\label{adding-noise}

When {[}{[}Additive-White-Gaussian-Noise-(AWGN){]}{]} is added:

\begin{verbatim}
I_received = I_transmitted + N_I
Q_received = Q_transmitted + N_Q
\end{verbatim}

Where \texttt{N\_I} and \texttt{N\_Q} are independent Gaussian random
variables. This is why you see \textbf{clouds} instead of
\textbf{points} in the RX constellation!

\subsection{See Also}\label{see-also}

\begin{itemize}
\tightlist
\item
  {[}{[}QPSK-Modulation{]}{]} - How bits map to I/Q values
\item
  {[}{[}Constellation-Diagrams{]}{]} - Visualizing I/Q space
\item
  {[}{[}Additive-White-Gaussian-Noise-(AWGN){]}{]} - How noise affects
  I/Q
\end{itemize}
