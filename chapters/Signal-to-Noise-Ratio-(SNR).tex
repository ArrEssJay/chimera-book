\section{Signal-to-Noise Ratio (SNR)}\label{signal-to-noise-ratio-snr}

\subsection{\texorpdfstring{ For Non-Technical
Readers}{ For Non-Technical Readers}}\label{for-non-technical-readers}

\textbf{SNR is like the difference between a conversation in a quiet
library (high SNR) vs a loud nightclub (low SNR)-\/-\/-higher SNR =
easier to understand the message!}

\textbf{The idea - Signal vs Background}: - \textbf{Signal}: The
information you want (voice, data, music) - \textbf{Noise}: Random
interference you don\textquotesingle t want (static, hiss, interference)
- \textbf{SNR}: How much stronger is signal than noise?

\textbf{Real-world analogies}:

\textbf{Good SNR} (Easy to hear): - \textbf{Quiet library conversation}:
Speech is 20\$\textbackslash times\$ louder than background
\$\textbackslash rightarrow\$ 26 dB SNR - \textbf{Clear radio station}:
Music is 100\$\textbackslash times\$ louder than static
\$\textbackslash rightarrow\$ 40 dB SNR - \textbf{Strong WiFi}: Data
signal is 1000\$\textbackslash times\$ stronger than noise
\$\textbackslash rightarrow\$ 60 dB SNR

\textbf{Bad SNR} (Hard to hear): - \textbf{Loud nightclub}: Trying to
talk, voice only 2\$\textbackslash times\$ louder than music
\$\textbackslash rightarrow\$ 6 dB SNR - \textbf{Weak radio station}:
Static almost as loud as music \$\textbackslash rightarrow\$ 3 dB SNR -
\textbf{Far from router}: WiFi signal barely stronger than interference
\$\textbackslash rightarrow\$ 5 dB SNR

\textbf{The dB scale} (why engineers use it): - \textbf{Linear}:
10\$\textbackslash times\$ stronger = 10 dB, 100\$\textbackslash times\$
stronger = 20 dB, 1000\$\textbackslash times\$ stronger = 30 dB -
\textbf{Logarithmic}: Makes huge ranges manageable - \textbf{Rule of
thumb}: +3 dB = double the power, +10 dB = 10\$\textbackslash times\$
the power

\textbf{SNR quality guide}:

\begin{verbatim}
60+ dB SNR:  Perfect - Laboratory quality
40-60 dB:    Excellent - WiFi close to router
20-40 dB:   🟡 Good - Cell phone normal use
10-20 dB:   🟠 Fair - Far from WiFi, slower speeds
0-10 dB:    🔴 Poor - Lots of errors, need error correction
Below 0 dB:  Terrible - Noise louder than signal!
\end{verbatim}

\textbf{Real examples you experience}:

\textbf{WiFi speed changes}: - \textbf{Next to router}: 60 dB SNR
\$\textbackslash rightarrow\$ Use 1024-QAM \$\textbackslash rightarrow\$
1200 Mbps - \textbf{One room away}: 35 dB SNR
\$\textbackslash rightarrow\$ Use 256-QAM \$\textbackslash rightarrow\$
600 Mbps - \textbf{Two rooms away}: 20 dB SNR
\$\textbackslash rightarrow\$ Use 64-QAM \$\textbackslash rightarrow\$
200 Mbps - \textbf{Far corner}: 10 dB SNR \$\textbackslash rightarrow\$
Use QPSK \$\textbackslash rightarrow\$ 50 Mbps - Your device
\textbf{automatically adjusts} based on SNR!

\textbf{Cell phone bars}: - \textbf{5 bars}: \textgreater20 dB SNR
\$\textbackslash rightarrow\$ Fast data, clear calls - \textbf{3 bars}:
\textasciitilde10 dB SNR \$\textbackslash rightarrow\$ Slower data,
occasional drop - \textbf{1 bar}: \textasciitilde5 dB SNR
\$\textbackslash rightarrow\$ Very slow, frequent errors - \textbf{No
bars}: \textless0 dB SNR \$\textbackslash rightarrow\$
Can\textquotesingle t connect

\textbf{Voice calls}: - \textbf{Landline}: \textasciitilde40 dB SNR
\$\textbackslash rightarrow\$ Crystal clear - \textbf{Good cell}:
\textasciitilde20 dB SNR \$\textbackslash rightarrow\$ Clear -
\textbf{Bad cell}: \textasciitilde10 dB SNR
\$\textbackslash rightarrow\$ ``Can you hear me now?'' -
\textbf{Terrible cell}: \textasciitilde5 dB SNR
\$\textbackslash rightarrow\$ Garbled, robotic voice

\textbf{Why SNR matters}:

\textbf{Data rate} (how fast): - High SNR \$\textbackslash rightarrow\$
Use complex modulation (256-QAM, 1024-QAM) \$\textbackslash rightarrow\$
Fast! - Low SNR \$\textbackslash rightarrow\$ Use simple modulation
(QPSK, BPSK) \$\textbackslash rightarrow\$ Slow but reliable

\textbf{Error rate} (how accurate): - High SNR
\$\textbackslash rightarrow\$ Few errors \$\textbackslash rightarrow\$
No retransmissions \$\textbackslash rightarrow\$ Efficient - Low SNR
\$\textbackslash rightarrow\$ Many errors \$\textbackslash rightarrow\$
Lots of retransmissions \$\textbackslash rightarrow\$ Inefficient

\textbf{Range} (how far): - Close distance \$\textbackslash rightarrow\$
High SNR \$\textbackslash rightarrow\$ Fast connection - Far distance
\$\textbackslash rightarrow\$ Low SNR \$\textbackslash rightarrow\$ Slow
or no connection

\textbf{Engineering trade-offs}:

\textbf{Increase SNR by}: - \textbf{More transmit power}: Stronger
signal (but uses battery, FCC limits) - \textbf{Bigger antennas}:
Collect more signal (but bulky) - \textbf{Get closer}: Reduce distance
(not always possible) - \textbf{Reduce noise}: Better receivers,
shielding (expensive)

\textbf{Shannon\textquotesingle s Law} (theoretical limit):

\begin{verbatim}
Max data rate = Bandwidth × log(1 + SNR)
\end{verbatim}

\begin{itemize}
\tightlist
\item
  Double SNR \$\textbackslash rightarrow\$ \textasciitilde40\% more data
  rate
\item
  10\$\textbackslash times\$ SNR \$\textbackslash rightarrow\$
  3\$\textbackslash times\$ more data rate
\item
  This is why 5G needs high SNR for multi-Gbps speeds!
\end{itemize}

\textbf{When you see SNR}:

\textbf{Router admin page}: ``SNR: 42 dB'' \$\textbackslash rightarrow\$
Excellent connection \textbf{WiFi diagnostics}: ``Signal: -45 dBm,
Noise: -95 dBm'' \$\textbackslash rightarrow\$ SNR = 50 dB \textbf{Cell
phone}: ``RSRP: -80 dBm, SINR: 15 dB'' \$\textbackslash rightarrow\$
Decent 4G signal \textbf{Audio recording}: ``SNR: 90 dB''
\$\textbackslash rightarrow\$ Professional studio quality

\textbf{The ultimate limit - Thermal noise}: - All electronics generate
noise from heat - Room temperature: Noise floor \textasciitilde-174
dBm/Hz - This sets fundamental limit for all communication -
Can\textquotesingle t go below this (without cooling to near absolute
zero!)

\textbf{Fun fact}: Deep space communications have SNR well below 0
dB-\/-\/-Voyager 1\textquotesingle s signal arriving at Earth is
\textbf{10,000\$\textbackslash times\$ weaker than the noise!} Engineers
use huge antennas, narrow filters, and sophisticated algorithms to
extract signal from noise. It\textquotesingle s like hearing a whisper
from 15 billion miles away!

\begin{center}\rule{0.5\linewidth}{0.5pt}\end{center}

\textbf{Signal-to-Noise Ratio (SNR)} measures the strength of the
desired signal relative to the background noise. It\textquotesingle s
typically expressed in decibels (dB).

\subsection{Understanding SNR Values}\label{understanding-snr-values}

\begin{verbatim}
Higher SNR = Better Signal Quality

SNR (dB)  |  Quality          |  Typical Use Case
----------|-------------------|----------------------------------
> 20 dB   |  Excellent        |  Clear reception, low error rate
10-20 dB  |  Good             |  Reliable communication
0-10 dB   |  Poor             |  Many errors, FEC required
< 0 dB    |  Very Poor        |  Noise stronger than signal
\end{verbatim}

\subsection{SNR Formula}\label{snr-formula}

\begin{verbatim}
SNR (linear) = Signal Power / Noise Power

SNR (dB) = 10 · log(Signal Power / Noise Power)
\end{verbatim}

\subsection{SNR in Chimera}\label{snr-in-chimera}

In Chimera\textquotesingle s simulation, you control the \textbf{channel
SNR}, which determines how much noise is added to the transmitted
signal:

{\def\LTcaptype{} % do not increment counter
\begin{longtable}[]{@{}lll@{}}
\toprule\noalign{}
Setting & Description & Constellation \\
\midrule\noalign{}
\endhead
\bottomrule\noalign{}
\endlastfoot
\textbf{High SNR} (-5 dB) & Minimal noise & Tight clusters \\
\textbf{Medium SNR} (-15 dB) & Moderate noise & Visible scatter \\
\textbf{Low SNR} (-25 dB) & Heavy noise & Large scatter, errors
likely \\
\end{longtable}
}

\subsubsection{Processing Gain}\label{processing-gain}

Chimera achieves approximately \textbf{35 dB of processing gain} through
symbol averaging and oversampling. This means:

\begin{verbatim}
Effective SNR = Channel SNR + Processing Gain
              = -25 dB + 35 dB
              = 10 dB (after processing)
\end{verbatim}

This is why the system can operate reliably even with very low channel
SNR values.

\subsection{SNR vs Es/N0}\label{snr-vs-esn0}

In Chimera\textquotesingle s UI, ``Channel SNR (dB)'' represents
\textbf{Es/N0} (symbol energy to noise ratio): - \textbf{Before
processing}: Low Es/N0 (e.g., -25 dB) - \textbf{After processing gain}:
Higher effective SNR (\textasciitilde10 dB) - \textbf{LDPC threshold}:
Fails below -27 dB channel SNR

\subsection{Impact on Performance}\label{impact-on-performance}

\subsubsection{High SNR (\textgreater15 dB)}\label{high-snr-15-db}

\begin{itemize}
\tightlist
\item
  Perfect constellation separation
\item
  Zero or near-zero bit errors
\item
  FEC not strictly needed
\item
  BER:
  \textless10\textbackslash textsuperscript\{-\}\textbackslash textsuperscript\{6\}
\end{itemize}

\subsubsection{Medium SNR (5-15 dB)}\label{medium-snr-5-15-db}

\begin{itemize}
\tightlist
\item
  Visible constellation scatter
\item
  Some bit errors occur
\item
  FEC recommended
\item
  BER:
  10\textbackslash textsuperscript\{-\}\textbackslash textsuperscript\{3\}
  to
  10\textbackslash textsuperscript\{-\}\textbackslash textsuperscript\{6\}
\end{itemize}

\subsubsection{Low SNR (\textless5 dB)}\label{low-snr-5-db}

\begin{itemize}
\tightlist
\item
  Heavy constellation scatter
\item
  Many bit errors
\item
  FEC required
\item
  BER:
  \textgreater10\textbackslash textsuperscript\{-\}\textbackslash textsuperscript\{3\}
\end{itemize}

\subsection{See Also}\label{see-also}

\begin{itemize}
\tightlist
\item
  {[}{[}Energy-Ratios-(Es-N0-and-Eb-N0){]}{]} - Related energy metrics
\item
  {[}{[}Additive-White-Gaussian-Noise-(AWGN){]}{]} - What creates the
  noise
\item
  {[}{[}Bit-Error-Rate-(BER){]}{]} - How SNR affects errors
\item
  {[}{[}Constellation-Diagrams{]}{]} - Visualizing SNR impact
\end{itemize}
