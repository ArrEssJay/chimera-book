\section{Amplitude-Shift Keying (ASK)}\label{amplitude-shift-keying-ask}

{[}{[}Home{]}{]} \textbar{} \textbf{Digital Modulation} \textbar{}
{[}{[}On-Off-Keying-(OOK){]}{]} \textbar{} {[}{[}QPSK-Modulation{]}{]}

\begin{center}\rule{0.5\linewidth}{0.5pt}\end{center}

\subsection{\texorpdfstring{ For Non-Technical
Readers}{ For Non-Technical Readers}}\label{for-non-technical-readers}

\textbf{ASK is like using a dimmer switch on a flashlight-\/-\/-bright =
1, dim = 0. Simple to understand, but noise makes it hard to tell bright
from dim!}

\textbf{The idea - Vary the volume}: - Want to send \textbf{1}? Transmit
at \textbf{full power} - Want to send \textbf{0}? Transmit at
\textbf{low power} (or off) - Receiver measures: ``How loud is the
signal right now?''

\textbf{Real-world analogy - Theater lights}: - \textbf{Full brightness}
= data bit ``1'' - \textbf{Dim/off} = data bit ``0'' - Audience
(receiver) can see which state you\textquotesingle re in - But if room
is smoky (noise), hard to tell bright from dim!

\textbf{Why it\textquotesingle s problematic}: - \textbf{Noise affects
amplitude}: Interference makes signal stronger/weaker - \textbf{Fading
affects amplitude}: Obstacles change signal strength - Hard to tell:
``Is this a dim signal, or a faded bright signal?'' - This is why ASK
isn\textquotesingle t used much in modern systems!

\textbf{Where ASK is still used}: - \textbf{RFID tags}: Extremely
simple, low power (backscatter modulation) - \textbf{Fiber optic}:
Optical fiber has low noise, so ASK works well - \textbf{Historical
modems}: Old 300 baud modems used ASK - \textbf{Combined with PSK}: QAM
= ASK + PSK together (best of both!)

\textbf{Advanced: M-ary ASK}: - Instead of 2 levels (on/off), use 4, 8,
or 16 levels - \textbf{4-ASK}: Four brightness levels = 2 bits/symbol -
\textbf{8-ASK}: Eight brightness levels = 3 bits/symbol - Even more
sensitive to noise!

\textbf{Why phase (PSK) or frequency (FSK) is better}: - \textbf{PSK}:
Noise changes amplitude, phase stays stable - \textbf{FSK}: Noise
changes amplitude, frequency stays stable - \textbf{ASK}: Noise directly
corrupts the information!

\textbf{Fun fact}: Your TV remote uses a form of ASK-\/-\/-infrared LED
blinks on/off to send button codes. It works because the path from
remote to TV is short and clean (low noise)!

\begin{center}\rule{0.5\linewidth}{0.5pt}\end{center}

\subsection{Overview}\label{overview}

\textbf{Amplitude-Shift Keying (ASK)} encodes digital data by varying
the \textbf{amplitude} of a carrier wave.

\textbf{Principle}: Different symbols represented by different amplitude
levels

\textbf{M-ary ASK}: M distinct amplitude levels encode \(\log_2(M)\)
bits per symbol

\textbf{Relationship to OOK}: OOK is binary ASK (M=2, one amplitude is
zero)

\begin{center}\rule{0.5\linewidth}{0.5pt}\end{center}

\subsection{Binary ASK (2-ASK)}\label{binary-ask-2-ask}

\textbf{Two amplitude levels}: \(A_1\) and \(A_2\) (typically 0 and A)

\textbf{Signal}:

\[
s(t) = \begin{cases}
A \cos(2\pi f_c t) & \text{bit = 1} \\
0 & \text{bit = 0}
\end{cases}
\]

\textbf{This is On-Off Keying (OOK)}

\textbf{See}: {[}{[}On-Off-Keying-(OOK){]}{]}

\begin{center}\rule{0.5\linewidth}{0.5pt}\end{center}

\subsection{M-ary ASK}\label{m-ary-ask}

\textbf{M amplitude levels}: \(A_m\) for \(m = 0, 1, \ldots, M-1\)

\textbf{Signal for symbol} \(m\):

\[
s_m(t) = A_m \cos(2\pi f_c t), \quad 0 \leq t < T_s
\]

\textbf{Amplitude levels} (equally spaced):

\[
A_m = A_0 + m \cdot \Delta A
\]

Where: - \(A_0\) = Minimum amplitude (often 0) - \(\Delta A\) =
Amplitude spacing - \(m\) = Symbol index (0 to M-1)

\begin{center}\rule{0.5\linewidth}{0.5pt}\end{center}

\subsubsection{4-ASK Example}\label{ask-example}

\textbf{4 amplitude levels}: 0, A, 2A, 3A

\textbf{Bits per symbol}: \(\log_2(4) = 2\) bits

\textbf{Mapping}:

{\def\LTcaptype{} % do not increment counter
\begin{longtable}[]{@{}lll@{}}
\toprule\noalign{}
Bits & Symbol & Amplitude \\
\midrule\noalign{}
\endhead
\bottomrule\noalign{}
\endlastfoot
00 & 0 & 0 \\
01 & 1 & A \\
10 & 2 & 2A \\
11 & 3 & 3A \\
\end{longtable}
}

\textbf{Constellation diagram} (1D):

\begin{verbatim}
      0   A   2A  3A
      |   |   |   |
     [] [] [] []
\end{verbatim}

\begin{center}\rule{0.5\linewidth}{0.5pt}\end{center}

\subsubsection{8-ASK Example}\label{ask-example-1}

\textbf{8 amplitude levels}: 0, A, 2A, \textbackslash ldots\{\}, 7A

\textbf{Bits per symbol}: \(\log_2(8) = 3\) bits

\textbf{Spectral efficiency}: 3 bits/symbol (3\$\textbackslash times\$
binary ASK for same bandwidth)

\begin{center}\rule{0.5\linewidth}{0.5pt}\end{center}

\subsection{Modulation \& Demodulation}\label{modulation-demodulation}

\subsubsection{ASK Modulator}\label{ask-modulator}

\textbf{Block diagram}:

\begin{verbatim}
Data bits --> [Serial to     --> [DAC] --> [×] --> ASK signal
               Parallel (log2M)]              |
                                          cos(2f_c t)
\end{verbatim}

\textbf{Steps}: 1. Group bits into symbols (\(\log_2(M)\) bits per
symbol) 2. Map symbol to amplitude level \(A_m\) 3. Multiply amplitude
by carrier

\textbf{Baseband equivalent}:

\[
s(t) = A_m(t) \cos(2\pi f_c t)
\]

Where \(A_m(t)\) = Pulse-shaped amplitude sequence

\begin{center}\rule{0.5\linewidth}{0.5pt}\end{center}

\subsubsection{Coherent Demodulation}\label{coherent-demodulation}

\textbf{Optimal detector} (requires carrier phase reference):

\textbf{Receiver}:

\begin{verbatim}
ASK signal --> [×] --> [LPF] --> [Sample] --> [Threshold] --> Bits
                |                             Detector
           cos(2f_c t)          t = kT_s
\end{verbatim}

\textbf{Steps}: 1. Multiply by synchronized carrier (coherent mixing) 2.
Low-pass filter \$\textbackslash rightarrow\$ Baseband amplitude 3.
Sample at symbol rate 4. Threshold detection (compare to M-1 thresholds)

\textbf{Decision thresholds}:

\[
\text{Threshold}_k = \frac{A_k + A_{k+1}}{2}, \quad k = 0, 1, \ldots, M-2
\]

\textbf{Example (4-ASK)}: Thresholds at A/2, 3A/2, 5A/2

\begin{center}\rule{0.5\linewidth}{0.5pt}\end{center}

\subsubsection{Non-Coherent
Demodulation}\label{non-coherent-demodulation}

\textbf{Envelope detector} (no phase reference needed):

\textbf{Block diagram}:

\begin{verbatim}
ASK signal --> [Envelope] --> [Sample] --> [Threshold] --> Bits
               Detector                   Detector
                             t = kT_s
\end{verbatim}

\textbf{Envelope detector}: Rectifier + low-pass filter (extracts
\textbar s(t)\textbar)

\textbf{Advantage}: Simple, no carrier recovery

\textbf{Disadvantage}: \textasciitilde3 dB worse SNR than coherent

\textbf{Used in}: AM radio, OOK (IR remote, RFID)

\begin{center}\rule{0.5\linewidth}{0.5pt}\end{center}

\subsection{Signal Space
Representation}\label{signal-space-representation}

\textbf{1-dimensional constellation} (real axis only):

\[
s_m(t) = A_m \phi(t)
\]

Where \(\phi(t) = \sqrt{\frac{2}{T_s}} \cos(2\pi f_c t)\) (orthonormal
basis)

\textbf{Energy per symbol}:

\[
E_m = \int_0^{T_s} s_m^2(t) dt = A_m^2
\]

\textbf{Average symbol energy} (assuming equal probability):

\[
\bar{E}_s = \frac{1}{M} \sum_{m=0}^{M-1} A_m^2
\]

\begin{center}\rule{0.5\linewidth}{0.5pt}\end{center}

\subsection{Performance Analysis}\label{performance-analysis}

\subsubsection{Bit Error Rate (BER) - Coherent
Detection}\label{bit-error-rate-ber---coherent-detection}

\textbf{Symbol error rate} (SER) for M-ASK in AWGN:

\[
P_s \approx 2\left(1 - \frac{1}{M}\right) Q\left(\sqrt{\frac{6\log_2(M)}{M^2 - 1} \cdot \frac{E_b}{N_0}}\right)
\]

\textbf{Gray coding assumed} (SER \$\textbackslash approx\$ BER /
\(\log_2(M)\)):

\[
\text{BER} \approx \frac{2}{M\log_2(M)}\left(1 - \frac{1}{M}\right) Q\left(\sqrt{\frac{6\log_2(M)}{M^2 - 1} \cdot \frac{E_b}{N_0}}\right)
\]

\textbf{Where}:
\(Q(x) = \frac{1}{\sqrt{2\pi}} \int_x^\infty e^{-t^2/2} dt\)

\begin{center}\rule{0.5\linewidth}{0.5pt}\end{center}

\subsubsection{BER for Specific M}\label{ber-for-specific-m}

\textbf{2-ASK (OOK)} @ 10 dB Eb/N0:

\[
\text{BER} = Q\left(\sqrt{2 \cdot 10}\right) \approx 3.9 \times 10^{-6}
\]

\textbf{4-ASK} @ 10 dB Eb/N0:

\[
\text{BER} \approx \frac{3}{4} Q\left(\sqrt{\frac{12}{15} \cdot 10}\right) = 0.75 \times Q(2.83) \approx 1.8 \times 10^{-3}
\]

\textbf{Observation}: Higher M \$\textbackslash rightarrow\$ Worse BER
(smaller amplitude spacing \$\textbackslash rightarrow\$ more noise
sensitivity)

\begin{center}\rule{0.5\linewidth}{0.5pt}\end{center}

\subsubsection{Required Eb/N0 for BER =
10\textbackslash textsuperscript\{-\}\textbackslash textsuperscript\{6\}}\label{required-ebn0-for-ber-10ux2076}

{\def\LTcaptype{} % do not increment counter
\begin{longtable}[]{@{}lll@{}}
\toprule\noalign{}
Modulation & Required Eb/N0 (dB) & Notes \\
\midrule\noalign{}
\endhead
\bottomrule\noalign{}
\endlastfoot
\textbf{2-ASK (OOK)} & 10.5 & Baseline \\
\textbf{4-ASK} & 14 & +3.5 dB penalty \\
\textbf{8-ASK} & 18 & +7.5 dB penalty \\
\textbf{16-ASK} & 22 & +11.5 dB penalty \\
\end{longtable}
}

\textbf{Pattern}: Each doubling of M adds \textasciitilde3.5-4 dB
penalty

\begin{center}\rule{0.5\linewidth}{0.5pt}\end{center}

\subsection{Power Efficiency}\label{power-efficiency}

\textbf{Average power} for M-ASK:

\[
\bar{P} = \frac{1}{M} \sum_{m=0}^{M-1} \frac{A_m^2}{2}
\]

\textbf{Peak-to-average power ratio (PAPR)}:

\[
\text{PAPR} = \frac{A_{\max}^2}{\bar{P}}
\]

\textbf{For M-ASK with amplitudes 0, A, 2A, \textbackslash ldots\{\},
(M-1)A}:

\[
\text{PAPR} = \frac{(M-1)^2}{\frac{1}{M}\sum_{m=0}^{M-1} m^2} = \frac{3(M-1)^2}{M(M-1)(2M-1)/3} = \frac{3(M-1)}{2M-1}
\]

\textbf{Example}: - 2-ASK: PAPR = 3 (4.8 dB) - 4-ASK: PAPR = 9/7
\$\textbackslash approx\$ 1.1 dB - 8-ASK: PAPR = 21/15 = 1.4 (1.5 dB)

\textbf{Problem}: High PAPR stresses power amplifier (requires large
backoff)

\begin{center}\rule{0.5\linewidth}{0.5pt}\end{center}

\subsection{Bandwidth Efficiency}\label{bandwidth-efficiency}

\textbf{Occupied bandwidth} (with pulse shaping):

\[
B = (1 + \alpha) R_s \quad (\text{Hz})
\]

Where: - \(R_s\) = Symbol rate (symbols/sec) - \(\alpha\) = Roll-off
factor (0-1, typically 0.35)

\textbf{Bit rate}:

\[
R_b = R_s \log_2(M) \quad (\text{bits/sec})
\]

\textbf{Spectral efficiency}:

\[
\eta = \frac{R_b}{B} = \frac{\log_2(M)}{1 + \alpha} \quad (\text{bits/sec/Hz})
\]

\begin{center}\rule{0.5\linewidth}{0.5pt}\end{center}

\subsubsection{Comparison}\label{comparison}

{\def\LTcaptype{} % do not increment counter
\begin{longtable}[]{@{}llll@{}}
\toprule\noalign{}
Modulation & Bits/symbol & \(\eta\) (\(\alpha\)=0.35) & Notes \\
\midrule\noalign{}
\endhead
\bottomrule\noalign{}
\endlastfoot
\textbf{2-ASK} & 1 & 0.74 & Same as BPSK \\
\textbf{4-ASK} & 2 & 1.48 & 2\$\textbackslash times\$ bandwidth
efficiency \\
\textbf{8-ASK} & 3 & 2.22 & 3\$\textbackslash times\$ bandwidth
efficiency \\
\textbf{16-ASK} & 4 & 2.96 & 4\$\textbackslash times\$ bandwidth
efficiency \\
\end{longtable}
}

\textbf{Trade-off}: Higher \$\textbackslash eta\$ but worse BER (need
higher SNR)

\begin{center}\rule{0.5\linewidth}{0.5pt}\end{center}

\subsection{ASK vs PSK vs QAM}\label{ask-vs-psk-vs-qam}

\textbf{At same bit rate}:

{\def\LTcaptype{} % do not increment counter
\begin{longtable}[]{@{}lllll@{}}
\toprule\noalign{}
Modulation & Bits/symbol & Constellation & BER (@ 10 dB) & Notes \\
\midrule\noalign{}
\endhead
\bottomrule\noalign{}
\endlastfoot
\textbf{BPSK} & 1 & 2 points (1D) & 3.9 \$\textbackslash times\$
10\textbackslash textsuperscript\{-\}\textbackslash textsuperscript\{6\}
& Best BER \\
\textbf{2-ASK} & 1 & 2 points (1D) & 3.9 \$\textbackslash times\$
10\textbackslash textsuperscript\{-\}\textbackslash textsuperscript\{6\}
& Same as BPSK \\
\textbf{QPSK} & 2 & 4 points (2D) & 3.9 \$\textbackslash times\$
10\textbackslash textsuperscript\{-\}\textbackslash textsuperscript\{6\}
& 2\$\textbackslash times\$ efficiency, same BER \\
\textbf{4-ASK} & 2 & 4 points (1D) & 1.8 \$\textbackslash times\$
10\textbackslash textsuperscript\{-\}\textbackslash textsuperscript\{3\}
& Worse BER (1D) \\
\textbf{16-QAM} & 4 & 16 points (2D) &
\textasciitilde10\textbackslash textsuperscript\{-\}\textbackslash textsuperscript\{4\}
& Better than 16-ASK \\
\textbf{16-ASK} & 4 & 16 points (1D) &
\textasciitilde10\textbackslash textsuperscript\{-\}\textbackslash textsuperscript\{2\}
& Worst BER (1D) \\
\end{longtable}
}

\textbf{Key insight}: \textbf{QAM (2D) outperforms ASK (1D)} for M
\textgreater{} 2

\textbf{Why QAM better}: Spreads points in 2D
\$\textbackslash rightarrow\$ Larger minimum distance for same average
power

\begin{center}\rule{0.5\linewidth}{0.5pt}\end{center}

\subsection{Practical Applications}\label{practical-applications}

\subsubsection{1. Optical Communications
(OOK)}\label{optical-communications-ook}

\textbf{Binary ASK (OOK)} dominates fiber optics: - \textbf{10 Gbps
Ethernet}: OOK with direct detection - \textbf{PON (Passive Optical
Networks)}: OOK upstream - \textbf{Simple}: LED/laser ON/OFF, photodiode
detection

\textbf{Higher-order}: 4-ASK (PAM-4) emerging for 100G+ (50 Gbaud PAM-4
= 100 Gbps)

\begin{center}\rule{0.5\linewidth}{0.5pt}\end{center}

\subsubsection{2. RFID (OOK/2-ASK)}\label{rfid-ook2-ask}

\textbf{Passive RFID tags}: OOK modulation - \textbf{Reader
\$\textbackslash rightarrow\$ Tag}: Continuous carrier (powers tag) -
\textbf{Tag \$\textbackslash rightarrow\$ Reader}: Load modulation (OOK)
- \textbf{Frequency}: 125 kHz (LF), 13.56 MHz (HF), 900 MHz (UHF)

\textbf{Advantage}: Non-coherent detection (simple tag circuit)

\begin{center}\rule{0.5\linewidth}{0.5pt}\end{center}

\subsubsection{3. Infrared (IR) Remote
Controls}\label{infrared-ir-remote-controls}

\textbf{Consumer IR}: OOK at 38 kHz carrier - \textbf{Protocol}:
Manchester-encoded OOK - \textbf{Range}: 5-10 meters - \textbf{Power}:
\textless10 mW (eye safety)

\begin{center}\rule{0.5\linewidth}{0.5pt}\end{center}

\subsubsection{4. DSL (Discrete Multi-Tone with
ASK)}\label{dsl-discrete-multi-tone-with-ask}

\textbf{ADSL/VDSL}: Multi-carrier system with per-tone QAM/ASK - Each
subcarrier uses 2-15 bit QAM (includes ASK as subset) - Adaptive bit
loading per tone (waterfilling)

\begin{center}\rule{0.5\linewidth}{0.5pt}\end{center}

\subsubsection{5. Visible Light Communication
(VLC)}\label{visible-light-communication-vlc}

\textbf{LED-based VLC}: OOK or M-ASK - \textbf{OOK}: Simple, high speed
(100+ Mbps) - \textbf{Dimming}: Adjust average amplitude (DC level) -
\textbf{Application}: Indoor positioning, LiFi

\begin{center}\rule{0.5\linewidth}{0.5pt}\end{center}

\subsection{Pulse Shaping}\label{pulse-shaping}

\textbf{Rectangular pulses} cause excessive ISI and spectral regrowth

\textbf{Raised cosine (RC)} pulse:

\[
p(t) = \frac{\sin(\pi t/T_s)}{\pi t/T_s} \cdot \frac{\cos(\alpha \pi t/T_s)}{1 - (2\alpha t/T_s)^2}
\]

\textbf{Properties}: - \textbf{\$\textbackslash alpha\$ = 0}: Sinc pulse
(infinite time, brick-wall spectrum) - \textbf{\$\textbackslash alpha\$
= 0.35}: Common (35\% excess bandwidth) -
\textbf{\$\textbackslash alpha\$ = 1}: Smoother time domain,
2\$\textbackslash times\$ bandwidth

\textbf{Root raised cosine (RRC)}: Split filtering between TX and RX
(matched filter)

\begin{center}\rule{0.5\linewidth}{0.5pt}\end{center}

\subsection{Noise Analysis}\label{noise-analysis}

\textbf{Additive noise} \(n(t)\) with power \(\sigma^2 = N_0/2\)
(single-sided PSD)

\textbf{After coherent demodulation}:

\[
r_m = A_m + n
\]

Where \(n \sim \mathcal{N}(0, \sigma^2)\) (Gaussian noise)

\textbf{Symbol error} if noise pushes sample past threshold:

\textbf{Example (4-ASK, symbol 1 at amplitude A)}: - Error if
\(r_1 < A/2\) (decide symbol 0) or \(r_1 > 3A/2\) (decide symbol 2)

\[
P_{e|1} = Q\left(\frac{A/2}{\sigma}\right) + Q\left(\frac{A/2}{\sigma}\right) = 2Q\left(\frac{A}{2\sigma}\right)
\]

\begin{center}\rule{0.5\linewidth}{0.5pt}\end{center}

\subsection{Implementation
Considerations}\label{implementation-considerations}

\subsubsection{1. Carrier Recovery}\label{carrier-recovery}

\textbf{Coherent ASK} requires carrier phase/frequency sync:

\textbf{Methods}: - \textbf{Squaring loop}: Square signal
\$\textbackslash rightarrow\$ 2f\_c component
\$\textbackslash rightarrow\$ PLL \$\textbackslash rightarrow\$ Divide
by 2 - \textbf{Costas loop}: Feedback loop with I/Q arms - \textbf{Pilot
tone}: Transmit unmodulated carrier (reduces efficiency)

\textbf{See}: {[}{[}Synchronization-(Carrier,-Timing,-Frame){]}{]}

\begin{center}\rule{0.5\linewidth}{0.5pt}\end{center}

\subsubsection{2. Automatic Gain Control
(AGC)}\label{automatic-gain-control-agc}

\textbf{Received amplitude varies} due to fading, path loss:

\textbf{AGC} adjusts receiver gain to maintain constant amplitude:

\[
\text{Gain}(t) = \frac{A_{\text{target}}}{\hat{A}_{\text{received}}(t)}
\]

\textbf{Critical for M-ASK} (M \textgreater{} 2) to maintain threshold
accuracy

\begin{center}\rule{0.5\linewidth}{0.5pt}\end{center}

\subsubsection{3. Nonlinear Distortion}\label{nonlinear-distortion}

\textbf{Power amplifier (PA) nonlinearity} compresses high amplitudes:

\textbf{Effect}: Amplitude levels not equally spaced
\$\textbackslash rightarrow\$ Increased BER

\textbf{Mitigation}: - \textbf{Backoff}: Operate PA below saturation
(reduces efficiency) - \textbf{Predistortion}: Digital or analog
linearization - \textbf{Consider PSK/FSK}: Constant envelope (less
sensitive to PA)

\begin{center}\rule{0.5\linewidth}{0.5pt}\end{center}

\subsubsection{4. Frequency-Selective
Fading}\label{frequency-selective-fading}

\textbf{Multipath fading} distorts amplitude:

\[
r(t) = h(t) \cdot s(t) + n(t)
\]

\textbf{Problem}: Fading gain \(|h(t)|\) multiplies signal amplitude
\$\textbackslash rightarrow\$ ASK especially vulnerable

\textbf{Mitigation}: - \textbf{Equalization}: Compensate for channel
(see {[}{[}Channel-Equalization{]}{]}) - \textbf{OFDM}: Flat fading per
subcarrier - \textbf{Consider PSK}: Less sensitive to amplitude fading
(phase-based)

\begin{center}\rule{0.5\linewidth}{0.5pt}\end{center}

\subsection{Advantages of ASK}\label{advantages-of-ask}

\begin{enumerate}
\def\labelenumi{\arabic{enumi}.}
\tightlist
\item
  \textbf{Simple modulator}: Single mixer, no phase shifter
\item
  \textbf{Non-coherent detection}: Envelope detector (OOK)
\item
  \textbf{Low cost}: Used in RFID, IR remotes
\item
  \textbf{Compatible with intensity modulation}: Optical, VLC (LED
  can\textquotesingle t do phase)
\end{enumerate}

\begin{center}\rule{0.5\linewidth}{0.5pt}\end{center}

\subsection{Disadvantages of ASK}\label{disadvantages-of-ask}

\begin{enumerate}
\def\labelenumi{\arabic{enumi}.}
\tightlist
\item
  \textbf{Poor power efficiency}: 1D constellation
  \$\textbackslash rightarrow\$ Worse BER than 2D (QAM)
\item
  \textbf{Susceptible to fading}: Amplitude-based (fading directly
  affects signal)
\item
  \textbf{Nonlinear PA distortion}: High PAPR, AM-AM conversion
\item
  \textbf{Threshold sensitivity}: AGC critical for M \textgreater{} 2
\item
  \textbf{No advantage over PSK}: For M \textgreater{} 2, PSK/QAM
  preferred (except optical)
\end{enumerate}

\begin{center}\rule{0.5\linewidth}{0.5pt}\end{center}

\subsection{Summary Table}\label{summary-table}

{\def\LTcaptype{} % do not increment counter
\begin{longtable}[]{@{}
  >{\raggedright\arraybackslash}p{(\linewidth - 6\tabcolsep) * \real{0.2286}}
  >{\raggedright\arraybackslash}p{(\linewidth - 6\tabcolsep) * \real{0.3714}}
  >{\raggedright\arraybackslash}p{(\linewidth - 6\tabcolsep) * \real{0.2000}}
  >{\raggedright\arraybackslash}p{(\linewidth - 6\tabcolsep) * \real{0.2000}}@{}}
\toprule\noalign{}
\begin{minipage}[b]{\linewidth}\raggedright
Aspect
\end{minipage} & \begin{minipage}[b]{\linewidth}\raggedright
2-ASK (OOK)
\end{minipage} & \begin{minipage}[b]{\linewidth}\raggedright
4-ASK
\end{minipage} & \begin{minipage}[b]{\linewidth}\raggedright
M-ASK
\end{minipage} \\
\midrule\noalign{}
\endhead
\bottomrule\noalign{}
\endlastfoot
\textbf{Bits/symbol} & 1 & 2 &
log\textbackslash textsubscript\{2\}(M) \\
\textbf{Complexity} & Very simple & Simple & Moderate \\
\textbf{BER @ 10 dB} & 3.9 \$\textbackslash times\$
10\textbackslash textsuperscript\{-\}\textbackslash textsuperscript\{6\}
& 1.8 \$\textbackslash times\$
10\textbackslash textsuperscript\{-\}\textbackslash textsuperscript\{3\}
& Degrades with M \\
\textbf{Detection} & Non-coherent OK & Coherent preferred & Coherent
required \\
\textbf{Applications} & RFID, IR, optical & Rarely used & Optical
(PAM-4) \\
\textbf{vs QAM} & Equivalent (M=2) & 3 dB worse & Much worse \\
\end{longtable}
}

\begin{center}\rule{0.5\linewidth}{0.5pt}\end{center}

\subsection{Related Topics}\label{related-topics}

\begin{itemize}
\tightlist
\item
  \textbf{{[}{[}On-Off-Keying-(OOK){]}{]}}: Binary ASK (M=2)
\item
  \textbf{{[}{[}QPSK-Modulation{]}{]}}: 2D alternative (better than
  4-ASK)
\item
  \textbf{{[}{[}Frequency-Shift-Keying-(FSK){]}{]}}: Frequency-based
  modulation
\item
  \textbf{{[}{[}Constellation-Diagrams{]}{]}}: Visualizing signal space
\item
  \textbf{{[}{[}Bit-Error-Rate-(BER){]}{]}}: Performance metric
\item
  \textbf{{[}{[}Synchronization-(Carrier,-Timing,-Frame){]}{]}}: Carrier
  recovery for coherent detection
\end{itemize}

\begin{center}\rule{0.5\linewidth}{0.5pt}\end{center}

\textbf{Key takeaway}: \textbf{ASK encodes data in amplitude levels.}
OOK (2-ASK) is simple and widely used (RFID, IR, optical). Higher-order
ASK (M \textgreater{} 2) is power-inefficient compared to PSK/QAM due to
1D constellation. QAM dominates RF, ASK dominates optical (intensity
modulation). Coherent detection needed for M \textgreater{} 2.
Susceptible to fading, PA nonlinearity, and amplitude noise.

\begin{center}\rule{0.5\linewidth}{0.5pt}\end{center}

\emph{This wiki is part of the {[}{[}Home\textbar Chimera Project{]}{]}
documentation.}
