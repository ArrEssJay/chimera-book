\section{Antenna Theory Basics}\label{antenna-theory-basics}

{[}{[}Home{]}{]} \textbar{} \textbf{Foundation} \textbar{}
{[}{[}Electromagnetic-Spectrum{]}{]} \textbar{}
{[}{[}Maxwell\textquotesingle s-Equations-\&-Wave-Propagation{]}{]}

\begin{center}\rule{0.5\linewidth}{0.5pt}\end{center}

\subsection{\texorpdfstring{ For Non-Technical
Readers}{ For Non-Technical Readers}}\label{for-non-technical-readers}

\textbf{An antenna is like a funnel for radio waves-\/-\/-it
concentrates energy in one direction (transmit) or collects it from many
directions (receive).}

\textbf{Simple analogies}: - \textbf{Flashlight vs.~bare bulb}: A
flashlight (directional antenna) focuses light. A bare bulb
(omnidirectional) lights up everything. - \textbf{Satellite dish}:
Curved shape collects weak space signals and focuses them onto a tiny
receiver - \textbf{Your cell phone}: Has multiple tiny antennas
inside-\/-\/-cellular, WiFi, GPS, Bluetooth (each tuned to different
frequencies)

\textbf{Key insights}: - \textbf{Bigger = stronger}: 10-meter dish
collects 100\$\textbackslash times\$ more energy than 1-meter dish -
\textbf{Shape matters}: Long wire for AM radio, small stub for WiFi,
dish for satellites - \textbf{Trade-off}: Omnidirectional (WiFi router)
covers whole area but weak. Directional (satellite dish) is strong but
must point exactly right.

\begin{center}\rule{0.5\linewidth}{0.5pt}\end{center}

\subsection{Overview}\label{overview}

An \textbf{antenna} is a transducer that converts \textbf{electrical
signals into electromagnetic waves} (transmit) and vice versa (receive).
Antennas are governed by \textbf{reciprocity}: their transmit and
receive properties are identical.

\textbf{Fundamental principle}: Accelerating charges radiate EM energy
({[}{[}Maxwell\textquotesingle s-Equations-\&-Wave-Propagation\textbar Maxwell\textquotesingle s
equations{]}{]}).

\begin{center}\rule{0.5\linewidth}{0.5pt}\end{center}

\subsection{Key Antenna Parameters}\label{key-antenna-parameters}

\subsubsection{1. Radiation Pattern}\label{radiation-pattern}

\textbf{The spatial distribution of radiated power}.

\textbf{Coordinate system}: - \textbf{Azimuth (\$\textbackslash phi\$)}:
Horizontal angle (0\$\^{}\textbackslash circ\$ -
360\$\^{}\textbackslash circ\$) - \textbf{Elevation
(\$\textbackslash theta\$)}: Vertical angle from zenith
(0\$\^{}\textbackslash circ\$ = straight up)

\textbf{Typical patterns}:

\paragraph{Isotropic Radiator
(Theoretical)}\label{isotropic-radiator-theoretical}

\begin{itemize}
\tightlist
\item
  Radiates \textbf{equally in all directions} (sphere)
\item
  \textbf{Power density} at distance \(r\):
\end{itemize}

\[
S = \frac{P_t}{4\pi r^2}
\]

\begin{itemize}
\tightlist
\item
  \textbf{Does not exist in reality} (used as reference for gain)
\end{itemize}

\begin{center}\rule{0.5\linewidth}{0.5pt}\end{center}

\paragraph{Dipole (\$\textbackslash lambda\$/2)}\label{dipole-ux3bb2}

\textbf{Classic antenna}: Half-wavelength wire

\textbf{Pattern}: - \textbf{Omnidirectional in azimuth}
(\$\textbackslash phi\$) - \textbf{Figure-8 in elevation}
(\$\textbackslash theta\$): Nulls along wire axis

\textbf{3D pattern}: Donut-shaped (toroid)

\textbf{Radiation resistance}: \(R_r = 73\ \Omega\) (lossless)

\begin{center}\rule{0.5\linewidth}{0.5pt}\end{center}

\paragraph{Directional Antennas}\label{directional-antennas}

\textbf{Yagi-Uda} (TV antenna): - Single driven element (dipole) -
Parasitic elements (directors + reflector) - \textbf{Gain}: 10-15 dBi -
\textbf{Beamwidth}: \textasciitilde30-60\$\^{}\textbackslash circ\$

\textbf{Parabolic Dish}: - Large aperture (diameter \(D \gg \lambda\)) -
\textbf{Gain}: 30-60 dBi (satellite comms) - \textbf{Beamwidth}:
\(\theta \approx 70 \lambda / D\) degrees

\textbf{Phased Array}: - Multiple elements with controllable phase -
\textbf{Electronically steerable} beam (no mechanical movement) - Used
in: Radar, 5G base stations, {[}{[}AID-Protocol-Case-Study\textbar AID
Protocol{]}{]} (THz)

\begin{center}\rule{0.5\linewidth}{0.5pt}\end{center}

\subsubsection{2. Antenna Gain (G)}\label{antenna-gain-g}

\textbf{Ratio of power density in preferred direction vs isotropic
radiator}.

\[
G = \frac{S(\theta, \phi)}{S_{\text{iso}}}
\]

\textbf{Units}: dBi (dB relative to isotropic)

\textbf{Typical gains}:

{\def\LTcaptype{} % do not increment counter
\begin{longtable}[]{@{}lll@{}}
\toprule\noalign{}
Antenna Type & Gain (dBi) & Beamwidth \\
\midrule\noalign{}
\endhead
\bottomrule\noalign{}
\endlastfoot
Isotropic (reference) & 0 dBi & 360\$\^{}\textbackslash circ\$ (all
directions) \\
Dipole (\$\textbackslash lambda\$/2) & 2.15 dBi &
\textasciitilde78\$\^{}\textbackslash circ\$ (elevation) \\
Monopole (\$\textbackslash lambda\$/4) & 5.15 dBi &
\textasciitilde30\$\^{}\textbackslash circ\$ (over ground plane) \\
Patch (microstrip) & 6-9 dBi &
\textasciitilde70-90\$\^{}\textbackslash circ\$ \\
Yagi (10 elements) & 12-15 dBi &
\textasciitilde30\$\^{}\textbackslash circ\$ \\
Parabolic dish (1 m @ 10 GHz) & \textasciitilde40 dBi &
\textasciitilde2\$\^{}\textbackslash circ\$ \\
Phased array (64 elements) & 18-24 dBi & Steerable \\
\end{longtable}
}

\textbf{Relationship to directivity}:

\[
G = \eta_{\text{ant}} \cdot D
\]

Where: - \(D\) = Directivity (concentrates power) -
\(\eta_{\text{ant}}\) = Antenna efficiency (0.5-0.95 typical, accounts
for ohmic losses)

\begin{center}\rule{0.5\linewidth}{0.5pt}\end{center}

\subsubsection{3. Directivity (D)}\label{directivity-d}

\textbf{Power concentration factor} (independent of losses):

\[
D = \frac{4\pi}{\Omega_A}
\]

Where \(\Omega_A\) is the \textbf{solid angle} of the main lobe
(steradians).

\textbf{Approximation} for narrow beams:

\[
D \approx \frac{41,253}{\theta_E \cdot \theta_H}
\]

Where: - \(\theta_E\) = Elevation beamwidth (degrees) - \(\theta_H\) =
Azimuth beamwidth (degrees)

\textbf{Example}: Beamwidth 10\$\^{}\textbackslash circ\$
\$\textbackslash times\$ 10\$\^{}\textbackslash circ\$
\$\textbackslash rightarrow\$
\(D = 41,253 / (10 \times 10) = 412.53 \approx 26.2\) dBi

\begin{center}\rule{0.5\linewidth}{0.5pt}\end{center}

\subsubsection{4. Beamwidth}\label{beamwidth}

\textbf{Angular width where power drops to half (-3 dB) of peak}.

\textbf{Half-power beamwidth (HPBW)}:

\[
\theta_{\text{HPBW}} \approx \frac{k \lambda}{D}
\]

Where: - \(D\) = Antenna diameter (aperture antennas) - \(k\) = Constant
(\textasciitilde70\$\^{}\textbackslash circ\$ for parabolic dishes)

\textbf{Example}: 1 m dish at 10 GHz (\(\lambda = 3\) cm):

\[
\theta_{\text{HPBW}} = \frac{70 \times 0.03}{1} = 2.1°
\]

\textbf{Implication}: Narrow beams require \textbf{precise pointing}
(satellites, radar)

\begin{center}\rule{0.5\linewidth}{0.5pt}\end{center}

\subsubsection{5. Polarization}\label{polarization}

\textbf{Orientation of electric field vector}.

\paragraph{Linear Polarization}\label{linear-polarization}

\begin{itemize}
\tightlist
\item
  \textbf{Vertical}: E-field parallel to ground (monopole, vertical
  dipole)
\item
  \textbf{Horizontal}: E-field perpendicular to ground (horizontal
  dipole)
\end{itemize}

\textbf{Cross-polarization loss}: 20-30 dB if TX and RX polarizations
are perpendicular

\begin{center}\rule{0.5\linewidth}{0.5pt}\end{center}

\paragraph{Circular Polarization}\label{circular-polarization}

\begin{itemize}
\tightlist
\item
  \textbf{E-field rotates} as wave propagates
\item
  \textbf{Right-hand circular (RHCP)}: Clockwise (looking at source)
\item
  \textbf{Left-hand circular (LHCP)}: Counter-clockwise
\end{itemize}

\textbf{Applications}: GPS, satellite comms (immune to Faraday rotation
in ionosphere)

\textbf{Axial ratio}: Measure of circularity (0 dB = perfect circular,
\textgreater3 dB = elliptical)

\begin{center}\rule{0.5\linewidth}{0.5pt}\end{center}

\paragraph{Elliptical Polarization}\label{elliptical-polarization}

\begin{itemize}
\tightlist
\item
  General case (between linear and circular)
\item
  Common when reflection/scattering depolarizes signal
\end{itemize}

\begin{center}\rule{0.5\linewidth}{0.5pt}\end{center}

\subsubsection{6. Impedance \& Matching}\label{impedance-matching}

\textbf{Antenna input impedance}:

\[
Z_{\text{ant}} = R_{\text{rad}} + R_{\text{loss}} + jX
\]

Where: - \(R_{\text{rad}}\) = Radiation resistance (power radiated) -
\(R_{\text{loss}}\) = Loss resistance (heat in conductors/dielectrics) -
\(X\) = Reactance (energy storage in near-field)

\textbf{Goal}: Match to transmission line (typically
50\$\textbackslash Omega\$ or 75\$\textbackslash Omega\$)

\begin{center}\rule{0.5\linewidth}{0.5pt}\end{center}

\paragraph{Standing Wave Ratio (SWR)}\label{standing-wave-ratio-swr}

\textbf{Mismatch metric}:

\[
\text{SWR} = \frac{1 + |\Gamma|}{1 - |\Gamma|}
\]

Where \(\Gamma = \frac{Z_{\text{ant}} - Z_0}{Z_{\text{ant}} + Z_0}\)
(reflection coefficient)

\textbf{Acceptable values}: - SWR \textless{} 1.5:1
\$\textbackslash rightarrow\$ Good match (\textless{} 4\% power
reflected) - SWR = 2:1 \$\textbackslash rightarrow\$ Marginal (11\%
reflected) - SWR \textgreater{} 3:1 \$\textbackslash rightarrow\$ Poor
(25\% reflected, may damage TX)

\textbf{Measurement}: Antenna analyzer, network analyzer, SWR meter

\begin{center}\rule{0.5\linewidth}{0.5pt}\end{center}

\subsubsection{7. Bandwidth}\label{bandwidth}

\textbf{Frequency range where antenna performs adequately}.

\textbf{Criteria}: - SWR \textless{} 2:1 - Gain variation \textless{} 3
dB - Pattern distortion minimal

\textbf{Narrowband antennas}: Dipole (2-5\%), loop (1-2\%)
\textbf{Wideband antennas}: Log-periodic (10:1 ratio), biconical
(octave), spiral (decade+)

\textbf{Example}: WiFi 2.4 GHz (2.4-2.5 GHz = 4\% bandwidth)
\$\textbackslash rightarrow\$ Simple patch works \textbf{Example}: UWB
radar (3-10 GHz = 107\% fractional BW) \$\textbackslash rightarrow\$
Needs spiral or horn

\begin{center}\rule{0.5\linewidth}{0.5pt}\end{center}

\subsubsection{\texorpdfstring{8. Effective Aperture
(\(A_e\))}{8. Effective Aperture (A\_e)}}\label{effective-aperture-a_e}

\textbf{Equivalent capture area} for receiving antennas:

\[
A_e = \frac{G \lambda^2}{4\pi}
\]

\textbf{Physical interpretation}: Power received = Incident power
density \$\textbackslash times\$ \(A_e\)

\[
P_r = S \cdot A_e
\]

\textbf{Example}: Dipole (\(G = 2.15\) dBi = 1.64 linear) at 1 GHz
(\(\lambda = 0.3\) m):

\[
A_e = \frac{1.64 \times (0.3)^2}{4\pi} = 0.0125\ \text{m}^2
\]

\textbf{Aperture efficiency}:
\(\eta_{\text{ap}} = A_e / A_{\text{phys}}\) (0.5-0.7 for dishes)

\begin{center}\rule{0.5\linewidth}{0.5pt}\end{center}

\subsection{Antenna Types by
Application}\label{antenna-types-by-application}

\subsubsection{1. Communication Antennas}\label{communication-antennas}

\paragraph{Dipole (VHF/UHF)}\label{dipole-vhfuhf}

\begin{itemize}
\tightlist
\item
  \textbf{Simple, cheap, omnidirectional}
\item
  \textbf{Use}: FM broadcast, amateur radio, WiFi (2.4 GHz diversity
  antennas)
\end{itemize}

\paragraph{Patch (Microstrip)}\label{patch-microstrip}

\begin{itemize}
\tightlist
\item
  \textbf{Flat, low-profile, easy to integrate}
\item
  \textbf{Use}: GPS, cellular, WiFi (5 GHz), IoT devices
\end{itemize}

\paragraph{Yagi-Uda}\label{yagi-uda}

\begin{itemize}
\tightlist
\item
  \textbf{Directional, moderate gain}
\item
  \textbf{Use}: TV reception, point-to-point links, amateur radio
\end{itemize}

\paragraph{Parabolic Dish}\label{parabolic-dish}

\begin{itemize}
\tightlist
\item
  \textbf{High gain, narrow beam}
\item
  \textbf{Use}: Satellite TV (12 GHz), deep-space comms (Ka-band), radio
  astronomy
\end{itemize}

\begin{center}\rule{0.5\linewidth}{0.5pt}\end{center}

\subsubsection{2. Mobile/Wearable
Antennas}\label{mobilewearable-antennas}

\paragraph{Monopole
(\$\textbackslash lambda\$/4)}\label{monopole-ux3bb4}

\begin{itemize}
\tightlist
\item
  \textbf{Requires ground plane} (vehicle roof, PCB)
\item
  \textbf{Use}: Car antennas, handheld radios
\end{itemize}

\paragraph{PIFA (Planar Inverted-F
Antenna)}\label{pifa-planar-inverted-f-antenna}

\begin{itemize}
\tightlist
\item
  \textbf{Compact, dual-band}
\item
  \textbf{Use}: Smartphones (cellular + WiFi)
\end{itemize}

\paragraph{Loop Antenna}\label{loop-antenna}

\begin{itemize}
\tightlist
\item
  \textbf{Small, magnetic field dominant}
\item
  \textbf{Use}: RFID tags, NFC, AM radio (ferrite bar)
\end{itemize}

\begin{center}\rule{0.5\linewidth}{0.5pt}\end{center}

\subsubsection{3. Phased Arrays}\label{phased-arrays}

\textbf{Multiple elements with controllable phase/amplitude}:

\textbf{Advantages}: - \textbf{Electronic beam steering} (no moving
parts) - \textbf{Adaptive nulling} (cancel interference) - \textbf{MIMO}
(spatial multiplexing)

\textbf{Beam steering}:

\[
\theta = \sin^{-1}\left(\frac{\phi \lambda}{2\pi d}\right)
\]

Where: - \(\phi\) = Phase shift between elements - \(d\) = Element
spacing

\textbf{Applications}: - Radar (military, automotive 77 GHz) - 5G base
stations (massive MIMO, 64-256 elements) -
{[}{[}AID-Protocol-Case-Study\textbar AID Protocol{]}{]} (THz phased
array for coherent combining)

\begin{center}\rule{0.5\linewidth}{0.5pt}\end{center}

\subsection{Friis Transmission
Equation}\label{friis-transmission-equation}

\textbf{Link budget fundamental} (connects antennas to
{[}{[}Free-Space-Path-Loss-(FSPL)\textbar path loss{]}{]}):

\[
P_r = P_t + G_t + G_r - L_{\text{FSPL}}
\]

(in dB)

Or in linear form:

\[
P_r = P_t \cdot G_t \cdot G_r \cdot \left(\frac{\lambda}{4\pi d}\right)^2
\]

\textbf{Derivation}:

\begin{enumerate}
\def\labelenumi{\arabic{enumi}.}
\tightlist
\item
  TX power \(P_t\) radiated isotropically \$\textbackslash rightarrow\$
  Power density at distance \(d\):
\end{enumerate}

\[
S_{\text{iso}} = \frac{P_t}{4\pi d^2}
\]

\begin{enumerate}
\def\labelenumi{\arabic{enumi}.}
\setcounter{enumi}{1}
\tightlist
\item
  TX antenna gain \(G_t\) concentrates power:
\end{enumerate}

\[
S = \frac{P_t G_t}{4\pi d^2}
\]

\begin{enumerate}
\def\labelenumi{\arabic{enumi}.}
\setcounter{enumi}{2}
\tightlist
\item
  RX antenna effective aperture \(A_e = G_r \lambda^2 / 4\pi\) captures
  power:
\end{enumerate}

\[
P_r = S \cdot A_e = \frac{P_t G_t}{4\pi d^2} \cdot \frac{G_r \lambda^2}{4\pi}
\]

\begin{enumerate}
\def\labelenumi{\arabic{enumi}.}
\setcounter{enumi}{3}
\tightlist
\item
  Simplify:
\end{enumerate}

\[
P_r = P_t G_t G_r \left(\frac{\lambda}{4\pi d}\right)^2
\]

\textbf{Key insight}: Antenna gain \textbf{adds} to link budget (in dB),
compensating for path loss.

\begin{center}\rule{0.5\linewidth}{0.5pt}\end{center}

\subsection{Antenna Design by
Frequency}\label{antenna-design-by-frequency}

\subsubsection{VLF/LF (\textless{} 300 kHz)}\label{vlflf-300-khz}

\textbf{Challenge}: Wavelength \textgreater\textgreater{} practical
antenna size

\textbf{Solution}: - \textbf{Electrically small antennas} (length
\(\ll \lambda\)) - \textbf{Low efficiency} (most power lost in ohmic
resistance) - \textbf{Loading coils} to resonate (match reactance)

\textbf{Example}: 100 kHz (\$\textbackslash lambda\$ = 3000 m), 10 m
vertical monopole: - Radiation resistance: \textasciitilde0.1
\$\textbackslash Omega\$ - Loss resistance: \textasciitilde10
\$\textbackslash Omega\$ - Efficiency: \textasciitilde1\%

\begin{center}\rule{0.5\linewidth}{0.5pt}\end{center}

\subsubsection{HF/VHF (3-300 MHz)}\label{hfvhf-3-300-mhz}

\textbf{Sweet spot}: Antennas are practical size

\textbf{Common types}: - Dipole (\$\textbackslash lambda\$/2): 50 m @ 3
MHz, 1 m @ 150 MHz - Monopole (\$\textbackslash lambda\$/4): 25 m @ 3
MHz (vertical tower) - Yagi-Uda: TV reception (VHF channels)

\textbf{Efficiency}: 50-90\% (good conductors, minimal loss)

\begin{center}\rule{0.5\linewidth}{0.5pt}\end{center}

\subsubsection{UHF/SHF (300 MHz - 30
GHz)}\label{uhfshf-300-mhz---30-ghz}

\textbf{Miniaturization}: Antennas fit on PCBs

\textbf{Common types}: - Patch (microstrip): 3 cm
\$\textbackslash times\$ 3 cm @ 2.4 GHz - Slot: Waveguide-based (radar,
satellite) - Horn: Wideband, calibration standard

\textbf{Phased arrays become feasible}: Element spacing
\(d \sim \lambda/2\)

\textbf{Example}: 10 GHz, \(\lambda = 3\) cm
\$\textbackslash rightarrow\$ 1.5 cm spacing
\$\textbackslash rightarrow\$ 100 elements in 15 cm
\$\textbackslash times\$ 15 cm

\begin{center}\rule{0.5\linewidth}{0.5pt}\end{center}

\subsubsection{EHF/THz (30 GHz - 10 THz)}\label{ehfthz-30-ghz---10-thz}

\textbf{Challenges}: - \textbf{Fabrication tolerance}
(\$\textbackslash mu\$m precision required) - \textbf{Surface roughness
losses} (skin depth at THz \textasciitilde{} nm) - \textbf{Impedance
matching} difficult (high frequencies)

\textbf{Solutions}: - \textbf{On-chip antennas} (silicon, III-V
semiconductors) - \textbf{Photolithography} (THz: \textless100
\$\textbackslash mu\$m features) - \textbf{Lens-coupled antennas} (match
impedance to free space)

\textbf{Example}: 1.875 THz (AID protocol), \(\lambda = 160\)
\$\textbackslash mu\$m: - Dipole: 80 \$\textbackslash mu\$m (fabricated
via e-beam lithography) - Phased array: 40 \$\textbackslash mu\$m
spacing, 1024 elements in 40 mm \$\textbackslash times\$ 40 mm

\begin{center}\rule{0.5\linewidth}{0.5pt}\end{center}

\subsection{Antenna Measurements}\label{antenna-measurements}

\subsubsection{Anechoic Chamber}\label{anechoic-chamber}

\textbf{Facility for measuring radiation patterns}:

\begin{itemize}
\tightlist
\item
  \textbf{Absorber walls}: Eliminate reflections (simulate free space)
\item
  \textbf{Turntable}: Rotate antenna under test (AUT)
\item
  \textbf{Reference antenna}: Known gain/pattern
\item
  \textbf{Network analyzer}: Measure
  S\textbackslash textsubscript\{2\}\textbackslash textsubscript\{1\}
  (transmission) vs angle
\end{itemize}

\textbf{Far-field distance}: \(d > 2D^2/\lambda\) (Fraunhofer region)

\textbf{Example}: 1 m dish @ 10 GHz \$\textbackslash rightarrow\$
\(d > 2 \times 1^2 / 0.03 = 67\) m (large chamber!)

\begin{center}\rule{0.5\linewidth}{0.5pt}\end{center}

\subsubsection{Near-Field Scanning}\label{near-field-scanning}

\textbf{For electrically large antennas} (where far-field distance is
impractical):

\begin{enumerate}
\def\labelenumi{\arabic{enumi}.}
\tightlist
\item
  \textbf{Scan E/H fields} on planar/cylindrical/spherical surface near
  antenna
\item
  \textbf{FFT transform} to compute far-field pattern
\item
  \textbf{Smaller chamber} required (1-2 m)
\end{enumerate}

\begin{center}\rule{0.5\linewidth}{0.5pt}\end{center}

\subsubsection{Gain Measurement (Comparison
Method)}\label{gain-measurement-comparison-method}

\begin{enumerate}
\def\labelenumi{\arabic{enumi}.}
\tightlist
\item
  Measure received power with \textbf{standard gain horn} (calibrated)
\item
  Replace with \textbf{antenna under test} (AUT)
\item
  Compare powers:
\end{enumerate}

\[
G_{\text{AUT}} = G_{\text{std}} + (P_{\text{AUT}} - P_{\text{std}})
\]

(in dB)

\begin{center}\rule{0.5\linewidth}{0.5pt}\end{center}

\subsection{Practical Design
Considerations}\label{practical-design-considerations}

\subsubsection{1. Matching Network}\label{matching-network}

\textbf{Goal}: Transform antenna impedance to 50\$\textbackslash Omega\$

\textbf{Techniques}: - \textbf{LC network}: Series/shunt
inductors/capacitors - \textbf{Quarter-wave transformer}:
\(Z_{\lambda/4} = \sqrt{Z_0 Z_{\text{ant}}}\) - \textbf{Stub matching}:
Open/short-circuited transmission line stubs

\textbf{Example}: Dipole (\(Z = 73 + j42.5\ \Omega\)) to
50\$\textbackslash Omega\$: - Add series capacitor to cancel reactance
(j42.5\$\textbackslash Omega\$) - Use transformer to match
73\$\textbackslash Omega\$ to 50\$\textbackslash Omega\$

\begin{center}\rule{0.5\linewidth}{0.5pt}\end{center}

\subsubsection{2. Balun (Balanced-Unbalanced
Transformer)}\label{balun-balanced-unbalanced-transformer}

\textbf{Problem}: Coaxial cable (unbalanced) feeding dipole (balanced)
\$\textbackslash rightarrow\$ Common-mode currents on outer shield
(pattern distortion)

\textbf{Solution}: Balun isolates antenna from feedline

\textbf{Types}: - \textbf{Choke balun}: Coil of coax (high impedance to
common-mode) - \textbf{Sleeve balun}: \$\textbackslash lambda\$/4 sleeve
over coax - \textbf{Transformer balun}: 1:1 or 4:1 turns ratio (ferrite
core)

\begin{center}\rule{0.5\linewidth}{0.5pt}\end{center}

\subsubsection{3. Environmental Effects}\label{environmental-effects}

\paragraph{Ground Plane}\label{ground-plane}

\begin{itemize}
\tightlist
\item
  \textbf{Monopole requires ground plane} (acts as mirror image)
\item
  \textbf{Poor ground} (dry soil, concrete)
  \$\textbackslash rightarrow\$ Reduced efficiency
\item
  \textbf{Elevated radials} (4-8 wires, \$\textbackslash lambda\$/4
  length) improve performance
\end{itemize}

\paragraph{Nearby Objects}\label{nearby-objects}

\begin{itemize}
\tightlist
\item
  \textbf{Metal structures}: Detune antenna (shift resonance), reflect
  energy
\item
  \textbf{Human body}: Lossy dielectric (especially at cellular
  frequencies) \$\textbackslash rightarrow\$ Detuning, absorption
\item
  \textbf{Solution}: Antenna placement away from body (smartphones:
  top/bottom), adaptive matching
\end{itemize}

\begin{center}\rule{0.5\linewidth}{0.5pt}\end{center}

\subsection{Summary: Key Antenna
Formulas}\label{summary-key-antenna-formulas}

{\def\LTcaptype{} % do not increment counter
\begin{longtable}[]{@{}
  >{\raggedright\arraybackslash}p{(\linewidth - 4\tabcolsep) * \real{0.4074}}
  >{\raggedright\arraybackslash}p{(\linewidth - 4\tabcolsep) * \real{0.3333}}
  >{\raggedright\arraybackslash}p{(\linewidth - 4\tabcolsep) * \real{0.2593}}@{}}
\toprule\noalign{}
\begin{minipage}[b]{\linewidth}\raggedright
Parameter
\end{minipage} & \begin{minipage}[b]{\linewidth}\raggedright
Formula
\end{minipage} & \begin{minipage}[b]{\linewidth}\raggedright
Units
\end{minipage} \\
\midrule\noalign{}
\endhead
\bottomrule\noalign{}
\endlastfoot
\textbf{Gain} & \(G = \eta_{\text{ant}} \cdot D\) & Linear or dBi \\
\textbf{Effective aperture} & \(A_e = \frac{G\lambda^2}{4\pi}\) &
m\textbackslash textsuperscript\{2\} \\
\textbf{Beamwidth (aperture)} & \(\theta \approx 70\lambda/D\) &
Degrees \\
\textbf{Directivity (narrow beam)} &
\(D \approx 41253/(\theta_E \theta_H)\) & Linear \\
\textbf{Friis equation} & \(P_r = P_t G_t G_r (\lambda/4\pi d)^2\) &
Watts \\
\textbf{FSPL} & \(L = 20\log(d) + 20\log(f) + 92.45\) & dB \\
\textbf{Radiation resistance (dipole)} & \(R_r = 73\ \Omega\) & Ohms \\
\textbf{SWR} & \(\text{SWR} = (1+|\Gamma|)/(1-|\Gamma|)\) & Ratio \\
\end{longtable}
}

\begin{center}\rule{0.5\linewidth}{0.5pt}\end{center}

\subsection{Related Topics}\label{related-topics}

\begin{itemize}
\tightlist
\item
  \textbf{{[}{[}Free-Space-Path-Loss-(FSPL){]}{]}}: Quantifies
  distance-dependent loss (uses antenna gains)
\item
  \textbf{{[}{[}Electromagnetic-Spectrum{]}{]}}: Frequency-dependent
  antenna design
\item
  \textbf{{[}{[}Maxwell\textquotesingle s-Equations-\&-Wave-Propagation{]}{]}}:
  Radiation mechanism
\item
  \textbf{{[}{[}Signal-to-Noise-Ratio-(SNR){]}{]}}: Antenna gain
  improves SNR
\item
  \textbf{{[}{[}AID-Protocol-Case-Study{]}{]}}: THz phased array example
  (1.875 THz, 40 dB gain)
\item
  \textbf{Propagation Modes}: How antennas couple to environment (TBD)
\item
  \textbf{Multipath \& Fading}: Antenna diversity, MIMO (TBD)
\end{itemize}

\begin{center}\rule{0.5\linewidth}{0.5pt}\end{center}

\textbf{Next}: \textbf{Binary Phase-Shift Keying (BPSK)} (TBD) -
Simplest phase modulation, bridge to {[}{[}QPSK-Modulation{]}{]}

\begin{center}\rule{0.5\linewidth}{0.5pt}\end{center}

\emph{This wiki is part of the {[}{[}Home\textbar Chimera Project{]}{]}
documentation.}
