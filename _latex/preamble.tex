% ==============================================================================
% Book Preamble
% ==============================================================================

% --- Font and Encoding ---
\usepackage{ebgaramond} % EB Garamond - elegant open-source old-style serif for body and headings
\usepackage[cmintegrals,cmbraces]{newtxmath} % Math support matching EB Garamond
\usepackage[T1]{fontenc}
\usepackage[utf8]{inputenc}
% --- Font and Encoding ---
\usepackage{fontspec} % For OTF font support (requires XeLaTeX or LuaLaTeX)

% Minion Pro with explicit font file names from fc-list
% Fallback to Latin Modern Roman if Minion Pro is not available
\IfFontExistsTF{MinionPro-Regular.otf}{
  \setmainfont{MinionPro}[
    Extension = .otf,
    Path = /Users/rowan/Library/Fonts/,
    UprightFont = *-Regular,
    ItalicFont = *-It,
    BoldFont = *-Bold,
    BoldItalicFont = *-BoldIt
  ]
}{
  \setmainfont{Latin Modern Roman}
}

% Myriad Pro for headings - fallback to Latin Modern Sans
\IfFontExistsTF{Myriad Pro}{
  \setsansfont{Myriad Pro}
}{
  \setsansfont{Latin Modern Sans}
}

% Monospace for code
\setmonofont{Latin Modern Mono}

\usepackage{textcomp} % For additional symbols
\usepackage{newunicodechar} % For handling Unicode characters

% Define remaining Unicode characters that may legitimately appear in content
\newunicodechar{✓}{\checkmark}
\newunicodechar{✗}{\texttimes}
\newunicodechar{°}{\ifmmode^\circ\else\textdegree\fi}  % Degree symbol (math/text mode)
\newunicodechar{⚠}{\textbf{!}}  % Warning symbol


% --- Page Layout ---
% 7x10 inch - standard for technical textbooks with diagrams and equations
\usepackage[paperwidth=7in, paperheight=10in, top=0.75in, bottom=0.75in, left=0.875in, right=0.875in, headheight=14.5pt]{geometry}
\usepackage{fancyhdr} % For custom headers and footers
\pagestyle{fancy}
\fancyhf{} % Clear all header and footer fields
\fancyhead[LE,RO]{\thepage}
\fancyhead[RE]{\nouppercase{\leftmark}}
\fancyhead[LO]{\nouppercase{\rightmark}}
\renewcommand{\headrulewidth}{0.4pt}
\renewcommand{\footrulewidth}{0pt}

% --- Language and Typography ---
\usepackage[english]{babel}
\usepackage{microtype} % Improves typography and justification

% --- Math and Symbols ---
\usepackage{amsmath, amssymb, amsfonts}

% --- Graphics and Colors ---
\usepackage{graphicx}
\usepackage{pdfpages} % For full-page PDF/image inclusion with no margins
\usepackage[dvipsnames, svgnames]{xcolor}
\setkeys{Gin}{width=\linewidth,height=0.8\textheight,keepaspectratio}

% TikZ for diagrams
\usepackage{tikz}
\usetikzlibrary{shapes.geometric, arrows.meta, positioning, calc, patterns, decorations.pathmorphing}

% --- Index ---
\usepackage{makeidx}
\makeindex

% --- Hyperlinks and Referencing ---
\usepackage[unicode=true, colorlinks=true, linkcolor=NavyBlue, urlcolor=NavyBlue, citecolor=Green]{hyperref}
\urlstyle{same}
\usepackage[nameinlink]{cleveref} % For smart cross-referencing

% --- Bibliography ---
\usepackage{csquotes} % Recommended by biblatex for proper quotation handling
\usepackage[backend=biber, style=numeric, sorting=none]{biblatex}
\addbibresource{bibliography.bib} % Assumes bibliography.bib in the same directory

% --- Code Blocks ---
\usepackage{minted} % For syntax highlighting, requires -shell-escape
\setminted{
  fontsize=\small,
  linenos,
  frame=lines,
  breaklines,
  breakanywhere,
  style=friendly
}

% Pandoc Shaded environment for code blocks
\usepackage{framed}
\newenvironment{Shaded}{\begin{snugshade}}{\end{snugshade}}
\definecolor{shadecolor}{RGB}{248,248,248}

% Pandoc Highlighting environment for syntax highlighting
\newenvironment{Highlighting}{\begin{Shaded}\small}{\end{Shaded}}

% --- Callouts/Framed Boxes ---
\usepackage[most]{tcolorbox}

% Professional callout box with optional custom styling
% Sharp corners for professional print appearance
\newtcolorbox{calloutbox}[2][]{
  colback=black!5!white,
  colframe=black!75!white,
  fonttitle=\bfseries\sffamily,
  title=#2,
  boxrule=0.75pt,
  sharp corners,
  left=8pt,
  right=8pt,
  top=6pt,
  bottom=6pt,
  toptitle=3pt,
  bottomtitle=3pt,
  #1
}

% Non-technical reader callout
\newtcolorbox{nontechnical}{
  colback=black!3!white,
  colframe=black!50!white,
  fonttitle=\bfseries\sffamily,
  title=FOR NON-TECHNICAL READERS,
  boxrule=1pt,
  sharp corners,
  left=10pt,
  right=10pt,
  top=8pt,
  bottom=8pt,
  toptitle=4pt,
  bottomtitle=4pt
}

% Key concept callout
\newtcolorbox{keyconcept}[1][]{
  colback=black!8!white,
  colframe=black!75!white,
  fonttitle=\bfseries\sffamily,
  title=KEY CONCEPT,
  boxrule=1.5pt,
  sharp corners,
  left=8pt,
  right=8pt,
  top=6pt,
  bottom=6pt,
  toptitle=3pt,
  bottomtitle=3pt,
  #1
}

% Warning callout (double box for emphasis, B/W compatible)
\newtcolorbox{warningbox}[1][]{
  colback=white,
  colframe=black,
  fonttitle=\bfseries\sffamily,
  title=\textbf{⚠}\ CRITICAL REQUIREMENT,
  boxrule=2pt,
  sharp corners,
  left=8pt,
  right=8pt,
  top=6pt,
  bottom=6pt,
  toptitle=3pt,
  bottomtitle=3pt,
  #1
}

% Important note callout
\newtcolorbox{importantbox}[1][]{
  colback=black!5!white,
  colframe=black,
  fonttitle=\bfseries\sffamily,
  title=IMPORTANT,
  boxrule=1pt,
  sharp corners,
  left=8pt,
  right=8pt,
  top=6pt,
  bottom=6pt,
  toptitle=3pt,
  bottomtitle=3pt,
  #1
}

% --- Tables ---
\usepackage{array} % Provides \arraybackslash
\usepackage{longtable}
\usepackage{booktabs}
\usepackage{tabularx} % For automatic column width calculation
\usepackage{calc} % For calculations in table column widths

% Define \real for pandoc-generated table column widths
\providecommand{\real}[1]{#1}

% --- Lists ---
\usepackage{enumitem}
\setlist{noitemsep, topsep=5pt}

% Pandoc compatibility - define \tightlist for compact lists
\providecommand{\tightlist}{%
  \setlength{\itemsep}{0pt}\setlength{\parskip}{0pt}}

% --- Chapter and Section Styling ---
\usepackage{titlesec}

% Myriad Pro for headings - clean, modern humanist sans-serif
% Minion Pro for body - refined old-style serif elegance
% Less is more - subtle, professional typography
\titleformat{\chapter}[display]
  {\sffamily\bfseries\Large}
  {\chaptertitlename\ \thechapter}{12pt}{\fontsize{24}{28}\selectfont}
\titlespacing*{\chapter}{0pt}{-30pt}{20pt}

\titleformat{\section}
  {\sffamily\bfseries\large}
  {\thesection}{1em}{}

\titleformat{\subsection}
  {\sffamily\bfseries\normalsize}
  {\thesubsection}{1em}{}

\titleformat{\subsubsection}
  {\sffamily\bfseries\normalsize}
  {\thesubsubsection}{1em}{}

% --- Numbering Depth ---
% Number chapters, sections, subsections, and subsubsections
\setcounter{secnumdepth}{3}

% --- Mermaid Diagrams ---
% This requires a build script to convert .mmd files to .pdf
% Usage: \includemermaid{diagram_filename_without_extension}
\newcommand{\includemermaid}[1]{%
  \begin{figure}[h!]
    \centering
    \includegraphics[width=0.8\textwidth]{#1.pdf}
  \end{figure}
}

% --- CC License Icon ---
\usepackage{ccicons}
