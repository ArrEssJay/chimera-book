\section{Weather Effects: Rain Fade \& Fog
Attenuation}\label{weather-effects-rain-fade-fog-attenuation}

{[}{[}Home{]}{]} \textbar{} \textbf{RF Propagation} \textbar{}
{[}{[}Free-Space-Path-Loss-(FSPL){]}{]} \textbar{}
{[}{[}Atmospheric-Effects-(Ionospheric,-Tropospheric){]}{]}

\begin{center}\rule{0.5\linewidth}{0.5pt}\end{center}

\subsection{For Non-Technical Readers}\label{for-non-technical-readers}

\textbf{Think of radio waves like light beams traveling through the
air.}

When it rains, you notice that: - \textbf{Headlights look dimmer}
through heavy rain - \textbf{You can\textquotesingle t see as far} in
fog - \textbf{Everything gets blurry} during a storm

\textbf{The exact same thing happens to satellite TV, 5G cell signals,
and WiFi} -\/-\/- but you can\textquotesingle t see it with your eyes.

\subsubsection{The Core Problem}\label{the-core-problem}

\textbf{Raindrops absorb and scatter radio waves}, weakening the signal.
The bigger the problem when:

\begin{enumerate}
\def\labelenumi{\arabic{enumi}.}
\tightlist
\item
  \textbf{Higher frequencies} (like 5G\textquotesingle s ``millimeter
  wave'') are used

  \begin{itemize}
  \tightlist
  \item
    Think: FM radio (low frequency) works fine in rain, but satellite
    internet (high frequency) struggles
  \end{itemize}
\item
  \textbf{Heavier rain} falls

  \begin{itemize}
  \tightlist
  \item
    Light drizzle: barely noticeable
  \item
    Thunderstorm: your satellite dish might lose connection entirely
  \end{itemize}
\item
  \textbf{Longer distances} through the weather

  \begin{itemize}
  \tightlist
  \item
    Short WiFi connection (30 feet): rain doesn\textquotesingle t matter
    much
  \item
    Satellite signal (22,000 miles up): crosses miles of rain clouds
  \end{itemize}
\end{enumerate}

\subsubsection{Real-World Examples You\textquotesingle ve
Experienced}\label{real-world-examples-youve-experienced}

\textbf{ Satellite TV going out during storms} - Your dish is trying to
receive a signal from space - Heavy rain blocks 50-90\% of the signal
strength - Below a threshold \$\textbackslash rightarrow\$ ``Searching
for signal\textbackslash ldots\{\}'' - This is called \textbf{rain fade}

\textbf{ Slower 5G during rain} - 5G ``mmWave'' uses very high
frequencies (like satellite dishes) - Rain weakens the signal between
tower and your phone - Phone automatically switches to slower but more
reliable 4G - You don\textquotesingle t notice the rain, just ``slower
internet''

\textbf{ Why your GPS still works} - GPS uses lower frequencies (1.5
GHz) than satellite TV (12+ GHz) - Rain barely affects it (like how
radio stations work in any weather)

\subsubsection{The Math Part (Optional)}\label{the-math-part-optional}

The technical sections below answer questions like: - \textbf{``Exactly
how much weaker?''} (e.g., 10 dB loss = 90\% power lost) - \textbf{``At
what frequency does rain start mattering?''} (\textasciitilde10 GHz
threshold) - \textbf{``How do engineers design systems that work in
rain?''} (Add backup power, use multiple frequencies, switch to lower
data rates)

\textbf{You don\textquotesingle t need to understand the equations} to
grasp the main point:

\begin{quote}
\textbf{Rain affects high-frequency radio signals a lot, low-frequency
signals barely at all. Engineers compensate by adding extra power, using
smarter antennas, or accepting slower speeds during storms.}
\end{quote}

\begin{center}\rule{0.5\linewidth}{0.5pt}\end{center}

\subsection{Overview}\label{overview}

\textbf{Weather significantly impacts RF propagation}, especially at
frequencies above 10 GHz. Rain, fog, snow, and clouds introduce
\textbf{frequency-dependent attenuation} that must be accounted for in
link budgets.

\textbf{Key principle}: Attenuation increases with: 1.
\textbf{Frequency} (higher frequencies = more attenuation) 2.
\textbf{Precipitation rate} (heavier rain = more loss) 3. \textbf{Path
length through weather} (longer distance = more cumulative loss)

\textbf{Critical for}: Satellite communications (Ku/Ka/V-band), 5G
mmWave (28/39 GHz), point-to-point microwave links

\begin{center}\rule{0.5\linewidth}{0.5pt}\end{center}

\subsection{Rain Attenuation}\label{rain-attenuation}

\subsubsection{Physical Mechanism}\label{physical-mechanism}

\textbf{Raindrops act as lossy dielectric spheres}:

\begin{enumerate}
\def\labelenumi{\arabic{enumi}.}
\tightlist
\item
  \textbf{Absorption}: EM energy heats water molecules (dielectric loss)
\item
  \textbf{Scattering}: Raindrops redirect energy out of main beam (Mie
  scattering when droplet size \$\textbackslash approx\$
  \$\textbackslash lambda\$)
\end{enumerate}

\textbf{Frequency dependence}: - \textbf{\textless{} 10 GHz}: Rain
effects negligible (\$\textbackslash lambda\$ \textgreater\textgreater{}
raindrop size) - \textbf{10-100 GHz}: Strong attenuation
(\$\textbackslash lambda\$ \$\textbackslash approx\$ raindrop size, 1-5
mm) - \textbf{\textgreater{} 100 GHz}: Extreme attenuation (THz
communications impossible in rain)

\begin{center}\rule{0.5\linewidth}{0.5pt}\end{center}

\subsubsection{ITU-R Rain Attenuation
Model}\label{itu-r-rain-attenuation-model}

\textbf{Standard method}: ITU-R P.838 and P.618

\textbf{Specific attenuation} (dB/km):

\[
\gamma_R = k \cdot R^\alpha
\]

Where: - \(\gamma_R\) = Specific attenuation (dB/km) - \(R\) = Rain rate
(mm/hr) - \(k, \alpha\) = Frequency-dependent coefficients (from ITU
tables)

\begin{center}\rule{0.5\linewidth}{0.5pt}\end{center}

\paragraph{Coefficients by Frequency}\label{coefficients-by-frequency}

\textbf{Selected values} (horizontal polarization):

{\def\LTcaptype{} % do not increment counter
\begin{longtable}[]{@{}llll@{}}
\toprule\noalign{}
Frequency & \(k\) & \(\alpha\) & Attenuation @ 25 mm/hr rain \\
\midrule\noalign{}
\endhead
\bottomrule\noalign{}
\endlastfoot
1 GHz & 0.0000387 & 0.912 & 0.0005 dB/km \\
4 GHz & 0.00065 & 1.121 & 0.025 dB/km \\
10 GHz & 0.0101 & 1.276 & 0.50 dB/km \\
12 GHz (Ku) & 0.0188 & 1.310 & 1.02 dB/km \\
20 GHz (Ka) & 0.0751 & 1.099 & 3.26 dB/km \\
30 GHz & 0.187 & 1.021 & 7.14 dB/km \\
40 GHz & 0.350 & 0.939 & 12.2 dB/km \\
50 GHz & 0.536 & 0.873 & 16.8 dB/km \\
60 GHz & 0.707 & 0.826 & 20.3 dB/km \\
80 GHz & 0.999 & 0.784 & 26.4 dB/km \\
100 GHz & 1.187 & 0.751 & 29.8 dB/km \\
\end{longtable}
}

\textbf{Note}: Vertical polarization has slightly different coefficients
(typically \textasciitilde10-20\% more attenuation)

\begin{center}\rule{0.5\linewidth}{0.5pt}\end{center}

\subsubsection{Rain Rate
Classifications}\label{rain-rate-classifications}

\textbf{ITU rain zones} (global climate regions):

{\def\LTcaptype{} % do not increment counter
\begin{longtable}[]{@{}
  >{\raggedright\arraybackslash}p{(\linewidth - 6\tabcolsep) * \real{0.0882}}
  >{\raggedright\arraybackslash}p{(\linewidth - 6\tabcolsep) * \real{0.1324}}
  >{\raggedright\arraybackslash}p{(\linewidth - 6\tabcolsep) * \real{0.5000}}
  >{\raggedright\arraybackslash}p{(\linewidth - 6\tabcolsep) * \real{0.2794}}@{}}
\toprule\noalign{}
\begin{minipage}[b]{\linewidth}\raggedright
Zone
\end{minipage} & \begin{minipage}[b]{\linewidth}\raggedright
Climate
\end{minipage} & \begin{minipage}[b]{\linewidth}\raggedright
Rain rate exceeded 0.01\% of year
\end{minipage} & \begin{minipage}[b]{\linewidth}\raggedright
Example locations
\end{minipage} \\
\midrule\noalign{}
\endhead
\bottomrule\noalign{}
\endlastfoot
A & Polar & 8 mm/hr & Arctic, Antarctic \\
B & Temperate & 12 mm/hr & Northern Europe, Canada \\
C & Subtropical & 22 mm/hr & Southern US, Mediterranean \\
D & Moderate tropical & 32 mm/hr & Southeast Asia, India \\
E & Equatorial & 42 mm/hr & Central Africa, Indonesia \\
F & Tropical maritime & 53 mm/hr & Amazon, Congo Basin \\
G & Monsoon & 63 mm/hr & Bangladesh, Myanmar \\
H & Intense tropical & 95 mm/hr & Extreme storms \\
\end{longtable}
}

\textbf{Design criterion}: Typically design for 99.9\% availability
(0.01\% outage time) - Temperate: 12 mm/hr - Tropical: 42-63 mm/hr

\begin{center}\rule{0.5\linewidth}{0.5pt}\end{center}

\subsubsection{Link Budget Impact: Satellite
Examples}\label{link-budget-impact-satellite-examples}

\paragraph{Example 1: Ku-Band Satellite (12 GHz
Downlink)}\label{example-1-ku-band-satellite-12-ghz-downlink}

\textbf{Scenario}: GEO satellite \$\textbackslash rightarrow\$ Home
receiver, temperate climate

\textbf{Path geometry}: - Elevation angle: 30\$\^{}\textbackslash circ\$
- Slant path through rain: \textasciitilde6 km effective length - Rain
rate (0.01\% time): 12 mm/hr

\textbf{Calculation}:

\[
\gamma_R = 0.0188 \times 12^{1.310} = 0.50\ \text{dB/km}
\]

\[
A_{\text{rain}} = \gamma_R \times d_{\text{eff}} = 0.50 \times 6 = 3\ \text{dB}
\]

\textbf{Impact}: 3 dB margin needed for 99.9\% availability

\textbf{With 95 mm/hr extreme storm} (H zone):

\[
\gamma_R = 0.0188 \times 95^{1.310} = 6.3\ \text{dB/km}
\]

\[
A_{\text{rain}} = 6.3 \times 6 = 38\ \text{dB}
\]

\textbf{Result}: Complete outage (exceeds typical 10-15 dB link margin)

\begin{center}\rule{0.5\linewidth}{0.5pt}\end{center}

\paragraph{Example 2: Ka-Band Satellite (20 GHz
Downlink)}\label{example-2-ka-band-satellite-20-ghz-downlink}

\textbf{Same scenario} as Ku-band:

\textbf{Temperate (12 mm/hr)}:

\[
\gamma_R = 0.0751 \times 12^{1.099} = 1.16\ \text{dB/km}
\]

\[
A_{\text{rain}} = 1.16 \times 6 = 7\ \text{dB}
\]

\textbf{Tropical (42 mm/hr)}:

\[
\gamma_R = 0.0751 \times 42^{1.099} = 4.4\ \text{dB/km}
\]

\[
A_{\text{rain}} = 4.4 \times 6 = 26\ \text{dB}
\]

\textbf{Comparison}: Ka-band suffers \textbf{2-3\$\textbackslash times\$
more rain fade} than Ku-band!

\textbf{Mitigation}: - Adaptive coding/modulation (ACM)
\$\textbackslash rightarrow\$ Lower data rate in rain - Site diversity
\$\textbackslash rightarrow\$ Multiple ground stations (rain cells are
localized) - Higher TX power margin

\begin{center}\rule{0.5\linewidth}{0.5pt}\end{center}

\paragraph{Example 3: V-Band Satellite (40
GHz)}\label{example-3-v-band-satellite-40-ghz}

\textbf{Next-gen satellite comms} (e.g., OneWeb, Starlink
inter-satellite links):

\textbf{Temperate (12 mm/hr)}:

\[
\gamma_R = 0.350 \times 12^{0.939} = 3.6\ \text{dB/km}
\]

\[
A_{\text{rain}} = 3.6 \times 6 = 22\ \text{dB}
\]

\textbf{Result}: \textbf{Severe rain fade}, requires 25+ dB margin or
advanced mitigation

\begin{center}\rule{0.5\linewidth}{0.5pt}\end{center}

\subsubsection{Terrestrial Path: 5G
mmWave}\label{terrestrial-path-5g-mmwave}

\paragraph{Example 4: 28 GHz 5G Link (Urban
Microcell)}\label{example-4-28-ghz-5g-link-urban-microcell}

\textbf{Scenario}: Base station \$\textbackslash rightarrow\$ UE, 200 m
range, light rain (5 mm/hr)

\textbf{Calculation}:

\[
\gamma_R = 0.187 \times 5^{1.021} = 0.98\ \text{dB/km}
\]

\[
A_{\text{rain}} = 0.98 \times 0.2 = 0.2\ \text{dB}
\]

\textbf{Impact}: Minimal (short path length)

\textbf{Heavy rain (25 mm/hr)}:

\[
\gamma_R = 0.187 \times 25^{1.021} = 5.2\ \text{dB/km}
\]

\[
A_{\text{rain}} = 5.2 \times 0.2 = 1\ \text{dB}
\]

\textbf{Conclusion}: 5G mmWave is \textbf{relatively rain-tolerant for
short ranges} (\textless{} 500 m)

\begin{center}\rule{0.5\linewidth}{0.5pt}\end{center}

\paragraph{Example 5: 60 GHz Point-to-Point
Link}\label{example-5-60-ghz-point-to-point-link}

\textbf{Scenario}: Building-to-building backhaul, 1 km, moderate rain
(15 mm/hr)

\textbf{Calculation}:

\[
\gamma_R = 0.707 \times 15^{0.826} = 6.4\ \text{dB/km}
\]

\[
A_{\text{rain}} = 6.4 \times 1 = 6.4\ \text{dB}
\]

\textbf{Plus oxygen absorption}: \textasciitilde15 dB/km at 60 GHz
(clear air)

\[
A_{\text{total}} = 15 + 6.4 = 21.4\ \text{dB}
\]

\textbf{Result}: \textbf{60 GHz is impractical for \textgreater1 km in
rain} (used for indoor/short-range only)

\begin{center}\rule{0.5\linewidth}{0.5pt}\end{center}

\subsection{Fog \& Cloud Attenuation}\label{fog-cloud-attenuation}

\textbf{Fog = suspended water droplets} (smaller than rain,
\textasciitilde10-100 \$\textbackslash mu\$m diameter)

\textbf{Attenuation mechanism}: Primarily absorption (droplets
\$\textbackslash ll\$ \$\textbackslash lambda\$ for most RF bands)

\begin{center}\rule{0.5\linewidth}{0.5pt}\end{center}

\subsubsection{Fog Attenuation Model}\label{fog-attenuation-model}

\textbf{Specific attenuation}:

\[
\gamma_{\text{fog}} = K_l \cdot M \quad (\text{dB/km})
\]

Where: - \(M\) = Liquid water content
(g/m\textbackslash textsuperscript\{3\}) - \(K_l\) = Frequency-dependent
coefficient

\textbf{Typical fog}: \(M = 0.05\)
g/m\textbackslash textsuperscript\{3\} (light fog) to \(M = 0.5\)
g/m\textbackslash textsuperscript\{3\} (dense fog)

\begin{center}\rule{0.5\linewidth}{0.5pt}\end{center}

\subsubsection{Coefficients by
Frequency}\label{coefficients-by-frequency-1}

{\def\LTcaptype{} % do not increment counter
\begin{longtable}[]{@{}
  >{\raggedright\arraybackslash}p{(\linewidth - 4\tabcolsep) * \real{0.1571}}
  >{\raggedright\arraybackslash}p{(\linewidth - 4\tabcolsep) * \real{0.3429}}
  >{\raggedright\arraybackslash}p{(\linewidth - 4\tabcolsep) * \real{0.5000}}@{}}
\toprule\noalign{}
\begin{minipage}[b]{\linewidth}\raggedright
Frequency
\end{minipage} & \begin{minipage}[b]{\linewidth}\raggedright
\(K_l\) (dB/km per g/m\textbackslash textsuperscript\{3\})
\end{minipage} & \begin{minipage}[b]{\linewidth}\raggedright
Attenuation (dense fog, 0.5 g/m\textbackslash textsuperscript\{3\})
\end{minipage} \\
\midrule\noalign{}
\endhead
\bottomrule\noalign{}
\endlastfoot
10 GHz & 0.01 & 0.005 dB/km \\
20 GHz & 0.07 & 0.035 dB/km \\
30 GHz & 0.20 & 0.10 dB/km \\
60 GHz & 1.0 & 0.50 dB/km \\
100 GHz & 2.5 & 1.25 dB/km \\
300 GHz & 15 & 7.5 dB/km \\
\end{longtable}
}

\textbf{Key insight}: Fog is \textbf{negligible below 30 GHz}, but
significant at THz frequencies.

\begin{center}\rule{0.5\linewidth}{0.5pt}\end{center}

\subsubsection{Comparison: Rain vs Fog}\label{comparison-rain-vs-fog}

\textbf{At 30 GHz, 1 km path}:

{\def\LTcaptype{} % do not increment counter
\begin{longtable}[]{@{}ll@{}}
\toprule\noalign{}
Condition & Attenuation \\
\midrule\noalign{}
\endhead
\bottomrule\noalign{}
\endlastfoot
Clear air & \textasciitilde0.1 dB \\
Dense fog (0.5 g/m\textbackslash textsuperscript\{3\}) & 0.10 dB \\
Light rain (5 mm/hr) & 3.7 dB \\
Moderate rain (12 mm/hr) & 7.1 dB \\
Heavy rain (25 mm/hr) & 12.5 dB \\
\end{longtable}
}

\textbf{Rain dominates} at microwave/mmWave frequencies.

\textbf{Fog becomes important} at THz (\textgreater{} 100 GHz):

\textbf{At 300 GHz (THz), 100 m path}:

{\def\LTcaptype{} % do not increment counter
\begin{longtable}[]{@{}ll@{}}
\toprule\noalign{}
Condition & Attenuation \\
\midrule\noalign{}
\endhead
\bottomrule\noalign{}
\endlastfoot
Clear air & \textasciitilde5 dB (water vapor) \\
Dense fog & 0.75 dB \\
Light rain (5 mm/hr) & \textbf{300+ dB} (complete blockage) \\
\end{longtable}
}

\begin{center}\rule{0.5\linewidth}{0.5pt}\end{center}

\subsection{Snow \& Ice Attenuation}\label{snow-ice-attenuation}

\textbf{Dry snow}: Very low attenuation (air + ice crystals, low loss)

\[
\gamma_{\text{dry snow}} \approx 0.0005 \times f^2 \times S \quad (\text{dB/km})
\]

Where: - \(f\) = Frequency (GHz) - \(S\) = Snowfall rate (mm/hr liquid
equivalent)

\textbf{At 20 GHz, 10 mm/hr dry snow}: \(\gamma \approx 0.2\) dB/km
(negligible)

\begin{center}\rule{0.5\linewidth}{0.5pt}\end{center}

\textbf{Wet snow} (melting): Much higher attenuation (comparable to
rain)

\textbf{Ice crystals} (cirrus clouds): Minimal attenuation (\textless{}
0.1 dB even at 100 GHz)

\textbf{Practical implication}: Snow is \textbf{far less problematic}
than rain for RF links.

\begin{center}\rule{0.5\linewidth}{0.5pt}\end{center}

\subsection{Hail Attenuation}\label{hail-attenuation}

\textbf{Hailstones}: Large (5-50 mm), but mostly ice (low loss tangent)

\textbf{Attenuation}: Typically \textbf{less than rain of equivalent
water content}

\textbf{Why?}: Ice has lower dielectric loss than liquid water
(\(\tan \delta_{\text{ice}} \ll \tan \delta_{\text{water}}\))

\textbf{Concern}: \textbf{Depolarization} (hailstones tumble, scatter
energy to cross-pol)

\begin{center}\rule{0.5\linewidth}{0.5pt}\end{center}

\subsection{Frequency-Specific
Considerations}\label{frequency-specific-considerations}

\subsubsection{Bands Most Affected by
Rain}\label{bands-most-affected-by-rain}

{\def\LTcaptype{} % do not increment counter
\begin{longtable}[]{@{}
  >{\raggedright\arraybackslash}p{(\linewidth - 6\tabcolsep) * \real{0.1250}}
  >{\raggedright\arraybackslash}p{(\linewidth - 6\tabcolsep) * \real{0.2292}}
  >{\raggedright\arraybackslash}p{(\linewidth - 6\tabcolsep) * \real{0.2708}}
  >{\raggedright\arraybackslash}p{(\linewidth - 6\tabcolsep) * \real{0.3750}}@{}}
\toprule\noalign{}
\begin{minipage}[b]{\linewidth}\raggedright
Band
\end{minipage} & \begin{minipage}[b]{\linewidth}\raggedright
Frequency
\end{minipage} & \begin{minipage}[b]{\linewidth}\raggedright
Primary Use
\end{minipage} & \begin{minipage}[b]{\linewidth}\raggedright
Rain Sensitivity
\end{minipage} \\
\midrule\noalign{}
\endhead
\bottomrule\noalign{}
\endlastfoot
C-band & 4-8 GHz & Satellite TV, radar & \textbf{Low} (0.05 dB/km @ 25
mm/hr) \\
X-band & 8-12 GHz & Military, radar & \textbf{Moderate} (0.5 dB/km) \\
Ku-band & 12-18 GHz & Satellite TV/broadband & \textbf{Moderate-High}
(1-2 dB/km) \\
Ka-band & 26.5-40 GHz & Satellite, 5G backhaul & \textbf{High} (3-12
dB/km) \\
V-band & 40-75 GHz & Next-gen satellite & \textbf{Very High} (12-20
dB/km) \\
W-band & 75-110 GHz & Automotive radar, imaging & \textbf{Extreme}
(20-30 dB/km) \\
\end{longtable}
}

\textbf{C-band advantage}: Widely used for tropical regions (low rain
fade)

\textbf{Ka-band challenge}: High data rates, but needs ACM and large
margins

\begin{center}\rule{0.5\linewidth}{0.5pt}\end{center}

\subsection{Mitigation Techniques}\label{mitigation-techniques}

\subsubsection{1. Link Margin}\label{link-margin}

\textbf{Add extra dB to link budget} for rain:

\begin{itemize}
\tightlist
\item
  Temperate climate, Ku-band: \textbf{+3-5 dB}
\item
  Tropical climate, Ka-band: \textbf{+8-15 dB}
\item
  mmWave terrestrial (\textless{} 1 km): \textbf{+2-3 dB}
\end{itemize}

\textbf{Tradeoff}: Higher TX power or larger antennas (more expensive)

\begin{center}\rule{0.5\linewidth}{0.5pt}\end{center}

\subsubsection{2. Adaptive Coding \& Modulation
(ACM)}\label{adaptive-coding-modulation-acm}

\textbf{Dynamically adjust modulation} based on link quality:

\begin{itemize}
\tightlist
\item
  Clear sky: 16-APSK (4 bits/symbol)
\item
  Light rain: QPSK (2 bits/symbol)
\item
  Heavy rain: BPSK + strong FEC (0.5 bits/symbol effective)
\end{itemize}

\textbf{Result}: \textbf{Graceful degradation} (lower data rate instead
of outage)

\textbf{Used in}: DVB-S2, 5G NR, satellite modems

\begin{center}\rule{0.5\linewidth}{0.5pt}\end{center}

\subsubsection{3. Site Diversity}\label{site-diversity}

\textbf{Multiple ground stations} separated by 5-20 km:

\textbf{Principle}: Rain cells are \textbf{localized}
(\textasciitilde5-10 km diameter) - Probability both sites in heavy rain
is low - Switch to non-rainy site

\textbf{Diversity gain}: 5-10 dB improvement in availability

\textbf{Example}: Satellite gateways often have 2-3 sites for 99.99\%
uptime

\begin{center}\rule{0.5\linewidth}{0.5pt}\end{center}

\subsubsection{4. Frequency Diversity}\label{frequency-diversity}

\textbf{Backup link at lower frequency}:

\begin{itemize}
\tightlist
\item
  Primary: Ka-band (high data rate, rain-sensitive)
\item
  Backup: Ku-band (lower rate, rain-tolerant)
\end{itemize}

\textbf{Switchover}: Automatic when Ka-band SNR drops

\begin{center}\rule{0.5\linewidth}{0.5pt}\end{center}

\subsubsection{5. Uplink Power Control
(UPC)}\label{uplink-power-control-upc}

\textbf{Increase TX power during rain} to compensate for attenuation:

\begin{itemize}
\tightlist
\item
  Monitor beacon signal from satellite
\item
  Detect fade, boost uplink power (up to \textasciitilde10 dB)
\item
  Avoid saturating satellite transponder
\end{itemize}

\textbf{Limitation}: Power amplifier headroom (can\textquotesingle t
boost infinitely)

\begin{center}\rule{0.5\linewidth}{0.5pt}\end{center}

\subsubsection{6. Orbit Selection}\label{orbit-selection}

\textbf{Low Earth Orbit (LEO)} satellites have shorter slant paths:

\begin{itemize}
\tightlist
\item
  GEO: \textasciitilde40,000 km, slant path through rain
  \textasciitilde6 km @ 30\$\^{}\textbackslash circ\$ elevation
\item
  LEO: \textasciitilde550 km, slant path \textasciitilde2 km @
  30\$\^{}\textbackslash circ\$ elevation
\end{itemize}

\textbf{Rain attenuation}:
\textbf{\textasciitilde3\$\textbackslash times\$ less for LEO} (shorter
path)

\textbf{Starlink/OneWeb advantage}: Better rain performance than GEO

\begin{center}\rule{0.5\linewidth}{0.5pt}\end{center}

\subsection{Depolarization Effects}\label{depolarization-effects}

\textbf{Rain also causes cross-polarization}:

\textbf{Mechanism}: Raindrops are \textbf{oblate} (flattened spheres) -
Horizontal and vertical polarizations experience different phase shifts
- Converts co-pol energy \$\textbackslash rightarrow\$ cross-pol

\textbf{Impact}: Degrades dual-polarization systems (e.g., V/H reuse for
2\$\textbackslash times\$ capacity)

\textbf{Cross-Polarization Discrimination (XPD)}:

\[
\text{XPD}_{\text{rain}} = U - V \log(A_{\text{rain}}) \quad (\text{dB})
\]

Where: - \(U, V\) = Frequency-dependent constants (\textasciitilde30-40
dB, \textasciitilde12-20 dB typical) - \(A_{\text{rain}}\) = Co-pol
attenuation (dB)

\textbf{Example}: If rain causes 10 dB attenuation
\$\textbackslash rightarrow\$ XPD degrades from 30 dB (clear) to
\textasciitilde20 dB

\begin{center}\rule{0.5\linewidth}{0.5pt}\end{center}

\subsection{Regional Considerations}\label{regional-considerations}

\subsubsection{Temperate Climates (Europe, Northern US,
Canada)}\label{temperate-climates-europe-northern-us-canada}

\textbf{Rain characteristics}: - Moderate intensity (12-22 mm/hr, 0.01\%
time) - Long-duration stratiform rain (hours) - Lower fade durations

\textbf{Design approach}: - Standard margins (3-5 dB for Ku, 8-12 dB for
Ka) - ACM effective (gradual fade)

\begin{center}\rule{0.5\linewidth}{0.5pt}\end{center}

\subsubsection{Tropical Climates (Southeast Asia, Equatorial Africa,
Amazon)}\label{tropical-climates-southeast-asia-equatorial-africa-amazon}

\textbf{Rain characteristics}: - High intensity (42-95 mm/hr, 0.01\%
time) - Short-duration convective storms (minutes) - High fade depths

\textbf{Design approach}: - Large margins (8-15 dB for Ku, 15-25 dB for
Ka) - Site diversity essential for Ka-band - C-band preferred for
critical services

\textbf{Case study}: Indonesia (equatorial) - Ku-band outages:
\textasciitilde0.5\% of time (annual) - Ka-band: \textasciitilde2-5\%
(unacceptable without mitigation) - C-band: \textless{} 0.01\%
(reliable)

\begin{center}\rule{0.5\linewidth}{0.5pt}\end{center}

\subsubsection{Coastal vs Inland}\label{coastal-vs-inland}

\textbf{Coastal regions}: Lower rain rates (maritime climate)
\textbf{Inland tropics}: Higher convective activity (more intense
storms)

\textbf{Elevation matters}: Higher altitude
\$\textbackslash rightarrow\$ shorter path through rain layer
(troposphere)

\begin{center}\rule{0.5\linewidth}{0.5pt}\end{center}

\subsection{Measurement \& Prediction}\label{measurement-prediction}

\subsubsection{Radiometer Method}\label{radiometer-method}

\textbf{Measure sky brightness temperature} \(T_B\):

\[
A_{\text{rain}} = 10 \log\left(\frac{T_{\text{sky}} - T_B}{T_{\text{sky}} - T_{\text{medium}}}\right) \quad (\text{dB})
\]

\textbf{Real-time fade monitoring} for UPC systems.

\begin{center}\rule{0.5\linewidth}{0.5pt}\end{center}

\subsubsection{Weather Radar
Integration}\label{weather-radar-integration}

\textbf{Use ground-based weather radar} to predict rain attenuation:

\begin{enumerate}
\def\labelenumi{\arabic{enumi}.}
\tightlist
\item
  Measure rain rate along path (3D map)
\item
  Apply ITU model
\item
  Predict fade 5-10 minutes ahead
\end{enumerate}

\textbf{Proactive ACM}: Adjust modulation before fade hits (minimize
disruption)

\begin{center}\rule{0.5\linewidth}{0.5pt}\end{center}

\subsection{Summary Table: Rain Attenuation by
Band}\label{summary-table-rain-attenuation-by-band}

\textbf{Path}: 6 km slant through rain (satellite,
30\$\^{}\textbackslash circ\$ elevation)

{\def\LTcaptype{} % do not increment counter
\begin{longtable}[]{@{}
  >{\raggedright\arraybackslash}p{(\linewidth - 8\tabcolsep) * \real{0.0759}}
  >{\raggedright\arraybackslash}p{(\linewidth - 8\tabcolsep) * \real{0.1392}}
  >{\raggedright\arraybackslash}p{(\linewidth - 8\tabcolsep) * \real{0.2785}}
  >{\raggedright\arraybackslash}p{(\linewidth - 8\tabcolsep) * \real{0.2658}}
  >{\raggedright\arraybackslash}p{(\linewidth - 8\tabcolsep) * \real{0.2405}}@{}}
\toprule\noalign{}
\begin{minipage}[b]{\linewidth}\raggedright
Band
\end{minipage} & \begin{minipage}[b]{\linewidth}\raggedright
Frequency
\end{minipage} & \begin{minipage}[b]{\linewidth}\raggedright
12 mm/hr (Temperate)
\end{minipage} & \begin{minipage}[b]{\linewidth}\raggedright
42 mm/hr (Tropical)
\end{minipage} & \begin{minipage}[b]{\linewidth}\raggedright
95 mm/hr (Extreme)
\end{minipage} \\
\midrule\noalign{}
\endhead
\bottomrule\noalign{}
\endlastfoot
C & 4 GHz & 0.15 dB & 0.7 dB & 2 dB \\
C & 6 GHz & 0.3 dB & 1.3 dB & 3.5 dB \\
X & 10 GHz & 0.5 dB & 2.5 dB & 7 dB \\
Ku & 12 GHz & \textbf{3 dB} & \textbf{9 dB} & \textbf{25 dB} \\
Ku & 14 GHz & 4 dB & 11 dB & 30 dB \\
Ka & 20 GHz & \textbf{7 dB} & \textbf{22 dB} & \textbf{55 dB} \\
Ka & 30 GHz & 13 dB & 38 dB & 90 dB \\
V & 40 GHz & 22 dB & 60 dB & 140 dB \\
V & 50 GHz & 30 dB & 80 dB & 180 dB \\
\end{longtable}
}

\textbf{Bold}: Typical link budget fades

\begin{center}\rule{0.5\linewidth}{0.5pt}\end{center}

\subsection{Related Topics}\label{related-topics}

\begin{itemize}
\tightlist
\item
  \textbf{{[}{[}Free-Space-Path-Loss-(FSPL){]}{]}}: Baseline propagation
  loss
\item
  \textbf{{[}{[}Atmospheric-Effects-(Ionospheric,-Tropospheric){]}{]}}:
  Clear-air propagation
\item
  \textbf{{[}{[}Multipath-Propagation-\&-Fading-(Rayleigh,-Rician){]}{]}}:
  Rayleigh/Rician fading (different mechanism)
\item
  \textbf{{[}{[}Signal-to-Noise-Ratio-(SNR){]}{]}}: Impact of
  attenuation on link quality
\item
  \textbf{{[}{[}QPSK-Modulation{]}{]}} /
  \textbf{{[}{[}LDPC-Codes{]}{]}}: ACM adapts these for rain conditions
\item
  \textbf{Antenna Theory}: Larger antennas provide more gain margin
\end{itemize}

\begin{center}\rule{0.5\linewidth}{0.5pt}\end{center}

\textbf{Key takeaway}: \textbf{Rain fade increases dramatically with
frequency}. C-band is robust but bandwidth-limited. Ka-band and above
require sophisticated mitigation (ACM, diversity, large margins) for
reliable service, especially in tropical regions.

\begin{center}\rule{0.5\linewidth}{0.5pt}\end{center}

\emph{This wiki is part of the {[}{[}Home\textbar Chimera Project{]}{]}
documentation.}
