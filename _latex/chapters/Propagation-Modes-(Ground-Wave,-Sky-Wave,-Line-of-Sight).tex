\section{Propagation Modes: Ground Wave, Sky Wave,
Line-of-Sight}\label{propagation-modes-ground-wave-sky-wave-line-of-sight}

{[}{[}Home{]}{]} \textbar{} \textbf{RF Propagation} \textbar{}
{[}{[}Free-Space-Path-Loss-(FSPL){]}{]} \textbar{}
{[}{[}Electromagnetic-Spectrum{]}{]}

\begin{center}\rule{0.5\linewidth}{0.5pt}\end{center}

\subsection{\texorpdfstring{ For Non-Technical
Readers}{ For Non-Technical Readers}}\label{for-non-technical-readers}

\textbf{Radio waves can travel three different ways-\/-\/-think of it
like: rolling on the ground, bouncing off the sky, or shooting straight
like a laser!}

\textbf{1. Ground Wave} (Surface Wave) - \textbf{What}: Radio wave hugs
the Earth\textquotesingle s surface and bends around the curve -
\textbf{Frequency}: Low (AM radio, 500 kHz - 1.5 MHz) - \textbf{Range}:
100-1000+ km depending on frequency - \textbf{Analogy}: Rolling a ball
on the ground-\/-\/-it follows the terrain

\textbf{Real example}: - \textbf{AM radio stations}: Travel hundreds of
miles, even over the horizon - \textbf{Why AM works everywhere}: Ground
wave bends around hills/buildings - \textbf{Maritime communications}:
Ships use ground wave to communicate over the ocean

\textbf{2. Sky Wave} (Ionospheric Bounce) - \textbf{What}: Radio wave
shoots up, bounces off ionosphere (layer of charged particles 100-400 km
up), comes back down - \textbf{Frequency}: Medium (shortwave radio, 3-30
MHz / HF) - \textbf{Range}: Global! Can bounce multiple times -
\textbf{Analogy}: Skipping a stone on water-\/-\/-one throw, many
bounces

\textbf{Real example}: - \textbf{Shortwave radio}: Broadcast from
London, heard in Australia! - \textbf{Amateur (ham) radio}: Talk to
people on other continents - \textbf{Why it works at night}: Ionosphere
gets stronger after sunset (no sun breaking it apart)

\textbf{3. Line-of-Sight} (LOS) - \textbf{What}: Radio wave travels
straight like light-\/-\/-if you can\textquotesingle t ``see'' the
tower, signal is blocked - \textbf{Frequency}: High (FM radio, TV, cell
phones, WiFi, 30 MHz+) - \textbf{Range}: Limited to visible horizon
(\textasciitilde5-50 km depending on height) - \textbf{Analogy}: Laser
pointer-\/-\/-must have clear path

\textbf{Real example}: - \textbf{Cell phones}: Need tower in view
(mostly) - \textbf{WiFi}: Walls/floors block it - \textbf{Satellite TV}:
Dish must point directly at satellite (trees in the way = no signal!) -
\textbf{5G mmWave}: Can\textquotesingle t even go through your hand!

\textbf{Why different modes?} - \textbf{Lower frequency}
\$\textbackslash rightarrow\$ bends around obstacles, long range, slow
data - \textbf{Higher frequency} \$\textbackslash rightarrow\$
straight-line only, shorter range, fast data

\textbf{Fun fact}: During the Cold War, governments used sky wave
propagation to broadcast radio into other countries-\/-\/-signals would
bounce off the ionosphere and arrive ``from above,'' impossible to
block!

\begin{center}\rule{0.5\linewidth}{0.5pt}\end{center}

\subsection{Overview}\label{overview}

\textbf{Electromagnetic waves propagate via different mechanisms}
depending on frequency, distance, and environment. Understanding
propagation modes is essential for predicting coverage and designing
wireless systems.

\textbf{Three primary modes}: 1. \textbf{Ground Wave} (Surface wave) -
LF/MF/HF, follows Earth\textquotesingle s curvature 2. \textbf{Sky Wave}
(Ionospheric) - HF, bounces off ionosphere, global reach 3.
\textbf{Line-of-Sight (LOS)} - VHF and above, direct path required

\begin{center}\rule{0.5\linewidth}{0.5pt}\end{center}

\subsection{Ground Wave Propagation}\label{ground-wave-propagation}

\subsubsection{Definition}\label{definition}

\textbf{Ground wave} = EM wave that travels along
Earth\textquotesingle s surface, guided by the ground-air interface.

\textbf{Mechanism}: - Electric field induces currents in ground
(conductive surface) - Ground acts as imperfect dielectric, slows wave
slightly - \textbf{Diffraction} allows wave to follow
Earth\textquotesingle s curvature (beyond horizon)

\textbf{Frequency range}: \textbf{LF to MF} (30 kHz - 3 MHz), limited
use in \textbf{HF} (3-30 MHz)

\begin{center}\rule{0.5\linewidth}{0.5pt}\end{center}

\subsubsection{Attenuation Factors}\label{attenuation-factors}

\textbf{Ground wave attenuation depends on}:

\begin{enumerate}
\def\labelenumi{\arabic{enumi}.}
\tightlist
\item
  \textbf{Frequency}: Higher frequency = more attenuation
\item
  \textbf{Ground conductivity}: Seawater (high \$\textbackslash sigma\$)
  \textless{} freshwater \textless{} wet soil \textless{} dry soil
  \textless{} ice
\item
  \textbf{Distance}: Exponential decay with range
\item
  \textbf{Polarization}: \textbf{Vertical polarization required}
  (horizontal is rapidly attenuated)
\end{enumerate}

\begin{center}\rule{0.5\linewidth}{0.5pt}\end{center}

\paragraph{Ground Conductivity}\label{ground-conductivity}

{\def\LTcaptype{} % do not increment counter
\begin{longtable}[]{@{}
  >{\raggedright\arraybackslash}p{(\linewidth - 6\tabcolsep) * \real{0.1728}}
  >{\raggedright\arraybackslash}p{(\linewidth - 6\tabcolsep) * \real{0.2346}}
  >{\raggedright\arraybackslash}p{(\linewidth - 6\tabcolsep) * \real{0.4321}}
  >{\raggedright\arraybackslash}p{(\linewidth - 6\tabcolsep) * \real{0.1605}}@{}}
\toprule\noalign{}
\begin{minipage}[b]{\linewidth}\raggedright
Surface Type
\end{minipage} & \begin{minipage}[b]{\linewidth}\raggedright
Conductivity (S/m)
\end{minipage} & \begin{minipage}[b]{\linewidth}\raggedright
Relative Permittivity \(\epsilon_r\)
\end{minipage} & \begin{minipage}[b]{\linewidth}\raggedright
Attenuation
\end{minipage} \\
\midrule\noalign{}
\endhead
\bottomrule\noalign{}
\endlastfoot
Seawater & 5 & 80 & \textbf{Very low} (best) \\
Freshwater & 0.01 & 80 & Low \\
Wet soil & 0.01-0.001 & 20-30 & Moderate \\
Dry soil & 0.001 & 3-10 & High \\
Urban/concrete & 0.001 & 5 & High \\
Ice/snow & 0.0001 & 3 & Very high \\
\end{longtable}
}

\textbf{Maritime advantage}: Ships can communicate over 1000+ km at
LF/MF (AM broadcast)

\textbf{Desert disadvantage}: Dry sand severely limits ground wave (100s
of meters)

\begin{center}\rule{0.5\linewidth}{0.5pt}\end{center}

\subsubsection{Range vs Frequency}\label{range-vs-frequency}

\textbf{Empirical range} (over average soil, vertical polarization):

{\def\LTcaptype{} % do not increment counter
\begin{longtable}[]{@{}
  >{\raggedright\arraybackslash}p{(\linewidth - 6\tabcolsep) * \real{0.2115}}
  >{\raggedright\arraybackslash}p{(\linewidth - 6\tabcolsep) * \real{0.2308}}
  >{\raggedright\arraybackslash}p{(\linewidth - 6\tabcolsep) * \real{0.2885}}
  >{\raggedright\arraybackslash}p{(\linewidth - 6\tabcolsep) * \real{0.2692}}@{}}
\toprule\noalign{}
\begin{minipage}[b]{\linewidth}\raggedright
Frequency
\end{minipage} & \begin{minipage}[b]{\linewidth}\raggedright
Wavelength
\end{minipage} & \begin{minipage}[b]{\linewidth}\raggedright
Typical Range
\end{minipage} & \begin{minipage}[b]{\linewidth}\raggedright
Applications
\end{minipage} \\
\midrule\noalign{}
\endhead
\bottomrule\noalign{}
\endlastfoot
50 kHz (VLF) & 6000 m & 500-1000 km & Navigation (LORAN) \\
150 kHz (LF) & 2000 m & 300-800 km & Longwave broadcast \\
500 kHz (MF) & 600 m & 200-500 km & Marine distress (SOS) \\
1 MHz (AM) & 300 m & 100-300 km & AM radio (nighttime skywave extends
this) \\
3 MHz (HF) & 100 m & 10-50 km & Limited ground wave, skywave dominant \\
\end{longtable}
}

\textbf{Key insight}: \textbf{Ground wave range decreases rapidly with
frequency}

\begin{center}\rule{0.5\linewidth}{0.5pt}\end{center}

\subsubsection{Path Loss Model}\label{path-loss-model}

\textbf{Norton ground wave equation} (simplified):

\[
L_{\text{ground}} = L_{\text{FS}} + A_{\text{ground}}(f, d, \sigma, \epsilon_r)
\]

Where: - \(L_{\text{FS}}\) = Free-space path loss -
\(A_{\text{ground}}\) = Additional ground attenuation (complex function,
see ITU-R P.368)

\textbf{Approximation} (MF band, average soil):

\[
A_{\text{ground}} \approx 0.1 \times \left(\frac{f}{1\ \text{MHz}}\right)^2 \times \left(\frac{d}{1\ \text{km}}\right) \quad (\text{dB})
\]

\textbf{Example}: 1 MHz, 100 km over average soil:

\[
A_{\text{ground}} \approx 0.1 \times 1^2 \times 100 = 10\ \text{dB additional loss}
\]

\begin{center}\rule{0.5\linewidth}{0.5pt}\end{center}

\subsubsection{Applications}\label{applications}

\paragraph{AM Radio (540-1600 kHz)}\label{am-radio-540-1600-khz}

\textbf{Daytime}: Ground wave only - Local coverage: 50-150 km - Limited
by ground conductivity

\textbf{Nighttime}: Skywave + ground wave - Extended coverage: 500-2000+
km (skywave reflection) - Interference common (multiple stations bounce
off ionosphere)

\begin{center}\rule{0.5\linewidth}{0.5pt}\end{center}

\paragraph{Maritime Communications
(LF/MF)}\label{maritime-communications-lfmf}

\textbf{VLF (20-100 kHz)}: Submarine comms - Penetrates seawater (10-20
m depth) - Global coverage (ground wave over ocean) - Very low data rate
(\textless{} 100 bps)

\textbf{MF (300-3000 kHz)}: Ship-to-shore - 200-500 km range over
seawater - Distress frequency: 500 kHz (historical SOS)

\begin{center}\rule{0.5\linewidth}{0.5pt}\end{center}

\paragraph{Aviation NDB (Non-Directional
Beacons)}\label{aviation-ndb-non-directional-beacons}

\textbf{Frequency}: 190-535 kHz \textbf{Range}: 50-100 km (ground wave)
\textbf{Use}: Aircraft homing (ADF receivers)

\begin{center}\rule{0.5\linewidth}{0.5pt}\end{center}

\subsection{Sky Wave (Ionospheric
Propagation)}\label{sky-wave-ionospheric-propagation}

\subsubsection{Definition}\label{definition-1}

\textbf{Sky wave} = EM wave refracted by ionosphere, returning to Earth
at distant location.

\textbf{Mechanism}: 1. HF wave travels upward at angle 2. Ionosphere
(charged plasma layer, 60-400 km altitude) acts as refractive medium 3.
Wave bends back toward Earth (if frequency/angle correct) 4. Can bounce
multiple times (multi-hop)

\textbf{Frequency range}: \textbf{HF (3-30 MHz)} primarily, some
\textbf{MF at night}

\begin{center}\rule{0.5\linewidth}{0.5pt}\end{center}

\subsubsection{Ionospheric Layers}\label{ionospheric-layers}

\textbf{Ionosphere = ionized by solar UV/X-rays}

{\def\LTcaptype{} % do not increment counter
\begin{longtable}[]{@{}
  >{\raggedright\arraybackslash}p{(\linewidth - 8\tabcolsep) * \real{0.1061}}
  >{\raggedright\arraybackslash}p{(\linewidth - 8\tabcolsep) * \real{0.1515}}
  >{\raggedright\arraybackslash}p{(\linewidth - 8\tabcolsep) * \real{0.1818}}
  >{\raggedright\arraybackslash}p{(\linewidth - 8\tabcolsep) * \real{0.2727}}
  >{\raggedright\arraybackslash}p{(\linewidth - 8\tabcolsep) * \real{0.2879}}@{}}
\toprule\noalign{}
\begin{minipage}[b]{\linewidth}\raggedright
Layer
\end{minipage} & \begin{minipage}[b]{\linewidth}\raggedright
Altitude
\end{minipage} & \begin{minipage}[b]{\linewidth}\raggedright
Ionization
\end{minipage} & \begin{minipage}[b]{\linewidth}\raggedright
Daytime Behavior
\end{minipage} & \begin{minipage}[b]{\linewidth}\raggedright
Nighttime Behavior
\end{minipage} \\
\midrule\noalign{}
\endhead
\bottomrule\noalign{}
\endlastfoot
\textbf{D} & 60-90 km & Low & \textbf{Absorbs HF} (attenuates MF/HF) &
\textbf{Disappears} (recombination fast) \\
\textbf{E} & 90-150 km & Moderate & Reflects some HF (\textless{} 10
MHz) & Weakens \\
\textbf{F1} & 150-250 km & Moderate & Reflects MF/HF & \textbf{Merges
with F2} \\
\textbf{F2} & 250-400 km & High & \textbf{Primary reflector} for HF &
Descends, remains strong \\
\end{longtable}
}

\textbf{Key concept}: \textbf{Critical frequency} \(f_c\) - maximum
frequency reflected at vertical incidence

\[
f_c = 9 \sqrt{N_e}
\]

Where \(N_e\) = electron density
(electrons/m\textbackslash textsuperscript\{3\})

\textbf{Typical values}: - Daytime F2: \(f_c = 10-15\) MHz - Nighttime
F2: \(f_c = 5-10\) MHz

\begin{center}\rule{0.5\linewidth}{0.5pt}\end{center}

\subsubsection{Skip Distance \& Hop}\label{skip-distance-hop}

\textbf{Skip distance} = minimum ground range for sky wave return

\[
d_{\text{skip}} = 2h \tan(\theta)
\]

Where: - \(h\) = Ionospheric layer height - \(\theta\) = Elevation angle
of departure

\textbf{For F2 layer} (h \$\textbackslash approx\$ 300 km): - Low angle
(5\$\^{}\textbackslash circ\$): Skip \textasciitilde3500 km (single hop)
- High angle (45\$\^{}\textbackslash circ\$): Skip \textasciitilde600 km

\textbf{Dead zone} = Region between ground wave limit and skip distance
(no coverage)

\begin{center}\rule{0.5\linewidth}{0.5pt}\end{center}

\subsubsection{Multi-Hop Propagation}\label{multi-hop-propagation}

\textbf{Wave bounces between ionosphere and ground}:

\begin{itemize}
\tightlist
\item
  \textbf{Single hop}: 2000-4000 km
\item
  \textbf{Two hops}: 4000-8000 km
\item
  \textbf{Multiple hops}: Global coverage possible (with sufficient
  power)
\end{itemize}

\textbf{Loss per hop}: 5-15 dB (depends on ionospheric conditions,
frequency)

\begin{center}\rule{0.5\linewidth}{0.5pt}\end{center}

\subsubsection{Frequency Selection}\label{frequency-selection}

\textbf{MUF (Maximum Usable Frequency)}: Highest frequency that refracts
back (not penetrating ionosphere)

\[
\text{MUF} = \frac{f_c}{\cos(\theta)}
\]

\textbf{LUF (Lowest Usable Frequency)}: Lowest frequency not absorbed by
D-layer

\textbf{Optimal Working Frequency (FOT)}: 80-90\% of MUF (safety margin)

\begin{center}\rule{0.5\linewidth}{0.5pt}\end{center}

\subsubsection{Diurnal Variations}\label{diurnal-variations}

\textbf{Daytime}: - D-layer absorbs lower HF (\textless{} 5 MHz) - F2
layer reflects higher HF (10-30 MHz) - \textbf{Best bands}: 15 MHz, 20
MHz (long-distance)

\textbf{Nighttime}: - D-layer disappears (no absorption) - F2 descends,
lower MUF - \textbf{Best bands}: 5 MHz, 7 MHz (medium-distance) - AM
broadcast skywave active (500-1600 kHz)

\begin{center}\rule{0.5\linewidth}{0.5pt}\end{center}

\subsubsection{Seasonal \& Solar Cycle
Effects}\label{seasonal-solar-cycle-effects}

\textbf{Solar cycle} (11 years): - \textbf{Solar max}: High ionization,
higher MUF (30 MHz+ usable) - \textbf{Solar min}: Lower MUF (often
\textless{} 20 MHz)

\textbf{Seasonal}: - \textbf{Summer}: Higher D-layer absorption
(daytime) - \textbf{Winter}: Lower absorption, better long-distance
(daytime)

\textbf{Sporadic E} (Es): - Unpredictable intense E-layer patches -
Reflects VHF (up to 150 MHz!) for short periods - Used opportunistically
by amateur radio

\begin{center}\rule{0.5\linewidth}{0.5pt}\end{center}

\subsubsection{Applications}\label{applications-1}

\paragraph{Shortwave Broadcast}\label{shortwave-broadcast}

\textbf{Frequency}: 3-30 MHz (HF bands) \textbf{Range}: 500-10,000+ km
(multi-hop) \textbf{Use}: International broadcasting (BBC World Service,
Voice of America)

\textbf{Schedule management}: Different frequencies for day/night,
seasons

\begin{center}\rule{0.5\linewidth}{0.5pt}\end{center}

\paragraph{Amateur Radio (Ham Radio)}\label{amateur-radio-ham-radio}

\textbf{HF bands}: 1.8, 3.5, 7, 10, 14, 18, 21, 24, 28 MHz
\textbf{Activity}: Global communication with \textless{} 100W (due to
skywave)

\textbf{80m (3.5 MHz)}: Nighttime, regional (500-2000 km) \textbf{20m
(14 MHz)}: Daytime, worldwide (DX)

\begin{center}\rule{0.5\linewidth}{0.5pt}\end{center}

\paragraph{Over-the-Horizon (OTH)
Radar}\label{over-the-horizon-oth-radar}

\textbf{Frequency}: 5-28 MHz \textbf{Range}: 1000-3500 km (beyond
line-of-sight) \textbf{Use}: Early warning, detection beyond horizon

\textbf{Principle}: Reflect radar signal off ionosphere to detect
aircraft/ships at great distance

\begin{center}\rule{0.5\linewidth}{0.5pt}\end{center}

\paragraph{Military HF Communications}\label{military-hf-communications}

\textbf{Strategic links}: Long-range, no satellite dependence
\textbf{Frequency hopping}: Adapt to ionospheric conditions
\textbf{Robustness}: Survives nuclear EMP (no infrastructure needed)

\begin{center}\rule{0.5\linewidth}{0.5pt}\end{center}

\subsection{Line-of-Sight (LOS)
Propagation}\label{line-of-sight-los-propagation}

\subsubsection{Definition}\label{definition-2}

\textbf{Line-of-sight} = Direct path from transmitter to receiver, no
obstructions.

\textbf{Frequency range}: \textbf{VHF and above} (\textgreater{} 30 MHz)

\textbf{Why?}: At VHF+, waves no longer refract around
Earth\textquotesingle s curvature (ionosphere transparent)

\begin{center}\rule{0.5\linewidth}{0.5pt}\end{center}

\subsubsection{Radio Horizon}\label{radio-horizon}

\textbf{Geometric horizon} (flat Earth): Distance where curvature blocks
LOS

\textbf{Radio horizon} (accounting for refraction):

\[
d_{\text{horizon}} = 3.57(\sqrt{h_t} + \sqrt{h_r}) \quad (\text{km})
\]

Where: - \(h_t\) = Transmitter antenna height (meters) - \(h_r\) =
Receiver antenna height (meters) - \textbf{4/3 Earth radius model}
accounts for atmospheric refraction

\begin{center}\rule{0.5\linewidth}{0.5pt}\end{center}

\paragraph{Examples}\label{examples}

\textbf{Mobile phone} (base station 30m, phone 1.5m):

\[
d = 3.57(\sqrt{30} + \sqrt{1.5}) = 3.57(5.48 + 1.22) = 24\ \text{km}
\]

\textbf{TV broadcast tower} (300m, home antenna 10m):

\[
d = 3.57(\sqrt{300} + \sqrt{10}) = 3.57(17.3 + 3.16) = 73\ \text{km}
\]

\textbf{Aircraft at 10,000m} (cruising altitude):

\[
d = 3.57 \sqrt{10000} = 357\ \text{km}
\]

\textbf{Satellite (LEO at 550 km)}: Horizon \textasciitilde2500 km
(covers \textasciitilde5\% of Earth)

\begin{center}\rule{0.5\linewidth}{0.5pt}\end{center}

\subsubsection{Fresnel Zone}\label{fresnel-zone}

\textbf{For reliable LOS, path must be clear not just geometrically, but
also volumetrically}.

\textbf{Fresnel zone} = Ellipsoidal region around direct path where
reflections can interfere

\textbf{First Fresnel zone radius} at midpoint:

\[
r_1 = \sqrt{\frac{\lambda d_1 d_2}{d_1 + d_2}}
\]

Where: - \(\lambda\) = Wavelength - \(d_1, d_2\) = Distances from TX and
RX to obstacle

\textbf{60\% clearance rule}: Keep first Fresnel zone 60\% clear for
reliable LOS

\textbf{Example}: 2 GHz (\$\textbackslash lambda\$ = 15 cm), 10 km link:

\[
r_1 = \sqrt{\frac{0.15 \times 5000 \times 5000}{10000}} = \sqrt{375} = 19\ \text{m}
\]

\textbf{Need}: 60\% \$\textbackslash times\$ 19m = \textbf{11m
clearance} at midpoint

\begin{center}\rule{0.5\linewidth}{0.5pt}\end{center}

\subsubsection{Applications}\label{applications-2}

\paragraph{FM Radio (VHF, 88-108 MHz)}\label{fm-radio-vhf-88-108-mhz}

\textbf{Range}: Line-of-sight limited - Transmitter tower: 100-300m
\$\textbackslash rightarrow\$ 40-70 km range - Terrain shadowing common
(mountains block signal)

\begin{center}\rule{0.5\linewidth}{0.5pt}\end{center}

\paragraph{TV Broadcast (VHF/UHF)}\label{tv-broadcast-vhfuhf}

\textbf{VHF}: Channels 2-13 (54-216 MHz) - legacy analog \textbf{UHF}:
Channels 14-51 (470-698 MHz) - digital TV (ATSC, DVB-T)

\textbf{Range}: 40-100 km (depends on tower height)

\begin{center}\rule{0.5\linewidth}{0.5pt}\end{center}

\paragraph{Cellular (800 MHz - 6 GHz)}\label{cellular-800-mhz---6-ghz}

\textbf{Macrocells}: LOS to horizon (\textasciitilde10-30 km)
\textbf{Microcells}: Urban, 200m-2km (NLOS due to buildings, but
diffraction/scattering help) \textbf{Picocells}: Indoor, 10-100m

\begin{center}\rule{0.5\linewidth}{0.5pt}\end{center}

\paragraph{Microwave Links (6-80 GHz)}\label{microwave-links-6-80-ghz}

\textbf{Point-to-point backhaul}: - Tower-to-tower links (10-50 km) -
Requires clear Fresnel zone - Rain fade significant (see
{[}{[}Weather-Effects-(Rain-Fade,-Fog-Attenuation){]}{]})

\begin{center}\rule{0.5\linewidth}{0.5pt}\end{center}

\paragraph{Satellite Communications}\label{satellite-communications}

\textbf{All satellite links are LOS}: - GEO (35,786 km): Always LOS if
above 10\$\^{}\textbackslash circ\$ elevation - LEO (400-1200 km): Pass
overhead, 5-15 min visibility windows - MEO (GPS, 20,200 km): 4-8 hours
visibility

\begin{center}\rule{0.5\linewidth}{0.5pt}\end{center}

\paragraph{5G mmWave (24-100 GHz)}\label{g-mmwave-24-100-ghz}

\textbf{Ultra-short range LOS}: - Range: 100-500m typical - Building
penetration: Poor (requires outdoor-to-outdoor LOS) - Use: Dense urban,
stadiums, fixed wireless access

\begin{center}\rule{0.5\linewidth}{0.5pt}\end{center}

\subsection{Comparison: Propagation
Modes}\label{comparison-propagation-modes}

{\def\LTcaptype{} % do not increment counter
\begin{longtable}[]{@{}
  >{\raggedright\arraybackslash}p{(\linewidth - 8\tabcolsep) * \real{0.1091}}
  >{\raggedright\arraybackslash}p{(\linewidth - 8\tabcolsep) * \real{0.2000}}
  >{\raggedright\arraybackslash}p{(\linewidth - 8\tabcolsep) * \real{0.1273}}
  >{\raggedright\arraybackslash}p{(\linewidth - 8\tabcolsep) * \real{0.3091}}
  >{\raggedright\arraybackslash}p{(\linewidth - 8\tabcolsep) * \real{0.2545}}@{}}
\toprule\noalign{}
\begin{minipage}[b]{\linewidth}\raggedright
Mode
\end{minipage} & \begin{minipage}[b]{\linewidth}\raggedright
Frequency
\end{minipage} & \begin{minipage}[b]{\linewidth}\raggedright
Range
\end{minipage} & \begin{minipage}[b]{\linewidth}\raggedright
Characteristics
\end{minipage} & \begin{minipage}[b]{\linewidth}\raggedright
Applications
\end{minipage} \\
\midrule\noalign{}
\endhead
\bottomrule\noalign{}
\endlastfoot
\textbf{Ground Wave} & LF/MF & 50-500 km & Follows curvature, stable,
vertical pol & AM radio, maritime, NDB \\
\textbf{Sky Wave} & HF & 500-10,000+ km & Ionospheric reflection,
variable & Shortwave, amateur, OTH radar \\
\textbf{LOS} & VHF+ & 10-100 km & Direct path, terrain-limited & FM, TV,
cellular, microwave \\
\textbf{Satellite LOS} & VHF-Ka & Global & Space path, rain fade
(\textgreater10 GHz) & GPS, satellite TV/internet \\
\textbf{Troposcatter} & UHF/SHF & 100-500 km & Beyond-horizon scatter &
Military long-haul \\
\end{longtable}
}

\begin{center}\rule{0.5\linewidth}{0.5pt}\end{center}

\subsection{Non-Line-of-Sight (NLOS)
Propagation}\label{non-line-of-sight-nlos-propagation}

\textbf{Even at VHF+, signals can reach beyond LOS via}:

\begin{enumerate}
\def\labelenumi{\arabic{enumi}.}
\tightlist
\item
  \textbf{Diffraction}: Bending around obstacles (buildings, hills)
\item
  \textbf{Reflection}: Bounce off surfaces (see
  {[}{[}Multipath-Propagation-\&-Fading-(Rayleigh,-Rician){]}{]})
\item
  \textbf{Scattering}: Random scatter from rough surfaces, rain, foliage
\item
  \textbf{Troposcatter}: Forward scatter from tropospheric turbulence
  (beyond-horizon, 100-500 km)
\end{enumerate}

\textbf{Result}: Cellular networks work in urban canyons (NLOS), but
with higher path loss and multipath fading.

\begin{center}\rule{0.5\linewidth}{0.5pt}\end{center}

\subsection{Ducting \& Anomalous
Propagation}\label{ducting-anomalous-propagation}

\subsubsection{Tropospheric Ducting}\label{tropospheric-ducting}

\textbf{Temperature inversion} creates refractive layer that traps
VHF/UHF waves:

\textbf{Mechanism}: Warm air over cool surface
\$\textbackslash rightarrow\$ gradient in refractive index
\$\textbackslash rightarrow\$ wave bends back to Earth

\textbf{Effect}: \textbf{VHF/UHF propagation far beyond horizon}
(500-2000 km)

\textbf{Conditions}: - Coastal regions (cool ocean, warm land) -
High-pressure systems (calm, clear weather) - Morning/evening
(temperature inversions)

\textbf{Impact}: - FM radio stations suddenly heard 1000 km away - TV
interference from distant stations - Cellular interference (distant
cells)

\begin{center}\rule{0.5\linewidth}{0.5pt}\end{center}

\subsubsection{Evaporation Ducts}\label{evaporation-ducts}

\textbf{Common over oceans}: Humidity gradient creates duct
\textasciitilde10-50m above sea surface

\textbf{Effect}: Ships can communicate VHF far beyond horizon (200-500
km)

\begin{center}\rule{0.5\linewidth}{0.5pt}\end{center}

\subsection{Propagation Models
Summary}\label{propagation-models-summary}

{\def\LTcaptype{} % do not increment counter
\begin{longtable}[]{@{}
  >{\raggedright\arraybackslash}p{(\linewidth - 6\tabcolsep) * \real{0.1842}}
  >{\raggedright\arraybackslash}p{(\linewidth - 6\tabcolsep) * \real{0.2632}}
  >{\raggedright\arraybackslash}p{(\linewidth - 6\tabcolsep) * \real{0.2895}}
  >{\raggedright\arraybackslash}p{(\linewidth - 6\tabcolsep) * \real{0.2632}}@{}}
\toprule\noalign{}
\begin{minipage}[b]{\linewidth}\raggedright
Model
\end{minipage} & \begin{minipage}[b]{\linewidth}\raggedright
Use Case
\end{minipage} & \begin{minipage}[b]{\linewidth}\raggedright
Frequency
\end{minipage} & \begin{minipage}[b]{\linewidth}\raggedright
Accuracy
\end{minipage} \\
\midrule\noalign{}
\endhead
\bottomrule\noalign{}
\endlastfoot
\textbf{Free-space} & Satellite, LOS & All & Baseline (ideal) \\
\textbf{Two-ray} & Flat terrain, LOS/reflection & VHF+ &
\$\textbackslash pm\$6 dB \\
\textbf{Okumura-Hata} & Urban/suburban cellular & 150 MHz - 2 GHz &
\$\textbackslash pm\$10 dB \\
\textbf{COST-231} & Urban microcells & 800 MHz - 2 GHz &
\$\textbackslash pm\$8 dB \\
\textbf{ITU-R P.1546} & Broadcast (TV/FM) & 30 MHz - 3 GHz &
\$\textbackslash pm\$10 dB \\
\textbf{ITU-R P.368} & Ground wave & LF/MF/HF & \$\textbackslash pm\$5
dB \\
\textbf{Longley-Rice} & Irregular terrain & 20 MHz - 20 GHz &
\$\textbackslash pm\$12 dB \\
\end{longtable}
}

\begin{center}\rule{0.5\linewidth}{0.5pt}\end{center}

\subsection{Related Topics}\label{related-topics}

\begin{itemize}
\tightlist
\item
  \textbf{{[}{[}Free-Space-Path-Loss-(FSPL){]}{]}}: Baseline loss for
  all propagation modes
\item
  \textbf{{[}{[}Multipath-Propagation-\&-Fading-(Rayleigh,-Rician){]}{]}}:
  Rayleigh/Rician fading in NLOS
\item
  \textbf{{[}{[}Atmospheric-Effects-(Ionospheric,-Tropospheric){]}{]}}:
  Ionospheric refraction, atmospheric absorption
\item
  \textbf{{[}{[}Weather-Effects-(Rain-Fade,-Fog-Attenuation){]}{]}}:
  Rain fade at high frequencies
\item
  \textbf{{[}{[}Electromagnetic-Spectrum{]}{]}}: Frequency-dependent
  propagation behavior
\item
  \textbf{{[}{[}Antenna-Theory-Basics{]}{]}}: Antenna height extends
  radio horizon
\end{itemize}

\begin{center}\rule{0.5\linewidth}{0.5pt}\end{center}

\textbf{Key takeaway}: \textbf{Propagation mode depends on frequency}.
LF/MF = ground wave, HF = skywave, VHF+ = LOS. Understanding which mode
applies is critical for predicting coverage and designing reliable
links.

\begin{center}\rule{0.5\linewidth}{0.5pt}\end{center}

\emph{This wiki is part of the {[}{[}Home\textbar Chimera Project{]}{]}
documentation.}
