\section{Constellation Diagrams}\label{constellation-diagrams}

\subsection{\texorpdfstring{ For Non-Technical
Readers}{ For Non-Technical Readers}}\label{for-non-technical-readers}

\textbf{A constellation diagram is like a visual map showing all the
``hand signals'' your WiFi/phone can use to send data-\/-\/-each dot is
a unique signal position!}

\textbf{The analogy - Lighthouse signals}: - Imagine
you\textquotesingle re communicating with lighthouse beams - You can
vary: \textbf{brightness} (amplitude) and \textbf{color} (phase) - Each
unique combination = one symbol (represents some bits) - Constellation
diagram = map showing all possible combinations!

\textbf{Real example - QPSK (4 dots)}:

\begin{verbatim}
     Q
     
    |      4 positions = 2 bits per symbol
-----+----- I
    |  
\end{verbatim}

\begin{itemize}
\tightlist
\item
  Top-right \$\textbackslash bullet\$ = ``00''
\item
  Top-left \$\textbackslash bullet\$ = ``01''
\item
  Bottom-left \$\textbackslash bullet\$ = ``10''
\item
  Bottom-right \$\textbackslash bullet\$ = ``11''
\end{itemize}

\textbf{Why dots matter}: - \textbf{More dots} = more data per symbol =
faster! - QPSK: 4 dots (2 bits/symbol) - 16-QAM: 16 dots (4 bits/symbol)
= 2\$\textbackslash times\$ faster - 256-QAM: 256 dots (8 bits/symbol) =
4\$\textbackslash times\$ faster! - \textbf{Dots closer together} =
harder to distinguish when noisy - Your phone uses fewer dots when
signal is weak (reliable) - Uses more dots when signal is strong (fast!)

\textbf{When you see it}: - \textbf{WiFi speed negotiation}:
``Constellation: 64-QAM'' = using 64-dot map - \textbf{Spectrum
analyzer}: Shows received dots scattered around ideal positions (noise!)
- \textbf{Signal quality}: Dots tight = good signal, dots spread out =
noisy channel

\textbf{Fun fact}: Your WiFi constantly monitors how scattered the
received dots are and automatically switches between constellations
(4/16/64/256/1024-QAM) to optimize speed vs reliability!

\begin{center}\rule{0.5\linewidth}{0.5pt}\end{center}

A \textbf{constellation diagram} is a visual representation of a digital
modulation scheme. It shows all possible symbol positions in the I/Q
plane.

\subsection{Reading a Constellation
Diagram}\label{reading-a-constellation-diagram}

\begin{verbatim}
      Q (Imaginary)
           
           |
      +----+----+
      |   |   |  Each dot represents
      |    |    |  a valid symbol position
   ---+----+----+--- I (Real)
      |    |    |
      |   |   |
      +----+----+
           |
\end{verbatim}

\subsection{Key Elements}\label{key-elements}

\begin{enumerate}
\def\labelenumi{\arabic{enumi}.}
\tightlist
\item
  \textbf{Ideal Points}: Perfect symbol positions (clean transmission)
\item
  \textbf{Received Cloud}: Actual received symbols scattered due to
  noise
\item
  \textbf{Decision Boundaries}: Regions that determine which symbol was
  sent
\end{enumerate}

\subsection{TX vs RX Constellations}\label{tx-vs-rx-constellations}

\subsubsection{TX Constellation
(Transmitter)}\label{tx-constellation-transmitter}

\begin{itemize}
\tightlist
\item
  Shows ideal symbol positions
\item
  Points are crisp and perfectly positioned
\item
  Represents what was intended to be transmitted
\end{itemize}

\subsubsection{RX Constellation
(Receiver)}\label{rx-constellation-receiver}

\begin{itemize}
\tightlist
\item
  Shows received symbol positions after channel effects
\item
  Points are scattered in clouds around ideal positions
\item
  Scattering indicates noise level and channel quality
\item
  Larger scatter = more noise = harder to decode correctly
\end{itemize}

\subsection{What the Constellation Tells
You}\label{what-the-constellation-tells-you}

{\def\LTcaptype{} % do not increment counter
\begin{longtable}[]{@{}lll@{}}
\toprule\noalign{}
Pattern & Meaning & Quality \\
\midrule\noalign{}
\endhead
\bottomrule\noalign{}
\endlastfoot
Tight clusters & Low noise, high SNR & Excellent \\
Scattered clouds & High noise, low SNR & Poor \\
Symbol overlap & Very poor signal quality & Critical \\
Pattern offset & Frequency or phase errors & Requires correction \\
\end{longtable}
}

\subsection{Example: QPSK Constellation at Different SNR
Levels}\label{example-qpsk-constellation-at-different-snr-levels}

\subsubsection{High SNR (-5 dB Channel)}\label{high-snr--5-db-channel}

\begin{verbatim}
  Q
  
 |       Tight clusters
--+-- I    Perfect separation
 | 
\end{verbatim}

\subsubsection{Medium SNR (-15 dB
Channel)}\label{medium-snr--15-db-channel}

\begin{verbatim}
  Q
  
|     Visible scatter
|    Still decodable
----+---- I
| 
|
\end{verbatim}

\subsubsection{Low SNR (-25 dB Channel)}\label{low-snr--25-db-channel}

\begin{verbatim}
  Q
  
 |     Heavy scatter
|   Errors likely
|  FEC required
-----+----- I
|
|
 |
\end{verbatim}

\subsection{Observing Constellations in
Chimera}\label{observing-constellations-in-chimera}

When you run a simulation:

\begin{enumerate}
\def\labelenumi{\arabic{enumi}.}
\tightlist
\item
  \textbf{TX Constellation Panel}: See the ideal QPSK symbol positions
\item
  \textbf{RX Constellation Panel}: See how noise scatters the received
  symbols
\item
  \textbf{Adjust SNR}: Watch how constellation quality changes in
  real-time
\end{enumerate}

The constellation is the most intuitive way to understand signal
quality!

\subsection{See Also}\label{see-also}

\begin{itemize}
\tightlist
\item
  {[}{[}QPSK-Modulation{]}{]} - The modulation scheme being visualized
\item
  {[}{[}Signal-to-Noise-Ratio-(SNR){]}{]} - What controls the scatter
\item
  {[}{[}IQ-Representation{]}{]} - The coordinate system
\item
  {[}{[}Reading the Constellation{]}{]} - Practical interpretation guide
\end{itemize}
