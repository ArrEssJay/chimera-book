\section{Frequency-Shift Keying (FSK)}\label{frequency-shift-keying-fsk}

\subsection{\texorpdfstring{ For Non-Technical
Readers}{ For Non-Technical Readers}}\label{for-non-technical-readers}

\textbf{FSK is like morse code with two different musical
notes-\/-\/-high note = 1, low note = 0. Simple, robust, and still used
everywhere!}

\textbf{The idea}: - Want to send a \textbf{1}? Transmit at \textbf{high
frequency} (e.g., 1200 Hz) - Want to send a \textbf{0}? Transmit at
\textbf{low frequency} (e.g., 1000 Hz) - Receiver listens for which tone
is present

\textbf{Musical analogy}: - Playing piano: \textbf{C note} = 0,
\textbf{E note} = 1 - Song: ``C C E C E E C'' = data: ``0 0 1 0 1 1 0''
- Your ear (receiver) easily distinguishes C from E! - FSK receiver does
the same with radio frequencies

\textbf{Why it\textquotesingle s great}: - \textbf{Super robust}: Noise
changes amplitude, but frequency stays clear! - \textbf{Simple}: Just
detect which frequency is present - \textbf{Immune to fading}: Signal
can get weaker, but frequency doesn\textquotesingle t change -
\textbf{Works in harsh environments}: Industrial, underwater, long-range

\textbf{Where you encounter FSK}: - \textbf{Caller ID}: Your phone uses
FSK to send caller info between rings! - \textbf{Old dial-up modems}:
1980s modems used FSK (remember the screeching sound?) -
\textbf{Bluetooth Low Energy}: Uses GFSK (Gaussian FSK) for low power -
\textbf{RFID tags}: Many use FSK for simplicity - \textbf{Weather
balloons}: FSK survives atmospheric interference - \textbf{Pagers}:
Remember pagers? FSK!

\textbf{Real-world sounds}: - \textbf{Fax machine}: That squawking noise
is FSK! Listen carefully-\/-\/-you can hear the two tones alternating -
\textbf{Dial-up internet}: BEEEE-doo-BEEEE-doo = FSK handshake -
\textbf{Emergency broadcast tones}: Two-tone alert = FSK

\textbf{Variants}: - \textbf{BFSK}: Binary (2 frequencies) = 1
bit/symbol - \textbf{MFSK}: Multiple frequencies (4, 8, 16, etc.) = more
bits/symbol - \textbf{GFSK}: Gaussian FSK (smooth transitions) = used in
Bluetooth

\textbf{Trade-off}: - \textbf{Advantage}: Extremely robust, immune to
amplitude variations - \textbf{Disadvantage}: Slow compared to QAM
(lower spectral efficiency) - Best for: Low-power, long-range, harsh
environments

\textbf{Fun fact}: Old telegraph operators could ``read'' morse code by
EAR at 40+ words/minute. FSK is the same idea-\/-\/-humans can literally
hear binary data if you slow it down!

\begin{center}\rule{0.5\linewidth}{0.5pt}\end{center}

\textbf{Frequency-Shift Keying (FSK)} is a digital modulation scheme
where binary data is represented by discrete changes in carrier
frequency.

\begin{center}\rule{0.5\linewidth}{0.5pt}\end{center}

\subsection{\texorpdfstring{ Basic
Principle}{ Basic Principle}}\label{basic-principle}

\textbf{Binary FSK (BFSK)}:

\begin{verbatim}
Bit "1": s(t) = A·cos(2f·t)
Bit "0": s(t) = A·cos(2f·t)

where:
- A = constant amplitude
- f = "mark" frequency (higher)
- f = "space" frequency (lower)
- f = f - f = frequency separation
\end{verbatim}

\textbf{Time-domain representation}:

\begin{verbatim}
Frequency
   
f |---|   |---|       |---|     Bit "1"
   |   |   |   |       |   |
f |   |---|   |-------|   |---  Bit "0"
   +-------------------------- Time
       1   0   1   0       1
\end{verbatim}

\begin{center}\rule{0.5\linewidth}{0.5pt}\end{center}

\subsection{\texorpdfstring{ Mathematical
Description}{ Mathematical Description}}\label{mathematical-description}

\textbf{Transmitted signal}:

\begin{verbatim}
s(t) = A·cos[2(f_c + b_k·f/2)·t]     for kT_b  t < (k+1)T_b

where:
- f_c = carrier frequency (center)
- b_k  {-1, +1} (or {0, 1})
- f = frequency deviation
- T_b = bit duration
\end{verbatim}

\textbf{Modulation index}:

\begin{verbatim}
h = 2f·T_b

Common values:
- h = 0.5 (Minimum Shift Keying - MSK)
- h = 1.0 (Sunde's FSK)
- h > 1 (Wideband FSK)
\end{verbatim}

\begin{center}\rule{0.5\linewidth}{0.5pt}\end{center}

\subsection{\texorpdfstring{ Spectral
Characteristics}{ Spectral Characteristics}}\label{spectral-characteristics}

\textbf{Bandwidth} (Carson\textquotesingle s rule):

\begin{verbatim}
B = 2(f + R_b)

where R_b = 1/T_b = bit rate
\end{verbatim}

\textbf{Examples}: - Narrowband FSK (h = 0.5): B
\$\textbackslash approx\$ 1.5 R\_b - Wideband FSK (h = 2): B
\$\textbackslash approx\$ 5 R\_b

\textbf{Power spectral density}: Two main lobes centered at
f\textbackslash textsubscript\{0\} and
f\textbackslash textsubscript\{1\}

\begin{center}\rule{0.5\linewidth}{0.5pt}\end{center}

\subsection{\texorpdfstring{ Demodulation
Methods}{ Demodulation Methods}}\label{demodulation-methods}

\subsubsection{1. Non-Coherent Detection (Envelope
Detector)}\label{non-coherent-detection-envelope-detector}

\textbf{Simple and practical} - no carrier phase recovery!

\begin{verbatim}
Architecture:

         +---------+
    r(t)-+ BPF @ f+--+
         +---------+  |
                      +- Envelope  Compare  Decision
         +---------+  |   Detectors
    r(t)-+ BPF @ f+--+
         +---------+

Decision:
If |output of f filter| > |output of f filter|: bit = 1
Else: bit = 0
\end{verbatim}

\textbf{Advantages}: Simple, no synchronization \textbf{Disadvantages}:
\textasciitilde1 dB worse than coherent

\begin{center}\rule{0.5\linewidth}{0.5pt}\end{center}

\subsubsection{2. Coherent Detection
(Correlation)}\label{coherent-detection-correlation}

\textbf{Optimal performance} but requires carrier synchronization:

\begin{verbatim}
Correlators:

         +----------+
    r(t)-+ × cos(2ft) +--   z
         +----------+      0 to Tb

         +----------+
    r(t)-+ × cos(2ft) +--   z
         +----------+      0 to Tb

Decision:
If z > z: bit = 1
Else: bit = 0
\end{verbatim}

\begin{center}\rule{0.5\linewidth}{0.5pt}\end{center}

\subsubsection{3. Frequency
Discriminator}\label{frequency-discriminator}

\textbf{Classic FM receiver approach}:

\begin{verbatim}
r(t)  [Limiter]  [Frequency Discriminator]  [LPF]  Decision
\end{verbatim}

\textbf{Converts frequency deviation to voltage}, then samples at bit
boundaries.

\begin{center}\rule{0.5\linewidth}{0.5pt}\end{center}

\subsection{\texorpdfstring{ Performance
Analysis}{ Performance Analysis}}\label{performance-analysis}

\subsubsection{Bit Error Rate (BER)}\label{bit-error-rate-ber}

\textbf{With non-coherent detection} (AWGN channel):

\begin{verbatim}
BER = (1/2)exp(-E_b/2N)      for orthogonal FSK

where:
- E_b = bit energy = (A²T_b)/2
- N = noise power spectral density
\end{verbatim}

\textbf{With coherent detection}:

\begin{verbatim}
BER = Q((E_b/N))            (1 dB better!)
\end{verbatim}

\textbf{For orthogonal FSK}: Frequencies
f\textbackslash textsubscript\{0\} and
f\textbackslash textsubscript\{1\} must satisfy:

\begin{verbatim}
(f - f)·T_b = n/2    (n = integer)

Minimum: f = 1/(2T_b)   h = 1 (Sunde's FSK)
\end{verbatim}

\begin{center}\rule{0.5\linewidth}{0.5pt}\end{center}

\subsection{\texorpdfstring{ Advantages \&
Disadvantages}{ Advantages \& Disadvantages}}\label{advantages-disadvantages}

\subsubsection{Advantages}\label{advantages}

\textbf{Constant envelope} - efficient power amplifiers (Class C)
\textbf{Non-coherent detection} - simple receivers \textbf{Robust to
fading} - amplitude variations don\textquotesingle t affect frequency
\textbf{Good for noisy channels} - frequency easier to detect than phase
\textbf{Legacy compatibility} - used in many older systems

\subsubsection{Disadvantages}\label{disadvantages}

\textbf{Poor spectral efficiency} - wider bandwidth than PSK
\textbf{Moderate power efficiency} - 1-2 dB worse than {[}{[}BPSK{]}{]}
\textbf{Frequency stability} - requires accurate oscillators
\textbf{Doppler sensitivity} - frequency shifts problematic

\begin{center}\rule{0.5\linewidth}{0.5pt}\end{center}

\subsection{\texorpdfstring{
Applications}{ Applications}}\label{applications}

\subsubsection{Historical \& Current}\label{historical-current}

\begin{itemize}
\tightlist
\item
  \textbf{Telephone modems} (Bell 103: 1962, 300 baud,
  f\textbackslash textsubscript\{0\}=1070 Hz,
  f\textbackslash textsubscript\{1\}=1270 Hz)
\item
  \textbf{Radio teletype} (RTTY, 1930s-)
\item
  \textbf{Caller ID} (Bell 202: 1200 bps,
  f\textbackslash textsubscript\{0\}=2200 Hz,
  f\textbackslash textsubscript\{1\}=1200 Hz)
\item
  \textbf{Pagers} (POCSAG, FLEX protocols)
\end{itemize}

\subsubsection{Modern}\label{modern}

\begin{itemize}
\tightlist
\item
  \textbf{LoRa} (sub-GHz IoT, chirp spread spectrum FSK)
\item
  \textbf{Bluetooth Low Energy} (GFSK - Gaussian FSK)
\item
  \textbf{Wireless sensor networks} - low power, simple receivers
\item
  \textbf{Optical fiber} (frequency-shifted laser)
\item
  \textbf{{[}{[}AID-Protocol-Case-Study{]}{]}} - 12 kHz FSK sub-carrier
  (11,999/12,001 Hz)
\end{itemize}

\begin{center}\rule{0.5\linewidth}{0.5pt}\end{center}

\subsection{\texorpdfstring{ FSK
Variants}{ FSK Variants}}\label{fsk-variants}

\subsubsection{1. Minimum Shift Keying
(MSK)}\label{minimum-shift-keying-msk}

\textbf{Special case}: h = 0.5 (minimum for orthogonality)

\begin{verbatim}
Properties:
- Continuous phase (no discontinuities)
- Constant envelope
- Bandwidth = 1.5 R_b (narrowest FSK)
- Equivalent to offset QPSK with sinusoidal pulse shaping
\end{verbatim}

\textbf{Used in}: GSM cellular (GMSK - Gaussian MSK)

\begin{center}\rule{0.5\linewidth}{0.5pt}\end{center}

\subsubsection{2. Gaussian FSK (GFSK)}\label{gaussian-fsk-gfsk}

\textbf{MSK + Gaussian pre-modulation filter}

\begin{verbatim}
Purpose: Further reduce spectral sidelobes
Bandwidth: ~1.2-1.5 R_b (depending on BT product)
BT product: Bandwidth × T_b (typical: 0.3-0.5)
\end{verbatim}

\textbf{Used in}: Bluetooth, Zigbee

\begin{center}\rule{0.5\linewidth}{0.5pt}\end{center}

\subsubsection{3. Continuous Phase FSK
(CPFSK)}\label{continuous-phase-fsk-cpfsk}

\textbf{Phase is continuous} across bit boundaries:

\begin{verbatim}
(t) = 2[f_c·t + (hf/2)·^{t} b()d]

Benefits:
- No spectral splatter
- Better spectral efficiency
- Smoother power envelope
\end{verbatim}

\begin{center}\rule{0.5\linewidth}{0.5pt}\end{center}

\subsubsection{4. Multi-Frequency FSK
(MFSK)}\label{multi-frequency-fsk-mfsk}

\textbf{M \textgreater{} 2 frequencies} for higher data rates:

\begin{verbatim}
M symbols  log(M) bits per symbol

Example (4-FSK):
- f: bits 00
- f: bits 01
- f: bits 10
- f: bits 11

Bandwidth: B = M·R_b (wider!)
Power efficiency: Better than BFSK for high M
\end{verbatim}

\textbf{Used in}: HF radio (MT63, Olivia modes)

\begin{center}\rule{0.5\linewidth}{0.5pt}\end{center}

\subsection{\texorpdfstring{ Constellation
Diagram}{ Constellation Diagram}}\label{constellation-diagram}

\textbf{BFSK in frequency space}:

\begin{verbatim}
Frequency
   
f |      Symbol "1"
   |
f_c|       (carrier)
   |
f |      Symbol "0"
   +------------ Time
\end{verbatim}

\textbf{Not a traditional I/Q constellation} (frequency, not
amplitude/phase).

\textbf{Equivalent I/Q representation} (for coherent detection):

\begin{verbatim}
      Q
      
      |
  0  |  1   On real axis, separated
      |
------+------ I
\end{verbatim}

\textbf{Distance between points}: d = \$\textbackslash sqrt\{\}\$(2E\_b)
(for orthogonal FSK)

\begin{center}\rule{0.5\linewidth}{0.5pt}\end{center}

\subsection{\texorpdfstring{ Comparison
Table}{ Comparison Table}}\label{comparison-table}

{\def\LTcaptype{} % do not increment counter
\begin{longtable}[]{@{}
  >{\raggedright\arraybackslash}p{(\linewidth - 10\tabcolsep) * \real{0.1579}}
  >{\raggedright\arraybackslash}p{(\linewidth - 10\tabcolsep) * \real{0.1711}}
  >{\raggedright\arraybackslash}p{(\linewidth - 10\tabcolsep) * \real{0.1447}}
  >{\raggedright\arraybackslash}p{(\linewidth - 10\tabcolsep) * \real{0.2500}}
  >{\raggedright\arraybackslash}p{(\linewidth - 10\tabcolsep) * \real{0.1316}}
  >{\raggedright\arraybackslash}p{(\linewidth - 10\tabcolsep) * \real{0.1447}}@{}}
\toprule\noalign{}
\begin{minipage}[b]{\linewidth}\raggedright
Modulation
\end{minipage} & \begin{minipage}[b]{\linewidth}\raggedright
Bits/Symbol
\end{minipage} & \begin{minipage}[b]{\linewidth}\raggedright
Bandwidth
\end{minipage} & \begin{minipage}[b]{\linewidth}\raggedright
Eb/N0 @ BER 10\^{}-6
\end{minipage} & \begin{minipage}[b]{\linewidth}\raggedright
Envelope
\end{minipage} & \begin{minipage}[b]{\linewidth}\raggedright
Detection
\end{minipage} \\
\midrule\noalign{}
\endhead
\bottomrule\noalign{}
\endlastfoot
{[}{[}On-Off-Keying-(OOK) & OOK{]}{]} & 1 & 2R\_b & 13.5 dB &
Variable \\
\textbf{FSK} & 1 & 2R\_b & 12.5 dB & Constant & Non-coherent \\
\textbf{MSK} & 1 & 1.5R\_b & 10.5 dB & Constant & Coherent \\
{[}{[}BPSK{]}{]} & 1 & R\_b & 10.5 dB & Constant & Coherent \\
{[}{[}QPSK-Modulation & QPSK{]}{]} & 2 & R\_b & 10.5 dB & Constant \\
\end{longtable}
}

\textbf{Key insight}: FSK trades bandwidth for simplicity.
{[}{[}BPSK{]}{]}/{[}{[}QPSK-Modulation\textbar QPSK{]}{]} are more
efficient but require phase synchronization.

\begin{center}\rule{0.5\linewidth}{0.5pt}\end{center}

\subsection{\texorpdfstring{ Key
Takeaways}{ Key Takeaways}}\label{key-takeaways}

\begin{enumerate}
\def\labelenumi{\arabic{enumi}.}
\tightlist
\item
  \textbf{Frequency switching}: Binary data
  \$\textbackslash rightarrow\$ two different frequencies
\item
  \textbf{Constant envelope}: Good for non-linear amplifiers
\item
  \textbf{Non-coherent detection}: Simple receivers, still good
  performance
\item
  \textbf{Bandwidth penalty}: \textasciitilde2\$\textbackslash times\$
  wider than PSK
\item
  \textbf{Robust}: Good for noisy, fading channels
\item
  \textbf{Still widely used}: Bluetooth, LoRa, pagers, caller ID
\item
  \textbf{Gateway to chirp spread spectrum}: LoRa uses frequency chirps
\end{enumerate}

\begin{center}\rule{0.5\linewidth}{0.5pt}\end{center}

\subsection{\texorpdfstring{ See Also}{ See Also}}\label{see-also}

\begin{itemize}
\tightlist
\item
  {[}{[}On-Off-Keying-(OOK){]}{]} - Simpler (amplitude modulation)
\item
  {[}{[}Binary-Phase-Shift-Keying-(BPSK){]}{]} - Alternative (phase
  modulation)
\item
  {[}{[}QPSK-Modulation{]}{]} - More bits per symbol (phase)
\item
  {[}{[}Constellation-Diagrams{]}{]} - Visualizing modulation schemes
\item
  {[}{[}AID-Protocol-Case-Study{]}{]} - Uses 1 bps FSK sub-carrier
  (11,999/12,001 Hz)
\end{itemize}

\begin{center}\rule{0.5\linewidth}{0.5pt}\end{center}

\subsection{\texorpdfstring{ References}{ References}}\label{references}

\begin{enumerate}
\def\labelenumi{\arabic{enumi}.}
\tightlist
\item
  \textbf{Sunde, E.D.} (1946) ``Ideal binary pulse transmission by AM
  and FM'' \emph{Bell Syst. Tech. J.} 25, 1067-1093
\item
  \textbf{de Jager, F. \& Dekker, C.B.} (1978) ``Tamed Frequency
  Modulation'' \emph{IEEE Trans. Comm.} COM-26, 534-542
\item
  \textbf{Proakis, J.G. \& Salehi, M.} (2008) \emph{Digital
  Communications} 5th ed.~(McGraw-Hill)
\item
  \textbf{Sklar, B.} (2001) \emph{Digital Communications} 2nd
  ed.~(Prentice Hall)
\end{enumerate}
