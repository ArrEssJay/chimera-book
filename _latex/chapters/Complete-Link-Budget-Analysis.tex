\section{Complete Link Budget
Analysis}\label{complete-link-budget-analysis}

{[}{[}Home{]}{]} \textbar{} \textbf{Link Budget \& System Performance}
\textbar{} {[}{[}Free-Space-Path-Loss-(FSPL){]}{]} \textbar{}
{[}{[}Signal-to-Noise-Ratio-(SNR){]}{]}

\begin{center}\rule{0.5\linewidth}{0.5pt}\end{center}

\subsection{\texorpdfstring{ For Non-Technical
Readers}{ For Non-Technical Readers}}\label{for-non-technical-readers}

\textbf{Link budget is like a financial budget for radio power-\/-\/-you
start with transmit power, subtract losses, add gains, and see if
there\textquotesingle s enough ``money'' (signal) left at the receiver!}

\textbf{The fundamental question}: \textgreater{} ``If I transmit from
HERE to THERE, will the receiver get enough signal?''

\textbf{The accounting}:

\textbf{START}: Transmit power - Your WiFi router: 100 mW (20 dBm) -
Your phone: 200 mW (23 dBm) - Cell tower: 20 W (43 dBm) - Satellite: 100
W (50 dBm)

\textbf{GAINS} (things that help): - \textbf{ Transmit antenna gain}:
Directional antenna focuses power - WiFi router: +2 dB (omnidirectional)
- Satellite dish: +40 dB (very focused!) - \textbf{ Receive antenna
gain}: Bigger receiver antenna collects more - Phone: +0 dB (tiny
antenna) - Satellite dish: +35 dB

\textbf{LOSSES} (things that hurt): - \textbf{ Free space path loss}:
Signal spreads out with distance - WiFi (50m): -74 dB - Cell tower (1
km): -100 dB\\
- Satellite (36,000 km): -206 dB! - \textbf{ Obstacles}: Walls, trees,
rain - One wall: -5 dB - Heavy rain: -10 dB - \textbf{ Cable losses}:
Connectors, imperfect cables - Typical: -1 to -3 dB

\textbf{END}: Received signal power - Must be stronger than noise floor!
- Typical requirement: Signal \textgreater{} Noise + 10 to 20 dB

\textbf{Example - WiFi Link Budget}:

\begin{verbatim}
Transmit power:          +20 dBm  (100 mW)
+ Transmit antenna:        +2 dB
------------------------------
EIRP (total radiated):   +22 dBm

- Free space loss (50m):  -74 dB
- Wall loss (2 walls):    -10 dB
------------------------------
Received power:          -62 dBm

Noise floor:             -90 dBm
SNR:                      28 dB   Good!
\end{verbatim}

\textbf{Real-world example - Satellite TV}:

\begin{verbatim}
Satellite transmit:      +50 dBm  (100 W)
+ Satellite antenna:      +35 dB   (huge!)
- Path loss (36,000 km): -206 dB   (ouch!)
+ Your dish:              +35 dB
------------------------------
Received power:          -86 dBm  (tiny!)

Noise floor:             -100 dBm
SNR:                      14 dB    Just enough!
\end{verbatim}

\textbf{Why link budget matters}: - \textbf{System design}: ``Do I need
a bigger antenna?'' - \textbf{Troubleshooting}: ``Why is my signal
weak?'' - \textbf{Standards}: ``What\textquotesingle s the maximum
range?'' - \textbf{Cost optimization}: ``Can I use cheaper components?''

\textbf{The critical moment}: - If received power \textgreater{} noise +
margin \$\textbackslash rightarrow\$ Link works! - If received power
\textless{} noise + margin \$\textbackslash rightarrow\$ Link fails! -
\textbf{Margin}: Extra dB for safety (rain, interference, fading) - Good
design: 10-20 dB margin - Marginal design: 3-5 dB margin (risky!)

\textbf{When you see it}: - \textbf{WiFi extender ads}: ``Extends range
by 20 dB!'' (link budget calc) - \textbf{Satellite dish size}: Bigger =
more gain = closes link budget - \textbf{Cell tower placement}:
Engineers run link budgets for coverage - \textbf{``Can you hear me
now?''}: That\textquotesingle s a link budget test!

\textbf{Fun fact}: The Voyager 1 spacecraft (13+ billion miles away)
transmits at 23 W, but by the time it reaches Earth, the signal is
10\^{}-16 watts-\/-\/-that\textquotesingle s 0.0000000000000001 watts!
Only massive 70-meter dishes with careful link budgets can hear it!

\begin{center}\rule{0.5\linewidth}{0.5pt}\end{center}

\subsection{Overview}\label{overview}

\textbf{Link budget} is a comprehensive accounting of \textbf{all gains
and losses} from transmitter to receiver, determining if a communication
link will work.

\textbf{Purpose}: Answer the critical question: \textgreater{} ``Will
the receiver get enough signal power to achieve the required data rate
and error rate?''

\textbf{Bottom line}:

\[
P_r = P_t + G_t - L_{\text{total}} + G_r \quad (\text{dBm or dBW})
\]

Where all gains/losses are in dB

\textbf{Link closes} if: \(P_r > P_{\text{min}}\) (receiver sensitivity)

\textbf{Margin}: \(M = P_r - P_{\text{min}}\) (dB of safety buffer)

\begin{center}\rule{0.5\linewidth}{0.5pt}\end{center}

\subsection{Link Budget Components}\label{link-budget-components}

\subsubsection{1. Transmitter Side}\label{transmitter-side}

\paragraph{Transmitted Power (P\_t)}\label{transmitted-power-p_t}

\textbf{RF power delivered to antenna} (after all TX losses):

\[
P_t = P_{\text{amp}} - L_{\text{TX}} \quad (\text{dB})
\]

Where: - \(P_{\text{amp}}\) = Power amplifier output (dBm) -
\(L_{\text{TX}}\) = TX losses (cables, filters, circulators)

\textbf{Example}: WiFi router - PA output: 20 dBm (100 mW) -
Cable/connector loss: 0.5 dB - \(P_t = 20 - 0.5 = 19.5\) dBm

\begin{center}\rule{0.5\linewidth}{0.5pt}\end{center}

\paragraph{Transmit Antenna Gain
(G\_t)}\label{transmit-antenna-gain-g_t}

\textbf{Gain relative to isotropic radiator} (dBi):

\textbf{EIRP} (Effective Isotropic Radiated Power):

\[
\text{EIRP} = P_t + G_t \quad (\text{dBm or dBW})
\]

\textbf{Example}: - \(P_t = 19.5\) dBm - \(G_t = 2\) dBi (WiFi router) -
EIRP = 19.5 + 2 = 21.5 dBm

\textbf{Regulatory limits}: FCC limits EIRP (e.g., 36 dBm for 2.4 GHz
WiFi)

\begin{center}\rule{0.5\linewidth}{0.5pt}\end{center}

\subsubsection{2. Propagation Path}\label{propagation-path}

\paragraph{Free-Space Path Loss (FSPL)}\label{free-space-path-loss-fspl}

\textbf{Loss due to spherical spreading}:

\[
L_{\text{FSPL}} = 20\log_{10}(d) + 20\log_{10}(f) + 20\log_{10}\left(\frac{4\pi}{c}\right)
\]

\textbf{Simplified}:

\[
L_{\text{FSPL}} = 32.45 + 20\log_{10}(d_{\text{km}}) + 20\log_{10}(f_{\text{MHz}}) \quad (\text{dB})
\]

\textbf{Example}: WiFi @ 2.4 GHz, 100 m - \(f = 2400\) MHz, \(d = 0.1\)
km

\[
L_{\text{FSPL}} = 32.45 + 20\log_{10}(0.1) + 20\log_{10}(2400)
\]

\[
= 32.45 - 20 + 67.6 = 80\ \text{dB}
\]

\textbf{See}: {[}{[}Free-Space-Path-Loss-(FSPL){]}{]}

\begin{center}\rule{0.5\linewidth}{0.5pt}\end{center}

\paragraph{Atmospheric Absorption}\label{atmospheric-absorption}

\textbf{Oxygen and water vapor absorption} (significant \textgreater{}
10 GHz):

\textbf{Zenith attenuation} (at sea level):

{\def\LTcaptype{} % do not increment counter
\begin{longtable}[]{@{}lll@{}}
\toprule\noalign{}
Frequency & Attenuation (dB/km) & Notes \\
\midrule\noalign{}
\endhead
\bottomrule\noalign{}
\endlastfoot
\textless{} 10 GHz & \textless{} 0.01 & Negligible \\
22.2 GHz & 0.2 & H\textbackslash textsubscript\{2\}O resonance \\
60 GHz & 15 & O\textbackslash textsubscript\{2\} resonance (peak) \\
120 GHz & 2 & Secondary O\textbackslash textsubscript\{2\} line \\
183 GHz & 5 & H\textbackslash textsubscript\{2\}O line \\
\end{longtable}
}

\textbf{Example}: Ka-band satellite @ 20 GHz,
5\$\^{}\textbackslash circ\$ elevation (path length \textasciitilde{} 11
km through atmosphere) - Attenuation: \textasciitilde0.05 dB/km
\$\textbackslash times\$ 11 km = \textbf{0.55 dB}

\textbf{See}:
{[}{[}Atmospheric-Effects-(Ionospheric,-Tropospheric){]}{]}

\begin{center}\rule{0.5\linewidth}{0.5pt}\end{center}

\paragraph{Rain Attenuation}\label{rain-attenuation}

\textbf{Dominant impairment for satellite Ku/Ka/V-band}:

\textbf{ITU-R model}: \(\gamma_R = k \cdot R^{\alpha}\) (dB/km)

\textbf{Example}: Ku-band @ 12 GHz, heavy rain (25 mm/hr), 4 km path - k
= 0.0188, \$\textbackslash alpha\$ = 1.217 -
\(\gamma_R = 0.0188 \times 25^{1.217} = 1.2\) dB/km - \textbf{Total
loss}: 1.2 \$\textbackslash times\$ 4 = 4.8 dB

\textbf{At 99\% availability}: Design for rain rate exceeded 1\% of time
(temperate: 12 mm/hr, tropical: 42 mm/hr)

\textbf{See}: {[}{[}Weather-Effects-(Rain-Fade,-Fog-Attenuation){]}{]}

\begin{center}\rule{0.5\linewidth}{0.5pt}\end{center}

\paragraph{Other Propagation Effects}\label{other-propagation-effects}

{\def\LTcaptype{} % do not increment counter
\begin{longtable}[]{@{}
  >{\raggedright\arraybackslash}p{(\linewidth - 4\tabcolsep) * \real{0.2051}}
  >{\raggedright\arraybackslash}p{(\linewidth - 4\tabcolsep) * \real{0.3590}}
  >{\raggedright\arraybackslash}p{(\linewidth - 4\tabcolsep) * \real{0.4359}}@{}}
\toprule\noalign{}
\begin{minipage}[b]{\linewidth}\raggedright
Effect
\end{minipage} & \begin{minipage}[b]{\linewidth}\raggedright
Typical Loss
\end{minipage} & \begin{minipage}[b]{\linewidth}\raggedright
When Applicable
\end{minipage} \\
\midrule\noalign{}
\endhead
\bottomrule\noalign{}
\endlastfoot
\textbf{Ionospheric scintillation} & 1-20 dB & L-band satellite,
equatorial, solar max \\
\textbf{Tropospheric scintillation} & 0.5-2 dB & Low elevation,
\textgreater{} 10 GHz \\
\textbf{Polarization mismatch} & 0-\$\textbackslash infty\$ dB & Antenna
misalignment, Faraday rotation \\
\textbf{Multipath fading} & 10-30 dB & Mobile, urban NLOS \\
\textbf{Foliage loss} & 0.3-1 dB/m & Trees, vegetation (VHF/UHF) \\
\textbf{Building penetration} & 5-20 dB & Indoor (depends on freq,
materials) \\
\end{longtable}
}

\textbf{See}:
{[}{[}Multipath-Propagation-\&-Fading-(Rayleigh,-Rician){]}{]},
{[}{[}Wave-Polarization{]}{]}

\begin{center}\rule{0.5\linewidth}{0.5pt}\end{center}

\subsubsection{3. Receiver Side}\label{receiver-side}

\paragraph{Receive Antenna Gain (G\_r)}\label{receive-antenna-gain-g_r}

\textbf{Same concept as TX antenna} (reciprocity):

\textbf{Example}: WiFi laptop - \(G_r = 0\) dBi (omnidirectional)

\textbf{Directional antenna}: - Parabolic dish: 30-60 dBi (satellite) -
Yagi: 10-15 dBi (TV, point-to-point)

\begin{center}\rule{0.5\linewidth}{0.5pt}\end{center}

\paragraph{Receiver Losses (L\_RX)}\label{receiver-losses-l_rx}

\textbf{Losses between antenna and receiver input}:

\begin{itemize}
\tightlist
\item
  Cable loss: 0.5-3 dB (depends on length, freq, cable type)
\item
  Connector loss: 0.1-0.5 dB per connector
\item
  Filter loss: 0.5-2 dB (bandpass filters)
\item
  Circulator/duplexer loss: 0.5-1 dB
\end{itemize}

\textbf{Example}: Satellite ground station - Cable: 2 dB (long run from
dish to equipment room) - Connectors: 0.3 dB - LNA inline: -40 dB (gain,
not loss!) - \textbf{Net}: 2 + 0.3 - 40 = -37.7 dB (LNA provides gain)

\begin{center}\rule{0.5\linewidth}{0.5pt}\end{center}

\paragraph{Receiver Sensitivity
(P\_min)}\label{receiver-sensitivity-p_min}

\textbf{Minimum signal power for acceptable performance}:

\[
P_{\text{min}} = -174 + 10\log_{10}(B) + \text{NF} + \text{SNR}_{\text{req}} + L_{\text{impl}} \quad (\text{dBm})
\]

Where: - \textbf{-174 dBm/Hz}: Thermal noise floor at 290 K -
\textbf{B}: Bandwidth (Hz) - \textbf{NF}: Noise figure (dB) -
\textbf{SNR\_req}: Required SNR for demodulation (dB) -
\textbf{L\_impl}: Implementation loss (quantization, imperfect sync,
etc.) \textasciitilde1-3 dB

\textbf{Example}: WiFi 802.11n, 20 MHz channel, QPSK 1/2 - Bandwidth: 20
MHz = 73 dBHz - NF: 6 dB (typical WiFi chipset) - SNR\_req: 5 dB (QPSK
with robust FEC) - L\_impl: 2 dB -
\(P_{\text{min}} = -174 + 73 + 6 + 5 + 2 = -88\) dBm

\textbf{See}: {[}{[}Noise-Sources-\&-Noise-Figure{]}{]}

\begin{center}\rule{0.5\linewidth}{0.5pt}\end{center}

\subsection{Complete Link Budget
Equation}\label{complete-link-budget-equation}

\[
P_r = \text{EIRP} - L_{\text{FSPL}} - L_{\text{atm}} - L_{\text{rain}} - L_{\text{other}} + G_r - L_{\text{RX}}
\]

\textbf{Expanded}:

\[
P_r = [P_t + G_t] - L_{\text{FSPL}} - L_{\text{atm}} - L_{\text{rain}} - L_{\text{multipath}} - L_{\text{misc}} + [G_r - L_{\text{cable}}]
\]

\textbf{Link margin}:

\[
M = P_r - P_{\text{min}}
\]

\textbf{Design guideline}: \(M \geq 10\) dB (provides fade margin,
interference tolerance)

\begin{center}\rule{0.5\linewidth}{0.5pt}\end{center}

\subsection{Example 1: WiFi Indoor
Link}\label{example-1-wifi-indoor-link}

\textbf{Scenario}: 2.4 GHz WiFi, 802.11n, 20 MHz, QPSK 1/2, 50 m indoor

\subsubsection{Transmitter}\label{transmitter}

\begin{itemize}
\tightlist
\item
  PA output: 20 dBm
\item
  Cable loss: 0.5 dB
\item
  \textbf{P\_t}: 19.5 dBm
\item
  Antenna gain: 2 dBi
\item
  \textbf{EIRP}: 21.5 dBm
\end{itemize}

\subsubsection{Path}\label{path}

\begin{itemize}
\tightlist
\item
  Free-space loss @ 50 m, 2.4 GHz:
\end{itemize}

\[
L_{\text{FSPL}} = 32.45 + 20\log_{10}(0.05) + 20\log_{10}(2400) = 32.45 - 26 + 67.6 = 74\ \text{dB}
\]

\begin{itemize}
\tightlist
\item
  Wall penetration (2 walls \$\textbackslash times\$ 5 dB): 10 dB
\item
  \textbf{Total path loss}: 74 + 10 = 84 dB
\end{itemize}

\subsubsection{Receiver}\label{receiver}

\begin{itemize}
\tightlist
\item
  Antenna gain: 0 dBi (laptop internal)
\item
  Cable loss: 0 dB (integrated)
\item
  \textbf{Received power}:
\end{itemize}

\[
P_r = 21.5 - 84 + 0 = -62.5\ \text{dBm}
\]

\subsubsection{Sensitivity}\label{sensitivity}

\begin{itemize}
\tightlist
\item
  Thermal noise: -174 dBm/Hz + 73 dBHz = -101 dBm
\item
  NF: 6 dB
\item
  SNR\_req: 5 dB (QPSK 1/2)
\item
  Impl loss: 2 dB
\item
  \textbf{P\_min}: -101 + 6 + 5 + 2 = -88 dBm
\end{itemize}

\subsubsection{Margin}\label{margin}

\[
M = -62.5 - (-88) = 25.5\ \text{dB}
\]

\textbf{Result}: Link \textbf{closes comfortably} with 25.5 dB margin
(can tolerate fading, interference)

\begin{center}\rule{0.5\linewidth}{0.5pt}\end{center}

\subsection{Example 2: GEO Satellite Ku-band
Downlink}\label{example-2-geo-satellite-ku-band-downlink}

\textbf{Scenario}: 12 GHz downlink, 36,000 km slant range, 1 m dish RX,
clear sky

\subsubsection{Transmitter (Satellite)}\label{transmitter-satellite}

\begin{itemize}
\tightlist
\item
  Satellite PA: 100 W = 50 dBm
\item
  TX antenna gain: 30 dBi (spot beam)
\item
  \textbf{EIRP}: 80 dBm = 80 dBW
\end{itemize}

\subsubsection{Path}\label{path-1}

\begin{itemize}
\tightlist
\item
  Distance: 36,000 km
\item
  Frequency: 12 GHz
\end{itemize}

\[
L_{\text{FSPL}} = 32.45 + 20\log_{10}(36,000) + 20\log_{10}(12,000)
\]

\[
= 32.45 + 91.1 + 81.6 = 205\ \text{dB}
\]

\begin{itemize}
\tightlist
\item
  Atmospheric absorption (5\$\^{}\textbackslash circ\$ elevation): 0.5
  dB
\item
  Clear-sky rain (0.01 mm/hr): 0.01 dB (negligible)
\item
  Ionospheric scintillation margin: 2 dB
\item
  \textbf{Total path loss}: 205 + 0.5 + 0 + 2 = 207.5 dB
\end{itemize}

\subsubsection{Receiver (Ground Station)}\label{receiver-ground-station}

\begin{itemize}
\tightlist
\item
  Dish diameter: 1 m
\item
  Antenna gain (eff 60\%):
\end{itemize}

\[
G_r = 10\log_{10}\left(0.6 \times \left(\frac{\pi \times 1}{0.025}\right)^2\right) = 10\log_{10}(0.6 \times 1580) = 37.8\ \text{dBi}
\]

\begin{itemize}
\tightlist
\item
  LNA noise figure: 0.8 dB (cryogenic)
\item
  Cable loss: 1 dB
\item
  \textbf{Net RX gain}: 37.8 - 1 = 36.8 dB
\end{itemize}

\textbf{Received power}:

\[
P_r = 80 - 207.5 + 36.8 = -90.7\ \text{dBm}
\]

\subsubsection{Sensitivity (DVB-S2, QPSK 3/4, 36 MHz
bandwidth)}\label{sensitivity-dvb-s2-qpsk-34-36-mhz-bandwidth}

\begin{itemize}
\tightlist
\item
  Bandwidth: 36 MHz = 75.6 dBHz
\item
  Thermal noise: -174 + 75.6 = -98.4 dBm
\item
  NF: 0.8 dB (LNA at antenna)
\item
  SNR\_req: 6.5 dB (QPSK 3/4 with LDPC)
\item
  Impl loss: 1.5 dB
\item
  \textbf{P\_min}: -98.4 + 0.8 + 6.5 + 1.5 = -89.6 dBm
\end{itemize}

\subsubsection{Margin (Clear Sky)}\label{margin-clear-sky}

\[
M = -90.7 - (-89.6) = -1.1\ \text{dB}
\]

\textbf{Uh oh!} Link \textbf{fails} in clear sky (need higher gain or
more TX power)

\textbf{Fix}: Increase dish to 1.8 m - New gain: 37.8 + 20log(1.8) =
42.9 dBi - New \(P_r\): 80 - 207.5 + 42.9 = -84.6 dBm - \textbf{New
margin}: -84.6 - (-89.6) = \textbf{5 dB} (marginal)

\textbf{With 99\% rain margin} (add 5 dB rain attenuation): - \(P_r\) in
rain: -84.6 - 5 = -89.6 dBm - \textbf{Rain margin}: 0 dB (link at
threshold!)

\textbf{Better fix}: Use 2.4 m dish - Gain: 47.5 dBi - Clear sky: -80
dBm, margin 10 dB - 99\% rain: -85 dBm, margin 5 dB

\begin{center}\rule{0.5\linewidth}{0.5pt}\end{center}

\subsection{Example 3: Cellular LTE (2.6
GHz)}\label{example-3-cellular-lte-2.6-ghz}

\textbf{Scenario}: eNodeB to UE, 2.6 GHz, 10 MHz RB, QPSK 1/2, 5 km
suburban

\subsubsection{Transmitter (Cell Tower)}\label{transmitter-cell-tower}

\begin{itemize}
\tightlist
\item
  PA per antenna: 43 dBm (20 W)
\item
  TX antenna: 17 dBi (sector antenna)
\item
  Cable loss: 2 dB
\item
  \textbf{EIRP}: 43 + 17 - 2 = 58 dBm
\end{itemize}

\subsubsection{Path}\label{path-2}

\begin{itemize}
\tightlist
\item
  FSPL @ 5 km, 2.6 GHz:
\end{itemize}

\[
L_{\text{FSPL}} = 32.45 + 20\log_{10}(5) + 20\log_{10}(2600) = 32.45 + 14 + 68.3 = 115\ \text{dB}
\]

\begin{itemize}
\tightlist
\item
  Shadowing margin (suburban log-normal): 8 dB (for 90\% coverage)
\item
  Building penetration: 10 dB (indoor UE)
\item
  \textbf{Total path loss}: 115 + 8 + 10 = 133 dB
\end{itemize}

\subsubsection{Receiver (UE)}\label{receiver-ue}

\begin{itemize}
\tightlist
\item
  Antenna gain: -2 dBi (internal, near body)
\item
  \textbf{Received power}:
\end{itemize}

\[
P_r = 58 - 133 - 2 = -77\ \text{dBm}
\]

\subsubsection{Sensitivity (10 MHz, QPSK
1/2)}\label{sensitivity-10-mhz-qpsk-12}

\begin{itemize}
\tightlist
\item
  Bandwidth: 10 MHz = 70 dBHz
\item
  Thermal noise: -174 + 70 = -104 dBm
\item
  NF: 9 dB (smartphone front-end)
\item
  SNR\_req: 4 dB (QPSK 1/2 Turbo code)
\item
  Impl loss: 2 dB
\item
  \textbf{P\_min}: -104 + 9 + 4 + 2 = -89 dBm
\end{itemize}

\subsubsection{Margin}\label{margin-1}

\[
M = -77 - (-89) = 12\ \text{dB}
\]

\textbf{Result}: Link \textbf{closes} with 12 dB margin (adequate for
mobile fading)

\textbf{With Rayleigh fading} (10 dB fade depth @ 10\% time): - Faded
\(P_r\): -77 - 10 = -87 dBm - \textbf{Faded margin}: -87 - (-89) = 2 dB
(still works, but error rate increases)

\textbf{Diversity RX} (2 antennas, max ratio combining): - Diversity
gain: 5 dB (typical for 2-branch) - Effective \(P_r\) in fade: -87 + 5 =
-82 dBm - \textbf{Margin with diversity}: -82 - (-89) = 7 dB (much
better!)

\begin{center}\rule{0.5\linewidth}{0.5pt}\end{center}

\subsection{Link Budget Table
Template}\label{link-budget-table-template}

{\def\LTcaptype{} % do not increment counter
\begin{longtable}[]{@{}
  >{\raggedright\arraybackslash}p{(\linewidth - 8\tabcolsep) * \real{0.2750}}
  >{\raggedright\arraybackslash}p{(\linewidth - 8\tabcolsep) * \real{0.2000}}
  >{\raggedright\arraybackslash}p{(\linewidth - 8\tabcolsep) * \real{0.1750}}
  >{\raggedright\arraybackslash}p{(\linewidth - 8\tabcolsep) * \real{0.1750}}
  >{\raggedright\arraybackslash}p{(\linewidth - 8\tabcolsep) * \real{0.1750}}@{}}
\toprule\noalign{}
\begin{minipage}[b]{\linewidth}\raggedright
Parameter
\end{minipage} & \begin{minipage}[b]{\linewidth}\raggedright
Symbol
\end{minipage} & \begin{minipage}[b]{\linewidth}\raggedright
Value
\end{minipage} & \begin{minipage}[b]{\linewidth}\raggedright
Units
\end{minipage} & \begin{minipage}[b]{\linewidth}\raggedright
Notes
\end{minipage} \\
\midrule\noalign{}
\endhead
\bottomrule\noalign{}
\endlastfoot
\textbf{TRANSMITTER} & & & & \\
TX power (PA) & \(P_{\text{amp}}\) & & dBm & \\
TX losses & \(L_{\text{TX}}\) & & dB & Cables, filters \\
Transmit power & \(P_t\) & & dBm & \(P_{\text{amp}} - L_{\text{TX}}\) \\
TX antenna gain & \(G_t\) & & dBi & \\
\textbf{EIRP} & & & dBm & \(P_t + G_t\) \\
\textbf{PROPAGATION} & & & & \\
Distance & \(d\) & & km & \\
Frequency & \(f\) & & GHz & \\
Free-space loss & \(L_{\text{FSPL}}\) & & dB & 32.45 + 20log(d) +
20log(f) \\
Atmospheric loss & \(L_{\text{atm}}\) & & dB & \\
Rain attenuation & \(L_{\text{rain}}\) & & dB & ITU model \\
Other losses & \(L_{\text{other}}\) & & dB & Multipath, penetration,
etc. \\
\textbf{Total path loss} & \(L_{\text{total}}\) & & dB & Sum \\
\textbf{RECEIVER} & & & & \\
RX antenna gain & \(G_r\) & & dBi & \\
RX losses & \(L_{\text{RX}}\) & & dB & Cables, connectors \\
\textbf{Received power} & \(P_r\) & & dBm & EIRP - \(L_{\text{total}}\)
+ \(G_r\) - \(L_{\text{RX}}\) \\
\textbf{PERFORMANCE} & & & & \\
Bandwidth & \(B\) & & MHz & \\
Thermal noise & \(N_0\) & -174 + 10log(B) & dBm & \\
Noise figure & NF & & dB & \\
Noise power & \(N\) & \(N_0\) + NF & dBm & \\
Required SNR & SNR\_req & & dB & For target BER \\
Impl loss & \(L_{\text{impl}}\) & & dB & Typically 1-3 dB \\
\textbf{Sensitivity} & \(P_{\text{min}}\) & & dBm & \(N\) + SNR\_req +
\(L_{\text{impl}}\) \\
\textbf{MARGIN} & \(M\) & & dB & \(P_r - P_{\text{min}}\) \\
\end{longtable}
}

\begin{center}\rule{0.5\linewidth}{0.5pt}\end{center}

\subsection{Fade Margin Design
Guidelines}\label{fade-margin-design-guidelines}

\textbf{Clear-sky margin} (no fading): - \textbf{Satellite (GEO Ku/Ka)}:
5-10 dB - \textbf{Terrestrial LOS}: 10-15 dB - \textbf{Mobile (NLOS)}:
15-20 dB

\textbf{Rain margin} (satellite): - \textbf{Availability}: 99\%
\$\textbackslash rightarrow\$ 5-8 dB, 99.9\%
\$\textbackslash rightarrow\$ 10-15 dB, 99.99\%
\$\textbackslash rightarrow\$ 20-30 dB - \textbf{Frequency}: Ku-band:
3-10 dB, Ka-band: 10-20 dB

\textbf{Multipath fading margin}: - \textbf{Rayleigh fading}: 20-30 dB
for 90\% location reliability - \textbf{Rician K=6 dB}: 10-15 dB

\textbf{Total design margin}:

\[
M_{\text{total}} = M_{\text{clear}} + M_{\text{rain}} + M_{\text{fade}}
\]

\textbf{Trade-off}: Higher margin \$\textbackslash rightarrow\$ More
expensive (bigger antennas, more power)

\begin{center}\rule{0.5\linewidth}{0.5pt}\end{center}

\subsection{Adaptive Techniques}\label{adaptive-techniques}

\textbf{Adaptive Coding and Modulation (ACM)}:

\textbf{Concept}: Change modulation/code rate based on channel
conditions

\textbf{Example}: DVB-S2X satellite - Clear sky: 32APSK 9/10
\$\textbackslash rightarrow\$ 3.5 bits/symbol, needs C/N = 16 dB - Light
rain: 8PSK 3/4 \$\textbackslash rightarrow\$ 2.25 bits/symbol, needs C/N
= 11 dB - Heavy rain: QPSK 1/2 \$\textbackslash rightarrow\$ 1
bit/symbol, needs C/N = 4 dB

\textbf{Benefit}: Maximize throughput when conditions good, maintain
connectivity when conditions poor

\begin{center}\rule{0.5\linewidth}{0.5pt}\end{center}

\subsection{Link Availability}\label{link-availability}

\textbf{Probability link meets performance requirement}:

\[
\text{Availability} = \frac{\text{Time link works}}{\text{Total time}} \times 100\%
\]

\textbf{Target availability} (depends on application): - \textbf{Data
networks}: 99.9\% (8.76 hours/year downtime) - \textbf{Voice}: 99.99\%
(52.6 minutes/year) - \textbf{Mission-critical}: 99.999\% (5.26
minutes/year)

\textbf{Dominated by rain} (for satellite Ku/Ka-band):

\textbf{ITU rain statistics}: - Temperate: 12 mm/hr exceeded 1\% of time
(3.65 days/year) - Tropical: 42 mm/hr exceeded 1\% of time

\textbf{Design procedure}: 1. Choose target availability (e.g., 99.9\%)
2. Find rain rate exceeded 0.1\% of time (e.g., 25 mm/hr temperate) 3.
Calculate rain attenuation for that rain rate 4. Ensure link margin
\textgreater{} rain attenuation

\begin{center}\rule{0.5\linewidth}{0.5pt}\end{center}

\subsection{Summary}\label{summary}

\textbf{Link budget essentials}:

\begin{enumerate}
\def\labelenumi{\arabic{enumi}.}
\tightlist
\item
  \textbf{EIRP} = TX power + TX gain (dBm)
\item
  \textbf{Path loss} = FSPL + atmospheric + rain + other (dB)
\item
  \textbf{RX power} = EIRP - path loss + RX gain - RX losses (dBm)
\item
  \textbf{Sensitivity} = Noise floor + NF + SNR\_req + impl loss (dBm)
\item
  \textbf{Margin} = RX power - sensitivity (dB, must be positive!)
\end{enumerate}

\textbf{Design targets}: - Clear-sky margin: 10+ dB - Rain margin: 5-20
dB (depends on frequency, availability) - Total margin: 15-30 dB typical

\textbf{Adaptive techniques} (ACM, diversity) improve spectral
efficiency and availability.

\begin{center}\rule{0.5\linewidth}{0.5pt}\end{center}

\subsection{Related Topics}\label{related-topics}

\begin{itemize}
\tightlist
\item
  \textbf{{[}{[}Free-Space-Path-Loss-(FSPL){]}{]}}: Dominant loss
  mechanism
\item
  \textbf{{[}{[}Signal-to-Noise-Ratio-(SNR){]}{]}}: Determines required
  C/N
\item
  \textbf{{[}{[}Noise-Sources-\&-Noise-Figure{]}{]}}: RX sensitivity
  calculation
\item
  \textbf{{[}{[}Energy-Ratios-(Es-N0-and-Eb-N0){]}{]}}: Alternative SNR
  metrics
\item
  \textbf{{[}{[}Weather-Effects-(Rain-Fade,-Fog-Attenuation){]}{]}}:
  Rain margin design
\item
  \textbf{{[}{[}Multipath-Propagation-\&-Fading-(Rayleigh,-Rician){]}{]}}:
  Fade margin for mobile
\item
  \textbf{{[}{[}Bit-Error-Rate-(BER){]}{]}}: Performance metric vs SNR
\end{itemize}

\begin{center}\rule{0.5\linewidth}{0.5pt}\end{center}

\textbf{Key takeaway}: \textbf{Link budget is systematic accounting of
all gains/losses from TX to RX.} Start with EIRP, subtract path losses,
add RX gain, compare to sensitivity. Margin = difference. Design for
10-30 dB total margin to handle fading, rain, interference. Adaptive
techniques maximize throughput while maintaining connectivity.

\begin{center}\rule{0.5\linewidth}{0.5pt}\end{center}

\emph{This wiki is part of the {[}{[}Home\textbar Chimera Project{]}{]}
documentation.}
