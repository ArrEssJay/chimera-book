\section{Energy Ratios: Es/N0 and
Eb/N0}\label{energy-ratios-esn0-and-ebn0}

\subsection{\texorpdfstring{ For Non-Technical
Readers}{ For Non-Technical Readers}}\label{for-non-technical-readers}

\textbf{Es/N0 and Eb/N0 are like measuring ``signal per bit of
information'' vs ``background noise''-\/-\/-higher ratio = cleaner
signal = fewer errors!}

\textbf{The idea - Signal vs Noise ratios}: - \textbf{Signal energy}:
How much ``juice'' each bit/symbol has - \textbf{Noise}: Random
interference (always present) - \textbf{Ratio}: Signal energy divided by
noise = quality measure!

\textbf{Two related measurements}:

\textbf{1. Eb/N0 (Energy per BIT)}: - How much energy in each individual
bit? - \textbf{Higher Eb/N0} = more energy per bit = easier to detect
correctly - Used to compare different systems fairly

\textbf{2. Es/N0 (Energy per SYMBOL)}: - How much energy in each symbol
(which may carry multiple bits)? - QPSK symbol carries 2 bits, so Es/N0
= 2 \$\textbackslash times\$ Eb/N0 - Used for actual system performance
calculations

\textbf{Why it matters - Quality threshold}: - \textbf{Low Eb/N0} (weak
signal): Many bit errors, unusable - \textbf{Medium Eb/N0}: Some errors,
FEC can fix them - \textbf{High Eb/N0} (strong signal): Nearly
error-free

\textbf{Real-world example}: - \textbf{WiFi close to router}: Eb/N0 = 25
dB \$\textbackslash rightarrow\$ use 256-QAM (fast!) - \textbf{WiFi far
from router}: Eb/N0 = 10 dB \$\textbackslash rightarrow\$ use QPSK
(reliable!) - \textbf{WiFi too far}: Eb/N0 = 5 dB
\$\textbackslash rightarrow\$ connection drops

\textbf{The relationship}:

\begin{verbatim}
Es/N0 = Eb/N0 × (bits per symbol)
\end{verbatim}

\begin{itemize}
\tightlist
\item
  BPSK: 1 bit/symbol \$\textbackslash rightarrow\$ Es/N0 = Eb/N0
\item
  QPSK: 2 bits/symbol \$\textbackslash rightarrow\$ Es/N0 = 2
  \$\textbackslash times\$ Eb/N0
\item
  16-QAM: 4 bits/symbol \$\textbackslash rightarrow\$ Es/N0 = 4
  \$\textbackslash times\$ Eb/N0
\end{itemize}

\textbf{Why engineers love Eb/N0}: - \textbf{Fair comparison}: Compares
systems with different modulations - \textbf{Theory matches practice}:
Theoretical limits use Eb/N0 - \textbf{Standards use it}: 3GPP, WiFi
specs quote Eb/N0 requirements

\textbf{When you see it}: - \textbf{Satellite link budget}: ``Requires
10 dB Eb/N0 for BER \textless{}
10\textbackslash textsuperscript\{-\}\textbackslash textsuperscript\{6\}''
- \textbf{Modem spec sheet}: ``Sensitivity: -95 dBm at 8 dB Eb/N0'' -
\textbf{Paper comparing modulations}: Always uses Eb/N0 for fair
comparison!

\textbf{Fun fact}: The theoretical minimum Eb/N0 for error-free
communication is -1.59 dB (Shannon limit)-\/-\/-but real systems need
5-15 dB due to practical limitations!

\begin{center}\rule{0.5\linewidth}{0.5pt}\end{center}

These ratios are fundamental measures of signal quality in digital
communications.

\subsection{Es/N0: Symbol Energy Ratio}\label{esn0-symbol-energy-ratio}

\textbf{Es/N0} measures the energy per symbol relative to the noise
power spectral density.

\begin{itemize}
\tightlist
\item
  \textbf{Es}: Energy per symbol
\item
  \textbf{N0}: Noise power per Hz (noise spectral density)
\item
  Used when analyzing symbol-level performance
\end{itemize}

\subsection{Eb/N0: Bit Energy Ratio}\label{ebn0-bit-energy-ratio}

\textbf{Eb/N0} measures the energy per bit relative to the noise power
spectral density.

\begin{itemize}
\tightlist
\item
  \textbf{Eb}: Energy per bit\\
\item
  \textbf{N0}: Noise power per Hz
\item
  More fundamental measure for comparing different modulation schemes
\end{itemize}

\subsection{Relationship Between Es/N0 and
Eb/N0}\label{relationship-between-esn0-and-ebn0}

The relationship depends on how many bits per symbol:

\begin{verbatim}
For QPSK (2 bits/symbol):
Eb/N0 = Es/N0 - 3.01 dB

General formula:
Eb/N0 (dB) = Es/N0 (dB) - 10·log(bits per symbol)
\end{verbatim}

\subsection{Example in Chimera}\label{example-in-chimera}

\begin{itemize}
\tightlist
\item
  If Channel Es/N0 = -15 dB
\item
  For QPSK (2 bits/symbol):
\item
  Then Eb/N0 = -15 dB - 3.01 dB = \textbf{-18.01 dB}
\end{itemize}

\subsection{Why These Ratios Matter}\label{why-these-ratios-matter}

\begin{enumerate}
\def\labelenumi{\arabic{enumi}.}
\tightlist
\item
  \textbf{Performance Comparison}: Allows fair comparison between
  different modulation schemes
\item
  \textbf{Link Budget Analysis}: Essential for designing communication
  systems
\item
  \textbf{BER Prediction}: Theoretical BER curves are plotted against
  Eb/N0
\item
  \textbf{Standard Metric}: Industry-standard way to specify
  communication system performance
\end{enumerate}

\subsection{Comparison Table}\label{comparison-table}

{\def\LTcaptype{} % do not increment counter
\begin{longtable}[]{@{}lll@{}}
\toprule\noalign{}
Modulation & Bits/Symbol & Es/N0 to Eb/N0 Conversion \\
\midrule\noalign{}
\endhead
\bottomrule\noalign{}
\endlastfoot
BPSK & 1 & Eb/N0 = Es/N0 (0 dB) \\
\textbf{QPSK} & \textbf{2} & \textbf{Eb/N0 = Es/N0 - 3.01 dB} \\
8PSK & 3 & Eb/N0 = Es/N0 - 4.77 dB \\
16QAM & 4 & Eb/N0 = Es/N0 - 6.02 dB \\
\end{longtable}
}

\subsection{SNR vs Es/N0 vs Eb/N0}\label{snr-vs-esn0-vs-ebn0}

These terms are related but distinct:

\begin{itemize}
\tightlist
\item
  \textbf{SNR}: Ratio of signal power to noise power (may include
  bandwidth effects)
\item
  \textbf{Es/N0}: Symbol energy to noise spectral density (symbol-level
  metric)
\item
  \textbf{Eb/N0}: Bit energy to noise spectral density (bit-level
  metric, most fundamental)
\end{itemize}

In many contexts (including Chimera\textquotesingle s simple channel
model), \textbf{SNR \$\textbackslash approx\$ Es/N0}.

\subsection{Theoretical BER for QPSK}\label{theoretical-ber-for-qpsk}

\begin{verbatim}
BER_QPSK  (1/2) · erfc((Eb/N0))

where erfc is the complementary error function
\end{verbatim}

\subsection{See Also}\label{see-also}

\begin{itemize}
\tightlist
\item
  {[}{[}Signal-to-Noise-Ratio-(SNR){]}{]} - Related power ratio
\item
  {[}{[}Bit-Error-Rate-(BER){]}{]} - Performance metric
\item
  {[}{[}QPSK-Modulation{]}{]} - 2 bits per symbol
\item
  {[}{[}Complete-Link-Budget-Analysis{]}{]} - Using energy ratios in
  system design
\end{itemize}
