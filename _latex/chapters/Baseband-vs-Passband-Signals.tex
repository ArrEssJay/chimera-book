\chapter{Baseband vs Passband Signals}
\label{ch:baseband-passband}

\begin{nontechnical}
\textbf{Baseband vs passband is like the difference between sheet music (the notes you play) and the actual sound coming out of a trumpet (shifted to a specific pitch range).}

\textbf{Musical analogy:}
\begin{itemize}
\item \textbf{Baseband}: The melody as written on paper---frequency near 0 Hz, the ``pure'' information
\item \textbf{Passband}: Same melody played on a flute (high pitch) vs tuba (low pitch)---shifted to different frequency ranges
\end{itemize}

\textbf{Your phone call journey:}
\begin{enumerate}
\item \textbf{Your voice}: Baseband (20 Hz -- 3.4 kHz)
\item \textbf{Cell phone transmitter}: Shifts to passband (1.9 GHz)
\item \textbf{Over the air}: Passband signal travels to tower
\item \textbf{Tower receiver}: Shifts back to baseband
\item \textbf{Recipient's phone}: Shifts to passband again (transmit)
\item \textbf{Recipient's speaker}: Back to baseband (audio)
\end{enumerate}

\textbf{Why antennas need passband:} Efficient antenna size $\approx \lambda/2$ (half wavelength). Audio at 20~Hz requires a 7,500~km antenna (impossible!). WiFi at 2.4~GHz needs only a 6~cm antenna.

\textbf{Fun fact:} Software Defined Radio (SDR) works by keeping signals in baseband as long as possible---only converting to passband at the last moment.
\end{nontechnical}

\section{Overview}

Understanding the distinction between baseband and passband signals is fundamental to radio communication system design. \textbf{Baseband signals} represent information at low frequencies (near DC), while \textbf{passband signals} are frequency-translated versions centered around a carrier frequency $f_c$.

\begin{keyconcept}
The transformation between baseband and passband domains enables \textbf{efficient wireless transmission} while preserving \textbf{computational simplicity} in digital signal processing. This duality is the cornerstone of modern software-defined radio architectures.
\end{keyconcept}

\textbf{Key operations:}
\begin{itemize}
\item \textbf{Upconversion}: Frequency translation from baseband to passband (modulation)
\item \textbf{Downconversion}: Frequency translation from passband to baseband (demodulation)
\end{itemize}

Modern communication systems leverage this separation: baseband processing handles complex modulation and coding algorithms in software, while passband conversion interfaces with the physical RF channel.

\section{Mathematical Definitions}

\subsection{Baseband Signal}

A \textbf{baseband signal} is defined as having frequency content centered around DC (0~Hz):
\begin{equation}
s_{\text{BB}}(t) = s_I(t) + js_Q(t)
\end{equation}
where:
\begin{itemize}
\item $s_I(t)$ = in-phase (real) component
\item $s_Q(t)$ = quadrature (imaginary) component
\item Complex representation enables efficient DSP
\end{itemize}

The \textbf{spectrum} of a baseband signal extends from approximately 0~Hz to $B$~Hz (bandwidth):
\begin{equation}
S_{\text{BB}}(f) \neq 0 \quad \text{for } |f| \leq B/2
\end{equation}

\textbf{Examples:}
\begin{itemize}
\item \textbf{Digital:} NRZ pulses ($\pm 1$), Manchester encoding, pulse-shaped symbols (raised cosine, RRC)
\item \textbf{Analog:} Voice (300--3400~Hz), audio (20~Hz -- 20~kHz), video (DC -- 6~MHz)
\end{itemize}

\subsection{Complex Baseband Representation}

For bandpass systems, the signal is represented as a \textbf{complex envelope} or \textbf{equivalent lowpass signal}. This representation simplifies analysis and implementation.

The relationship between the real passband signal and complex baseband is:
\begin{equation}
s_{\text{RF}}(t) = \text{Re}\{s_{\text{BB}}(t) \cdot e^{j2\pi f_c t}\}
\end{equation}
where:
\begin{itemize}
\item $s_{\text{RF}}(t)$ = real passband signal
\item $s_{\text{BB}}(t) = s_I(t) + js_Q(t)$ = complex baseband signal
\item $f_c$ = carrier frequency (Hz)
\end{itemize}

Expanding using Euler's formula:
\begin{equation}
s_{\text{RF}}(t) = s_I(t)\cos(2\pi f_c t) - s_Q(t)\sin(2\pi f_c t)
\end{equation}

\textbf{Advantages of complex baseband:}
\begin{itemize}
\item Simplifies DSP (single complex signal vs two real signals)
\item Natural representation for IQ modulation
\item Halves sampling rate requirement (no negative frequencies)
\item Enables efficient software-defined radio implementations
\end{itemize}

\subsection{Baseband Bandwidth}

The \textbf{occupied bandwidth} of a baseband signal depends on the symbol rate $R_s$ and pulse shaping filter characteristics.

For \textbf{ideal rectangular pulses} (no pulse shaping):
\begin{equation}
B_{\text{BB}} = R_s \quad \text{(Hz)}
\end{equation}

For \textbf{raised cosine pulse shaping} with roll-off factor $\alpha$:
\begin{equation}
B_{\text{BB}} = R_s(1 + \alpha) \quad \text{(Hz)}
\end{equation}
where:
\begin{itemize}
\item $R_s$ = symbol rate (symbols/second)
\item $\alpha \in [0, 1]$ = roll-off factor ($\alpha = 0$ is ideal, $\alpha = 1$ is 100\% excess bandwidth)
\item Practical values: $\alpha = 0.22$ (LTE), $\alpha = 0.35$ (satellite), $\alpha = 0.5$ (legacy systems)
\end{itemize}

\textbf{Example:} QPSK modulation at 1~Msps with $\alpha = 0.35$:
\begin{equation}
B_{\text{BB}} = 1.0 \times (1 + 0.35) = 1.35~\text{MHz}
\end{equation}

\subsection{Passband Signal}

A \textbf{passband signal} has frequency content centered around a carrier frequency $f_c$:
\begin{equation}
s_{\text{RF}}(t) = A(t)\cos[2\pi f_c t + \phi(t)]
\end{equation}
where:
\begin{itemize}
\item $A(t)$ = time-varying amplitude (envelope)
\item $f_c$ = carrier frequency (Hz)
\item $\phi(t)$ = time-varying phase
\end{itemize}

The \textbf{spectrum} extends from $f_c - B/2$ to $f_c + B/2$:
\begin{equation}
S_{\text{RF}}(f) \neq 0 \quad \text{for } f_c - B/2 \leq f \leq f_c + B/2
\end{equation}

\subsection{Why Passband Transmission?}

\textbf{1. Antenna Efficiency}

Efficient antenna length is approximately $\lambda/4$ (quarter-wavelength):
\begin{equation}
L_{\text{ant}} \approx \frac{\lambda}{4} = \frac{c}{4f}
\end{equation}
where:
\begin{itemize}
\item $c = 3 \times 10^8$~m/s (speed of light)
\item $f$ = operating frequency (Hz)
\end{itemize}

\begin{calloutbox}{Antenna Size Comparison}
\begin{itemize}
\item \textbf{100~Hz baseband:} $\lambda = 3000$~km $\rightarrow$ antenna = 750~km (impossible!)
\item \textbf{2.4~GHz WiFi:} $\lambda = 12.5$~cm $\rightarrow$ antenna = 3.1~cm (practical)
\end{itemize}
\end{calloutbox}

\textbf{2. Spectrum Allocation}

Regulatory bodies (FCC, ITU) assign non-overlapping frequency bands to different services, enabling coexistence without interference.

\textbf{3. Propagation Characteristics}

Different frequency bands exhibit distinct propagation properties:
\begin{itemize}
\item \textbf{HF (3--30~MHz):} Ionospheric reflection, long-range
\item \textbf{VHF (30--300~MHz):} Line-of-sight, FM radio
\item \textbf{UHF (300~MHz -- 3~GHz):} Building penetration, cellular
\item \textbf{SHF (3--30~GHz):} Satellite links, high capacity
\end{itemize}

\textbf{4. Frequency Division Multiplexing (FDM)}

Multiple baseband signals can be upconverted to different carriers, enabling simultaneous transmission over a single physical channel.

\section{Frequency Translation: Baseband to Passband}

\subsection{Spectrum Visualization}

The transformation from baseband to passband is a frequency shift operation. Consider a baseband signal with bandwidth $B$ centered at DC:

\begin{center}
\begin{tikzpicture}[scale=1.2]
% Baseband spectrum
\draw[->] (-0.5,0) -- (4.5,0) node[right,font=\small] {$f$ (Hz)};
\draw[->] (0,-0.3) -- (0,2.5) node[above,font=\small] {$|S_{\text{BB}}(f)|$};

% Baseband signal (triangular spectrum)
\draw[thick,blue] (0,0) -- (0.5,2) -- (2,2) -- (2.5,0);
\draw[dashed] (0.5,0) -- (0.5,2);
\draw[dashed] (2,0) -- (2,2);

% Labels
\node[below,font=\scriptsize] at (0,0) {0};
\node[below,font=\scriptsize] at (0.5,0) {$-B/2$};
\node[below,font=\scriptsize] at (2,0) {$B/2$};
\node[above,font=\small,blue] at (1.25,2.2) {Baseband};

% Arrow indicating upconversion
\node[right,font=\large] at (5,1) {$\xrightarrow{\text{Upconvert to } f_c}$};

% Passband spectrum
\begin{scope}[xshift=9cm]
\draw[->] (-0.5,0) -- (8.5,0) node[right,font=\small] {$f$ (Hz)};
\draw[->] (0,-0.3) -- (0,2.5) node[above,font=\small] {$|S_{\text{RF}}(f)|$};

% Passband signal (shifted spectrum)
\draw[thick,red] (3.5,0) -- (4,2) -- (5.5,2) -- (6,0);
\draw[dashed] (4,0) -- (4,2);
\draw[dashed] (5.5,0) -- (5.5,2);

% Labels
\node[below,font=\scriptsize] at (0,0) {0};
\node[below,font=\scriptsize] at (4,0) {$f_c - B/2$};
\node[below,font=\scriptsize] at (4.75,0) {$f_c$};
\node[below,font=\scriptsize] at (5.5,0) {$f_c + B/2$};
\node[above,font=\small,red] at (4.75,2.2) {Passband};

\draw[<->,thick] (4,-0.8) -- (5.5,-0.8) node[midway,below,font=\scriptsize] {Bandwidth $B$};
\end{scope}
\end{tikzpicture}
\end{center}

\textbf{Key insight:} Upconversion preserves bandwidth while shifting the signal to a higher frequency suitable for wireless transmission.

\section{Upconversion (Modulation)}

\subsection{IQ Modulator Architecture}

The \textbf{quadrature modulator} (IQ modulator) performs the frequency translation from complex baseband to real passband.

\begin{center}
\begin{tikzpicture}[
  block/.style={rectangle, draw, minimum width=2cm, minimum height=1cm, font=\sffamily\small},
  node distance=2.2cm,
  font=\small
]
% Input signals
\node (input_i) {\sffamily $s_I(t)$};
\node[below=1.5cm of input_i] (input_q) {\sffamily $s_Q(t)$};

% Mixers
\node[block, right of=input_i, node distance=2.5cm] (mix_i) {$\times$};
\node[block, right of=input_q, node distance=2.5cm] (mix_q) {$\times$};

% Local oscillator
\node[below=2.8cm of mix_i, font=\scriptsize] (lo) {Local\\Oscillator\\$f_c$};
\node[above=0.3cm of mix_i, font=\scriptsize] (cos) {$\cos(2\pi f_c t)$};
\node[above=0.3cm of mix_q, font=\scriptsize] (sin) {$-\sin(2\pi f_c t)$};

% Summer
\node[block, right of=mix_i, node distance=3.5cm, yshift=-0.75cm] (sum) {$\Sigma$};

% Bandpass filter
\node[block, right of=sum, node distance=3cm] (bpf) {Bandpass\\Filter};

% Output
\node[right of=bpf, node distance=3cm] (output) {\sffamily $s_{\text{RF}}(t)$};

% Connections
\draw[->,thick] (input_i) -- (mix_i);
\draw[->,thick] (input_q) -- (mix_q);
\draw[->,thick] (lo) -- ($(lo)+(0,1.2)$) -| (mix_i);
\draw[->,thick] (lo) -- ($(lo)+(0,0.5)$) -| (mix_q);
\draw[->,thick] (mix_i) -| (sum);
\draw[->,thick] (mix_q) -| (sum);
\draw[->,thick] (sum) -- (bpf);
\draw[->,thick] (bpf) -- (output);
\end{tikzpicture}
\end{center}

\subsection{Mathematical Description}

The I-channel is multiplied by $\cos(2\pi f_c t)$:
\begin{equation}
s_{\text{I,up}}(t) = s_I(t) \cos(2\pi f_c t)
\end{equation}

The Q-channel is multiplied by $-\sin(2\pi f_c t)$:
\begin{equation}
s_{\text{Q,up}}(t) = -s_Q(t) \sin(2\pi f_c t)
\end{equation}

The two components are summed to produce the real RF output:
\begin{equation}
s_{\text{RF}}(t) = s_I(t)\cos(2\pi f_c t) - s_Q(t)\sin(2\pi f_c t)
\end{equation}
where:
\begin{itemize}
\item $s_I(t), s_Q(t)$ = baseband in-phase and quadrature components
\item $f_c$ = carrier frequency (Hz)
\item $s_{\text{RF}}(t)$ = real-valued passband signal
\end{itemize}

This can be expressed compactly using complex notation:
\begin{equation}
s_{\text{RF}}(t) = \text{Re}\{[s_I(t) + js_Q(t)] \cdot e^{j2\pi f_c t}\}
\end{equation}

\subsection{Frequency Domain Analysis}

In the frequency domain, upconversion shifts the baseband spectrum to the carrier frequency:
\begin{equation}
S_{\text{RF}}(f) = \frac{1}{2}[S_{\text{BB}}(f - f_c) + S_{\text{BB}}^*(-f - f_c)]
\end{equation}
where:
\begin{itemize}
\item $S_{\text{BB}}(f)$ = complex baseband spectrum
\item $S_{\text{BB}}^*(f)$ = complex conjugate
\item The conjugate term ensures $s_{\text{RF}}(t)$ is real-valued
\end{itemize}

\begin{warningbox}
\textbf{Image Rejection:} Real mixers (single-ended) produce both upper sideband (USB) at $f_c + f_{\text{BB}}$ and lower sideband (LSB) at $f_c - f_{\text{BB}}$. The IQ modulator provides natural image rejection by controlling the relative phase between I and Q channels.
\end{warningbox}

\section{Downconversion (Demodulation)}

\subsection{IQ Demodulator Architecture}

The \textbf{quadrature demodulator} (IQ demodulator) performs the reverse operation: frequency translation from real passband back to complex baseband.

\begin{center}
\begin{tikzpicture}[
  block/.style={rectangle, draw, minimum width=2cm, minimum height=1cm, font=\sffamily\small},
  node distance=2.2cm,
  font=\small
]
% Input
\node (input) {\sffamily $s_{\text{RF}}(t)$};

% Splitter
\node[right of=input, node distance=2cm] (split) {};

% Mixers
\node[block, above right=0.5cm and 2cm of split] (mix_i) {$\times$};
\node[block, below right=0.5cm and 2cm of split] (mix_q) {$\times$};

% Local oscillator
\node[below=3cm of split, font=\scriptsize] (lo) {Local\\Oscillator\\$f_c$};
\node[above=0.3cm of mix_i, font=\scriptsize] (cos) {$\cos(2\pi f_c t)$};
\node[above=0.3cm of mix_q, font=\scriptsize] (sin) {$-\sin(2\pi f_c t)$};

% Lowpass filters
\node[block, right of=mix_i, node distance=3cm] (lpf_i) {Lowpass\\Filter};
\node[block, right of=mix_q, node distance=3cm] (lpf_q) {Lowpass\\Filter};

% Outputs
\node[right of=lpf_i, node distance=2.8cm] (output_i) {\sffamily $s_I(t)$};
\node[right of=lpf_q, node distance=2.8cm] (output_q) {\sffamily $s_Q(t)$};

% Connections
\draw[->,thick] (input) -- (split);
\draw[->,thick] (split) |- (mix_i);
\draw[->,thick] (split) |- (mix_q);
\draw[->,thick] (lo) -- ($(lo)+(0,1.8)$) -| (mix_i);
\draw[->,thick] (lo) -- ($(lo)+(0,1)$) -| (mix_q);
\draw[->,thick] (mix_i) -- (lpf_i);
\draw[->,thick] (mix_q) -- (lpf_q);
\draw[->,thick] (lpf_i) -- (output_i);
\draw[->,thick] (lpf_q) -- (output_q);
\end{tikzpicture}
\end{center}

\subsection{Mathematical Description}

The received RF signal is split and mixed with both cosine and sine local oscillator signals.

\textbf{I-channel downconversion:}
\begin{equation}
s_I(t) = \text{LPF}\{s_{\text{RF}}(t) \cdot \cos(2\pi f_c t)\}
\end{equation}

\textbf{Q-channel downconversion:}
\begin{equation}
s_Q(t) = \text{LPF}\{s_{\text{RF}}(t) \cdot [-\sin(2\pi f_c t)]\}
\end{equation}
where:
\begin{itemize}
\item $s_{\text{RF}}(t)$ = received passband signal
\item LPF = lowpass filter (removes $2f_c$ mixing products)
\item $s_I(t), s_Q(t)$ = recovered baseband components
\end{itemize}

\subsection{Derivation of Downconversion}

Consider the received RF signal:
\begin{equation}
s_{\text{RF}}(t) = s_I^{\text{TX}}(t)\cos(2\pi f_c t) - s_Q^{\text{TX}}(t)\sin(2\pi f_c t)
\end{equation}

\textbf{I-channel after mixing:}
\begin{equation}
s_I^{\text{mix}}(t) = s_{\text{RF}}(t) \cdot \cos(2\pi f_c t)
\end{equation}

Expanding:
\begin{equation}
s_I^{\text{mix}}(t) = s_I^{\text{TX}}\cos^2(2\pi f_c t) - s_Q^{\text{TX}}\sin(2\pi f_c t)\cos(2\pi f_c t)
\end{equation}

Using trigonometric identities:
\begin{itemize}
\item $\cos^2\theta = \frac{1 + \cos(2\theta)}{2}$
\item $\sin\theta\cos\theta = \frac{\sin(2\theta)}{2}$
\end{itemize}

This yields:
\begin{equation}
s_I^{\text{mix}}(t) = s_I^{\text{TX}}\frac{1 + \cos(4\pi f_c t)}{2} - s_Q^{\text{TX}}\frac{\sin(4\pi f_c t)}{2}
\end{equation}

After lowpass filtering (removes $2f_c$ components):
\begin{equation}
s_I(t) = \frac{1}{2}s_I^{\text{TX}}(t)
\end{equation}

Similarly for the Q-channel:
\begin{equation}
s_Q(t) = \frac{1}{2}s_Q^{\text{TX}}(t)
\end{equation}

The recovered complex baseband signal is:
\begin{equation}
s_{\text{BB}}(t) = s_I(t) + js_Q(t) = \frac{1}{2}[s_I^{\text{TX}}(t) + js_Q^{\text{TX}}(t)]
\end{equation}

The factor of $1/2$ is easily compensated by digital gain adjustment.

\section{Worked Example: WiFi 802.11n Upconversion}

\subsection{System Specifications}

\textbf{Given:}
\begin{itemize}
\item Modulation: 64-QAM OFDM
\item Channel bandwidth: 20~MHz
\item Symbol rate: 20~Msps
\item Carrier frequency: 2.412~GHz (Channel 1)
\item Baseband oversampling: 2$\times$
\end{itemize}

\subsection{Baseband Signal Characteristics}

The complex baseband OFDM signal occupies:
\begin{equation}
f_{\text{BB}} \in [-10~\text{MHz}, +10~\text{MHz}]
\end{equation}

Sampling rate (with 2$\times$ oversampling):
\begin{equation}
f_s = 2 \times 20~\text{Msps} = 40~\text{Msps}
\end{equation}

\subsection{Upconversion Process}

The IQ modulator shifts the baseband spectrum to the carrier:
\begin{equation}
s_{\text{RF}}(t) = s_I(t)\cos(2\pi \cdot 2.412~\text{GHz} \cdot t) - s_Q(t)\sin(2\pi \cdot 2.412~\text{GHz} \cdot t)
\end{equation}

Resulting RF spectrum:
\begin{equation}
f_{\text{RF}} \in [2.402~\text{GHz}, 2.422~\text{GHz}]
\end{equation}

\subsection{Transmit Chain}

\textbf{Signal processing flow:}
\begin{enumerate}
\item \textbf{Generate OFDM symbols:} 64-QAM constellation mapping, IFFT
\item \textbf{Digital-to-analog conversion:} DAC at 40~Msps
\item \textbf{IQ upconversion:} Mixer with 2.412~GHz LO
\item \textbf{Power amplification:} PA drives antenna
\end{enumerate}

\subsection{Key Design Parameters}

Antenna quarter-wavelength at 2.412~GHz:
\begin{equation}
L_{\text{ant}} = \frac{c}{4f_c} = \frac{3 \times 10^8}{4 \times 2.412 \times 10^9} = 3.1~\text{cm}
\end{equation}

Required RF bandwidth matches baseband:
\begin{equation}
B_{\text{RF}} = B_{\text{BB}} = 20~\text{MHz}
\end{equation}

\textbf{Conclusion:} The upconversion preserves signal bandwidth while enabling practical antenna dimensions and regulatory compliance with ISM band allocations.

\section{Complete Transceiver System}

A modern wireless transceiver integrates both baseband processing and passband RF conversion:

\begin{center}
\begin{tikzpicture}[
  block/.style={rectangle, draw, minimum width=2.2cm, minimum height=0.9cm, font=\sffamily\scriptsize},
  node distance=1.8cm,
  font=\scriptsize
]
% Transmit path (top)
\node[align=center] (tx_data) {\sffamily Baseband\\Data};
\node[block, right of=tx_data, node distance=2.5cm] (tx_dsp) {DSP\\Modulator};
\node[block, right of=tx_dsp, node distance=2.5cm] (tx_dac) {DAC};
\node[block, right of=tx_dac, node distance=2.5cm] (tx_iq) {IQ\\Upconvert};
\node[block, right of=tx_iq, node distance=2.5cm] (tx_pa) {PA};
\node[right of=tx_pa, node distance=2.2cm] (tx_ant) {\sffamily Antenna};

% Receive path (bottom)
\node[below=3cm of tx_data, align=center] (rx_data) {\sffamily Baseband\\Data};
\node[block, right of=rx_data, node distance=2.5cm] (rx_dsp) {DSP\\Demod};
\node[block, right of=rx_dsp, node distance=2.5cm] (rx_adc) {ADC};
\node[block, right of=rx_adc, node distance=2.5cm] (rx_iq) {IQ\\Downconvert};
\node[block, right of=rx_iq, node distance=2.5cm] (rx_lna) {LNA};
\node[right of=rx_lna, node distance=2.2cm] (rx_ant) {\sffamily Antenna};

% Local oscillator (shared)
\node[block, below=1.5cm of tx_iq] (lo) {LO\\$f_c$};

% Transmit path connections
\draw[->,thick] (tx_data) -- (tx_dsp);
\draw[->,thick] (tx_dsp) -- node[above,font=\tiny] {I/Q} (tx_dac);
\draw[->,thick] (tx_dac) -- node[above,font=\tiny] {analog} (tx_iq);
\draw[->,thick] (tx_iq) -- node[above,font=\tiny] {RF} (tx_pa);
\draw[->,thick] (tx_pa) -- (tx_ant);

% Receive path connections
\draw[->,thick] (rx_ant) -- (rx_lna);
\draw[->,thick] (rx_lna) -- node[above,font=\tiny] {RF} (rx_iq);
\draw[->,thick] (rx_iq) -- node[above,font=\tiny] {analog} (rx_adc);
\draw[->,thick] (rx_adc) -- node[above,font=\tiny] {I/Q} (rx_dsp);
\draw[->,thick] (rx_dsp) -- (rx_data);

% LO connections
\draw[->,thick] (lo) -- (tx_iq);
\draw[->,thick] (lo) -- (rx_iq);

% Baseband/Passband labels
\draw[dashed,gray,thick] ($(tx_dac.east)!0.5!(tx_iq.west)$) ++(-0.3,1.5) -- ++(-0.3,-5) node[midway,left,align=center,font=\tiny] {Baseband\\Domain};
\draw[dashed,gray,thick] ($(tx_dac.east)!0.5!(tx_iq.west)$) ++(0.3,1.5) -- ++(0.3,-5) node[midway,right,align=center,font=\tiny] {Passband\\Domain};
\end{tikzpicture}
\end{center}

\textbf{Key observations:}
\begin{itemize}
\item \textbf{Baseband domain:} All processing in I/Q (complex) format
\item \textbf{Passband domain:} Real RF signals at carrier frequency $f_c$
\item \textbf{Frequency translation:} IQ modulators/demodulators bridge the domains
\item \textbf{Shared LO:} Ensures TX and RX use same frequency reference
\end{itemize}

\section{Receiver Architectures}

\subsection{Superheterodyne Receiver}

The \textbf{superheterodyne} (superhet) architecture uses an intermediate frequency (IF) stage between RF and baseband.

\textbf{Architecture:} RF $\rightarrow$ IF $\rightarrow$ Baseband

\textbf{Advantages:}
\begin{itemize}
\item \textbf{Image rejection:} High-Q IF filters provide excellent selectivity
\item \textbf{Fixed IF:} Optimized filters independent of RF tuning
\item \textbf{Gain distribution:} Prevents oscillation through staged amplification
\end{itemize}

\textbf{Example---FM radio receiver:}
\begin{itemize}
\item RF input: 88--108~MHz (tunable)
\item Local oscillator: $f_{\text{LO}} = f_{\text{RF}} + 10.7$~MHz
\item IF: 10.7~MHz (fixed)
\item IF filter: 150~kHz ceramic/crystal filter
\end{itemize}

\subsection{Zero-IF (Direct Conversion) Receiver}

\textbf{Zero-IF} receivers directly downconvert RF to baseband, eliminating the IF stage.

\textbf{Architecture:} RF $\rightarrow$ Baseband (one step)

\textbf{Advantages:}
\begin{itemize}
\item Fewer components (no IF filters, single LO)
\item Compact, low power (ideal for mobile)
\item Software-defined bandwidth and filtering
\end{itemize}

\textbf{Challenges:}
\begin{itemize}
\item \textbf{DC offset:} LO leakage creates DC component
\item \textbf{Flicker noise:} 1/f noise near DC degrades SNR
\item \textbf{I/Q imbalance:} Gain/phase mismatch between channels
\end{itemize}

\begin{calloutbox}{Modern Trend}
Modern SDR transceivers (e.g., AD9361, LMS7002M) use zero-IF with digital compensation for DC offset and I/Q imbalance. This enables single-chip integration and software reconfigurability.
\end{calloutbox}

\section{Sampling Considerations}

\subsection{Bandpass Sampling Theorem}

For a real passband signal $s_{\text{RF}}(t)$ centered at $f_c$ with bandwidth $B$:
\begin{equation}
f_s \geq 2B
\end{equation}
where:
\begin{itemize}
\item $f_s$ = sampling rate (Hz)
\item $B$ = signal bandwidth (Hz)
\item Note: $f_s < 2f_c$ is permitted (undersampling)
\end{itemize}

\textbf{Advantage:} Can sample at much lower rate than $2f_c$ (Nyquist for baseband).

\textbf{Example---WiFi at 2.4~GHz:}
\begin{itemize}
\item Carrier: $f_c = 2.4$~GHz
\item Bandwidth: $B = 20$~MHz
\item Minimum sampling rate: $f_s = 40$~MHz (60$\times$ less than $2f_c = 4.8$~GHz)
\end{itemize}

\subsection{Complex Baseband Sampling}

For complex baseband $s_{\text{BB}}(t) = s_I(t) + js_Q(t)$:
\begin{equation}
f_s \geq B
\end{equation}

This is \textbf{half the rate} required for real passband sampling because negative frequencies are represented in the complex signal and need not be avoided.

\textbf{Practical consideration:} Typical oversampling of 2$\times$ or more accommodates pulse shaping filter roll-off:
\begin{equation}
f_s = (2 \text{ to } 4) \times B
\end{equation}

\section{Practical Impairments}

\subsection{Carrier Frequency Offset (CFO)}

Transmitter and receiver oscillators are never perfectly matched:
\begin{equation}
\Delta f = f_{\text{TX}} - f_{\text{RX}}
\end{equation}

This causes a rotating phase error in the received baseband:
\begin{equation}
s_{\text{RX}}(t) = s(t) \cdot e^{j2\pi \Delta f t}
\end{equation}
where:
\begin{itemize}
\item $\Delta f$ = frequency offset (Hz)
\item Constellation rotates at $2\pi \Delta f$ rad/s
\end{itemize}

\textbf{Typical oscillator accuracy:} $\pm 10$~ppm
\begin{itemize}
\item At 2.4~GHz: $\Delta f \approx \pm 24$~kHz
\item At 28~GHz (5G mmWave): $\Delta f \approx \pm 280$~kHz
\end{itemize}

\subsection{Phase Noise}

Local oscillator jitter introduces random phase variations:
\begin{equation}
s_{\text{RF}}(t) = A(t)\cos[2\pi f_c t + \phi_n(t)]
\end{equation}
where $\phi_n(t)$ is a random phase noise process.

\textbf{Effects:}
\begin{itemize}
\item Constellation point spreading
\item Inter-carrier interference (ICI) in OFDM systems
\item BER floor at high SNR
\end{itemize}

\textbf{Specification:} Phase noise power spectral density $\mathcal{L}(f_m)$ at offset $f_m$ from carrier (units: dBc/Hz).

\subsection{I/Q Imbalance}

Non-ideal hardware causes gain and phase mismatches between I and Q channels:
\begin{equation}
s_{\text{imb}}(t) = G_I s_I(t) + G_Q e^{j(\pi/2 + \epsilon)} s_Q(t)
\end{equation}
where:
\begin{itemize}
\item $G_I, G_Q$ = channel gains (typically $\pm 0.5$~dB mismatch)
\item $\epsilon$ = phase error (typically $\pm 2°$ from ideal 90°)
\end{itemize}

\textbf{Consequence:} Image sideband leakage and constellation distortion.

\textbf{Mitigation:} Digital calibration using known pilot symbols or factory calibration tables.

\subsection{DC Offset}

In zero-IF receivers, local oscillator leakage mixes with itself, creating a DC component:
\begin{equation}
s_{\text{DC}} = k \cdot P_{\text{LO}}
\end{equation}

\textbf{Mitigation strategies:}
\begin{itemize}
\item AC coupling (high-pass filter)
\item Blank center subcarrier (OFDM)
\item Adaptive DC offset cancellation
\end{itemize}

\section{Applications}

\subsection{Cellular Networks}

Modern cellular systems (LTE, 5G NR) process signals in complex baseband for:
\begin{itemize}
\item Channel estimation and equalization
\item MIMO spatial processing
\item Turbo/LDPC decoding
\end{itemize}

RF upconversion/downconversion occurs at the antenna interface, with carrier frequencies ranging from 600~MHz (LTE Band 71) to 39~GHz (5G FR2).

\subsection{Software-Defined Radio (SDR)}

SDR platforms (USRP, BladeRF, HackRF) maximize baseband processing time:
\begin{itemize}
\item Flexible modulation schemes (BPSK, QPSK, QAM, OFDM) in software
\item Reconfigurable bandwidth and center frequency
\item Digital filtering and pulse shaping
\end{itemize}

Only the final RF stage is fixed hardware; all signal processing occurs in complex baseband on CPU/FPGA.

\subsection{Satellite Communications}

Satellites receive uplink at one passband (e.g., 14~GHz), downconvert to IF, process/amplify, then upconvert to downlink passband (e.g., 12~GHz). The IF stage enables:
\begin{itemize}
\item Channel filtering and routing
\item Transponder switching
\item Power efficient amplification
\end{itemize}

\subsection{WiFi and WLAN}

IEEE 802.11 devices use zero-IF transceivers for compact integration:
\begin{itemize}
\item OFDM generation/detection in complex baseband
\item 20/40/80/160~MHz channel bandwidths
\item Carrier frequencies: 2.4~GHz, 5~GHz, 6~GHz (WiFi 6E)
\end{itemize}

\section{Summary}

\begin{center}
\begin{tabular}{@{}lll@{}}
\toprule
\textbf{Aspect} & \textbf{Baseband} & \textbf{Passband} \\
\midrule
Frequency range & $\sim 0$ to $B$ Hz & $f_c - B/2$ to $f_c + B/2$ \\
Signal type & Complex or real & Real only \\
Sampling rate & $\geq B$ (complex) & $\geq 2B$ (bandpass) \\
Processing & Digital (DSP) & Analog RF or SDR \\
Transmission & Wired (Ethernet) & Wireless (antenna) \\
Representation & $s(t) = s_I + js_Q$ & $s_{\text{RF}} = s_I\cos\omega_c t - s_Q\sin\omega_c t$ \\
\bottomrule
\end{tabular}
\end{center}

\textbf{Key points:}
\begin{itemize}
\item Baseband signals represent information at low frequency (near DC)
\item Passband signals are frequency-translated for wireless transmission
\item IQ modulation enables efficient complex-to-real conversion
\item Modern systems maximize baseband processing for flexibility
\item Hardware impairments (CFO, phase noise, I/Q imbalance) require compensation
\end{itemize}

\section{Further Reading}

\begin{itemize}
\item \textbf{IQ Representation:} Complex baseband signal structure and properties
\item \textbf{QPSK Modulation:} Practical example of baseband-to-passband transformation
\item \textbf{OFDM:} Multi-carrier modulation using complex baseband per subcarrier
\item \textbf{Synchronization:} Carrier frequency and phase recovery techniques
\item \textbf{Free-Space Path Loss:} Link budget analysis for passband transmission
\item \textbf{Software-Defined Radio:} Modern baseband processing architectures
\end{itemize}
