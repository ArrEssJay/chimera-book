\section{THz Propagation in Biological
Tissue}\label{thz-propagation-in-biological-tissue}

{[}{[}Home{]}{]} \textbar{} {[}{[}Terahertz-(THz)-Technology{]}{]}
\textbar{} {[}{[}THz-Bioeffects-Thermal-and-Non-Thermal{]}{]} \textbar{}
{[}{[}mmWave-\&-THz-Communications{]}{]}

\begin{center}\rule{0.5\linewidth}{0.5pt}\end{center}

\subsection{Overview}\label{overview}

\textbf{Terahertz (THz) radiation} (0.1-10 THz, 30
\$\textbackslash mu\$m-3 mm wavelength) occupies the spectral gap
between microwaves and infrared. THz waves interact strongly with
biological tissue due to resonances with molecular vibrations and
rotations, particularly water. Understanding THz propagation is critical
for: - \textbf{Medical imaging}: Cancer detection, burn assessment -
\textbf{Security}: Concealed weapons/explosives detection through
clothing - \textbf{Neuromodulation}: Speculative applications in neural
stimulation (see {[}{[}AID-Protocol-Case-Study{]}{]})

\begin{center}\rule{0.5\linewidth}{0.5pt}\end{center}

\subsection{Quick Start for Non-Technical
Readers}\label{quick-start-for-non-technical-readers}

\textbf{What is THz radiation?} Think of it as invisible light that sits
between microwaves (used in your microwave oven) and infrared (heat you
feel from a fireplace). It can pass through some materials but not
others.

\textbf{Why study it in tissue?} Scientists and doctors want to use THz
waves to see inside skin without cutting it open-\/-\/-like an X-ray,
but safer and better for soft tissue.

\textbf{The main challenge: Water blocks THz waves.} Your body is mostly
water, and water absorbs THz radiation very strongly. This means: -
\textbf{Good news}: THz waves can create detailed images of skin and
surface tissues - \textbf{Bad news}: They can\textquotesingle t
penetrate deep (only a fraction of a millimeter)

\textbf{Real-world analogy}: Imagine shining a flashlight through fog.
The light gets absorbed quickly, so you can only see a short distance.
THz waves in wet tissue behave the same way.

\textbf{Key takeaways}: 1. \textbf{THz imaging works for skin}: Doctors
can detect skin cancer, assess burn depth, or check dental cavities 2.
\textbf{THz cannot image deep organs}: Unlike X-rays, THz stops at the
surface (it can\textquotesingle t see your heart or brain through skin)
3. \textbf{Safety}: THz is non-ionizing (unlike X-rays), so it
doesn\textquotesingle t damage DNA. At low power, it\textquotesingle s
considered safe 4. \textbf{The physics}: Water molecules spin in
response to THz waves, absorbing energy and turning it into heat

\textbf{Who uses this?} - Dermatologists (skin doctors) for cancer
detection - Security personnel for airport body scanners - Researchers
exploring futuristic applications (like non-invasive brain
stimulation-\/-\/-still very experimental)

\textbf{Want more detail?} The sections below explain the physics, but
you now have the gist!

\begin{center}\rule{0.5\linewidth}{0.5pt}\end{center}

\subsection{1. Electromagnetic Properties of Biological Tissue at THz
Frequencies}\label{electromagnetic-properties-of-biological-tissue-at-thz-frequencies}

\subsubsection{1.1 Complex Permittivity}\label{complex-permittivity}

Biological tissue\textquotesingle s response to EM waves is
characterized by \textbf{complex relative permittivity}:
\[\epsilon_r(\omega) = \epsilon'(\omega) - i\epsilon''(\omega)\] where:
- \(\epsilon'\): Real part (polarization, refractive index) -
\(\epsilon''\): Imaginary part (absorption, dissipation)

\textbf{Refractive index}:
\(n = \sqrt{\epsilon' \mu_r} \approx \sqrt{\epsilon'}\) (assuming
\(\mu_r \approx 1\) for non-magnetic tissue)

\textbf{Absorption coefficient}:
\[\alpha(\omega) = \frac{\omega}{c} \sqrt{\frac{\epsilon'}{2} \left( \sqrt{1 + \left(\frac{\epsilon''}{\epsilon'}\right)^2} - 1 \right)} \approx \frac{\omega \epsilon''}{2cn}\]
for \(\epsilon'' \ll \epsilon'\).

\subsubsection{1.2 Frequency Dependence}\label{frequency-dependence}

\textbf{Water dominates} (tissue is \textasciitilde70-90\% water by
mass):

\textbf{Debye relaxation model} (single relaxation time):
\[\epsilon_r(\omega) = \epsilon_\infty + \frac{\epsilon_s - \epsilon_\infty}{1 + i\omega \tau}\]
where: - \(\epsilon_s\): Static permittivity (\textasciitilde80 for
water at DC) - \(\epsilon_\infty\): High-frequency limit
(\textasciitilde5 for water) - \(\tau\): Relaxation time
(\textasciitilde8 ps for bulk water at 20\$\^{}\textbackslash circ\$C)

\textbf{Relaxation frequency}: \(f_0 = 1/(2\pi \tau) \approx 20\) GHz

\textbf{At THz frequencies} (0.1-10 THz \textgreater\textgreater{} 20
GHz): - \(\epsilon'\) approaches \(\epsilon_\infty \approx 5\) -
\(\epsilon''\) increases linearly with frequency (Lorentzian tail) -
\textbf{Absorption increases with frequency}:
\(\alpha \propto \omega \epsilon''(\omega)\)

\subsubsection{1.3 Hydration State}\label{hydration-state}

\textbf{Free vs.~bound water}: - \textbf{Free water}: Bulk-like
rotational dynamics, strong THz absorption - \textbf{Bound water}: Near
protein surfaces, restricted rotation, reduced absorption

\textbf{Tissue hydration} varies: - \textbf{Skin (stratum corneum)}:
\textasciitilde20\% water \$\textbackslash rightarrow\$ lower absorption
- \textbf{Muscle}: \textasciitilde75\% water
\$\textbackslash rightarrow\$ high absorption - \textbf{Fat (adipose)}:
\textasciitilde10\% water \$\textbackslash rightarrow\$ low absorption
(relatively transparent at THz)

\textbf{Temperature dependence}: Absorption increases with temperature
(faster molecular relaxation)

\begin{center}\rule{0.5\linewidth}{0.5pt}\end{center}

\subsection{2. Absorption Mechanisms}\label{absorption-mechanisms}

\subsubsection{2.1 Water Rotational Modes}\label{water-rotational-modes}

\textbf{Dominant mechanism}: Dipolar water molecules rotate in response
to THz electric field.

\textbf{Debye absorption peak}: \textasciitilde20 GHz (microwave range)
\textbf{THz tail}: Absorption continues into THz due to: - Hindered
rotations (librational modes) - Collective hydrogen bond network
dynamics

\textbf{Absorption coefficient} (water at 1 THz,
20\$\^{}\textbackslash circ\$C): \(\alpha \approx 250\)
cm\textbackslash textsuperscript\{-\}\textbackslash textsuperscript\{1\}
\textbf{Penetration depth}: \(\delta = 1/\alpha \approx 40\)
\$\textbackslash mu\$m (very shallow!)

\subsubsection{2.2 Protein Vibrational
Modes}\label{protein-vibrational-modes}

\textbf{Proteins contribute} secondary absorption: -
\textbf{Low-frequency modes} (0.1-3 THz): Collective vibrations, domain
motions - \textbf{Amide bands}: \textasciitilde6 THz (C=O stretch
overtones)

\textbf{Effect}: Protein-rich tissues (e.g., collagen in dermis) have
enhanced absorption at specific frequencies.

\subsubsection{2.3 Lipid and Membrane
Absorption}\label{lipid-and-membrane-absorption}

\textbf{Lipids}: Lower absorption than water/protein - \textbf{Fatty
acids}: CH\textbackslash textsubscript\{2\} rocking modes at
\textasciitilde2-4 THz - \textbf{Phospholipid headgroups}: Hydrated,
contribute dielectric relaxation

\textbf{Cell membranes}: - Thin (\textasciitilde7 nm lipid bilayer)
\$\textbackslash rightarrow\$ minimal direct absorption - But
membrane-associated water has altered dynamics

\begin{center}\rule{0.5\linewidth}{0.5pt}\end{center}

\subsection{3. Scattering Mechanisms}\label{scattering-mechanisms}

\subsubsection{3.1 Rayleigh Scattering (Small
Particles)}\label{rayleigh-scattering-small-particles}

\textbf{Condition}: Particle size \(d \ll \lambda\) (THz wavelength
\textasciitilde100 \$\textbackslash mu\$m)

\textbf{Scattering cross-section}:
\[\sigma_s \propto \left(\frac{d}{\lambda}\right)^4 \propto \omega^4\]

\textbf{In tissue}: - \textbf{Organelles} (mitochondria \textasciitilde1
\$\textbackslash mu\$m, lysosomes \textasciitilde0.5
\$\textbackslash mu\$m): Weak Rayleigh scattering - \textbf{Cellular
nuclei} (\textasciitilde10 \$\textbackslash mu\$m): Transition to Mie
regime

\textbf{Result}: Scattering is weak compared to absorption at THz
frequencies (unlike visible light, where scattering dominates in tissue)

\subsubsection{3.2 Mie Scattering (Comparable
Size)}\label{mie-scattering-comparable-size}

\textbf{Condition}: Particle size \(d \approx \lambda\)

\textbf{Applicable to}: - Cells (\textasciitilde10-20
\$\textbackslash mu\$m diameter) at low THz (\textasciitilde0.3 THz,
\(\lambda \approx 1000\) \$\textbackslash mu\$m
\$\textbackslash rightarrow\$ Rayleigh regime) - Cells at high THz
(\textasciitilde3 THz, \(\lambda \approx 100\) \$\textbackslash mu\$m
\$\textbackslash rightarrow\$ Mie regime)

\textbf{Mie theory}: Complex calculation; depends on refractive index
contrast and particle geometry.

\subsubsection{3.3 Interface Reflections}\label{interface-reflections}

\textbf{Fresnel reflection} at interfaces with refractive index
mismatch: \[R = \left| \frac{n_1 - n_2}{n_1 + n_2} \right|^2\]

\textbf{Tissue interfaces}: - \textbf{Air-skin}: \(n_1 = 1\),
\(n_2 \approx 2.2\) \$\textbackslash rightarrow\$ \(R \approx 0.15\)
(15\% reflected) - \textbf{Dermis-fat}: Small contrast
\$\textbackslash rightarrow\$ \(R < 0.01\) - \textbf{Tissue-bone}: Large
contrast \$\textbackslash rightarrow\$ \(R \approx 0.2\)

\textbf{Implication}: Surface reflections significant; impedance
matching needed for efficient coupling

\begin{center}\rule{0.5\linewidth}{0.5pt}\end{center}

\subsection{4. Penetration Depth}\label{penetration-depth}

\subsubsection{4.1 Beer-Lambert Law}\label{beer-lambert-law}

Intensity decays exponentially: \[I(z) = I_0 e^{-\alpha z}\]

\textbf{Penetration depth} (1/e attenuation):
\[\delta = \frac{1}{\alpha}\]

\subsubsection{4.2 Frequency Dependence}\label{frequency-dependence-1}

\textbf{Typical values} (in vivo human tissue):

{\def\LTcaptype{} % do not increment counter
\begin{longtable}[]{@{}lll@{}}
\toprule\noalign{}
Frequency & Absorption (\(\text{cm}^{-1}\)) & Penetration depth \\
\midrule\noalign{}
\endhead
\bottomrule\noalign{}
\endlastfoot
0.1 THz & \textasciitilde10 & 1 mm \\
0.5 THz & \textasciitilde50 & 200 \$\textbackslash mu\$m \\
1.0 THz & \textasciitilde200 & 50 \$\textbackslash mu\$m \\
3.0 THz & \textasciitilde700 & 15 \$\textbackslash mu\$m \\
\end{longtable}
}

\textbf{Key trend}: Penetration decreases rapidly with frequency. At 1
THz, THz waves barely penetrate beyond epidermis (\textasciitilde100
\$\textbackslash mu\$m thick).

\subsubsection{4.3 Tissue-Specific
Penetration}\label{tissue-specific-penetration}

\textbf{Stratum corneum (dry outer skin)}: Lower water content
\$\textbackslash rightarrow\$ deeper penetration (\textasciitilde500
\$\textbackslash mu\$m at 0.5 THz) \textbf{Muscle}: High water
\$\textbackslash rightarrow\$ shallow (\textasciitilde50
\$\textbackslash mu\$m at 1 THz) \textbf{Fat}: Low water
\$\textbackslash rightarrow\$ relatively transparent (\textasciitilde2
mm at 1 THz) \textbf{Bone}: High mineral content, low water
\$\textbackslash rightarrow\$ moderate absorption

\textbf{Clinical implication}: THz imaging ideal for skin pathology
(basal cell carcinoma, burns); poor for deep tissue imaging.

\begin{center}\rule{0.5\linewidth}{0.5pt}\end{center}

\subsection{5. Wave Propagation Models}\label{wave-propagation-models}

\subsubsection{5.1 Plane Wave
Approximation}\label{plane-wave-approximation}

\textbf{Assumption}: Infinite homogeneous medium

\textbf{Electric field}:
\[E(z,t) = E_0 e^{-\alpha z/2} e^{i(kz - \omega t)}\] where
\(k = \omega n/c\) is the wavenumber.

\textbf{Limitations}: Ignores interfaces, scattering, finite beam
effects

\subsubsection{5.2 Stratified Media Model}\label{stratified-media-model}

\textbf{Skin structure}: Multi-layer (stratum corneum, epidermis,
dermis, fat)

\textbf{Transfer matrix method}: 1. Divide tissue into \(N\) layers 2.
Apply boundary conditions at each interface (Fresnel
reflection/transmission) 3. Multiply transfer matrices:
\(\mathbf{M}_{\text{total}} = \mathbf{M}_N \cdots \mathbf{M}_2 \mathbf{M}_1\)
4. Calculate total reflection/transmission

\textbf{Result}: Oscillatory reflection spectrum due to interference
(etalon effect in thin layers)

\subsubsection{5.3 Diffusion
Approximation}\label{diffusion-approximation}

\textbf{When scattering dominates} (rare in THz):
\[\nabla^2 U - \frac{U}{L^2} = -S\] where \(U\) is fluence rate,
\(L = 1/\sqrt{3\mu_a \mu_s'}\) is diffusion length (\(\mu_a\) =
absorption, \(\mu_s'\) = reduced scattering)

\textbf{Not applicable} to most THz tissue scenarios (absorption
\textgreater\textgreater{} scattering)

\begin{center}\rule{0.5\linewidth}{0.5pt}\end{center}

\subsection{6. Applications}\label{applications}

\subsubsection{\texorpdfstring{6.1 Medical Imaging
(Established)}{6.1 Medical Imaging  (Established)}}\label{medical-imaging-established}

\textbf{THz time-domain spectroscopy (THz-TDS)}: - Ultrafast THz pulse
(\(\sim\)ps duration) transmitted/reflected from tissue - Time-of-flight
\$\textbackslash rightarrow\$ layer thickness - Spectral features
\$\textbackslash rightarrow\$ molecular composition

\textbf{Clinical applications}: - \textbf{Skin cancer detection}: Basal
cell carcinoma has altered water content \$\textbackslash rightarrow\$
contrast - \textbf{Burn depth assessment}: Damaged tissue has different
THz signature - \textbf{Dental imaging}: Caries detection (demineralized
enamel has higher water)

\textbf{Limitations}: Shallow penetration, slow acquisition (raster
scanning), requires contact or near-contact

\subsubsection{\texorpdfstring{6.2 Security Screening
(Established)}{6.2 Security Screening  (Established)}}\label{security-screening-established}

\textbf{Through-clothing imaging}: - THz transparent to fabrics (low
water) - Opaque to skin (high water) - Detect concealed weapons,
explosives

\textbf{Systems}: Passive THz cameras (detect natural thermal emission)
or active illumination

\textbf{Privacy concerns}: Can image body contours
\$\textbackslash rightarrow\$ ``virtual strip search''

\subsubsection{\texorpdfstring{6.3 Neuromodulation
(Speculative)}{6.3 Neuromodulation  (Speculative)}}\label{neuromodulation-speculative}

\textbf{Hypothesis}: THz pulses could stimulate neurons non-invasively.

\textbf{Mechanisms} (proposed): - \textbf{Thermal}: Localized heating
triggers temperature-sensitive ion channels (TRPV1) -
\textbf{Non-thermal}: THz resonates with protein vibrational modes
\$\textbackslash rightarrow\$ conformational changes - \textbf{Quantum}:
Vibronic coupling in microtubules (see
{[}{[}THz-Resonances-in-Microtubules{]}{]})

\textbf{Challenges}: - Penetration: THz cannot reach deep brain
structures transcranially (skull absorption + scalp) - Power: High
intensity needed \$\textbackslash rightarrow\$ thermal damage risk -
Specificity: How to target specific neuron populations?

\textbf{Current status}: In vitro studies show weak effects; in vivo
neuromodulation not demonstrated

\begin{center}\rule{0.5\linewidth}{0.5pt}\end{center}

\subsection{7. Measurement Techniques}\label{measurement-techniques}

\subsubsection{7.1 THz Time-Domain Spectroscopy
(THz-TDS)}\label{thz-time-domain-spectroscopy-thz-tds}

\textbf{Setup}: 1. Femtosecond laser generates THz pulse via
photoconductive antenna or nonlinear crystal 2. THz pulse transmitted
through sample 3. Detected via electro-optic sampling (time-resolved
electric field)

\textbf{Advantages}: Phase-sensitive (extract \(\epsilon'\) and
\(\epsilon''\)), broadband (\textasciitilde0.1-5 THz)
\textbf{Disadvantages}: Slow (mechanical delay line), expensive

\subsubsection{7.2 Continuous-Wave (CW) THz
Systems}\label{continuous-wave-cw-thz-systems}

\textbf{Quantum cascade lasers (QCLs)}: High-power, narrow-band THz
sources

\textbf{Advantage}: Fast imaging, compact \textbf{Disadvantage}: Single
frequency (must scan laser to get spectrum)

\subsubsection{7.3 In Vivo Reflection
Geometry}\label{in-vivo-reflection-geometry}

\textbf{Challenge}: Most tissue measurements done in transmission; in
vivo requires reflection mode.

\textbf{Reflection coefficient}:
\[r(\omega) = \frac{E_{\text{ref}}(\omega)}{E_{\text{inc}}(\omega)}\]
Extract optical properties via model fitting (stratified media)

\textbf{Confounding factors}: Surface roughness, sweat, air gaps

\begin{center}\rule{0.5\linewidth}{0.5pt}\end{center}

\subsection{8. Safety Considerations}\label{safety-considerations}

\textbf{Thermal effects}: Dominant safety concern

\textbf{ICNIRP guidelines} (International Commission on Non-Ionizing
Radiation Protection): - \textbf{Power density limit}: 10
mW/cm\textbackslash textsuperscript\{2\} (averaged over 6 minutes) for
0.3-3 THz - \textbf{Temperature rise}: Must stay below
1\$\^{}\textbackslash circ\$C for prolonged exposure

\textbf{Non-thermal effects}: Controversial (see
{[}{[}THz-Bioeffects-Thermal-and-Non-Thermal{]}{]})

\textbf{Current assessment}: THz radiation is non-ionizing (photon
energy \textasciitilde4 meV \textless\textless{} 13.6 eV ionization
potential). At low intensities (\textless1
W/cm\textbackslash textsuperscript\{2\}), considered safe for brief
exposure.

\begin{center}\rule{0.5\linewidth}{0.5pt}\end{center}

\subsection{9. Connections to Other Wiki
Pages}\label{connections-to-other-wiki-pages}

\begin{itemize}
\tightlist
\item
  {[}{[}Terahertz-(THz)-Technology{]}{]} -\/-\/- THz sources and
  detectors
\item
  {[}{[}THz-Bioeffects-Thermal-and-Non-Thermal{]}{]} -\/-\/- Biological
  effects
\item
  {[}{[}THz-Resonances-in-Microtubules{]}{]} -\/-\/- Speculative quantum
  effects
\item
  {[}{[}mmWave-\&-THz-Communications{]}{]} -\/-\/- Wireless applications
\item
  {[}{[}Free-Space-Path-Loss-(FSPL){]}{]} -\/-\/- Propagation
  fundamentals
\end{itemize}

\begin{center}\rule{0.5\linewidth}{0.5pt}\end{center}

\subsection{10. References}\label{references}

\subsubsection{Tissue Optical
Properties}\label{tissue-optical-properties}

\begin{enumerate}
\def\labelenumi{\arabic{enumi}.}
\tightlist
\item
  \textbf{Pickwell \& Wallace, \emph{J. Phys. D} 39, R301 (2006)}
  -\/-\/- THz tissue review
\item
  \textbf{Smye et al., \emph{Phys. Med. Biol.} 46, R101 (2001)} -\/-\/-
  Tissue dielectric properties
\end{enumerate}

\subsubsection{Medical Imaging}\label{medical-imaging}

\begin{enumerate}
\def\labelenumi{\arabic{enumi}.}
\setcounter{enumi}{2}
\tightlist
\item
  \textbf{Woodward et al., \emph{Phys. Med. Biol.} 47, 3853 (2002)}
  -\/-\/- THz skin imaging
\item
  \textbf{Wallace et al., \emph{Br. J. Dermatol.} 151, 424 (2004)}
  -\/-\/- Basal cell carcinoma detection
\end{enumerate}

\subsubsection{Propagation Models}\label{propagation-models}

\begin{enumerate}
\def\labelenumi{\arabic{enumi}.}
\setcounter{enumi}{4}
\tightlist
\item
  \textbf{Pickwell et al., \emph{Appl. Phys. Lett.} 84, 2190 (2004)}
  -\/-\/- Stratified media modeling
\end{enumerate}

\subsubsection{Safety}\label{safety}

\begin{enumerate}
\def\labelenumi{\arabic{enumi}.}
\setcounter{enumi}{5}
\tightlist
\item
  \textbf{ICNIRP, \emph{Health Phys.} 99, 818 (2010)} -\/-\/- THz
  exposure guidelines
\end{enumerate}

\begin{center}\rule{0.5\linewidth}{0.5pt}\end{center}

\textbf{Last updated}: October 2025
