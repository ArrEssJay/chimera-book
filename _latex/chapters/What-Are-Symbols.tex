\section{What Are Symbols?}\label{what-are-symbols}

\subsection{\texorpdfstring{ For Non-Technical
Readers}{ For Non-Technical Readers}}\label{for-non-technical-readers}

\textbf{A symbol is like a ``word'' in radio language-\/-\/-instead of
sending individual letters (bits), you send whole words (symbols) to go
faster!}

\textbf{The idea}: - \textbf{Bit}: Single 0 or 1 (like one letter) -
\textbf{Symbol}: Group of bits (like a whole word) - \textbf{Why}:
Sending ``words'' is faster than spelling letter-by-letter!

\textbf{Simple analogy - Semaphore flags}:

\textbf{Sending bit-by-bit}: - Flag up = 1 - Flag down = 0 - Message
``HI'' (8 bits): Up, down, down, down, up, down, down, up - Takes 8
``signals''

\textbf{Sending symbol-by-symbol}: - 4 flag positions = 1 symbol = 2
bits! - Up-right = ``00'' - Up-left = ``01'' - Down-left = ``10''\\
- Down-right = ``11'' - Message ``HI'' (8 bits
\$\textbackslash rightarrow\$ 4 symbols): - Takes only 4 ``signals'' =
\textbf{2\$\textbackslash times\$ faster!}

\textbf{Real examples by modulation}:

\textbf{BPSK (Binary PSK)}: - 1 symbol = 1 bit - 2 possible positions
(up or down) - Like: Light switch (on/off)

\textbf{QPSK}: - 1 symbol = 2 bits - 4 possible positions - Like: 4-way
hand gesture

\textbf{16-QAM}: - 1 symbol = 4 bits - 16 possible positions - Like:
Showing fingers (0-15)

\textbf{256-QAM}: - 1 symbol = 8 bits - 256 possible positions! - Like:
Sign language with 256 distinct signs

\textbf{Why symbols matter}:

\textbf{Symbol rate vs bit rate}: - \textbf{Symbol rate}: How many
symbols per second? - \textbf{Bit rate}: How many bits per second? -
\textbf{Relationship}: Bit rate = Symbol rate \$\textbackslash times\$
bits/symbol

\textbf{Example - WiFi}: - Symbol rate: 250,000 symbols/second -
Modulation: 256-QAM (8 bits/symbol) - Bit rate: 250,000
\$\textbackslash times\$ 8 = 2 Mbps

\textbf{Bandwidth connection}: - Symbols take time/frequency (bandwidth)
- More symbols/second = more bandwidth needed - But: More bits/symbol =
free speed boost!

\textbf{The constellation diagram}: - Visual map of all possible symbols
- Each dot = one unique symbol - More dots = more bits/symbol = faster!

\textbf{Real-world impact}: - Your phone: Switches between modulations
(QPSK \$\textbackslash rightarrow\$ 64-QAM) = switching symbol mappings
- Strong signal: Use symbols with 8 bits each (fast!) - Weak signal: Use
symbols with 2 bits each (reliable!)

\textbf{Fun fact}: When your WiFi says ``MCS 9, 256-QAM'', that means
each symbol carries 8 bits. If you move away and it drops to ``MCS 0,
BPSK'', each symbol now carries only 1 bit-\/-\/-that\textquotesingle s
an 8\$\textbackslash times\$ speed difference, just by changing how many
bits each symbol represents!

\begin{center}\rule{0.5\linewidth}{0.5pt}\end{center}

In digital communication, a \textbf{symbol} is a fundamental unit of
information transmitted over the channel. Think of symbols as the
``words'' of a digital communication system.

\subsection{The Symbol Hierarchy}\label{the-symbol-hierarchy}

\begin{verbatim}
Raw Data (Bits)
    
Grouped into Symbols
    
Mapped to Signal States
    
Transmitted over Channel
\end{verbatim}

\subsection{Example: From Bits to
Symbols}\label{example-from-bits-to-symbols}

Imagine you want to transmit the binary data:
\texttt{0\ 0\ 1\ 1\ 0\ 1\ 1\ 0}

Instead of sending each bit individually, we group them into pairs (for
QPSK): - Bits \texttt{0\ 0} \$\textbackslash rightarrow\$ Symbol 1 -
Bits \texttt{1\ 1} \$\textbackslash rightarrow\$ Symbol 2\\
- Bits \texttt{0\ 1} \$\textbackslash rightarrow\$ Symbol 3 - Bits
\texttt{1\ 0} \$\textbackslash rightarrow\$ Symbol 4

This grouping allows us to transmit more efficiently and makes the
signal more robust to noise.

\subsection{Why Use Symbols?}\label{why-use-symbols}

\begin{enumerate}
\def\labelenumi{\arabic{enumi}.}
\tightlist
\item
  \textbf{Efficiency}: Transmitting symbols (groups of bits) can be more
  bandwidth-efficient than transmitting individual bits
\item
  \textbf{Robustness}: Symbol-based schemes can be designed to be more
  resistant to noise and interference
\item
  \textbf{Flexibility}: Different modulation schemes can encode
  different numbers of bits per symbol
\end{enumerate}

\subsection{Bits Per Symbol in Common Modulation
Schemes}\label{bits-per-symbol-in-common-modulation-schemes}

{\def\LTcaptype{} % do not increment counter
\begin{longtable}[]{@{}lll@{}}
\toprule\noalign{}
Modulation & Bits/Symbol & Total States \\
\midrule\noalign{}
\endhead
\bottomrule\noalign{}
\endlastfoot
BPSK & 1 & 2 \\
\textbf{QPSK} & \textbf{2} & \textbf{4} \\
8PSK & 3 & 8 \\
16QAM & 4 & 16 \\
64QAM & 6 & 64 \\
\end{longtable}
}

\textbf{Chimera uses QPSK} (2 bits per symbol, 4 states)

\subsection{See Also}\label{see-also}

\begin{itemize}
\tightlist
\item
  {[}{[}QPSK-Modulation{]}{]} - How symbols are mapped to phase states
\item
  {[}{[}Constellation-Diagrams{]}{]} - Visual representation of symbols
\item
  {[}{[}IQ-Representation{]}{]} - Mathematical representation of symbols
\end{itemize}
