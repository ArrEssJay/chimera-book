\section{Atmospheric Effects: Ionospheric \&
Tropospheric}\label{atmospheric-effects-ionospheric-tropospheric}

{[}{[}Home{]}{]} \textbar{} \textbf{RF Propagation} \textbar{}
{[}{[}Propagation-Modes-(Ground-Wave,-Sky-Wave,-Line-of-Sight){]}{]}
\textbar{} {[}{[}Weather-Effects-(Rain-Fade,-Fog-Attenuation){]}{]}

\begin{center}\rule{0.5\linewidth}{0.5pt}\end{center}

\subsection{\texorpdfstring{ For Non-Technical
Readers}{ For Non-Technical Readers}}\label{for-non-technical-readers}

\textbf{Think of the atmosphere as a giant, invisible lens and filter
for radio waves.}

Imagine you\textquotesingle re trying to shine a flashlight across a
room: - \textbf{On a clear day}, the light travels straight and far -
\textbf{Through fog}, the light gets scattered and dimmer - \textbf{With
a curved mirror}, the light bends and can reach around corners

Radio waves behave similarly through Earth\textquotesingle s atmosphere:

\begin{enumerate}
\def\labelenumi{\arabic{enumi}.}
\tightlist
\item
  \textbf{The Ionosphere} (60-400 km up) is like a \textbf{curved mirror
  in space}

  \begin{itemize}
  \tightlist
  \item
    Acts like a reflector for AM radio and shortwave (HF) signals
  \item
    This is why you can hear distant AM radio stations at
    night-\/-\/-the signal bounces off this invisible mirror!
  \item
    Created by the sun\textquotesingle s energy ionizing air molecules
  \end{itemize}
\item
  \textbf{The Troposphere} (0-15 km up, where weather happens) is like
  \textbf{fog or water vapor}

  \begin{itemize}
  \tightlist
  \item
    Bends and absorbs radio waves, especially at high frequencies
  \item
    This is why 5G signals don\textquotesingle t travel as far as
    4G-\/-\/-they\textquotesingle re more easily absorbed by air
    humidity
  \item
    Weather (rain, fog) makes this worse
  \end{itemize}
\end{enumerate}

\textbf{Real-world impact}: - \textbf{GPS errors}: The ionosphere slows
down GPS signals, causing \textasciitilde10-30 meter errors (your phone
corrects for this) - \textbf{Satellite TV in rain}: Signal drops out
because raindrops absorb the microwaves - \textbf{Shortwave radio at
night}: Can receive stations from across the globe because the
ionosphere reflects signals back to Earth

\textbf{The key insight}: Different radio frequencies interact with the
atmosphere in completely different ways-\/-\/-AM radio bounces off the
ionosphere, while 5G gets absorbed by humidity.

\begin{center}\rule{0.5\linewidth}{0.5pt}\end{center}

\subsection{Overview}\label{overview}

\textbf{Earth\textquotesingle s atmosphere profoundly affects RF
propagation} through:

\begin{enumerate}
\def\labelenumi{\arabic{enumi}.}
\tightlist
\item
  \textbf{Ionosphere} (60-1000 km altitude): \textbf{Refracts HF},
  enables sky wave
\item
  \textbf{Troposphere} (0-15 km altitude): \textbf{Absorbs/refracts
  VHF+}, causes ducting
\end{enumerate}

\textbf{Key distinction}: - \textbf{Below \textasciitilde30 MHz}:
Ionosphere dominates (enables long-distance HF comms) - \textbf{Above
\textasciitilde1 GHz}: Troposphere/weather dominates (absorption, rain
fade)

\begin{center}\rule{0.5\linewidth}{0.5pt}\end{center}

\subsection{Ionospheric Effects}\label{ionospheric-effects}

\subsubsection{Structure of the
Ionosphere}\label{structure-of-the-ionosphere}

\textbf{Ionosphere = layers of ionized gas} (free electrons and ions
created by solar UV/X-rays)

{\def\LTcaptype{} % do not increment counter
\begin{longtable}[]{@{}
  >{\raggedright\arraybackslash}p{(\linewidth - 6\tabcolsep) * \real{0.1250}}
  >{\raggedright\arraybackslash}p{(\linewidth - 6\tabcolsep) * \real{0.1786}}
  >{\raggedright\arraybackslash}p{(\linewidth - 6\tabcolsep) * \real{0.3929}}
  >{\raggedright\arraybackslash}p{(\linewidth - 6\tabcolsep) * \real{0.3036}}@{}}
\toprule\noalign{}
\begin{minipage}[b]{\linewidth}\raggedright
Layer
\end{minipage} & \begin{minipage}[b]{\linewidth}\raggedright
Altitude
\end{minipage} & \begin{minipage}[b]{\linewidth}\raggedright
Peak Density (\(N_e\))
\end{minipage} & \begin{minipage}[b]{\linewidth}\raggedright
Characteristics
\end{minipage} \\
\midrule\noalign{}
\endhead
\bottomrule\noalign{}
\endlastfoot
\textbf{D} & 60-90 km &
10\textbackslash textsuperscript\{8\}-10\textbackslash textsuperscript\{9\}
e\textbackslash textsuperscript\{-\}/m\textbackslash textsuperscript\{3\}
& \textbf{Absorbs MF/HF} (daytime only) \\
\textbf{E} & 90-150 km &
10\textbackslash textsuperscript\{1\}\textbackslash textsuperscript\{0\}-10\textbackslash textsuperscript\{1\}\textbackslash textsuperscript\{1\}
e\textbackslash textsuperscript\{-\}/m\textbackslash textsuperscript\{3\}
& Reflects MF, low HF \\
\textbf{F1} & 150-250 km &
10\textbackslash textsuperscript\{1\}\textbackslash textsuperscript\{1\}
e\textbackslash textsuperscript\{-\}/m\textbackslash textsuperscript\{3\}
& Daytime only, merges with F2 at night \\
\textbf{F2} & 250-400 km &
10\textbackslash textsuperscript\{1\}\textbackslash textsuperscript\{1\}-10\textbackslash textsuperscript\{1\}\textbackslash textsuperscript\{2\}
e\textbackslash textsuperscript\{-\}/m\textbackslash textsuperscript\{3\}
& \textbf{Primary HF reflector}, highest density \\
\end{longtable}
}

\textbf{Formation}: Solar UV photons ionize
O\textbackslash textsubscript\{2\}, N\textbackslash textsubscript\{2\}
\$\textbackslash rightarrow\$
O\textbackslash textsubscript\{2\}\textbackslash textsuperscript\{+\},
N\textbackslash textsubscript\{2\}\textbackslash textsuperscript\{+\},
e\textbackslash textsuperscript\{-\}

\textbf{Recombination}: Electrons recombine with ions (faster at lower
altitudes due to higher density)

\begin{center}\rule{0.5\linewidth}{0.5pt}\end{center}

\subsubsection{Refractive Index}\label{refractive-index}

\textbf{Plasma refractive index}:

\[
n = \sqrt{1 - \frac{f_p^2}{f^2}}
\]

Where: - \(f_p\) = Plasma frequency = \(9\sqrt{N_e}\) Hz (\(N_e\) in
electrons/m\textbackslash textsuperscript\{3\}) - \(f\) = Signal
frequency

\textbf{Key behaviors}:

\begin{enumerate}
\def\labelenumi{\arabic{enumi}.}
\tightlist
\item
  \textbf{\(f \ll f_p\)}: Wave is \textbf{reflected} (HF sky wave)
\item
  \textbf{\(f \approx f_p\)}: Wave \textbf{refracts} (bends back to
  Earth)
\item
  \textbf{\(f \gg f_p\)}: Wave \textbf{penetrates} (VHF+ passes through
  ionosphere)
\end{enumerate}

\textbf{Typical \(f_p\) values}: - D-layer: \textasciitilde1 MHz -
E-layer: \textasciitilde3-5 MHz - F2-layer (day): \textasciitilde10-15
MHz - F2-layer (night): \textasciitilde5-10 MHz

\textbf{Implication}: \textbf{VHF and above (\textgreater30 MHz) always
penetrate} ionosphere \$\textbackslash rightarrow\$ No skywave, only
LOS.

\begin{center}\rule{0.5\linewidth}{0.5pt}\end{center}

\subsubsection{Critical Frequency \& Skip
Distance}\label{critical-frequency-skip-distance}

\textbf{Critical frequency} \(f_c\): Maximum frequency reflected at
\textbf{vertical incidence}

\[
f_c = 9\sqrt{N_{e,\text{max}}}
\]

\textbf{At oblique angles}, higher frequencies can be reflected:

\[
\text{MUF} = \frac{f_c}{\sin(\theta)}
\]

Where \(\theta\) = elevation angle

\textbf{Example}: If \(f_c = 10\) MHz, and wave launched at
10\$\^{}\textbackslash circ\$ elevation:

\[
\text{MUF} = \frac{10}{\sin(10°)} = \frac{10}{0.174} = 57\ \text{MHz}
\]

(But practical MUF limited by absorption and other factors to
\textasciitilde30 MHz)

\begin{center}\rule{0.5\linewidth}{0.5pt}\end{center}

\subsubsection{Absorption}\label{absorption}

\textbf{D-layer absorption} (collisional damping):

\[
A = K \cdot \frac{N_e \cdot \nu}{f^2} \quad (\text{dB})
\]

Where: - \(\nu\) = Collision frequency
(\textasciitilde10\textbackslash textsuperscript\{6\} Hz in D-layer) -
\(N_e\) = Electron density - \(f\) = Signal frequency

\textbf{Key insight}: \textbf{Absorption \$\textbackslash propto\$
1/f\textbackslash textsuperscript\{2\}} \$\textbackslash rightarrow\$
Lower frequencies absorbed more

\textbf{Impact}: - \textbf{Daytime}: D-layer absorbs 1-5 MHz (MF/LF
severe absorption) - \textbf{Nighttime}: D-layer disappears
\$\textbackslash rightarrow\$ Lower frequencies propagate (AM broadcast
skywave)

\textbf{Typical absorption} (HF, daytime): - 3 MHz: 10-20 dB - 7 MHz:
3-6 dB - 14 MHz: 1-2 dB - 28 MHz: \textless1 dB

\begin{center}\rule{0.5\linewidth}{0.5pt}\end{center}

\subsubsection{Faraday Rotation}\label{faraday-rotation}

\textbf{Ionosphere is magnetized} (Earth\textquotesingle s magnetic
field):

\textbf{Effect}: \textbf{Polarization rotates} as wave propagates
through ionosphere

\[
\Omega = \frac{2.36 \times 10^4}{f^2} \int N_e B_\parallel \, dl \quad (\text{radians})
\]

Where: - \(f\) = Frequency (Hz) - \(N_e\) = Electron density
(e\textbackslash textsuperscript\{-\}/m\textbackslash textsuperscript\{3\})
- \(B_\parallel\) = Magnetic field component along path (Tesla) -
Integral over path length

\textbf{Impact}: - \textbf{Linear polarized signals} experience rotation
(can cause \textgreater20 dB loss if RX antenna wrong orientation) -
\textbf{Circular polarization immune} (GPS, satellite comms use
RHCP/LHCP to mitigate)

\textbf{Example}: GPS L1 (1575 MHz) experiences
\textasciitilde10-50\$\^{}\textbackslash circ\$ rotation (varies with
solar activity, latitude)

\begin{center}\rule{0.5\linewidth}{0.5pt}\end{center}

\subsubsection{Ionospheric
Scintillation}\label{ionospheric-scintillation}

\textbf{Irregularities in ionosphere} (plasma turbulence) cause:

\begin{enumerate}
\def\labelenumi{\arabic{enumi}.}
\tightlist
\item
  \textbf{Amplitude scintillation}: Rapid fading (seconds to minutes)
\item
  \textbf{Phase scintillation}: Phase jitter (disrupts carrier tracking)
\end{enumerate}

\textbf{Causes}: - Equatorial plasma bubbles (post-sunset) - Auroral
activity (high latitudes) - Solar flares (sudden ionospheric
disturbances)

\textbf{Impact}: - GPS errors (meter-level positioning errors) -
Satellite comms outages (L-band, 1-2 GHz) - Most severe near magnetic
equator and auroral zones

\textbf{Mitigation}: Dual-frequency GPS (L1 + L5) corrects ionospheric
delay

\begin{center}\rule{0.5\linewidth}{0.5pt}\end{center}

\subsubsection{Solar Activity Effects}\label{solar-activity-effects}

\paragraph{Solar Flares}\label{solar-flares}

\textbf{X-ray burst ionizes D-layer}:

\begin{itemize}
\tightlist
\item
  \textbf{Sudden Ionospheric Disturbance (SID)}: HF absorption increases
  10-30 dB instantly
\item
  Duration: Minutes to hours
\item
  Daytime only (needs sunlight)
\end{itemize}

\textbf{Result}: HF blackout on sunlit side of Earth

\begin{center}\rule{0.5\linewidth}{0.5pt}\end{center}

\paragraph{Geomagnetic Storms}\label{geomagnetic-storms}

\textbf{Coronal mass ejection (CME) hits Earth}:

\begin{itemize}
\tightlist
\item
  \textbf{Auroral electrojet}: Intense ionization at high latitudes
\item
  \textbf{Ionospheric storm}: TEC (total electron content) increases
  globally
\item
  Duration: Days
\end{itemize}

\textbf{Result}: - HF propagation unpredictable - GPS errors increase
(10-100m) - Satellite operations affected

\begin{center}\rule{0.5\linewidth}{0.5pt}\end{center}

\paragraph{11-Year Solar Cycle}\label{year-solar-cycle}

\textbf{Solar maximum}: - Higher ionization (F2 peak density
2-3\$\textbackslash times\$ higher) - MUF increases (30+ MHz usable for
long-distance) - Better long-distance HF propagation

\textbf{Solar minimum}: - Lower MUF (\textasciitilde15-20 MHz) - 10m
band (28 MHz) often ``dead'' - More reliance on lower HF bands (7, 14
MHz)

\textbf{Current cycle}: Solar Cycle 25 (2019-2030), peak
\textasciitilde2025

\begin{center}\rule{0.5\linewidth}{0.5pt}\end{center}

\subsubsection{Ionospheric Delay}\label{ionospheric-delay}

\textbf{Group delay} (signal travels slower than speed of light):

\[
\Delta t = \frac{40.3}{c f^2} \int N_e \, dl \quad (\text{seconds})
\]

\textbf{Impact}: - GPS ranging errors (10-100m if uncorrected) -
Two-frequency correction: Measure delay at L1 and L5, compute
ionospheric TEC

\textbf{TEC (Total Electron Content)}:

\[
\text{TEC} = \int N_e \, dl \quad (\text{electrons/m}^2)
\]

\textbf{Typical values}: - Nighttime:
10\textbackslash textsuperscript\{1\}\textbackslash textsuperscript\{6\}
e\textbackslash textsuperscript\{-\}/m\textbackslash textsuperscript\{2\}
- Daytime:
10\textbackslash textsuperscript\{1\}\textbackslash textsuperscript\{7\}
e\textbackslash textsuperscript\{-\}/m\textbackslash textsuperscript\{2\}
- Solar max:
10\textbackslash textsuperscript\{1\}\textbackslash textsuperscript\{8\}
e\textbackslash textsuperscript\{-\}/m\textbackslash textsuperscript\{2\}
(equatorial)

\begin{center}\rule{0.5\linewidth}{0.5pt}\end{center}

\subsection{Tropospheric Effects}\label{tropospheric-effects}

\textbf{Troposphere} = Lower atmosphere (0-15 km altitude), where
weather occurs

\textbf{Key effects}: 1. \textbf{Refraction} (bending, ducting) 2.
\textbf{Absorption} (oxygen, water vapor) 3. \textbf{Scattering} (rain,
turbulence)

\begin{center}\rule{0.5\linewidth}{0.5pt}\end{center}

\subsubsection{Atmospheric Refraction}\label{atmospheric-refraction}

\textbf{Refractive index} depends on temperature, pressure, humidity:

\[
n = 1 + N \times 10^{-6}
\]

Where \textbf{refractivity} \(N\):

\[
N = 77.6 \frac{P}{T} + 3.73 \times 10^5 \frac{e}{T^2}
\]

\begin{itemize}
\tightlist
\item
  \(P\) = Pressure (hPa)
\item
  \(T\) = Temperature (K)
\item
  \(e\) = Water vapor partial pressure (hPa)
\end{itemize}

\textbf{Typical values}: - Sea level: \(N \approx 300-400\)
\$\textbackslash rightarrow\$ \(n \approx 1.0003\) - 10 km altitude:
\(N \approx 100\) \$\textbackslash rightarrow\$ \(n \approx 1.0001\)

\begin{center}\rule{0.5\linewidth}{0.5pt}\end{center}

\subsubsection{Ray Bending}\label{ray-bending}

\textbf{Gradient in \(N\) bends rays downward}:

\textbf{Standard atmosphere}: \(dN/dh \approx -40\) N-units/km

\textbf{Effect}: \textbf{Radio horizon extended} beyond geometric
horizon

\textbf{4/3 Earth radius model}:

\[
d_{\text{radio}} = 1.33 \times d_{\text{geometric}}
\]

\textbf{Example}: Geometric horizon for 30m antenna = 20 km
\$\textbackslash rightarrow\$ Radio horizon = 26 km

\begin{center}\rule{0.5\linewidth}{0.5pt}\end{center}

\subsubsection{Tropospheric Ducting}\label{tropospheric-ducting}

\textbf{Temperature inversion} (warm air over cool) creates refractive
layer:

\textbf{Super-refraction}: Wave bends more than normal
\$\textbackslash rightarrow\$ Trapped in duct

\textbf{Effect}: \textbf{VHF/UHF signals propagate 500-2000 km} (far
beyond normal LOS)

\textbf{Conditions}: - Coastal regions (cool ocean, warm land) -
High-pressure systems (stable, clear weather) - Nighttime (radiative
cooling)

\textbf{Impact}: - FM/TV interference from distant stations - Cellular
network interference (distant cells suddenly visible) - Opportunistic
long-range VHF communications

\textbf{Duct height}: Typically 10-100m (depends on inversion strength)

\begin{center}\rule{0.5\linewidth}{0.5pt}\end{center}

\subsubsection{Atmospheric Absorption}\label{atmospheric-absorption}

\textbf{Gases absorb RF energy}:

\begin{enumerate}
\def\labelenumi{\arabic{enumi}.}
\tightlist
\item
  \textbf{Oxygen (O\textbackslash textsubscript\{2\})}: Peak at
  \textbf{60 GHz}, secondary at 118 GHz
\item
  \textbf{Water vapor (H\textbackslash textsubscript\{2\}O)}: Peaks at
  \textbf{22.2 GHz, 183 GHz, 325 GHz}, plus continuum absorption
\end{enumerate}

\begin{center}\rule{0.5\linewidth}{0.5pt}\end{center}

\paragraph{Oxygen Absorption}\label{oxygen-absorption}

\textbf{60 GHz resonance}:

\[
\alpha_{O_2} \approx 15\ \text{dB/km} \quad \text{(at sea level, 60 GHz)}
\]

\textbf{Frequency dependence} (0-100 GHz):

{\def\LTcaptype{} % do not increment counter
\begin{longtable}[]{@{}ll@{}}
\toprule\noalign{}
Frequency & Attenuation (dB/km) \\
\midrule\noalign{}
\endhead
\bottomrule\noalign{}
\endlastfoot
10 GHz & 0.01 \\
30 GHz & 0.05 \\
50 GHz & 0.3 \\
\textbf{60 GHz} & \textbf{15} (peak) \\
70 GHz & 1 \\
100 GHz & 0.5 \\
\end{longtable}
}

\textbf{Application}: 60 GHz used for \textbf{secure short-range comms}
(signals don\textquotesingle t propagate far)

\begin{center}\rule{0.5\linewidth}{0.5pt}\end{center}

\paragraph{Water Vapor Absorption}\label{water-vapor-absorption}

\textbf{22.2 GHz resonance}:

\[
\alpha_{H_2O} = k \cdot \rho \quad (\text{dB/km})
\]

Where: - \(\rho\) = Water vapor density
(g/m\textbackslash textsuperscript\{3\}) - \(k\) = Frequency-dependent
coefficient

\textbf{Typical humidity} (7.5 g/m\textbackslash textsuperscript\{3\} at
sea level, temperate):

{\def\LTcaptype{} % do not increment counter
\begin{longtable}[]{@{}ll@{}}
\toprule\noalign{}
Frequency & Attenuation (dB/km) \\
\midrule\noalign{}
\endhead
\bottomrule\noalign{}
\endlastfoot
10 GHz & 0.01 \\
\textbf{22.2 GHz} & \textbf{0.2} (peak) \\
30 GHz & 0.08 \\
50 GHz & 0.15 \\
100 GHz & 1.0 \\
300 GHz & 10+ (THz region) \\
\end{longtable}
}

\textbf{Implication}: \textbf{THz communications limited to
indoor/short-range} (water vapor + rain = severe attenuation)

\begin{center}\rule{0.5\linewidth}{0.5pt}\end{center}

\subsubsection{Atmospheric Windows}\label{atmospheric-windows}

\textbf{Frequency ranges with low absorption} (clear air):

{\def\LTcaptype{} % do not increment counter
\begin{longtable}[]{@{}
  >{\raggedright\arraybackslash}p{(\linewidth - 6\tabcolsep) * \real{0.2162}}
  >{\raggedright\arraybackslash}p{(\linewidth - 6\tabcolsep) * \real{0.2973}}
  >{\raggedright\arraybackslash}p{(\linewidth - 6\tabcolsep) * \real{0.3514}}
  >{\raggedright\arraybackslash}p{(\linewidth - 6\tabcolsep) * \real{0.1351}}@{}}
\toprule\noalign{}
\begin{minipage}[b]{\linewidth}\raggedright
Window
\end{minipage} & \begin{minipage}[b]{\linewidth}\raggedright
Frequency
\end{minipage} & \begin{minipage}[b]{\linewidth}\raggedright
Attenuation
\end{minipage} & \begin{minipage}[b]{\linewidth}\raggedright
Use
\end{minipage} \\
\midrule\noalign{}
\endhead
\bottomrule\noalign{}
\endlastfoot
\textbf{HF} & 3-30 MHz & Negligible & Skywave comms \\
\textbf{VHF/UHF} & 30-3000 MHz & \textless{} 0.01 dB/km & Broadcast,
cellular \\
\textbf{L/S-band} & 1-4 GHz & \textless{} 0.01 dB/km & GPS, mobile
satellite \\
\textbf{C-band} & 4-8 GHz & 0.01 dB/km & Satellite (robust) \\
\textbf{X/Ku-band} & 8-18 GHz & 0.05-0.5 dB/km & Satellite, radar \\
\textbf{Ka-band} & 26.5-40 GHz & 0.1-0.3 dB/km & High-rate satellite \\
\textbf{V/W-band} & 40-100 GHz & 0.3-15 dB/km & Point-to-point (watch 60
GHz!) \\
\end{longtable}
}

\textbf{Avoid}: 22 GHz (H\textbackslash textsubscript\{2\}O), 60 GHz
(O\textbackslash textsubscript\{2\}), 183 GHz
(H\textbackslash textsubscript\{2\}O)

\begin{center}\rule{0.5\linewidth}{0.5pt}\end{center}

\subsubsection{Tropospheric
Scintillation}\label{tropospheric-scintillation}

\textbf{Turbulence in troposphere} causes refractive index fluctuations:

\textbf{Effect}: Amplitude/phase scintillation (similar to ionospheric,
but different mechanism)

\textbf{Severity}: Increases with: - Frequency (\textgreater{} 10 GHz) -
Low elevation angles (longer path through troposphere) - Daytime
(convective turbulence)

\textbf{Impact}: - Satellite links \textgreater{} 20 GHz: 1-3 dB
peak-to-peak fading - Typically slower than rain fade (seconds to
minutes)

\textbf{Mitigation}: Less critical than rain fade (lower magnitude)

\begin{center}\rule{0.5\linewidth}{0.5pt}\end{center}

\subsection{Path Loss Models with Atmospheric
Effects}\label{path-loss-models-with-atmospheric-effects}

\subsubsection{Satellite Link Budget (with
Atmosphere)}\label{satellite-link-budget-with-atmosphere}

\textbf{Total path loss}:

\[
L_{\text{total}} = L_{\text{FS}} + L_{\text{atm}} + L_{\text{rain}} + L_{\text{scint}}
\]

Where: - \(L_{\text{FS}}\) = Free-space path loss (see
{[}{[}Free-Space-Path-Loss-(FSPL){]}{]}) - \(L_{\text{atm}}\) =
Clear-air atmospheric absorption (O\textbackslash textsubscript\{2\},
H\textbackslash textsubscript\{2\}O) - \(L_{\text{rain}}\) = Rain
attenuation (see
{[}{[}Weather-Effects-(Rain-Fade,-Fog-Attenuation){]}{]}) -
\(L_{\text{scint}}\) = Tropospheric scintillation (margin for fading)

\begin{center}\rule{0.5\linewidth}{0.5pt}\end{center}

\paragraph{Example: Ku-Band Satellite (12
GHz)}\label{example-ku-band-satellite-12-ghz}

\textbf{Path}: GEO (36,000 km), 30\$\^{}\textbackslash circ\$ elevation

\textbf{Free-space loss}:

\[
L_{\text{FS}} = 20\log(36000 \times 10^3) + 20\log(12 \times 10^9) + 92.45 = 205.5\ \text{dB}
\]

\textbf{Clear-air atmospheric} (O\textbackslash textsubscript\{2\} +
H\textbackslash textsubscript\{2\}O, zenith):

\[
L_{\text{atm}} \approx 0.3\ \text{dB}
\]

\textbf{At 30\$\^{}\textbackslash circ\$ elevation} (longer slant path):

\[
L_{\text{atm}} = 0.3 / \sin(30°) = 0.6\ \text{dB}
\]

\textbf{Rain fade} (99.9\% availability, temperate):

\[
L_{\text{rain}} = 3\ \text{dB} \quad \text{(see weather effects page)}
\]

\textbf{Scintillation margin}:

\[
L_{\text{scint}} = 1\ \text{dB}
\]

\textbf{Total}:

\[
L_{\text{total}} = 205.5 + 0.6 + 3 + 1 = 210.1\ \text{dB}
\]

\begin{center}\rule{0.5\linewidth}{0.5pt}\end{center}

\paragraph{Example: Ka-Band (30 GHz)}\label{example-ka-band-30-ghz}

\textbf{Same geometry}:

\[
L_{\text{FS}} = 20\log(36000 \times 10^3) + 20\log(30 \times 10^9) + 92.45 = 213.5\ \text{dB}
\]

\textbf{Clear-air atmospheric}:

\[
L_{\text{atm}} = 0.8 / \sin(30°) = 1.6\ \text{dB}
\]

\textbf{Rain fade} (99.9\%, temperate):

\[
L_{\text{rain}} = 13\ \text{dB}
\]

\textbf{Scintillation}:

\[
L_{\text{scint}} = 2\ \text{dB}
\]

\textbf{Total}:

\[
L_{\text{total}} = 213.5 + 1.6 + 13 + 2 = 230.1\ \text{dB}
\]

\textbf{Comparison}: Ka-band suffers \textbf{20 dB more loss} than
Ku-band (mostly rain!)

\begin{center}\rule{0.5\linewidth}{0.5pt}\end{center}

\subsection{Practical Considerations}\label{practical-considerations}

\subsubsection{Elevation Angle Matters}\label{elevation-angle-matters}

\textbf{Low elevation} (\textless{} 10\$\^{}\textbackslash circ\$): -
Longer path through troposphere - More atmospheric absorption - Worse
rain fade (factor of 2-3\$\textbackslash times\$ vs
30\$\^{}\textbackslash circ\$ elevation) - Higher scintillation

\textbf{Design guideline}: Avoid elevations \textless{}
10\$\^{}\textbackslash circ\$ if possible (especially for Ka-band+)

\begin{center}\rule{0.5\linewidth}{0.5pt}\end{center}

\subsubsection{Frequency Selection
Trade-offs}\label{frequency-selection-trade-offs}

{\def\LTcaptype{} % do not increment counter
\begin{longtable}[]{@{}
  >{\raggedright\arraybackslash}p{(\linewidth - 10\tabcolsep) * \real{0.1071}}
  >{\raggedright\arraybackslash}p{(\linewidth - 10\tabcolsep) * \real{0.1071}}
  >{\raggedright\arraybackslash}p{(\linewidth - 10\tabcolsep) * \real{0.2321}}
  >{\raggedright\arraybackslash}p{(\linewidth - 10\tabcolsep) * \real{0.1071}}
  >{\raggedright\arraybackslash}p{(\linewidth - 10\tabcolsep) * \real{0.2500}}
  >{\raggedright\arraybackslash}p{(\linewidth - 10\tabcolsep) * \real{0.1964}}@{}}
\toprule\noalign{}
\begin{minipage}[b]{\linewidth}\raggedright
Band
\end{minipage} & \begin{minipage}[b]{\linewidth}\raggedright
FSPL
\end{minipage} & \begin{minipage}[b]{\linewidth}\raggedright
Atmospheric
\end{minipage} & \begin{minipage}[b]{\linewidth}\raggedright
Rain
\end{minipage} & \begin{minipage}[b]{\linewidth}\raggedright
Antenna Size
\end{minipage} & \begin{minipage}[b]{\linewidth}\raggedright
Bandwidth
\end{minipage} \\
\midrule\noalign{}
\endhead
\bottomrule\noalign{}
\endlastfoot
\textbf{C (4-8 GHz)} & Low & Very low & \textbf{Very low} & Large &
Moderate \\
\textbf{Ku (12-18 GHz)} & Moderate & Low & \textbf{Moderate} & Moderate
& Good \\
\textbf{Ka (26.5-40 GHz)} & High & Moderate & \textbf{High} & Small &
Excellent \\
\textbf{V (40-75 GHz)} & Very high & High & \textbf{Very high} & Very
small & Huge \\
\end{longtable}
}

\textbf{Tropical regions}: C-band preferred (rain-robust, 99.99\%
availability achievable) \textbf{Temperate regions}: Ku-band good
compromise (rain manageable with margins) \textbf{Ka-band}: Requires
ACM, site diversity, or large margins

\begin{center}\rule{0.5\linewidth}{0.5pt}\end{center}

\subsubsection{Time-of-Day Effects}\label{time-of-day-effects}

\textbf{Ionosphere} (HF): - \textbf{Daytime}: Higher MUF, D-layer
absorption - \textbf{Nighttime}: Lower MUF, no D-layer, skywave active

\textbf{Troposphere} (VHF+): - \textbf{Daytime}: More turbulence
(scintillation), convective clouds (rain) - \textbf{Nighttime}: Calmer,
potential ducting (temperature inversions)

\textbf{GPS errors}: - \textbf{Noon}: Peak TEC, highest ionospheric
delay (\textasciitilde10-30m error) - \textbf{Midnight}: Minimum TEC,
lower error (\textasciitilde5-10m)

\begin{center}\rule{0.5\linewidth}{0.5pt}\end{center}

\subsection{Regional Variations}\label{regional-variations}

\subsubsection{Equatorial Regions}\label{equatorial-regions}

\textbf{Ionosphere}: - High TEC
(10\textbackslash textsuperscript\{1\}\textbackslash textsuperscript\{8\}
e\textbackslash textsuperscript\{-\}/m\textbackslash textsuperscript\{2\})
- Plasma bubbles (scintillation) - GPS errors
2-3\$\textbackslash times\$ worse than mid-latitudes

\textbf{Troposphere}: - High humidity (water vapor absorption) - Intense
rain (42-95 mm/hr)

\textbf{Recommendation}: C-band for satellite, robust GPS receivers

\begin{center}\rule{0.5\linewidth}{0.5pt}\end{center}

\subsubsection{High Latitudes}\label{high-latitudes}

\textbf{Ionosphere}: - Auroral activity (scintillation, blackouts) -
Solar proton events (polar cap absorption)

\textbf{Troposphere}: - Low humidity (less absorption) - Moderate rain

\textbf{Recommendation}: HF comms challenging during storms, but low
rain fade

\begin{center}\rule{0.5\linewidth}{0.5pt}\end{center}

\subsubsection{Temperate Mid-Latitudes}\label{temperate-mid-latitudes}

\textbf{Ionosphere}: - Moderate TEC - Stable conditions (less
scintillation)

\textbf{Troposphere}: - Seasonal variations (summer rain, winter
ducting) - Moderate humidity

\textbf{Recommendation}: Best overall conditions for
satellite/terrestrial

\begin{center}\rule{0.5\linewidth}{0.5pt}\end{center}

\subsection{Summary Table: Atmospheric Effects by
Frequency}\label{summary-table-atmospheric-effects-by-frequency}

{\def\LTcaptype{} % do not increment counter
\begin{longtable}[]{@{}
  >{\raggedright\arraybackslash}p{(\linewidth - 6\tabcolsep) * \real{0.2075}}
  >{\raggedright\arraybackslash}p{(\linewidth - 6\tabcolsep) * \real{0.2264}}
  >{\raggedright\arraybackslash}p{(\linewidth - 6\tabcolsep) * \real{0.2453}}
  >{\raggedright\arraybackslash}p{(\linewidth - 6\tabcolsep) * \real{0.3208}}@{}}
\toprule\noalign{}
\begin{minipage}[b]{\linewidth}\raggedright
Frequency
\end{minipage} & \begin{minipage}[b]{\linewidth}\raggedright
Ionosphere
\end{minipage} & \begin{minipage}[b]{\linewidth}\raggedright
Troposphere
\end{minipage} & \begin{minipage}[b]{\linewidth}\raggedright
Dominant Effect
\end{minipage} \\
\midrule\noalign{}
\endhead
\bottomrule\noalign{}
\endlastfoot
\textbf{LF/MF} & Absorbed by D-layer (day), reflected (night) &
Negligible & \textbf{Ionospheric absorption} \\
\textbf{HF (3-30 MHz)} & Sky wave (F2 reflection) & Negligible &
\textbf{Ionospheric refraction} \\
\textbf{VHF (30-300 MHz)} & Penetrates (no reflection) & Refraction,
ducting & \textbf{Tropospheric refraction} \\
\textbf{UHF (300-3000 MHz)} & Faraday rotation, delay & Minimal
absorption & \textbf{Ionospheric delay (GPS)} \\
\textbf{L/S (1-4 GHz)} & TEC delay (GPS error) & \textless{} 0.01 dB/km
& \textbf{Ionospheric scintillation} \\
\textbf{C (4-8 GHz)} & Negligible & 0.01 dB/km & \textbf{Rain fade
(minor)} \\
\textbf{Ku (12-18 GHz)} & Negligible & 0.05 dB/km & \textbf{Rain fade
(moderate)} \\
\textbf{Ka (26.5-40 GHz)} & Negligible & 0.1-0.3 dB/km & \textbf{Rain
fade (severe)} \\
\textbf{V (40-75 GHz)} & Negligible & 0.3-15 dB/km (60 GHz peak) &
\textbf{O\textbackslash textsubscript\{2\} absorption, rain} \\
\textbf{W (75-110 GHz)} & Negligible & 1-5 dB/km &
\textbf{H\textbackslash textsubscript\{2\}O absorption, rain} \\
\textbf{THz (\textgreater300 GHz)} & Negligible & 10-100+ dB/km &
\textbf{H\textbackslash textsubscript\{2\}O absorption (severe)} \\
\end{longtable}
}

\begin{center}\rule{0.5\linewidth}{0.5pt}\end{center}

\subsection{Related Topics}\label{related-topics}

\begin{itemize}
\tightlist
\item
  \textbf{{[}{[}Propagation-Modes-(Ground-Wave,-Sky-Wave,-Line-of-Sight){]}{]}}:
  How ionosphere enables HF skywave
\item
  \textbf{{[}{[}Weather-Effects-(Rain-Fade,-Fog-Attenuation){]}{]}}:
  Rain dominates at high frequencies
\item
  \textbf{{[}{[}Free-Space-Path-Loss-(FSPL){]}{]}}: Baseline loss before
  atmospheric effects
\item
  \textbf{{[}{[}Signal-to-Noise-Ratio-(SNR){]}{]}}: Atmospheric loss
  reduces SNR
\item
  \textbf{{[}{[}Electromagnetic-Spectrum{]}{]}}: Frequency-dependent
  atmospheric behavior
\end{itemize}

\begin{center}\rule{0.5\linewidth}{0.5pt}\end{center}

\textbf{Key takeaway}: \textbf{Ionosphere enables HF long-distance}
(refraction), but \textbf{disrupts GPS/satellite L-band} (delay,
scintillation). \textbf{Troposphere absorbs high frequencies}
(O\textbackslash textsubscript\{2\} @ 60 GHz,
H\textbackslash textsubscript\{2\}O @ 22 GHz), and \textbf{weather
dominates above 10 GHz} (rain fade). Choose frequency based on
application and climate.

\begin{center}\rule{0.5\linewidth}{0.5pt}\end{center}

\emph{This wiki is part of the {[}{[}Home\textbar Chimera Project{]}{]}
documentation.}
