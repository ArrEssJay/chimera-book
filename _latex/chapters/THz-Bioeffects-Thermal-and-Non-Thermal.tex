\section{THz Bioeffects: Thermal and
Non-Thermal}\label{thz-bioeffects-thermal-and-non-thermal}

{[}{[}Home{]}{]} \textbar{} {[}{[}Terahertz-(THz)-Technology{]}{]}
\textbar{} {[}{[}THz-Propagation-in-Biological-Tissue{]}{]} \textbar{}
{[}{[}THz-Resonances-in-Microtubules{]}{]}

\begin{center}\rule{0.5\linewidth}{0.5pt}\end{center}

\subsection{Overview}\label{overview}

Terahertz (THz) radiation (0.1-10 THz) interacts with biological systems
through \textbf{thermal} (heating) and potentially \textbf{non-thermal}
(resonant or quantum) mechanisms. Understanding these effects is
critical for: - \textbf{Safety standards}: Protecting workers and
patients from excessive exposure - \textbf{Therapeutic applications}:
Exploiting beneficial effects (if any) - \textbf{Fundamental
biophysics}: Understanding molecule-THz interactions

\textbf{Current consensus} : Thermal effects well-established;
non-thermal effects controversial.

\begin{center}\rule{0.5\linewidth}{0.5pt}\end{center}

\subsection{\texorpdfstring{For Non-Technical Readers
}{For Non-Technical Readers }}\label{for-non-technical-readers}

\textbf{What is terahertz radiation?}\\
Think of it as invisible light that sits between microwaves (used in
your microwave oven) and infrared (what you feel as heat from the sun).
It\textquotesingle s completely different from dangerous ionizing
radiation like X-rays-\/-\/-THz waves don\textquotesingle t have enough
energy to damage DNA or cause cancer directly.

\textbf{Why does this matter?}\\
THz technology is being developed for: - \textbf{Security scanning}
(airport body scanners) - \textbf{Medical imaging} (seeing skin cancer
without biopsies) - \textbf{Quality control} (checking pills for
defects) - \textbf{Communication} (future 6G wireless networks)

As these applications grow, we need to know: \emph{Is THz radiation
safe?}

\textbf{The Two Types of Effects:}

\begin{enumerate}
\def\labelenumi{\arabic{enumi}.}
\tightlist
\item
  \textbf{Thermal Effects (Heating)} -\/-\/- \textbf{Well Understood}

  \begin{itemize}
  \tightlist
  \item
    \textbf{What happens}: THz waves make water molecules jiggle faster,
    creating heat (like a microwave oven, but much weaker)
  \item
    \textbf{Is it dangerous?}: Only at high power. Safety guidelines
    keep exposure low enough that heating is less than
    1\$\^{}\textbackslash circ\$C-\/-\/-similar to walking from shade
    into sunlight
  \item
    \textbf{Analogy}: Standing near a campfire. Get too close = you feel
    heat and can get burned. Stay at a safe distance = perfectly fine
  \item
    \textbf{Current safety standards}: Designed to prevent any
    significant heating
  \end{itemize}
\item
  \textbf{Non-Thermal Effects (Molecular Resonance?)} -\/-\/-
  \textbf{Controversial and Unproven}

  \begin{itemize}
  \tightlist
  \item
    \textbf{What\textquotesingle s claimed}: Some scientists hypothesize
    that THz might affect biology \emph{without} heating-\/-\/-by
    vibrating specific molecules at their ``natural frequencies'' (like
    shattering a wine glass with sound)
  \item
    \textbf{Why it\textquotesingle s controversial}:

    \begin{itemize}
    \tightlist
    \item
      Hard to prove the effects aren\textquotesingle t just from tiny
      amounts of heating we can\textquotesingle t measure
    \item
      Many studies can\textquotesingle t be replicated by other labs
    \item
      No agreed-upon mechanism for how it would work
    \end{itemize}
  \item
    \textbf{Analogy}: Imagine claiming a dog whistle (which humans
    can\textquotesingle t hear) gives you headaches. Is it the sound
    frequency, or stress from thinking about it? Hard to prove.
  \item
    \textbf{Current consensus}: Most scientists are skeptical; safety
    standards ignore non-thermal effects because evidence is weak
  \end{itemize}
\end{enumerate}

\textbf{Should I worry about THz exposure?}\\
\textbf{No, if exposure is within guidelines.} Current safety limits are
conservative-\/-\/-they\textquotesingle re set well below levels that
cause heating. It\textquotesingle s like speed limits: if everyone
follows them, the risk is minimal.

\textbf{What about long-term effects?}\\
We don\textquotesingle t have 50-year studies yet (THz tech is
relatively new), but: - No mechanism for cancer (THz photons are too
weak to break chemical bonds) - No evidence of cumulative damage in
animal studies - Similar to concerns about cell phones 20 years
ago-\/-\/-still being studied, but no confirmed harm at safe levels

\textbf{The Bottom Line:}\\
THz technology is probably safe at low power, but research continues.
The document below dives into the science for those who want details.

\begin{center}\rule{0.5\linewidth}{0.5pt}\end{center}

\subsection{1. Thermal Effects (Established
)}\label{thermal-effects-established}

\subsubsection{1.1 Absorption and Heating}\label{absorption-and-heating}

\textbf{Mechanism}: THz radiation absorbed by tissue
\$\textbackslash rightarrow\$ molecular kinetic energy
\$\textbackslash rightarrow\$ temperature rise

\textbf{Governing equation} (heat diffusion):
\[\rho c_p \frac{\partial T}{\partial t} = \nabla \cdot (k \nabla T) + Q\]
where: - \(\rho\): Tissue density (\textasciitilde1
g/cm\textbackslash textsuperscript\{3\}) - \(c_p\): Specific heat
capacity (\textasciitilde3.6 J/g/K for tissue) - \(k\): Thermal
conductivity (\textasciitilde0.5 W/m/K) - \(Q\): Heat source =
\(\alpha I\) (absorption coefficient \$\textbackslash times\$ intensity)

\textbf{Temperature rise} (steady-state, no blood flow):
\[\Delta T \approx \frac{\alpha I \delta^2}{k}\] where
\(\delta = 1/\alpha\) is penetration depth.

\textbf{Example}: 1 W/cm\textbackslash textsuperscript\{2\} at 1 THz,
\(\alpha = 200\)
cm\textbackslash textsuperscript\{-\}\textbackslash textsuperscript\{1\},
\(\delta = 50\) \$\textbackslash mu\$m:
\[\Delta T \approx \frac{200 \times 10^4 \times 10^{-6} \times (50 \times 10^{-6})^2}{0.005} \approx 1^\circ\text{C}\]

\textbf{Safety threshold}: \(\Delta T < 1^\circ\)C for prolonged
exposure (ICNIRP guideline)

\subsubsection{1.2 Depth Dependence}\label{depth-dependence}

\textbf{Shallow heating}: THz absorption strongest at surface
\$\textbackslash rightarrow\$ temperature peak at skin surface

\textbf{Thermal diffusion time}:
\[\tau_{\text{th}} = \frac{L^2}{\kappa}\] where
\(\kappa = k/(\rho c_p)\) is thermal diffusivity (\textasciitilde1.3
\$\textbackslash times\$
10\textbackslash textsuperscript\{-\}\textbackslash textsuperscript\{3\}
cm\textbackslash textsuperscript\{2\}/s for tissue).

For \(L = 100\) \$\textbackslash mu\$m: \(\tau_{\text{th}} \approx 0.1\)
s (heat dissipates quickly)

\textbf{Pulsed exposure}: Short pulses (\textless1
\$\textbackslash mu\$s) create transient temperature spikes that relax
before tissue damage.

\subsubsection{1.3 Biological Consequences of
Heating}\label{biological-consequences-of-heating}

\textbf{Mild heating} (1-2\$\^{}\textbackslash circ\$C): - Increased
metabolic rate - Altered enzyme kinetics - Enhanced blood flow
(vasodilation)

\textbf{Moderate heating} (5-10\$\^{}\textbackslash circ\$C): - Protein
denaturation (irreversible above
\textasciitilde50\$\^{}\textbackslash circ\$C) - Cell membrane
disruption - Apoptosis (programmed cell death)

\textbf{Severe heating} (\textgreater20\$\^{}\textbackslash circ\$C): -
Tissue ablation - Burns

\textbf{Threshold for damage}:
\textasciitilde43\$\^{}\textbackslash circ\$C for prolonged exposure
(\textgreater1 hour) \$\textbackslash rightarrow\$ cumulative equivalent
minutes (CEM43)

\begin{center}\rule{0.5\linewidth}{0.5pt}\end{center}

\subsection{2. Non-Thermal Effects (Speculative
)}\label{non-thermal-effects-speculative}

\subsubsection{2.1 Definition}\label{definition}

\textbf{Non-thermal effect}: Biological response that occurs at
intensities too low to cause measurable heating
(\(\Delta T < 0.1^\circ\)C) OR that persists after heating stops.

\textbf{Challenge}: Distinguishing non-thermal from: - \textbf{Localized
heating}: Hot spots due to field enhancement - \textbf{Transient
heating}: Temporary temperature spikes below detection threshold -
\textbf{Indirect thermal effects}: Heat-activated signaling cascades

\subsubsection{2.2 Proposed Mechanisms}\label{proposed-mechanisms}

\paragraph{2.2.1 Resonant Absorption by
Biomolecules}\label{resonant-absorption-by-biomolecules}

\textbf{Hypothesis}: THz frequencies match vibrational modes of
proteins, DNA, or membranes \$\textbackslash rightarrow\$ selective
excitation.

\textbf{Evidence}: - Proteins have collective vibrational modes at 0.1-3
THz (low-frequency Raman, THz-TDS) - DNA backbone vibrations at
\textasciitilde1 THz (B-form helix breathing modes)

\textbf{Problem}: In solution, these modes are heavily broadened
(lifetime \textasciitilde ps) \$\textbackslash rightarrow\$ weak
resonance peak. Excitation is non-selective.

\textbf{Counterpoint}: \emph{In vitro} studies show altered protein
function at sub-thermal intensities (see Section 3.1)

\paragraph{2.2.2 Membrane
Electroporation}\label{membrane-electroporation}

\textbf{Hypothesis}: THz electric fields induce transmembrane voltage
\$\textbackslash rightarrow\$ pore formation.

\textbf{Induced voltage}: \[V_m = 1.5 r E \cos\theta\] where \(r\) is
cell radius, \(E\) is external field, \(\theta\) is angle.

For \(r = 10\) \$\textbackslash mu\$m, \(E = 10\) kV/cm:
\(V_m \approx 15\) mV (below electroporation threshold \textasciitilde1
V)

\textbf{Conclusion}: Unlikely at THz (frequency too high for membrane
charging; shielded by ionic double layer)

\paragraph{2.2.3 Microtubule Resonances}\label{microtubule-resonances}

\textbf{Hypothesis}: THz resonates with microtubule vibrational modes
\$\textbackslash rightarrow\$ alters quantum coherence
\$\textbackslash rightarrow\$ affects neural function (see
{[}{[}THz-Resonances-in-Microtubules{]}{]}).

\textbf{Predicted frequencies}: 0.5-10 THz (acoustic phonons, optical
phonons)

\textbf{Quantum mechanism}: Vibronic coupling (electron-phonon) sustains
coherence at 310 K; THz drives transitions between vibronic states.

\textbf{Status}: No direct experimental test; theoretical models exist
but lack validation.

\paragraph{2.2.4 Water Structuring}\label{water-structuring}

\textbf{Hypothesis}: THz alters hydrogen bond network dynamics in
vicinal water (near protein/membrane surfaces)
\$\textbackslash rightarrow\$ affects protein function.

\textbf{Mechanism}: THz drives librational modes (hindered rotations)
\$\textbackslash rightarrow\$ transiently disrupts H-bond network
\$\textbackslash rightarrow\$ lowers activation barrier for
conformational changes.

\textbf{Evidence}: Simulations suggest THz can perturb water structure
on \textasciitilde ps timescales; biological relevance unclear.

\begin{center}\rule{0.5\linewidth}{0.5pt}\end{center}

\subsection{3. Experimental Evidence}\label{experimental-evidence}

\subsubsection{3.1 Cell-Level Studies}\label{cell-level-studies}

\textbf{Gene expression} : - \textbf{Observation}: Altered mRNA levels
after THz exposure (0.1-2.5 THz, \textless1
mW/cm\textbackslash textsuperscript\{2\},
\textless1\$\^{}\textbackslash circ\$C heating) - \textbf{Example}:
Upregulation of heat shock proteins (HSP70) in human keratinocytes
(Wilmink et al., 2010) - \textbf{Interpretation}: Could be indirect
thermal effect (transient microheating) OR non-thermal stress response

\textbf{Membrane permeability} : - \textbf{Observation}: Increased
uptake of fluorescent dyes after THz pulse exposure (Bock et al., 2010)
- \textbf{Interpretation}: Pore formation? Or thermal disruption? -
\textbf{Control needed}: Measure temperature with high spatial/temporal
resolution

\textbf{Calcium signaling} : - \textbf{Observation}: Transient
Ca\textbackslash textsuperscript\{2\}\textbackslash textsuperscript\{+\}
influx in neurons after THz exposure (Zhao et al., 2019) -
\textbf{Mechanism}: THz-sensitive ion channels? Or indirect heating? -
\textbf{Problem}: Calcium-sensitive dyes themselves have temperature
dependence

\subsubsection{3.2 Protein Studies}\label{protein-studies}

\textbf{Enzyme activity} (thermal) / (non-thermal?): -
\textbf{Observation}: Altered kinetics of lysozyme, alkaline phosphatase
at sub-thermal intensities (Cherkasova et al., 2009) -
\textbf{Interpretation}: Possible resonant excitation of active site
modes; but thermal artifacts not fully ruled out

\textbf{Protein unfolding} (thermal): - Clear correlation with
temperature; follows Arrhenius kinetics

\subsubsection{3.3 DNA Studies}\label{dna-studies}

\textbf{Strand breaks} (thermal at high intensity): - Observed at
\textgreater100 W/cm\textbackslash textsuperscript\{2\} (ablation
regime); clearly thermal

\textbf{Transcription} : - \emph{In vitro} transcription assays: Some
studies report altered transcription rates at \textless1
W/cm\textbackslash textsuperscript\{2\} - \textbf{Problem}: DNA
polymerase highly temperature-sensitive; even
0.1\$\^{}\textbackslash circ\$C affects rate

\subsubsection{3.4 Whole-Animal Studies}\label{whole-animal-studies}

\textbf{Developmental effects} : - \textbf{Zebrafish embryos}: Some
studies report abnormal development after THz exposure (Titova et al.,
2013) - \textbf{Confounding factors}: Dehydration, handling stress,
temperature gradients in aquarium

\textbf{Behavioral effects} : - \textbf{Mice}: No consistent behavioral
changes at sub-thermal intensities - \textbf{Drosophila}: Some reports
of altered locomotion; not reproduced independently

\textbf{Conclusion}: No robust, reproducible whole-animal non-thermal
effects demonstrated.

\begin{center}\rule{0.5\linewidth}{0.5pt}\end{center}

\subsection{4. Critical Analysis: Are Non-Thermal Effects
Real?}\label{critical-analysis-are-non-thermal-effects-real}

\subsubsection{\texorpdfstring{4.1 Arguments For
}{4.1 Arguments For }}\label{arguments-for}

\begin{enumerate}
\def\labelenumi{\arabic{enumi}.}
\tightlist
\item
  \textbf{Molecular resonances exist}: Proteins, DNA have THz
  vibrational modes
\item
  \textbf{Some cellular effects at low intensity}: Not all studies show
  strict temperature correlation
\item
  \textbf{Precedent in other bands}: RF/microwave ``non-thermal
  effects'' debated for decades
\end{enumerate}

\subsubsection{\texorpdfstring{4.2 Arguments Against
}{4.2 Arguments Against }}\label{arguments-against}

\begin{enumerate}
\def\labelenumi{\arabic{enumi}.}
\tightlist
\item
  \textbf{No consensus mechanism}: Multiple proposed mechanisms, none
  with strong evidence
\item
  \textbf{Reproducibility issues}: Many studies lack independent
  replication
\item
  \textbf{Thermal artifacts}: Hard to rule out localized or transient
  heating
\item
  \textbf{Lack of dose-response}: No clear threshold or saturation
  behavior for ``non-thermal'' effects
\item
  \textbf{Evolutionary perspective}: If THz resonances were functionally
  important, natural selection would have exploited or shielded them
\end{enumerate}

\subsubsection{4.3 Current Scientific
Consensus}\label{current-scientific-consensus}

\textbf{ICNIRP position} (2013): ``There is no consistent evidence for
non-thermal effects at intensities below thermal damage thresholds.''

\textbf{WHO position}: THz safety guidelines based on thermal effects
only.

\textbf{Research community}: Divided; ongoing studies but skepticism
high.

\begin{center}\rule{0.5\linewidth}{0.5pt}\end{center}

\subsection{5. Safety Standards}\label{safety-standards}

\subsubsection{5.1 ICNIRP Guidelines
(2013)}\label{icnirp-guidelines-2013}

\textbf{Frequency range}: 0.3-3 THz

\textbf{Power density limits}: - \textbf{Occupational exposure}: 10
mW/cm\textbackslash textsuperscript\{2\} (averaged over
68/f\textbackslash textsuperscript\{1\}\$\textbackslash cdot\$\textbackslash textsuperscript\{0\}\textbackslash textsuperscript\{5\}
minutes, \(f\) in THz) - \textbf{General public exposure}: 2
mW/cm\textbackslash textsuperscript\{2\} (same averaging)

\textbf{Rationale}: Keep \(\Delta T < 1^\circ\)C

\subsubsection{5.2 IEEE Standards
(C95.1-2019)}\label{ieee-standards-c95.1-2019}

\textbf{Similar limits}: \textasciitilde10
mW/cm\textbackslash textsuperscript\{2\} for controlled environments

\textbf{Frequency gaps}: Standards less developed for 3-10 THz (far-IR
overlap)

\subsubsection{5.3 Medical Device
Regulations}\label{medical-device-regulations}

\textbf{THz imaging systems}: Require FDA clearance (USA) or CE mark
(EU)

\textbf{Approval criteria}: - Demonstrate temperature rise
\textless1\$\^{}\textbackslash circ\$C in vivo - No evidence of
long-term effects (mutagenicity, carcinogenicity)

\begin{center}\rule{0.5\linewidth}{0.5pt}\end{center}

\subsection{6. Therapeutic Potential (Speculative
)}\label{therapeutic-potential-speculative}

\subsubsection{6.1 THz-Induced
Neuromodulation}\label{thz-induced-neuromodulation}

\textbf{Hypothesis}: THz pulses could activate neurons non-invasively.

\textbf{Mechanisms} (proposed): - \textbf{TRPV channels}:
Temperature-sensitive ion channels activated by localized heating -
\textbf{Microtubule resonances}: Quantum effects alter neuronal
excitability

\textbf{Challenges}: Penetration (THz doesn\textquotesingle t reach deep
brain), specificity (heating is non-selective)

\subsubsection{6.2 Cancer Therapy}\label{cancer-therapy}

\textbf{Hypothesis}: Cancer cells more sensitive to THz due to altered
water content or membrane properties.

\textbf{Evidence}: Minimal; no clinical trials

\textbf{Alternative}: THz imaging for cancer detection (established)
vs.~THz ablation (speculative)

\subsubsection{6.3 Wound Healing}\label{wound-healing}

\textbf{Hypothesis}: Low-intensity THz stimulates cell proliferation.

\textbf{Evidence}: \emph{In vitro} studies show increased fibroblast
migration at \textless1 mW/cm\textbackslash textsuperscript\{2\};
mechanism unknown.

\begin{center}\rule{0.5\linewidth}{0.5pt}\end{center}

\subsection{7. Future Directions}\label{future-directions}

\subsubsection{7.1 What Experiments Are
Needed?}\label{what-experiments-are-needed}

\textbf{To prove non-thermal effects exist}: 1. \textbf{High-resolution
thermometry}: Measure temperature with
\$\textbackslash pm\$0.01\$\^{}\textbackslash circ\$C accuracy,
\textless10 \$\textbackslash mu\$m spatial resolution 2. \textbf{Isotope
substitution}: Deuterate proteins (H \$\textbackslash rightarrow\$ D
shifts vibrational modes); predict frequency-dependent effects 3.
\textbf{Molecular dynamics simulations}: Model THz-biomolecule
interactions at atomic resolution 4. \textbf{Dose-response curves}:
Establish clear thresholds and saturation 5. \textbf{Blind studies}:
Eliminate experimenter bias

\textbf{To understand thermal effects better}: 1. \textbf{Pulsed vs.~CW
comparison}: Do transient spikes matter more than average temperature?
2. \textbf{Tissue-specific thresholds}: Map safe exposure limits for
skin, eye, brain

\subsubsection{7.2 Proposed Mechanisms to
Test}\label{proposed-mechanisms-to-test}

\begin{itemize}
\tightlist
\item
  \textbf{Vibronic coupling in microtubules}: Measure quantum variance
  (see {[}{[}Quantum-Coherence-in-Biological-Systems{]}{]}); test if THz
  modulates coherence time
\item
  \textbf{Water structuring}: Time-resolved spectroscopy of vicinal
  water during THz exposure
\item
  \textbf{Resonant protein excitation}: Site-directed mutagenesis to
  shift vibrational frequencies; predict altered THz sensitivity
\end{itemize}

\begin{center}\rule{0.5\linewidth}{0.5pt}\end{center}

\subsection{8. Connections to Other Wiki
Pages}\label{connections-to-other-wiki-pages}

\begin{itemize}
\tightlist
\item
  {[}{[}THz-Propagation-in-Biological-Tissue{]}{]} -\/-\/- Absorption
  and penetration depth
\item
  {[}{[}THz-Resonances-in-Microtubules{]}{]} -\/-\/- Speculative quantum
  mechanism
\item
  {[}{[}Terahertz-(THz)-Technology{]}{]} -\/-\/- Sources and detectors
\item
  {[}{[}Quantum-Coherence-in-Biological-Systems{]}{]} -\/-\/-
  Theoretical framework for non-thermal effects
\item
  {[}{[}Frey-Microwave-Auditory-Effect{]}{]} -\/-\/- Analogous RF
  non-thermal effect (pulsed microwaves \$\textbackslash rightarrow\$
  auditory perception)
\end{itemize}

\begin{center}\rule{0.5\linewidth}{0.5pt}\end{center}

\subsection{9. References}\label{references}

\subsubsection{Thermal Effects
(Established)}\label{thermal-effects-established-1}

\begin{enumerate}
\def\labelenumi{\arabic{enumi}.}
\tightlist
\item
  \textbf{ICNIRP, \emph{Health Phys.} 105, 171 (2013)} -\/-\/- THz
  exposure guidelines
\item
  \textbf{Pickwell \& Wallace, \emph{J. Phys. D} 39, R301 (2006)}
  -\/-\/- THz-tissue interactions
\end{enumerate}

\subsubsection{Non-Thermal Effects
(Speculative)}\label{non-thermal-effects-speculative-1}

\begin{enumerate}
\def\labelenumi{\arabic{enumi}.}
\setcounter{enumi}{2}
\tightlist
\item
  \textbf{Wilmink et al., \emph{J. Infrared Millim. THz Waves} 31, 1234
  (2010)} -\/-\/- Gene expression changes
\item
  \textbf{Titova et al., \emph{Sci. Rep.} 3, 2363 (2013)} -\/-\/-
  Zebrafish developmental effects
\item
  \textbf{Zhao et al., \emph{Neurophotonics} 6, 011004 (2019)} -\/-\/-
  Calcium signaling in neurons
\end{enumerate}

\subsubsection{Critical Reviews}\label{critical-reviews}

\begin{enumerate}
\def\labelenumi{\arabic{enumi}.}
\setcounter{enumi}{5}
\tightlist
\item
  \textbf{Alexandrov et al., \emph{Phys. Lett. A} 374, 1214 (2010)}
  -\/-\/- DNA resonances (controversial)
\item
  \textbf{Foster, \emph{Radiat. Res.} 162, 492 (2004)} -\/-\/- Critique
  of non-thermal RF/THz effects
\end{enumerate}

\subsubsection{Vibronic Coupling}\label{vibronic-coupling}

\begin{enumerate}
\def\labelenumi{\arabic{enumi}.}
\setcounter{enumi}{7}
\tightlist
\item
  \textbf{Bao et al., \emph{J. Chem. Theory Comput.} 20, 4377 (2024)}
  -\/-\/- VE-TFCC theory (thermal coherence)
\end{enumerate}

\begin{center}\rule{0.5\linewidth}{0.5pt}\end{center}

\textbf{Last updated}: October 2025
