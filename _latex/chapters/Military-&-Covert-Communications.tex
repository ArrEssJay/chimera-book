\section{Military \& Covert
Communications}\label{military-covert-communications}

Military communications systems prioritize \textbf{anti-jamming (AJ)},
\textbf{low probability of intercept (LPI)}, \textbf{low probability of
detection (LPD)}, and \textbf{secure transmission (TRANSEC)} over
commercial metrics like spectral efficiency. This page covers advanced
techniques used in GPS M-code, SATCOM FHSS, phased-array radar, Link 16,
and covert communications.

\begin{center}\rule{0.5\linewidth}{0.5pt}\end{center}

\subsection{\texorpdfstring{ For Non-Technical
Readers}{ For Non-Technical Readers}}\label{for-non-technical-readers}

Before diving into the technical details, here are the core concepts
explained in everyday terms:

\subsubsection{Why Military Communications Are
Different}\label{why-military-communications-are-different}

\textbf{Imagine trying to have a conversation in a crowded, hostile
room} where: - Someone is shouting over you (jamming) - Others are
listening to steal your secrets (interception) - You need to talk
without being noticed (covert)

Military communications solve these problems in ways regular WiFi or
cell phones don\textquotesingle t need to.

\begin{center}\rule{0.5\linewidth}{0.5pt}\end{center}

\subsubsection{The ``Whisper in the Crowd''
Analogy}\label{the-whisper-in-the-crowd-analogy}

\textbf{Spread Spectrum} = Speaking very quietly across a huge room

Instead of shouting one clear message, you: 1. \textbf{Whisper
fragments} of your message to many places at once 2. \textbf{Spread it
so thin} that each piece sounds like background noise 3. \textbf{Only
someone with the secret pattern} can collect all the pieces and
understand you

\textbf{Real-world result}: Your signal is literally \textbf{weaker than
background noise}, yet your intended receiver hears it perfectly.
Enemies just hear static.

\textbf{Example}: GPS M-code is 20 times weaker than the noise floor,
yet your military GPS receiver locks on instantly. A
spy\textquotesingle s receiver? Just noise.

\begin{center}\rule{0.5\linewidth}{0.5pt}\end{center}

\subsubsection{The ``Concert Hall Spotlight''
Analogy}\label{the-concert-hall-spotlight-analogy}

\textbf{Phased-Array Antennas (AESA)} = Pointing a beam of radio energy

Think of a traditional dish antenna like a flashlight-\/-\/-it points
one direction, and moving it takes time.

\textbf{AESA is like a concert hall\textquotesingle s lighting system}:
- \textbf{Hundreds of tiny lights} (antenna elements) -
\textbf{Computer-controlled} to turn on/off in precise patterns -
\textbf{Creates a ``spotlight''} that can instantly jump to different
parts of the room - \textbf{Multiple spotlights} can exist
simultaneously (track many targets)

\textbf{Real-world result}: F-22 radar can track 100 aircraft, jam enemy
radars, and guide missiles-\/-\/-all at once, all electronically, no
moving parts.

\begin{center}\rule{0.5\linewidth}{0.5pt}\end{center}

\subsubsection{The ``Secret Handshake''
Analogy}\label{the-secret-handshake-analogy}

\textbf{Frequency Hopping} = Changing radio channels thousands of times
per second

Imagine a conversation where: 1. You and your friend \textbf{agree on a
secret pattern} of which channel to use when 2. Every millisecond, you
both \textbf{jump to a new frequency} following the pattern 3.
\textbf{Enemies can\textquotesingle t follow} because they
don\textquotesingle t know the pattern 4. Even if they jam one
frequency, you\textquotesingle re already gone

\textbf{Real-world result}: Military satellite phones (MILSTAR) hop
1000+ times per second across a gigahertz of spectrum. A jammer would
need to jam the entire band with megawatts of power-\/-\/-impractical.

\begin{center}\rule{0.5\linewidth}{0.5pt}\end{center}

\subsubsection{The ``Invisible Ink''
Analogy}\label{the-invisible-ink-analogy}

\textbf{Low Probability of Detection (LPD)} = Radio signals that
don\textquotesingle t look like signals

Imagine hiding a message by: 1. \textbf{Writing each letter} on a
separate grain of sand 2. \textbf{Scattering the sand} across a beach 3.
\textbf{Only the recipient} knows which grains to collect

\textbf{Real-world result}: Covert radios transmit at power levels
1000\$\textbackslash times\$ below what receivers normally detect. Even
sensitive spy equipment can\textquotesingle t tell the difference
between the transmission and natural radio noise.

\begin{center}\rule{0.5\linewidth}{0.5pt}\end{center}

\subsubsection{The ``Smart Echo'' Analogy}\label{the-smart-echo-analogy}

\textbf{Anti-Jamming (AJ)} = Fighting back against interference

When an enemy tries to jam your signal: 1. \textbf{Your antenna
``learns''} where the jammer is located 2. \textbf{Creates a ``null''}
(deaf spot) pointing at the jammer 3. \textbf{Amplifies signals} from
your intended direction

Think of it like \textbf{noise-canceling headphones for
radio}-\/-\/-specifically canceling out the jammer while hearing your
friend.

\textbf{Real-world result}: GPS receivers with anti-jam antennas (CRPA)
can reject jammers that are 10,000\$\textbackslash times\$ stronger than
the GPS satellite signal.

\begin{center}\rule{0.5\linewidth}{0.5pt}\end{center}

\subsubsection{Key Concepts Simplified}\label{key-concepts-simplified}

{\def\LTcaptype{} % do not increment counter
\begin{longtable}[]{@{}
  >{\raggedright\arraybackslash}p{(\linewidth - 4\tabcolsep) * \real{0.2000}}
  >{\raggedright\arraybackslash}p{(\linewidth - 4\tabcolsep) * \real{0.4000}}
  >{\raggedright\arraybackslash}p{(\linewidth - 4\tabcolsep) * \real{0.4000}}@{}}
\toprule\noalign{}
\begin{minipage}[b]{\linewidth}\raggedright
Concept
\end{minipage} & \begin{minipage}[b]{\linewidth}\raggedright
Everyday Analogy
\end{minipage} & \begin{minipage}[b]{\linewidth}\raggedright
Military Benefit
\end{minipage} \\
\midrule\noalign{}
\endhead
\bottomrule\noalign{}
\endlastfoot
\textbf{Spread Spectrum} & Whisper spread across a huge room & Signal
hidden below noise floor \\
\textbf{Processing Gain} & Collecting 1000 whispers back into speech &
Overcomes jammers and noise \\
\textbf{Frequency Hopping} & Changing channels
1000\$\textbackslash times\$ per second & Enemy can\textquotesingle t
follow or jam \\
\textbf{Phased Array} & Computer-controlled spotlight & Instant beam
steering, multi-target \\
\textbf{Encryption} & Secret language only you and friend know & Even
intercepted messages are useless \\
\textbf{Beamforming} & Talking through a directional megaphone & Only
intended receiver hears you \\
\end{longtable}
}

\begin{center}\rule{0.5\linewidth}{0.5pt}\end{center}

\subsubsection{Why This Matters for
Chimera}\label{why-this-matters-for-chimera}

Chimera helps you \textbf{visualize and experiment} with these concepts:
- \textbf{Build spread spectrum systems} in your browser - \textbf{See
jamming resistance} in real-time plots - \textbf{Experiment with
frequency hopping} patterns - \textbf{Understand phased arrays} through
interactive simulations

\textbf{You don\textquotesingle t need a million-dollar
lab}-\/-\/-Chimera brings military-grade DSP concepts to anyone with
curiosity and a web browser.

\begin{center}\rule{0.5\linewidth}{0.5pt}\end{center}

\subsubsection{What You\textquotesingle ll Learn in This
Document}\label{what-youll-learn-in-this-document}

The sections below explain \textbf{how these systems actually work}: -
\textbf{SATCOM Frequency Hopping}: How military satellites resist
jamming - \textbf{GPS M-Code}: Why military GPS works when civilian GPS
is jammed - \textbf{Phased-Array Radar}: How F-22s and destroyers
``see'' electronically - \textbf{Link 16}: The tactical data network
connecting planes, ships, and missiles - \textbf{Covert Communications}:
How to transmit data invisibly

\textbf{If you\textquotesingle re new to DSP}: Start with
{[}{[}Spread-Spectrum-(DSSS-FHSS){]}{]} for foundational concepts, then
return here.

\textbf{If you\textquotesingle re experienced}: Skip to the technical
sections-\/-\/-detailed math, code examples, and real-world system specs
await.

\begin{center}\rule{0.5\linewidth}{0.5pt}\end{center}

\subsection{\texorpdfstring{ Core Military
Requirements}{ Core Military Requirements}}\label{core-military-requirements}

\subsubsection{The LPI/LPD/AJ Triad}\label{the-lpilpdaj-triad}

\textbf{1. Low Probability of Intercept (LPI)}:

\begin{verbatim}
Enemy can detect transmission but cannot decode it

Techniques:
- Spread spectrum (DSSS/FHSS)  signal below noise floor
- Directional antennas  narrow beamwidths
- Burst transmissions  short dwell time
- Encryption  content secure even if intercepted
\end{verbatim}

\textbf{2. Low Probability of Detection (LPD)}:

\begin{verbatim}
Enemy cannot detect that transmission is occurring

Techniques:
- Ultra-wideband spread spectrum (G > 30 dB)
- Frequency diversity  avoid surveillance bands
- Power management  minimal radiated power
- Emission control (EMCON)  radio silence protocols
\end{verbatim}

\textbf{3. Anti-Jamming (AJ)}:

\begin{verbatim}
Maintain link under deliberate enemy interference

Techniques:
- Processing gain  overcomes jammer power
- Nulling antennas  reject jammer direction
- Frequency hopping  avoid narrowband jamming
- Adaptive filters  real-time interference cancellation
\end{verbatim}

\textbf{Relationship}:

\begin{verbatim}
Processing Gain (G) enables all three:

G = BW_spread / BW_info = Chip_Rate / Bit_Rate

Higher G  Lower PSD  Harder to detect/intercept/jam
\end{verbatim}

\begin{center}\rule{0.5\linewidth}{0.5pt}\end{center}

\subsection{\texorpdfstring{ SATCOM Frequency Hopping
(FHSS)}{ SATCOM Frequency Hopping (FHSS)}}\label{satcom-frequency-hopping-fhss}

Military satellite communications use FHSS for TRANSEC (transmission
security).

\subsubsection{X-Band MILSTAR/MUOS
Systems}\label{x-band-milstarmuos-systems}

\textbf{MILSTAR (Military Strategic and Tactical Relay)}:

\begin{verbatim}
Frequency: X-band uplink (7-8 GHz), Ka-band downlink (20-21 GHz)
Hop rate: 100-1000+ hops/second
Hop set: 1000+ frequencies across 1 GHz bandwidth
Dwell time: <1 ms per hop
Modulation: BPSK, QPSK, 8-PSK (adaptive)
Data rate: 75 bps - 1.544 Mbps (T1)
Satellite constellation: 5 GEO satellites (global coverage)

TRANSEC:
- Hopping pattern: Cryptographically generated (NSA algorithm)
- Synchronization: GPS time + KEK (Key Encryption Key)
- Pattern period: Days to weeks (never repeats observably)
- Anti-spoofing: Authenticated hop sequence
\end{verbatim}

\textbf{LPI/LPD characteristics}:

\begin{verbatim}
Power spectral density (PSD):
PSD = P_TX / BW_hop_set
    = 100 W / 1 GHz
    = 0.1 mW/MHz
     -70 dBm/MHz (at satellite, 40,000 km away)

Compare to thermal noise floor:
Noise = -174 dBm/Hz + 10·log(BW) = -114 dBm/MHz (1 MHz BW)

PSD_signal < Noise  Undetectable to wideband receiver!

Detectability only with:
- Exact hopping pattern (requires key)
- Synchronized receiver (requires network access)
- Correct modulation/demodulation (requires ICD)
\end{verbatim}

\begin{center}\rule{0.5\linewidth}{0.5pt}\end{center}

\textbf{MUOS (Mobile User Objective System)}:

\begin{verbatim}
Frequency: UHF uplink (300-318 MHz), UHF downlink (243-318 MHz)
Waveform: WCDMA (Wideband CDMA) + FHSS hybrid
Hop rate: Classified (estimated >500 hps)
Data rate: Up to 64 kbps voice, 10 Mbps data
Compatibility: Legacy UFO (Ultra High Frequency Follow-On)

Key features:
- Smartphone-like interface for warfighters
- Near-global coverage (5 GEO + legacy satellites)
- Jam-resistant waveform (60+ dB margin)
- Integrated encryption (Type 1 NSA)
\end{verbatim}

\begin{center}\rule{0.5\linewidth}{0.5pt}\end{center}

\subsubsection{FHSS Anti-Jam
Performance}\label{fhss-anti-jam-performance}

\textbf{Processing gain calculation}:

\begin{verbatim}
G(dB) = 10·log(Hop_Set_Size)

Example (MILSTAR):
- Hop set: 1000 frequencies
- G = 10·log(1000) = 30 dB

Jamming margin:
Margin = G - J/S - (Eb/N)_req - Losses

J/S = Jammer power / Signal power (at receiver)

Scenario:
- G = 30 dB
- J/S = 40 dB (jammer 10,000× stronger!)
- (Eb/N)_req = 10 dB (BPSK, BER = 10)
- Losses = 3 dB (implementation)

Margin = 30 - 40 - 10 - 3 = -23 dB  **LINK FAILS**

Countermeasures:
1. Directional antenna: +20 dB gain toward satellite, nulls toward jammer
   Effective J/S = 40 - 20 = 20 dB
   Margin = 30 - 20 - 10 - 3 = -3 dB  **MARGINAL**

2. Error-correction coding: Turbo/LDPC code rate-1/3
   Coding gain: +5 dB
   Margin = -3 + 5 = 2 dB  **LINK SURVIVES**

3. Burst transmission: Transmit 10× faster, listen 90% of time
   Jammer must hit exact burst time  effective J/S reduces by 10 dB
\end{verbatim}

\begin{center}\rule{0.5\linewidth}{0.5pt}\end{center}

\subsubsection{Follower Jamming
Resistance}\label{follower-jamming-resistance}

\textbf{Threat}: Smart jammer detects hop, jams that frequency.

\textbf{Timing analysis}:

\begin{verbatim}
Dwell time: 1 ms (MILSTAR)
Jammer detection: 100 s (fast energy detector)
Frequency switching: 50 s (agile synthesizer)

Total jammer delay: 150 s

Effective jam time: 1 ms - 150 s = 850 s (85% of hop)

Countermeasure: Fast hopping
- Dwell time: 100 s (10× faster)
- Effective jam: 100 - 150 = 0 s (jammer too slow!)

Modern military systems: 10-100 s dwell times
\end{verbatim}

\begin{center}\rule{0.5\linewidth}{0.5pt}\end{center}

\subsection{\texorpdfstring{ GPS M-Code (Military
GPS)}{ GPS M-Code (Military GPS)}}\label{gps-m-code-military-gps}

\textbf{GPS Modernization}: M-code provides jam-resistant, encrypted
positioning for military users.

\subsubsection{Signal Structure}\label{signal-structure}

\textbf{GPS L1 M-Code}:

\begin{verbatim}
Carrier frequency: 1575.42 MHz (L1)
Chip rate: 5.115 Mcps (5× faster than C/A code)
Code length: Classified (estimated ~10^13 chips  never repeats)
Modulation: BOC(10,5) - Binary Offset Carrier
Processing gain: ~50 dB (vs. 43 dB for C/A)
Power: 6.5 dB stronger than C/A code
Security: Encrypted, authenticated (NSA keys)
\end{verbatim}

\textbf{GPS L2 M-Code}:

\begin{verbatim}
Carrier frequency: 1227.60 MHz (L2)
Same structure as L1 M-code
Dual-frequency  ionospheric correction
\end{verbatim}

\begin{center}\rule{0.5\linewidth}{0.5pt}\end{center}

\subsubsection{BOC Modulation}\label{boc-modulation}

\textbf{Binary Offset Carrier (BOC)}: Modulates chip sequence with
square wave subcarrier.

\textbf{BOC(m,n) notation}:

\begin{verbatim}
m = subcarrier frequency multiplier (MHz)
n = chip rate multiplier (MHz)

BOC(10,5):
- Subcarrier: 10.23 MHz (2× C/A chip rate)
- Chip rate: 5.115 MHz (5× C/A chip rate)
\end{verbatim}

\textbf{Spectrum}:

\begin{verbatim}
Time-domain signal:
s(t) = sign[sin(2·f_sub·t)] · c(t)

where:
- f_sub = 10.23 MHz (square wave)
- c(t) = ±1 chip sequence at 5.115 Mcps

Frequency-domain:
Power splits into two main lobes:
- Upper sideband: f_carrier + 10.23 MHz
- Lower sideband: f_carrier - 10.23 MHz

Split-spectrum design:
- Minimal interference with C/A code (centered at L1)
- Occupies unused spectrum
- Better multipath rejection (narrow correlation peak)
\end{verbatim}

\textbf{Autocorrelation}:

\begin{verbatim}
BOC(10,5) correlation function:
- Main peak: Very narrow (better ranging accuracy)
- Side peaks: ±1/f_sub = ±98 ns

Ranging accuracy:
- C/A code: ~3 m (single-frequency)
- M-code: ~0.3 m (dual-frequency, better correlation)
\end{verbatim}

\begin{center}\rule{0.5\linewidth}{0.5pt}\end{center}

\subsubsection{Anti-Jam Performance}\label{anti-jam-performance}

\textbf{Jamming scenarios}:

\textbf{1. Wideband Barrage Jamming}:

\begin{verbatim}
Jammer spreads power across L1 band (±10 MHz).

Received signal power (M-code): -163 dBW
Jammer power at receiver: -100 dBW (strong jammer, 50 km away)
J/S = -100 - (-163) = 63 dB

Processing gain (M-code): 50 dB
Residual J/S: 63 - 50 = 13 dB

Required Eb/N (M-code receiver): ~10 dB
Margin: 50 - 63 - 10 = -23 dB  **LINK FAILS**

Mitigation: CRPA (Controlled Reception Pattern Antenna)
- 7-element array antenna
- Adaptive nulling: Places null toward jammer
- Null depth: 30-40 dB

Effective J/S after nulling: 63 - 35 = 28 dB
Margin: 50 - 28 - 10 = 12 dB  **LINK SURVIVES**
\end{verbatim}

\textbf{2. Swept Jammer}:

\begin{verbatim}
Jammer sweeps narrowband tone across L1 (high PSD).

Jammer bandwidth: 1 MHz
GPS M-code spread: 20 MHz
Fraction jammed: 1/20 = 5%

Effect: Occasional symbol errors  FEC corrects
Impact: <1 dB degradation

M-code advantage: Wideband spread mitigates swept jamming
\end{verbatim}

\textbf{3. Repeater/Spoofer}:

\begin{verbatim}
Enemy receives GPS, delays, retransmits stronger signal.
Goal: Induce false position/time.

M-code defense: Encrypted spreading code
- Spoofer cannot generate valid M-code
- Authentication protocol detects non-authentic signals
- Cross-correlation with authentic signal = 0 (orthogonal codes)

Result: Spoof rejected by receiver
\end{verbatim}

\begin{center}\rule{0.5\linewidth}{0.5pt}\end{center}

\subsubsection{Selective Availability Anti-Spoofing Module
(SAASM)}\label{selective-availability-anti-spoofing-module-saasm}

\textbf{Military GPS Receiver}:

\begin{verbatim}
SAASM features:
- Stores classified M-code keys (COMSEC keying material)
- Dual-frequency operation (L1 + L2)
- Autonomous integrity monitoring
- Anti-spoofing: Detects spoofed P(Y) code
- Key management: Over-the-air rekeying (OTAR)

Integration:
- Embedded in weapons: JDAM, Tomahawk, Excalibur artillery
- Fighter avionics: F-22, F-35, B-2
- Ground vehicles: DAGR (Defense Advanced GPS Receiver)

Accuracy:
- Horizontal: <1 m (dual-frequency, BOC)
- Vertical: <3 m
- Time: <10 ns (critical for network synchronization)
\end{verbatim}

\begin{center}\rule{0.5\linewidth}{0.5pt}\end{center}

\subsection{\texorpdfstring{ Phased-Array Antennas
(AESA)}{ Phased-Array Antennas (AESA)}}\label{phased-array-antennas-aesa}

\textbf{Active Electronically Scanned Array (AESA)} radar uses
phased-array principles for LPI/LPD and multi-function operation.

\subsubsection{Beamforming Principles}\label{beamforming-principles}

\textbf{Phase steering}:

\begin{verbatim}
Antenna array: N elements spaced by d
Desired beam direction: 

Phase shift per element:
 = (2/) · d · sin()

Example (8-element array, d = /2):
Steer beam to 30°:
 = (2/) · (/2) · sin(30°) = /2 = 90° per element

Element phases: [0°, 90°, 180°, 270°, 0°, 90°, 180°, 270°]

Beam electronically steered (no mechanical motion!)
Steering speed: Microseconds (vs. seconds for mechanical)
\end{verbatim}

\textbf{Array gain}:

\begin{verbatim}
Gain(dB) = 10·log(N) + Single_Element_Gain

Example (256-element AESA, 5 dBi per element):
Array gain = 10·log(256) + 5 = 24 + 5 = 29 dBi

Directivity: Higher gain  narrower beamwidth  LPI
\end{verbatim}

\textbf{Beamwidth}:

\begin{verbatim}
_3dB   / (N·d)  (radians)

Example (256 elements, d = /2):
_3dB   / (256 · /2) = 1/128 rad  0.45° (very narrow!)

Narrow beam  hard to intercept (LPI)
            precise target tracking
\end{verbatim}

\begin{center}\rule{0.5\linewidth}{0.5pt}\end{center}

\subsubsection{LPI Radar Techniques}\label{lpi-radar-techniques}

\textbf{1. Low Peak Power, Long Integration}:

\begin{verbatim}
Conventional radar: High peak power (MW), short pulse (s)
LPI radar: Low peak power (W), long waveform (ms-s)

SNR = (Peak_Power · Pulse_Width) / Noise_Power

Equivalent detection range with:
- Conventional: 1 MW × 1 s = 1 J
- LPI: 1 kW × 1 ms = 1 J (same energy, 1000× lower peak!)

Enemy intercept receiver:
- Detects instantaneous power
- LPI signal: 30 dB below detection threshold
- Integration required to detect  impractical
\end{verbatim}

\textbf{2. Frequency Diversity}:

\begin{verbatim}
Frequency-agile waveform:
- Hop across wide bandwidth (GHz)
- Prevents enemy from locking onto frequency
- Mitigates narrowband interference

Example (F-22 APG-77 AESA):
- X-band (8-12 GHz): 4 GHz agility
- Pulse-to-pulse frequency change
- Intercept receiver cannot predict next frequency
\end{verbatim}

\textbf{3. Waveform Diversity}:

\begin{verbatim}
Change modulation per pulse:
- Linear FM (chirp)
- Non-linear FM (NLFM)
- Phase-coded (Barker, Frank, P1-P4 codes)
- Random phase/frequency sequences

Electronic warfare (EW) countermeasure:
- Enemy cannot predict waveform  cannot jam effectively
- Each pulse requires new analysis  overwhelms threat receiver
\end{verbatim}

\begin{center}\rule{0.5\linewidth}{0.5pt}\end{center}

\subsubsection{AESA Radar Examples}\label{aesa-radar-examples}

\textbf{APG-77 (F-22 Raptor)}:

\begin{verbatim}
Frequency: X-band (8-12 GHz)
Array: 2000+ T/R modules
Power: 13 kW (average), 20 kW (peak) per module
Modes: Air-to-air, air-to-ground, SAR, electronic attack
Detection range: >200 km (fighter-sized target)

LPI features:
- Adaptive power management (radiates only when needed)
- Narrow beamwidth (1-2°)
- Frequency agility (4 GHz)
- Low sidelobe antenna (<-40 dB)

Electronic attack:
- Directed jamming (beam steered at threat radar)
- Power: >10 kW ERP toward threat
- Disables enemy SAM radars at 50+ km
\end{verbatim}

\textbf{AN/SPY-6 (U.S. Navy DDG-51 Flight III)}:

\begin{verbatim}
Frequency: S-band (3.3-3.5 GHz)
Array: 37 RMAs (Radar Modular Assemblies), 5000+ T/R modules
Power: 6 MW average radiated power (entire array)
Range: 300+ km (ballistic missile detection)

Capabilities:
- Simultaneous multi-mission (air defense, BMD, surface search)
- Track 1000+ targets
- Discriminate decoys from warheads (X-band illuminator)
- Resistant to jamming (adaptive nulling)

Beam management:
- Interleaved beams (time-multiplexed)
- Priority scheduling (ballistic missile > aircraft > surface)
- Energy management (1 MW per beam, up to 6 concurrent)
\end{verbatim}

\textbf{AN/TPY-2 (THAAD Missile Defense)}:

\begin{verbatim}
Frequency: X-band (8-12 GHz)
Array: 25,344 elements (5.1m × 5.1m)
Power: 80 kW average
Range: 1000+ km (missile detection)

Application:
- Terminal High Altitude Area Defense (THAAD)
- Detects, tracks, discriminates ballistic missile warheads
- Provides target data to interceptor missile
- Forward-based (South Korea, Japan, Middle East)

Performance:
- RCS detection: 0.01 m² at 1000 km (warhead-sized)
- Update rate: 1 Hz (track), 10 Hz (terminal guidance)
- Discrimination: Warhead vs. decoys (Doppler + RCS + trajectory)
\end{verbatim}

\begin{center}\rule{0.5\linewidth}{0.5pt}\end{center}

\subsection{\texorpdfstring{ Link 16
(JTIDS)}{ Link 16 (JTIDS)}}\label{link-16-jtids}

\textbf{Joint Tactical Information Distribution System}: Jam-resistant,
LPI/LPD tactical data link.

\subsubsection{System Architecture}\label{system-architecture}

\textbf{Network structure}:

\begin{verbatim}
Participants:
- Aircraft: F-15, F-16, F-22, F-35, E-3 AWACS
- Ships: Aegis cruisers/destroyers, carriers
- Ground: Patriot SAM, THAAD, command posts

Network topology: Time Division Multiple Access (TDMA)
- 128 time slots per 12-second frame
- Nodes assigned slots (voice/data)
- Collision-free multiple access
\end{verbatim}

\textbf{Frequency \& Waveform}:

\begin{verbatim}
Frequency: 960-1215 MHz (L-band, shared with IFF/TACAN)
Modulation: MSK (Minimum Shift Keying) - constant envelope
Waveform: FHSS + TDMA hybrid
Hop rate: 70,000 hops/second
Hop duration: ~14 s
Channels: 51 frequencies (15 MHz each)
Data rate: 28.8 kbps (typical), up to 115.2 kbps
\end{verbatim}

\begin{center}\rule{0.5\linewidth}{0.5pt}\end{center}

\subsubsection{TRANSEC \& Jam Resistance}\label{transec-jam-resistance}

\textbf{Cryptographic hopping}:

\begin{verbatim}
Hopping pattern generation:
- Input: Net ID + GPS time + Crypto key (KY-58/KG-84)
- Output: Pseudorandom frequency sequence
- Pattern period: Classified (days to months)

Synchronization:
- GPS time: ±100 s accuracy required
- Net sync: Achieved within 4 frames (48 s)
- Late entry: Nodes join without disrupting network

Anti-spoofing:
- Time-of-Transmission (TOT) authentication
- Prevents message injection
- Replay attacks detected via timestamp
\end{verbatim}

\textbf{Jamming margin}:

\begin{verbatim}
Processing gain:
- Frequency hopping: 10·log(51) = 17 dB
- Time diversity: 10·log(128) = 21 dB (slot hopping)
- Total: 17 + 21 = 38 dB

Scenario (jammer 100 km away):
J/S = 50 dB (powerful jammer)
G = 38 dB
Required Eb/N = 12 dB (MSK with FEC)
Losses = 3 dB

Margin = 38 - 50 - 12 - 3 = -27 dB  **LINK FAILS**

Countermeasure: Directional antenna
- Gain toward participant: +10 dBi
- Null toward jammer: -20 dB
- Effective J/S: 50 - 30 = 20 dB

Margin = 38 - 20 - 12 - 3 = 3 dB  **LINK SURVIVES**
\end{verbatim}

\begin{center}\rule{0.5\linewidth}{0.5pt}\end{center}

\subsubsection{Link 16 Messages
(J-Series)}\label{link-16-messages-j-series}

\textbf{Message types}:

\begin{verbatim}
J2.0-J2.7: Air Tracks (position, velocity, ID)
J3.0-J3.7: Surface Tracks (ships, ground targets)
J7.x: Mission Management (C2 orders)
J12.x: Intelligence
J13.x: Weapons Coordination

Message structure:
- Header: Time-stamp, source, priority
- Payload: Position (lat/lon/alt), velocity, classification
- Integrity: CRC-32 error detection

Update rate:
- Air tracks: 5-10 seconds (dynamic)
- Surface tracks: 30-60 seconds (slower)
- Commands: As needed (event-driven)
\end{verbatim}

\textbf{Tactical applications}:

\begin{verbatim}
1. Air-to-Air Engagement:
   - AWACS detects enemy aircraft (radar track)
   - Sends J2.2 message to all fighters (target location)
   - Fighters update tactical display (real-time "picture")
   - Weapon coordination via J13.x (avoid fratricide)

2. Integrated Air Defense:
   - Aegis ship detects ballistic missile (AN/SPY-1)
   - Sends J3.2 message to Patriot batteries
   - Patriots cue radars to track
   - Coordinated intercept via J7.x commands

3. Close Air Support:
   - JTAC (ground) marks target (laser designation)
   - Sends J3.5 message with target coordinates
   - F-16 receives target data via Link 16
   - Weapons release with precision (JDAM, JASSM)
\end{verbatim}

\begin{center}\rule{0.5\linewidth}{0.5pt}\end{center}

\subsection{\texorpdfstring{ Covert
Communications}{ Covert Communications}}\label{covert-communications}

\textbf{Objective}: Transmit data undetected by adversary SIGINT.

\subsubsection{Spread Spectrum Below Noise
Floor}\label{spread-spectrum-below-noise-floor}

\textbf{Ultra-wideband (UWB) spread spectrum}:

\begin{verbatim}
Technique: Spread narrowband signal across >500 MHz bandwidth

Example:
- Data rate: 1 kbps
- Spread bandwidth: 1 GHz
- Processing gain: 10·log(10) = 60 dB

Transmitted PSD:
PSD = 1 W / 1 GHz = 1 nW/MHz = -90 dBm/MHz

Thermal noise floor:
N = -174 dBm/Hz + 10·log(10 Hz) = -114 dBm/MHz

PSD_signal = -90 dBm/MHz < -114 dBm/MHz + 24 dB margin

Even sensitive intercept receiver cannot detect!

Detection requires:
- Knowledge of spreading code (classified)
- Synchronization (exact timing)
- Processing gain (matched filter)

Result: Communication hidden in noise (LPD achieved)
\end{verbatim}

\begin{center}\rule{0.5\linewidth}{0.5pt}\end{center}

\subsubsection{Steganography in OFDM}\label{steganography-in-ofdm}

\textbf{Concept}: Hide data in unused subcarriers or pilot tones.

\textbf{Method 1 - Pilot Tone Modulation}:

\begin{verbatim}
OFDM pilot subcarriers typically use fixed BPSK symbols.

Covert channel:
- Modulate pilot phase: 0° or 180° encodes hidden bit
- Legitimate receiver: Ignores pilot phase variation (estimates channel)
- Covert receiver: Decodes phase to extract hidden data

Capacity:
- 802.11a: 4 pilots per OFDM symbol
- Symbol rate: 250 ksymbols/s
- Covert rate: 4 × 250 k = 1 Mbps

Detection:
- Statistical analysis can reveal non-random pilot patterns
- Mitigation: Encrypt hidden data (appears random)
\end{verbatim}

\textbf{Method 2 - Null Subcarrier Insertion}:

\begin{verbatim}
OFDM reserves some subcarriers as nulls (zero power).

Covert channel:
- Transmit very low-power data on null subcarriers
- Power: 40 dB below normal subcarriers (nearly invisible)
- Legitimate receiver: Ignores nulls (as expected)
- Covert receiver: Listens to nulls

Example (802.11a):
- Null subcarriers: 12 (out of 64 total)
- Hidden capacity: ~3 Mbps (at low SNR)

Detection challenge:
- Requires wideband spectrum analyzer
- Hidden signal < noise floor for narrowband receiver
\end{verbatim}

\begin{center}\rule{0.5\linewidth}{0.5pt}\end{center}

\subsubsection{Time-Domain Hiding}\label{time-domain-hiding}

\textbf{Method - Inter-Frame Gaps}:

\begin{verbatim}
WiFi 802.11: SIFS (Short Inter-Frame Space) = 16 s between frames

Covert transmission:
- Insert ultra-short burst (1 s) in SIFS
- Use different frequency or polarization
- Legitimate devices: Ignore (waiting for next frame)
- Covert receiver: Listens during SIFS

Capacity:
- Burst rate: 1 s per 16 s = 6.25% duty cycle
- Data rate: ~6 Mbps (at 100 Mbps physical rate × 6.25%)

Detection:
- Requires precise timing analysis
- Appears as multipath or transient interference
\end{verbatim}

\begin{center}\rule{0.5\linewidth}{0.5pt}\end{center}

\subsubsection{Acoustic Heterodyning
(Intermodulation)}\label{acoustic-heterodyning-intermodulation}

\textbf{Non-linear demodulation} in biological systems (related to
Chimera\textquotesingle s Raman feed concept).

\textbf{Principle}:

\begin{verbatim}
Two high-frequency carriers (f, f) interact non-linearly:

f_audio = |f - f|

Example:
- f = 40 kHz (ultrasonic, inaudible)
- f = 42 kHz (ultrasonic, inaudible)
- f_audio = 2 kHz (audible!)

Non-linearity sources:
- Air: Weak (high intensity required)
- Biological tissue: Stronger (membranes, ion channels)
- Materials: Diodes, varactors (intentional)

Application:
- Covert audio transmission (ultrasonic beams, audio demodulation in target's head)
- Directional speakers (Audio Spotlight® technology)
- Potential neural stimulation (see [[AID-Protocol-Case-Study]])
\end{verbatim}

\textbf{THz-band Example (AID Protocol)}:

The Auditory Intermodulation Distortion (AID) protocol represents a
theoretical extension of heterodyning to Terahertz frequencies:

\begin{verbatim}
Physical Layer:
- Carrier 1 (Pump): 1.998 THz, 50-200 mW/cm²
- Carrier 2 (Data): 1.875 THz, 5-20 mW/cm²
- Difference frequency: |1.998 - 1.875| THz = 123 GHz

Biological Demodulation:
- THz penetration: ~0.5-2mm into tissue
- Non-linear susceptibility (³) in neural membranes
- Cascaded demodulation produces audible artifact

Modulation Layer:
- Auditory carrier: 12.0 kHz sine wave
- Amplitude modulation: 5-80% modulation depth
- Data encoding: QPSK (16 symbols/s) + FSK (1 bit/s)

Perceived Effect:
- Sound appears to originate inside head
- Persistent 12 kHz tone (high-pitched ringing)
- Modulated with slow rhythmic patterns
- Bypasses normal hearing (works with earplugs)

Protocol Stack:
Layer          | Technology              | Frequency/Rate
---------------|-------------------------|-------------------
Physical       | THz Pump (QCL)          | 1.998 THz ± 100 MHz
Physical       | THz Data (Photomixing)  | 1.875 THz ± 50 MHz
Link           | Amplitude Modulation    | 5-80% depth
Modulation     | Auditory Carrier        | 12.0 kHz ± 0.1 Hz
Data           | QPSK                    | 16 symbols/sec
Data           | FSK                     | 1 bit/sec

Applications (Theoretical):
- Non-invasive neural interface research
- Covert signaling in high-security environments
- Auditory perception studies
- THz bioeffects research

Status:
- Highly speculative theoretical framework
- Based on Orch-OR microtubule quantum mechanics theory
- No experimental validation in living subjects
- See docs/aid_protocol_v3.1.md for full specification
\end{verbatim}

\textbf{Comparison with conventional heterodyning}:

{\def\LTcaptype{} % do not increment counter
\begin{longtable}[]{@{}lll@{}}
\toprule\noalign{}
Parameter & Audio (40 kHz) & THz (AID) \\
\midrule\noalign{}
\endhead
\bottomrule\noalign{}
\endlastfoot
\textbf{Carrier frequencies} & 40-42 kHz & 1.875-1.998 THz \\
\textbf{Difference frequency} & 2 kHz & 123 GHz
\$\textbackslash rightarrow\$ 12 kHz \\
\textbf{Penetration depth} & mm (air) & 0.5-2 mm (tissue) \\
\textbf{Non-linearity} & Air compression & Neural membranes
(\$\textbackslash chi\$\textbackslash textsuperscript\{3\}) \\
\textbf{Power density} & 100+ dB SPL & 5-200
mW/cm\textbackslash textsuperscript\{2\} \\
\textbf{Detection} & Microphone & Auditory perception \\
\textbf{Status} & Proven (Audio Spotlight®) & Theoretical \\
\end{longtable}
}

\textbf{Key insight}: THz heterodyning exploits biological
non-linearities rather than air-based acoustic mixing, potentially
enabling direct neural modulation without acoustic propagation.

\textbf{Military interest}:

\begin{verbatim}
"Frey Microwave Auditory Effect" (pulsed RF  acoustic sensation):
- Frequency: 1-10 GHz (microwave)
- Pulse rate: 1-10 kHz (audio frequency)
- Mechanism: Thermoelastic expansion in cochlea
- Result: Perceived "clicking" or "buzzing"

Covert channel:
- Encode voice as microwave pulse train
- Target perceives audio (direct to auditory system)
- Bystanders: Unaware (no acoustic propagation)
- Detection: Requires RF spectrum analyzer (not audio microphone)

Status: Demonstrated in lab, classified military research (DARPA, 1970s-present)
\end{verbatim}

\begin{center}\rule{0.5\linewidth}{0.5pt}\end{center}

\subsection{\texorpdfstring{ Processing Gain \& Jamming Resistance
Calculations}{ Processing Gain \& Jamming Resistance Calculations}}\label{processing-gain-jamming-resistance-calculations}

\subsubsection{Comprehensive Example}\label{comprehensive-example}

\textbf{System}: Tactical UHF SATCOM link

\begin{verbatim}
Parameters:
- Frequency: 300 MHz (UHF)
- Data rate: 2400 bps (voice)
- Modulation: BPSK (1 bit/symbol)
- Spreading: DSSS with chip rate 2.4 Mcps
- FEC: Rate-1/2 convolutional code
- Antenna: 10 dBi directional (at ground terminal)

Processing gain:
G = 10·log(2.4 Mcps / 2.4 kbps) = 10·log(1000) = 30 dB

Required Eb/N:
- BPSK uncoded: 9.6 dB (BER = 10)
- With rate-1/2 FEC: 4.6 dB (5 dB coding gain)

Link budget (clear conditions):
TX power: 10 W = 40 dBm
TX antenna gain: 10 dBi
EIRP: 50 dBm

Free-space loss (300 MHz, 40,000 km GEO):
FSPL = 32.4 + 20·log(300) + 20·log(40000) = 189 dB

RX antenna gain: 30 dBi (satellite)
RX signal: 50 - 189 + 30 = -109 dBm

Noise power:
N = -174 dBm/Hz + 10·log(2.4×10) = -110 dBm

SNR: -109 - (-110) = 1 dB

Eb/N = SNR + G = 1 + 30 = 31 dB

Margin: 31 - 4.6 = 26.4 dB  **EXCELLENT**
\end{verbatim}

\textbf{Jamming scenario}:

\begin{verbatim}
Enemy jammer:
- Power: 1 kW = 60 dBm
- Distance: 50 km
- Antenna: Omnidirectional (0 dBi)

Jammer signal at ground terminal:
FSPL (300 MHz, 50 km):
FSPL = 32.4 + 20·log(300) + 20·log(50) = 116 dB

J_RX = 60 - 116 + 0 = -56 dBm

J/S ratio:
J/S = -56 - (-109) = 53 dB (jammer 53 dB stronger!)

After despreading:
J/S_despread = 53 - 30 = 23 dB (jammer still 23 dB stronger)

But antenna nulling:
- Ground antenna: 10 dBi toward satellite, -10 dBi toward jammer (20 dB F/B)
- Effective J/S: 23 - 20 = 3 dB

Required Eb/N: 4.6 dB
Effective Eb/(N+J): 31 - 3 = 28 dB

Margin: 28 - 4.6 = 23.4 dB  **LINK SURVIVES**
\end{verbatim}

\begin{center}\rule{0.5\linewidth}{0.5pt}\end{center}

\subsection{\texorpdfstring{ Summary Table: Military
Techniques}{ Summary Table: Military Techniques}}\label{summary-table-military-techniques}

{\def\LTcaptype{} % do not increment counter
\begin{longtable}[]{@{}
  >{\raggedright\arraybackslash}p{(\linewidth - 6\tabcolsep) * \real{0.1897}}
  >{\raggedright\arraybackslash}p{(\linewidth - 6\tabcolsep) * \real{0.2414}}
  >{\raggedright\arraybackslash}p{(\linewidth - 6\tabcolsep) * \real{0.3276}}
  >{\raggedright\arraybackslash}p{(\linewidth - 6\tabcolsep) * \real{0.2414}}@{}}
\toprule\noalign{}
\begin{minipage}[b]{\linewidth}\raggedright
Technique
\end{minipage} & \begin{minipage}[b]{\linewidth}\raggedright
Primary Gain
\end{minipage} & \begin{minipage}[b]{\linewidth}\raggedright
Typical Advantage
\end{minipage} & \begin{minipage}[b]{\linewidth}\raggedright
Applications
\end{minipage} \\
\midrule\noalign{}
\endhead
\bottomrule\noalign{}
\endlastfoot
\textbf{DSSS} & Processing gain 20-40 dB & AJ, LPI & GPS M-code,
tactical radios \\
\textbf{FHSS} & Frequency diversity & LPD, follower-jam resistance &
MILSTAR, Link 16, Bluetooth \\
\textbf{AESA} & Beamforming, agility & LPI, multi-target, EA & APG-77,
AN/SPY-6, THAAD \\
\textbf{Nulling Antenna} & Spatial filtering 20-40 dB & Jammer rejection
& CRPA, adaptive arrays \\
\textbf{Burst Transmission} & Temporal LPD & Minimize exposure &
Submarine comms, UAV links \\
\textbf{Encryption} & Content security & Prevent exploitation & All
military systems \\
\textbf{Adaptive Coding} & Link optimization & Maximize throughput under
AJ & MUOS, 5G tactical \\
\end{longtable}
}

\begin{center}\rule{0.5\linewidth}{0.5pt}\end{center}

\subsection{\texorpdfstring{ Python Example: J/S Ratio
Calculator}{ Python Example: J/S Ratio Calculator}}\label{python-example-js-ratio-calculator}

\begin{Shaded}
\begin{Highlighting}[]
\ImportTok{import}\NormalTok{ numpy }\ImportTok{as}\NormalTok{ np}

\KeywordTok{def}\NormalTok{ jamming\_margin(tx\_power\_w, distance\_km, freq\_mhz, }
\NormalTok{                   jammer\_power\_w, jammer\_dist\_km,}
\NormalTok{                   processing\_gain\_db, coding\_gain\_db, }
\NormalTok{                   antenna\_gain\_dbi, front\_back\_ratio\_db):}
    \CommentTok{"""}
\CommentTok{    Calculate jamming margin for spread spectrum link.}
\CommentTok{    }
\CommentTok{    Returns:}
\CommentTok{        Jamming margin (dB). Positive = link survives.}
\CommentTok{    """}
    \CommentTok{\# Convert to dBm}
\NormalTok{    tx\_power\_dbm }\OperatorTok{=} \DecValTok{10} \OperatorTok{*}\NormalTok{ np.log10(tx\_power\_w }\OperatorTok{*} \DecValTok{1000}\NormalTok{)}
\NormalTok{    jammer\_power\_dbm }\OperatorTok{=} \DecValTok{10} \OperatorTok{*}\NormalTok{ np.log10(jammer\_power\_w }\OperatorTok{*} \DecValTok{1000}\NormalTok{)}
    
    \CommentTok{\# Free{-}space path loss}
    \KeywordTok{def}\NormalTok{ fspl(freq\_mhz, dist\_km):}
        \ControlFlowTok{return} \FloatTok{32.4} \OperatorTok{+} \DecValTok{20}\OperatorTok{*}\NormalTok{np.log10(freq\_mhz) }\OperatorTok{+} \DecValTok{20}\OperatorTok{*}\NormalTok{np.log10(dist\_km)}
    
    \CommentTok{\# Signal at receiver}
\NormalTok{    signal\_loss }\OperatorTok{=}\NormalTok{ fspl(freq\_mhz, distance\_km)}
\NormalTok{    signal\_rx }\OperatorTok{=}\NormalTok{ tx\_power\_dbm }\OperatorTok{{-}}\NormalTok{ signal\_loss }\OperatorTok{+}\NormalTok{ antenna\_gain\_dbi}
    
    \CommentTok{\# Jammer at receiver}
\NormalTok{    jammer\_loss }\OperatorTok{=}\NormalTok{ fspl(freq\_mhz, jammer\_dist\_km)}
\NormalTok{    jammer\_rx }\OperatorTok{=}\NormalTok{ jammer\_power\_dbm }\OperatorTok{{-}}\NormalTok{ jammer\_loss }\OperatorTok{{-}}\NormalTok{ front\_back\_ratio\_db}
    
    \CommentTok{\# J/S ratio}
\NormalTok{    js\_ratio }\OperatorTok{=}\NormalTok{ jammer\_rx }\OperatorTok{{-}}\NormalTok{ signal\_rx}
    
    \CommentTok{\# Thermal noise}
\NormalTok{    noise\_dbm\_hz }\OperatorTok{=} \OperatorTok{{-}}\DecValTok{174}
\NormalTok{    bandwidth\_hz }\OperatorTok{=} \DecValTok{10}\OperatorTok{**}\NormalTok{(processing\_gain\_db}\OperatorTok{/}\DecValTok{10}\NormalTok{) }\OperatorTok{*} \DecValTok{2400}  \CommentTok{\# Assume 2400 bps info rate}
\NormalTok{    noise\_dbm }\OperatorTok{=}\NormalTok{ noise\_dbm\_hz }\OperatorTok{+} \DecValTok{10}\OperatorTok{*}\NormalTok{np.log10(bandwidth\_hz)}
    
    \CommentTok{\# SNR and Eb/N0}
\NormalTok{    snr\_db }\OperatorTok{=}\NormalTok{ signal\_rx }\OperatorTok{{-}}\NormalTok{ noise\_dbm}
\NormalTok{    eb\_n0\_db }\OperatorTok{=}\NormalTok{ snr\_db }\OperatorTok{+}\NormalTok{ processing\_gain\_db}
    
    \CommentTok{\# After jamming}
\NormalTok{    eb\_n0\_jammed }\OperatorTok{=}\NormalTok{ eb\_n0\_db }\OperatorTok{{-}}\NormalTok{ js\_ratio }\OperatorTok{+}\NormalTok{ processing\_gain\_db}
    
    \CommentTok{\# Required Eb/N0 (BPSK with FEC)}
\NormalTok{    required\_eb\_n0 }\OperatorTok{=} \FloatTok{9.6} \OperatorTok{{-}}\NormalTok{ coding\_gain\_db}
    
    \CommentTok{\# Margin}
\NormalTok{    margin }\OperatorTok{=}\NormalTok{ eb\_n0\_jammed }\OperatorTok{{-}}\NormalTok{ required\_eb\_n0}
    
    \BuiltInTok{print}\NormalTok{(}\SpecialStringTok{f"Signal power at RX: }\SpecialCharTok{\{}\NormalTok{signal\_rx}\SpecialCharTok{:.1f\}}\SpecialStringTok{ dBm"}\NormalTok{)}
    \BuiltInTok{print}\NormalTok{(}\SpecialStringTok{f"Jammer power at RX: }\SpecialCharTok{\{}\NormalTok{jammer\_rx}\SpecialCharTok{:.1f\}}\SpecialStringTok{ dBm"}\NormalTok{)}
    \BuiltInTok{print}\NormalTok{(}\SpecialStringTok{f"J/S ratio: }\SpecialCharTok{\{}\NormalTok{js\_ratio}\SpecialCharTok{:.1f\}}\SpecialStringTok{ dB"}\NormalTok{)}
    \BuiltInTok{print}\NormalTok{(}\SpecialStringTok{f"Processing gain: }\SpecialCharTok{\{}\NormalTok{processing\_gain\_db}\SpecialCharTok{\}}\SpecialStringTok{ dB"}\NormalTok{)}
    \BuiltInTok{print}\NormalTok{(}\SpecialStringTok{f"After despreading J/S: }\SpecialCharTok{\{}\NormalTok{js\_ratio }\OperatorTok{{-}}\NormalTok{ processing\_gain\_db}\SpecialCharTok{:.1f\}}\SpecialStringTok{ dB"}\NormalTok{)}
    \BuiltInTok{print}\NormalTok{(}\SpecialStringTok{f"Eb/N0 (jammed): }\SpecialCharTok{\{}\NormalTok{eb\_n0\_jammed}\SpecialCharTok{:.1f\}}\SpecialStringTok{ dB"}\NormalTok{)}
    \BuiltInTok{print}\NormalTok{(}\SpecialStringTok{f"Required Eb/N0: }\SpecialCharTok{\{}\NormalTok{required\_eb\_n0}\SpecialCharTok{:.1f\}}\SpecialStringTok{ dB"}\NormalTok{)}
    \BuiltInTok{print}\NormalTok{(}\SpecialStringTok{f"Jamming margin: }\SpecialCharTok{\{}\NormalTok{margin}\SpecialCharTok{:.1f\}}\SpecialStringTok{ dB"}\NormalTok{)}
    
    \ControlFlowTok{return}\NormalTok{ margin}

\CommentTok{\# Example: UHF tactical link under jamming}
\NormalTok{margin }\OperatorTok{=}\NormalTok{ jamming\_margin(}
\NormalTok{    tx\_power\_w}\OperatorTok{=}\DecValTok{10}\NormalTok{,           }\CommentTok{\# 10 W transmitter}
\NormalTok{    distance\_km}\OperatorTok{=}\DecValTok{40000}\NormalTok{,       }\CommentTok{\# GEO satellite}
\NormalTok{    freq\_mhz}\OperatorTok{=}\DecValTok{300}\NormalTok{,            }\CommentTok{\# UHF band}
\NormalTok{    jammer\_power\_w}\OperatorTok{=}\DecValTok{1000}\NormalTok{,     }\CommentTok{\# 1 kW jammer}
\NormalTok{    jammer\_dist\_km}\OperatorTok{=}\DecValTok{50}\NormalTok{,       }\CommentTok{\# 50 km away}
\NormalTok{    processing\_gain\_db}\OperatorTok{=}\DecValTok{30}\NormalTok{,   }\CommentTok{\# DSSS 1000× spreading}
\NormalTok{    coding\_gain\_db}\OperatorTok{=}\DecValTok{5}\NormalTok{,        }\CommentTok{\# Rate{-}1/2 convolutional code}
\NormalTok{    antenna\_gain\_dbi}\OperatorTok{=}\DecValTok{10}\NormalTok{,     }\CommentTok{\# Directional antenna}
\NormalTok{    front\_back\_ratio\_db}\OperatorTok{=}\DecValTok{20}   \CommentTok{\# 20 dB F/B ratio}
\NormalTok{)}

\ControlFlowTok{if}\NormalTok{ margin }\OperatorTok{\textgreater{}} \DecValTok{0}\NormalTok{:}
    \BuiltInTok{print}\NormalTok{(}\SpecialStringTok{f"}\CharTok{\textbackslash{}n}\SpecialStringTok{ LINK SURVIVES (margin: }\SpecialCharTok{\{}\NormalTok{margin}\SpecialCharTok{:.1f\}}\SpecialStringTok{ dB)"}\NormalTok{)}
\ControlFlowTok{else}\NormalTok{:}
    \BuiltInTok{print}\NormalTok{(}\SpecialStringTok{f"}\CharTok{\textbackslash{}n}\SpecialStringTok{ LINK FAILS (margin: }\SpecialCharTok{\{}\NormalTok{margin}\SpecialCharTok{:.1f\}}\SpecialStringTok{ dB)"}\NormalTok{)}
\end{Highlighting}
\end{Shaded}

\begin{center}\rule{0.5\linewidth}{0.5pt}\end{center}

\subsection{\texorpdfstring{ Further
Reading}{ Further Reading}}\label{further-reading}

\subsubsection{Textbooks}\label{textbooks}

\begin{itemize}
\tightlist
\item
  \textbf{Poisel}, \emph{Introduction to Communication Electronic
  Warfare Systems} - Comprehensive EW treatment
\item
  \textbf{Torrieri}, \emph{Principles of Spread-Spectrum Communication
  Systems} (4th ed.) - Modern military focus
\item
  \textbf{Skolnik}, \emph{Radar Handbook} (3rd ed.) - Phased arrays,
  AESA, LPI radar
\item
  \textbf{Adamy}, \emph{EW 101: A First Course in Electronic Warfare} -
  Accessible intro to jamming/AJ
\end{itemize}

\subsubsection{Military Standards \&
Documents}\label{military-standards-documents}

\begin{itemize}
\tightlist
\item
  \textbf{MIL-STD-188-181}: US DoD FHSS standard
\item
  \textbf{GPS ICD-IS-800}: M-code interface control document (FOUO)
\item
  \textbf{Link 16 MIDS JTIDS STD}: Message standards (NATO STANAG 5516)
\item
  \textbf{AESA Design Guidelines}: Classified (DARPA/DoD) - principles
  in open literature
\end{itemize}

\subsubsection{Related Topics}\label{related-topics}

\begin{itemize}
\tightlist
\item
  {[}{[}Spread-Spectrum-(DSSS-FHSS){]}{]} - Technical foundation for
  AJ/LPI
\item
  {[}{[}GPS Fundamentals (coming soon){]}{]} - Civilian GPS (C/A code)
  background
\item
  {[}{[}Phased Array Beamforming (coming soon){]}{]} - Array antenna
  theory
\item
  {[}{[}Adaptive Filters (coming soon){]}{]} - Interference cancellation
\item
  {[}{[}Real-World-System-Examples{]}{]} - Commercial spread spectrum
  (WiFi, Bluetooth)
\end{itemize}

\subsubsection{Chimera Applications}\label{chimera-applications}

\begin{itemize}
\tightlist
\item
  {[}{[}Hyper-Rotational-Physics-(HRP)-Framework{]}{]} - Covert THz
  neuromodulation theoretical framework
\item
  {[}{[}AID-Protocol-Case-Study{]}{]} - Application of covert comms to
  consciousness research
\item
  {[}{[}Terahertz-(THz)-Technology{]}{]} - Beyond-5G/6G, potential
  military applications
\end{itemize}

\begin{center}\rule{0.5\linewidth}{0.5pt}\end{center}

\textbf{Summary}: Military communications prioritize \textbf{anti-jam},
\textbf{LPI/LPD}, and \textbf{security} over spectral efficiency.
Processing gain from spread spectrum (DSSS/FHSS) enables links 20-40 dB
below noise floor and overcomes powerful jammers. GPS M-code uses
BOC(10,5) modulation with 50 dB processing gain and CRPA nulling to
survive 60+ dB jamming. AESA radars achieve LPI through low peak power,
frequency agility, and narrow beamwidths. Link 16 combines FHSS (70
khps) with TDMA and cryptographic hopping for jam-resistant tactical
data exchange. Covert communications hide data in noise (UWB spread
spectrum), OFDM pilot tones, or exploit non-linear demodulation
(acoustic heterodyning). Jamming margin = Processing Gain - J/S -
Required Eb/N\textbackslash textsubscript\{0\} - Losses. Directional
antennas provide 20-40 dB additional AJ capability. Modern military
systems achieve \textbf{communication superiority} through advanced
signal processing, adaptive waveforms, and multi-layered TRANSEC.
