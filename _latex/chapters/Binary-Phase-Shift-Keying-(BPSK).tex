\section{Binary Phase-Shift Keying
(BPSK)}\label{binary-phase-shift-keying-bpsk}

{[}{[}Home{]}{]} \textbar{} \textbf{Modulation} \textbar{}
{[}{[}On-Off-Keying-(OOK){]}{]} \textbar{}
{[}{[}Frequency-Shift-Keying-(FSK){]}{]} \textbar{}
{[}{[}QPSK-Modulation{]}{]}

\begin{center}\rule{0.5\linewidth}{0.5pt}\end{center}

\subsection{\texorpdfstring{ For Non-Technical
Readers}{ For Non-Technical Readers}}\label{for-non-technical-readers}

\textbf{BPSK is like Morse code with a twist-\/-\/-instead of on/off,
you flip the wave upside-down to send 1s and 0s.}

\textbf{Simple idea}: - Bit 0 = wave pointing ``up''
\$\textbackslash uparrow\$\\
- Bit 1 = wave pointing ``down'' \$\textbackslash downarrow\$ (flipped
180\$\^{}\textbackslash circ\$)

\textbf{Real use}: GPS satellites use BPSK! Your phone detects whether
the signal is normal or flipped.

\textbf{Why flip instead of on/off?} More reliable in noise, works with
constant power, less interference. Trade-off: Simple but slow (1 bit per
symbol).

\begin{center}\rule{0.5\linewidth}{0.5pt}\end{center}

\subsection{Overview}\label{overview}

\textbf{Binary Phase-Shift Keying (BPSK)} is the simplest form of
\textbf{phase modulation}, where binary data is encoded by
\textbf{shifting the carrier phase} between two states:
0\$\^{}\textbackslash circ\$ and 180\$\^{}\textbackslash circ\$.

\textbf{Key advantage over {[}{[}On-Off-Keying-(OOK)\textbar OOK{]}{]}
and {[}{[}Frequency-Shift-Keying-(FSK)\textbar FSK{]}{]}}: BPSK uses
\textbf{coherent detection} and provides \textbf{3 dB better
performance} (lower BER for same SNR).

\textbf{Foundation for}: {[}{[}QPSK-Modulation{]}{]} (4 phases), 8PSK (8
phases), and higher-order modulation.

\begin{center}\rule{0.5\linewidth}{0.5pt}\end{center}

\subsection{Mathematical Description}\label{mathematical-description}

\subsubsection{Time-Domain Signal}\label{time-domain-signal}

\textbf{BPSK waveform}:

\[
s(t) = A \cos(2\pi f_c t + \phi_n)
\]

Where: - \(A\) = Carrier amplitude - \(f_c\) = Carrier frequency -
\(\phi_n \in \{0°, 180°\}\) = Phase for bit \(n\)

\textbf{Phase encoding}:

\[
\phi_n = \begin{cases}
0° & \text{if bit = 0} \\
180° & \text{if bit = 1}
\end{cases}
\]

\textbf{Alternative representation} (using cosine identity):

\[
s(t) = A \cdot d_n \cdot \cos(2\pi f_c t)
\]

Where: - \(d_n \in \{+1, -1\}\) = Bipolar data symbol - Bit 0
\$\textbackslash rightarrow\$ \(d_n = +1\) \$\textbackslash rightarrow\$
0\$\^{}\textbackslash circ\$ phase - Bit 1 \$\textbackslash rightarrow\$
\(d_n = -1\) \$\textbackslash rightarrow\$
180\$\^{}\textbackslash circ\$ phase (inverted carrier)

\textbf{Key insight}: BPSK is \textbf{amplitude modulation with bipolar
data} (carrier polarity flips).

\begin{center}\rule{0.5\linewidth}{0.5pt}\end{center}

\subsection{{[}{[}IQ-Representation{]}{]}}\label{iq-representation}

\textbf{Baseband complex representation}:

\[
s(t) = \text{Re}\{A \cdot d_n \cdot e^{j2\pi f_c t}\}
\]

\textbf{IQ components}: - \textbf{I (In-phase)}: \(I_n = A \cdot d_n\)
(either \(+A\) or \(-A\)) - \textbf{Q (Quadrature)}: \(Q_n = 0\) (BPSK
uses only I axis)

\textbf{{[}{[}Constellation-Diagrams\textbar Constellation{]}{]}}:

\begin{verbatim}
Q (Imaginary)
     |
     |
-----+-----+-----  I (Real)
  -A | 0   | +A
     |
     |

Two constellation points:
- Bit 0: (+A, 0)   0° phase
- Bit 1: (-A, 0)   180° phase
\end{verbatim}

\textbf{Distance between symbols}: \(d = 2A\)

\begin{center}\rule{0.5\linewidth}{0.5pt}\end{center}

\subsection{Modulation \& Demodulation}\label{modulation-demodulation}

\subsubsection{Transmitter (Modulator)}\label{transmitter-modulator}

\textbf{Block diagram}:

\begin{verbatim}
Binary data  Bipolar NRZ  [×]  Bandpass  BPSK signal
 {0, 1}      {+1, -1}       |     filter
                            |
                         cos(2f_c t)
                         (Carrier)
\end{verbatim}

\textbf{Steps}: 1. \textbf{NRZ encoding}: Map bits to symbols - Bit 0
\$\textbackslash rightarrow\$ \(d_n = +1\) - Bit 1
\$\textbackslash rightarrow\$ \(d_n = -1\) 2. \textbf{Multiply by
carrier}: \(s(t) = A d_n \cos(2\pi f_c t)\) 3. \textbf{Pulse shaping}:
Apply raised-cosine filter (limit bandwidth, prevent ISI)

\begin{center}\rule{0.5\linewidth}{0.5pt}\end{center}

\subsubsection{Receiver (Coherent
Detector)}\label{receiver-coherent-detector}

\textbf{Block diagram}:

\begin{verbatim}
BPSK signal  [×]  Lowpass  Sample  Threshold  Binary data
               |     filter    at T_s    (> 0?)      {0, 1}
               |
            cos(2f_c t + )
       (Local oscillator, must be phase-locked!)
\end{verbatim}

\textbf{Steps}: 1. \textbf{Multiply by local carrier} (same frequency
and phase as TX):

\[
r(t) = s(t) \cdot 2\cos(2\pi f_c t)
\]

\begin{enumerate}
\def\labelenumi{\arabic{enumi}.}
\setcounter{enumi}{1}
\tightlist
\item
  \textbf{Product}:
\end{enumerate}

\[
r(t) = A d_n \cos(2\pi f_c t) \cdot 2\cos(2\pi f_c t)
\]

\begin{enumerate}
\def\labelenumi{\arabic{enumi}.}
\setcounter{enumi}{2}
\tightlist
\item
  \textbf{Trig identity}: \(\cos(x)\cos(x) = \frac{1}{2}[1 + \cos(2x)]\)
\end{enumerate}

\[
r(t) = A d_n [1 + \cos(4\pi f_c t)]
\]

\begin{enumerate}
\def\labelenumi{\arabic{enumi}.}
\setcounter{enumi}{3}
\tightlist
\item
  \textbf{Lowpass filter} removes \(2f_c\) term:
\end{enumerate}

\[
r(t) = A d_n
\]

\begin{enumerate}
\def\labelenumi{\arabic{enumi}.}
\setcounter{enumi}{4}
\item
  \textbf{Sample at bit period}: \(y_n = A d_n + n(t)\)
\item
  \textbf{Threshold decision}:
\end{enumerate}

\[
\hat{d}_n = \begin{cases}
+1 & \text{if } y_n > 0 \quad (\text{bit } 0) \\
-1 & \text{if } y_n < 0 \quad (\text{bit } 1)
\end{cases}
\]

\textbf{Critical requirement}: \textbf{Phase synchronization} (carrier
recovery circuit needed)

\begin{center}\rule{0.5\linewidth}{0.5pt}\end{center}

\subsection{Carrier Recovery}\label{carrier-recovery}

\textbf{Problem}: Receiver must generate local oscillator
\textbf{exactly in phase} with TX carrier.

\textbf{Phase offset \(\phi_e\) causes errors}:

\[
r(t) = A d_n \cos(\phi_e)
\]

If \(\phi_e = 90°\): \(r(t) = 0\) (complete signal loss!)

\begin{center}\rule{0.5\linewidth}{0.5pt}\end{center}

\subsubsection{Solutions}\label{solutions}

\paragraph{1. Pilot Tone}\label{pilot-tone}

\begin{itemize}
\tightlist
\item
  TX sends unmodulated carrier alongside data
\item
  RX phase-locks to pilot
\item
  \textbf{Overhead}: Wastes power/bandwidth
\end{itemize}

\paragraph{2. Costas Loop}\label{costas-loop}

\begin{itemize}
\tightlist
\item
  \textbf{PLL-based carrier recovery} from modulated signal
\item
  Multiplies signal by \(\sin(2\pi f_c t)\) and \(\cos(2\pi f_c t)\)
\item
  Adjusts phase until Q-channel (sine branch) = 0
\item
  \textbf{Advantage}: No pilot needed
\end{itemize}

\paragraph{3. Squaring Loop}\label{squaring-loop}

\begin{itemize}
\tightlist
\item
  Square BPSK signal:
  \((d_n \cos(\theta))^2 = \frac{1}{2}d_n^2[1 + \cos(2\theta)]\)
\item
  Since \(d_n^2 = 1\): Doubled-frequency carrier emerges
\item
  PLL locks to \(2f_c\), then divide by 2
\item
  \textbf{Advantage}: Removes data modulation
\item
  \textbf{Disadvantage}: 180\$\^{}\textbackslash circ\$ phase ambiguity
  (need differential encoding)
\end{itemize}

\begin{center}\rule{0.5\linewidth}{0.5pt}\end{center}

\subsubsection{Differential BPSK (DBPSK)}\label{differential-bpsk-dbpsk}

\textbf{Solution to phase ambiguity}: Encode data in \textbf{phase
transitions}, not absolute phase.

\textbf{Encoding}:

\[
\phi_n = \phi_{n-1} + \Delta\phi_n
\]

Where: - Bit 0 \$\textbackslash rightarrow\$ No phase change
(\(\Delta\phi = 0°\)) - Bit 1 \$\textbackslash rightarrow\$ Phase change
(\(\Delta\phi = 180°\))

\textbf{Decoding}: Compare consecutive symbols:

\[
\hat{b}_n = \begin{cases}
0 & \text{if } \text{sgn}(y_n) = \text{sgn}(y_{n-1}) \\
1 & \text{if } \text{sgn}(y_n) \neq \text{sgn}(y_{n-1})
\end{cases}
\]

\textbf{Advantage}: No carrier recovery needed (differential detection)
\textbf{Disadvantage}: \textasciitilde3 dB worse than coherent BPSK
(errors propagate)

\begin{center}\rule{0.5\linewidth}{0.5pt}\end{center}

\subsection{Bit Error Rate (BER)
Performance}\label{bit-error-rate-ber-performance}

\subsubsection{Coherent BPSK (Ideal)}\label{coherent-bpsk-ideal}

\textbf{In AWGN channel}:

\[
\text{BER} = Q\left(\sqrt{\frac{2E_b}{N_0}}\right) = \frac{1}{2}\text{erfc}\left(\sqrt{\frac{E_b}{N_0}}\right)
\]

Where: - \(E_b\) = Energy per bit = \(\frac{A^2 T_b}{2}\) - \(N_0\) =
Noise power spectral density - \(Q(x)\) = Tail probability of Gaussian:
\(Q(x) = \frac{1}{\sqrt{2\pi}}\int_x^\infty e^{-t^2/2}dt\)

\textbf{Key values}:

{\def\LTcaptype{} % do not increment counter
\begin{longtable}[]{@{}ll@{}}
\toprule\noalign{}
\(E_b/N_0\) (dB) & BER \\
\midrule\noalign{}
\endhead
\bottomrule\noalign{}
\endlastfoot
0 dB & 7.9 \$\textbackslash times\$
10\textbackslash textsuperscript\{-\}\textbackslash textsuperscript\{2\}
(1 error in 13 bits) \\
5 dB & 9.7 \$\textbackslash times\$
10\textbackslash textsuperscript\{-\}\textbackslash textsuperscript\{4\}
(1 in 1,000) \\
10 dB & 3.9 \$\textbackslash times\$
10\textbackslash textsuperscript\{-\}\textbackslash textsuperscript\{6\}
(1 in 250,000) \\
15 dB & 6.9 \$\textbackslash times\$
10\textbackslash textsuperscript\{-\}\textbackslash textsuperscript\{1\}\textbackslash textsuperscript\{0\}
(1 in 1.4 billion) \\
\end{longtable}
}

\begin{center}\rule{0.5\linewidth}{0.5pt}\end{center}

\subsubsection{Comparison: BPSK vs OOK}\label{comparison-bpsk-vs-ook}

\textbf{At same \(E_b/N_0\)}:

{\def\LTcaptype{} % do not increment counter
\begin{longtable}[]{@{}ll@{}}
\toprule\noalign{}
Modulation & BER @ 10 dB \(E_b/N_0\) \\
\midrule\noalign{}
\endhead
\bottomrule\noalign{}
\endlastfoot
{[}{[}On-Off-Keying-(OOK) & OOK{]}{]} (non-coherent) \\
\textbf{BPSK (coherent)} & \textbf{3.9 \$\textbackslash times\$
10\textbackslash textsuperscript\{-\}\textbackslash textsuperscript\{6\}} \\
\end{longtable}
}

\textbf{BPSK is \textasciitilde1000\$\textbackslash times\$ better} at
10 dB!

\textbf{Why?} 1. \textbf{BPSK uses both halves of signal space}
(\$\textbackslash pm\$A vs OOK\textquotesingle s 0/A) 2.
\textbf{Coherent detection} (correlates with carrier, optimal) 3.
\textbf{Maximum Euclidean distance} between symbols

\begin{center}\rule{0.5\linewidth}{0.5pt}\end{center}

\subsubsection{Differential BPSK
(DBPSK)}\label{differential-bpsk-dbpsk-1}

\textbf{Slightly worse than coherent BPSK}:

\[
\text{BER}_{\text{DBPSK}} \approx \frac{1}{2}e^{-E_b/N_0}
\]

\textbf{At 10 dB}: BER \$\textbackslash approx\$ 5
\$\textbackslash times\$
10\textbackslash textsuperscript\{-\}\textbackslash textsuperscript\{6\}
(\textasciitilde1.3 dB penalty vs coherent)

\begin{center}\rule{0.5\linewidth}{0.5pt}\end{center}

\subsection{Bandwidth Efficiency}\label{bandwidth-efficiency}

\textbf{Occupied bandwidth} (99\% power):

\[
B \approx \frac{1}{T_b} = R_b
\]

Where: - \(R_b\) = Bit rate (bps) - \(T_b\) = Bit period

\textbf{With raised-cosine pulse shaping} (roll-off \(\alpha\)):

\[
B = R_b(1 + \alpha)
\]

\textbf{Typical}: \(\alpha = 0.35\) \$\textbackslash rightarrow\$
\(B = 1.35 R_b\)

\textbf{Spectral efficiency}:

\[
\eta = \frac{R_b}{B} = \frac{1}{1+\alpha} \approx 0.74\ \text{bps/Hz}
\]

\textbf{Example}: 1 Mbps BPSK with \(\alpha = 0.35\) requires
\textbf{1.35 MHz bandwidth}.

\begin{center}\rule{0.5\linewidth}{0.5pt}\end{center}

\subsection{Practical Implementations}\label{practical-implementations}

\subsubsection{1. IEEE 802.15.4 (Zigbee, Low-Rate
WPAN)}\label{ieee-802.15.4-zigbee-low-rate-wpan}

\textbf{PHY layer} (868/915 MHz bands): - \textbf{Modulation}: BPSK
(optional O-QPSK) - \textbf{Chip rate}: 300 kcps (868 MHz), 600 kcps
(915 MHz) - \textbf{Data rate}: 20 kbps (868), 40 kbps (915) -
\textbf{Spreading}: DSSS (Direct-Sequence Spread Spectrum)

\begin{center}\rule{0.5\linewidth}{0.5pt}\end{center}

\subsubsection{2. Satellite Telemetry}\label{satellite-telemetry}

\textbf{Deep-space missions} (Voyager, Mars rovers): -
\textbf{Modulation}: BPSK or QPSK - \textbf{Coding}: Convolutional +
Reed-Solomon (concatenated FEC) - \textbf{Data rate}: 10 bps - 10 kbps
(extreme distances) - \textbf{Why BPSK?}: Maximum power efficiency
(every dB counts)

\textbf{Example}: Voyager 1 (24 billion km): - TX power: 23 W - Antenna
gain: 48 dBi (dish) - RX antenna: 70 m DSN dish (74 dBi) - Link budget:
Barely positive with FEC (BER
10\textbackslash textsuperscript\{-\}\textbackslash textsuperscript\{5\})

\begin{center}\rule{0.5\linewidth}{0.5pt}\end{center}

\subsubsection{3. RFID (Passive Tags)}\label{rfid-passive-tags}

\textbf{Backscatter modulation}: - Tag reflects/absorbs carrier energy -
\textbf{Binary encoding}: Reflection = bit 0, absorption = bit 1 -
\textbf{Effectively BPSK} (from reader\textquotesingle s perspective) -
\textbf{Data rate}: 40-640 kbps (EPC Gen2)

\begin{center}\rule{0.5\linewidth}{0.5pt}\end{center}

\subsection{Advantages of BPSK}\label{advantages-of-bpsk}

\begin{enumerate}
\def\labelenumi{\arabic{enumi}.}
\tightlist
\item
  \textbf{Best BER performance} for binary modulation (3 dB better than
  OOK)
\item
  \textbf{Constant envelope} (nonlinear amplifiers OK, no AM-PM
  distortion)
\item
  \textbf{Simple constellation} (two points, easy visualization)
\item
  \textbf{Foundation for higher-order PSK}
  ({[}{[}QPSK-Modulation\textbar QPSK{]}{]}, 8PSK)
\end{enumerate}

\begin{center}\rule{0.5\linewidth}{0.5pt}\end{center}

\subsection{Disadvantages of BPSK}\label{disadvantages-of-bpsk}

\begin{enumerate}
\def\labelenumi{\arabic{enumi}.}
\tightlist
\item
  \textbf{Requires carrier synchronization} (Costas loop, squaring loop
  = complex)
\item
  \textbf{Differential BPSK} (DBPSK) avoids this but has 3 dB penalty
\item
  \textbf{Low spectral efficiency} (1 bit/symbol = 1 bps/Hz max)
\item
  \textbf{Higher-order modulation} (QPSK, 16-QAM) more efficient for
  high SNR
\end{enumerate}

\begin{center}\rule{0.5\linewidth}{0.5pt}\end{center}

\subsection{Transition to QPSK}\label{transition-to-qpsk}

\textbf{BPSK uses one axis} (I-axis) with two constellation points.

\textbf{Natural extension}: Use \textbf{both I and Q axes}
\$\textbackslash rightarrow\$ {[}{[}QPSK-Modulation\textbar QPSK{]}{]}:

\begin{verbatim}
BPSK constellation:
   Q
   |
---+---+---  I
  -A   +A
   |

QPSK constellation (4 points):
   Q
   |  +A
   | / \
---+-----+---  I
   | \ /
   | -A
   |

4 phases: 45°, 135°, 225°, 315°
2 bits per symbol  Double spectral efficiency
\end{verbatim}

\textbf{QPSK} = Two independent BPSK channels (I and Q) in parallel.

\textbf{See}: {[}{[}QPSK-Modulation{]}{]} for details

\begin{center}\rule{0.5\linewidth}{0.5pt}\end{center}

\subsection{Worked Example: BPSK Link
Budget}\label{worked-example-bpsk-link-budget}

\textbf{Scenario}: Satellite downlink

\textbf{Given}: - TX power: \(P_t = 10\) W (40 dBm) - TX antenna gain:
\(G_t = 30\) dBi - Distance: \(d = 36,000\) km (GEO) - Frequency:
\(f = 12\) GHz (Ku-band) - RX antenna gain: \(G_r = 40\) dBi (1 m dish)
- System noise temperature: \(T_s = 150\) K - Bandwidth: \(B = 1\) MHz -
Required BER: \(10^{-6}\)

\begin{center}\rule{0.5\linewidth}{0.5pt}\end{center}

\subsubsection{Step 1: Calculate FSPL}\label{step-1-calculate-fspl}

\[
\text{FSPL} = 20\log(36 \times 10^6) + 20\log(12 \times 10^9) + 92.45 = 205.5\ \text{dB}
\]

\begin{center}\rule{0.5\linewidth}{0.5pt}\end{center}

\subsubsection{Step 2: Received Power}\label{step-2-received-power}

\[
P_r = P_t + G_t + G_r - \text{FSPL}
\]

\[
P_r = 40 + 30 + 40 - 205.5 = -95.5\ \text{dBm}
\]

\begin{center}\rule{0.5\linewidth}{0.5pt}\end{center}

\subsubsection{Step 3: Noise Power}\label{step-3-noise-power}

\[
N = kT_sB = (1.38 \times 10^{-23})(150)(10^6) = 2.07 \times 10^{-15}\ \text{W} = -117\ \text{dBm}
\]

\begin{center}\rule{0.5\linewidth}{0.5pt}\end{center}

\subsubsection{Step 4: SNR}\label{step-4-snr}

\[
\text{SNR} = P_r - N = -95.5 - (-117) = 21.5\ \text{dB}
\]

\begin{center}\rule{0.5\linewidth}{0.5pt}\end{center}

\subsubsection{Step 5: Check BER
Requirement}\label{step-5-check-ber-requirement}

\textbf{For BPSK}, BER \(= 10^{-6}\) requires:

\[
\frac{E_b}{N_0} \approx 10.5\ \text{dB}
\]

\textbf{Convert SNR to \(E_b/N_0\)}:

\[
\frac{E_b}{N_0} = \text{SNR} + 10\log\left(\frac{B}{R_b}\right)
\]

\textbf{If data rate} \(R_b = 500\) kbps:

\[
\frac{E_b}{N_0} = 21.5 + 10\log\left(\frac{10^6}{5 \times 10^5}\right) = 21.5 + 3 = 24.5\ \text{dB}
\]

\textbf{Margin}: \(24.5 - 10.5 = 14\) dB (comfortable margin for rain
fade, implementation loss)

\textbf{Link closes!}

\begin{center}\rule{0.5\linewidth}{0.5pt}\end{center}

\subsection{Summary}\label{summary}

{\def\LTcaptype{} % do not increment counter
\begin{longtable}[]{@{}ll@{}}
\toprule\noalign{}
Aspect & BPSK \\
\midrule\noalign{}
\endhead
\bottomrule\noalign{}
\endlastfoot
\textbf{Bits per symbol} & 1 \\
\textbf{Constellation points} & 2 (0\$\^{}\textbackslash circ\$,
180\$\^{}\textbackslash circ\$) \\
\textbf{Spectral efficiency} & \textasciitilde1 bps/Hz (with pulse
shaping) \\
\textbf{BER @ 10 dB \(E_b/N_0\)} & 3.9 \$\textbackslash times\$
10\textbackslash textsuperscript\{-\}\textbackslash textsuperscript\{6\} \\
\textbf{Carrier recovery} & Required (Costas loop, squaring loop) \\
\textbf{Complexity} & Moderate (coherent detection) \\
\textbf{Best for} & Power-limited channels (satellite, deep-space) \\
\end{longtable}
}

\begin{center}\rule{0.5\linewidth}{0.5pt}\end{center}

\subsection{Related Topics}\label{related-topics}

\begin{itemize}
\tightlist
\item
  \textbf{{[}{[}On-Off-Keying-(OOK){]}{]}}: Simpler but 3 dB worse
  performance
\item
  \textbf{{[}{[}Frequency-Shift-Keying-(FSK){]}{]}}: Alternative binary
  modulation (non-coherent detection)
\item
  \textbf{{[}{[}QPSK-Modulation{]}{]}}: Extension to 4 phases (2
  bits/symbol)
\item
  \textbf{{[}{[}Constellation-Diagrams{]}{]}}: Visual representation of
  modulation
\item
  \textbf{{[}{[}IQ-Representation{]}{]}}: Complex baseband notation
\item
  \textbf{{[}{[}Bit-Error-Rate-(BER){]}{]}}: Performance metric for
  digital modulation
\item
  \textbf{{[}{[}Forward-Error-Correction-(FEC){]}{]}}: Coding to improve
  BER
\end{itemize}

\begin{center}\rule{0.5\linewidth}{0.5pt}\end{center}

\textbf{Next}: \textbf{8PSK \& Higher-Order Modulation} (TBD) - More
bits per symbol, trades SNR for bandwidth

\begin{center}\rule{0.5\linewidth}{0.5pt}\end{center}

\emph{This wiki is part of the {[}{[}Home\textbar Chimera Project{]}{]}
documentation.}
