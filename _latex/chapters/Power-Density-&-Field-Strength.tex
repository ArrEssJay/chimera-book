\section{Power Density \& Field
Strength}\label{power-density-field-strength}

{[}{[}Home{]}{]} \textbar{} \textbf{EM Fundamentals} \textbar{}
{[}{[}Maxwell\textquotesingle s-Equations-\&-Wave-Propagation{]}{]}
\textbar{} {[}{[}Wave-Polarization{]}{]}

\begin{center}\rule{0.5\linewidth}{0.5pt}\end{center}

\subsection{\texorpdfstring{ For Non-Technical
Readers}{ For Non-Technical Readers}}\label{for-non-technical-readers}

\textbf{Power density is like measuring sunlight intensity-\/-\/-how
much energy hits each square meter. Field strength is like measuring the
``force'' of the electromagnetic wave at a point.}

\textbf{Power Density (W/m\textbackslash textsuperscript\{2\})}: - How
much power passes through 1 square meter? - Like sunlight:
\textasciitilde1000 W/m\textbackslash textsuperscript\{2\} at noon
(hot!) - Radio waves: 0.000001 W/m\textbackslash textsuperscript\{2\} =
1 µW/m\textbackslash textsuperscript\{2\} (typical WiFi)

\textbf{Field Strength (V/m)}: - How strong is the electric field at
this point? - Higher V/m = stronger ``electromagnetic force'' - Think of
it like wind speed vs wind power

\textbf{Real-world examples}:

\textbf{Sunlight} (for comparison): - \textbf{Power density}: 1000
W/m\textbackslash textsuperscript\{2\} at Earth\textquotesingle s
surface - This is why solar panels work!

\textbf{WiFi router} (1 meter away): - \textbf{Power density}:
\textasciitilde10 µW/m\textbackslash textsuperscript\{2\} (0.00001
W/m\textbackslash textsuperscript\{2\}) - \textbf{Field strength}:
\textasciitilde2 V/m - 100 million times weaker than sunlight!

\textbf{Cell tower} (100 meters away): - \textbf{Power density}:
\textasciitilde0.1 µW/m\textbackslash textsuperscript\{2\} -
\textbf{Field strength}: \textasciitilde0.6 V/m - Regulations limit max
exposure to \textasciitilde10 W/m\textbackslash textsuperscript\{2\}

\textbf{Why they\textquotesingle re related}: - \textbf{Power density}
\$\textbackslash propto\$ (Field
strength)\textbackslash textsuperscript\{2\} - Double the field strength
\$\textbackslash rightarrow\$ 4\$\textbackslash times\$ the power
density! - This is why moving closer to WiFi helps so much

\textbf{Inverse square law}: - Double the distance
\$\textbackslash rightarrow\$ \$\textbackslash frac\{1\}\{4\}\$ the
power density - This is why: - WiFi works at 50m but not 200m - Cell
towers need to be closer in cities - Satellites need huge power
(they\textquotesingle re 36,000 km away!)

\textbf{Safety limits}: - \textbf{Sunlight}: 1000
W/m\textbackslash textsuperscript\{2\} (safe for limited time) -
\textbf{FCC RF limit}: 10 W/m\textbackslash textsuperscript\{2\} (safe
for general public) - \textbf{Typical WiFi}: 0.00001
W/m\textbackslash textsuperscript\{2\} (100,000\$\textbackslash times\$
below limit!) - \textbf{Your phone}: 0.001
W/m\textbackslash textsuperscript\{2\} at 1cm
(10,000\$\textbackslash times\$ below limit)

\textbf{When you encounter it}: - \textbf{RF safety assessments}: ``Max
power density: 5 W/m\textbackslash textsuperscript\{2\}'' -
\textbf{Antenna specifications}: ``Field strength: 50 V/m at 1 meter'' -
\textbf{EMC testing}: Measuring field strength for interference -
\textbf{Link budget}: Converting transmit power to received power

\textbf{Fun fact}: The power density from the sun is so high (1000
W/m\textbackslash textsuperscript\{2\}) that if WiFi routers were as
powerful, a 100W router at 1 meter would deliver the same power
density-\/-\/-but would violate FCC limits by
1000\$\textbackslash times\$ and cook you like a microwave oven!

\begin{center}\rule{0.5\linewidth}{0.5pt}\end{center}

\subsection{Overview}\label{overview}

\textbf{Power density} and \textbf{field strength} quantify the
\textbf{intensity of electromagnetic radiation} at a given point in
space.

\textbf{Key relationships}: - \textbf{Field strength} (E, H)
\$\textbackslash rightarrow\$ Measured in V/m, A/m - \textbf{Power
density} (S) \$\textbackslash rightarrow\$ Measured in
W/m\textbackslash textsuperscript\{2\} - \textbf{Relationship}: Power
density proportional to E\textbackslash textsuperscript\{2\}

\textbf{Why it matters}: - \textbf{Link budget calculations}: Determine
received signal strength - \textbf{Safety standards}: RF exposure limits
(FCC, ICNIRP) - \textbf{Antenna performance}: Radiated power
distribution - \textbf{Radar range equation}: Detection capability vs
distance

\begin{center}\rule{0.5\linewidth}{0.5pt}\end{center}

\subsection{Electric Field Strength
(E)}\label{electric-field-strength-e}

\textbf{Electric field} \(\vec{E}\) describes the \textbf{force per unit
charge} exerted on a test charge:

\[
\vec{E} = \frac{\vec{F}}{q} \quad (\text{V/m or N/C})
\]

\textbf{In electromagnetic wave} (plane wave, propagating in +z):

\[
E(z,t) = E_0 \cos(\omega t - kz + \phi)
\]

Where: - \(E_0\) = Peak electric field amplitude (V/m) - Often use
\textbf{RMS value}: \(E_{\text{rms}} = E_0 / \sqrt{2}\)

\begin{center}\rule{0.5\linewidth}{0.5pt}\end{center}

\subsubsection{Typical Values}\label{typical-values}

{\def\LTcaptype{} % do not increment counter
\begin{longtable}[]{@{}lll@{}}
\toprule\noalign{}
Source & Distance & E-field (V/m) \\
\midrule\noalign{}
\endhead
\bottomrule\noalign{}
\endlastfoot
\textbf{AM broadcast} (50 kW) & 1 km & \textasciitilde0.1 \\
\textbf{FM broadcast} (100 kW) & 1 km & \textasciitilde0.2 \\
\textbf{Cell tower} (40 W ERP) & 100 m & \textasciitilde1-2 \\
\textbf{WiFi router} (100 mW) & 1 m & \textasciitilde3 \\
\textbf{Microwave oven} leak & 5 cm & \textasciitilde10-50 (max
allowed) \\
\textbf{Lightning} & Near channel &
\textasciitilde10\textbackslash textsuperscript\{6\} \\
\end{longtable}
}

\begin{center}\rule{0.5\linewidth}{0.5pt}\end{center}

\subsection{Magnetic Field Strength
(H)}\label{magnetic-field-strength-h}

\textbf{Magnetic field} \(\vec{H}\) describes the \textbf{magnetizing
force}:

\[
\vec{H} = \frac{\vec{B}}{\mu} \quad (\text{A/m})
\]

Where: - \(\vec{B}\) = Magnetic flux density (Tesla) - \(\mu\) =
Permeability (H/m)

\textbf{In free space}: \(\mu = \mu_0 = 4\pi \times 10^{-7}\) H/m

\begin{center}\rule{0.5\linewidth}{0.5pt}\end{center}

\subsubsection{Relationship Between E and H (Far
Field)}\label{relationship-between-e-and-h-far-field}

\textbf{In plane wave} (far from source), E and H are related by
\textbf{wave impedance}:

\[
\frac{E}{H} = \eta_0 = \sqrt{\frac{\mu_0}{\epsilon_0}} \approx 377\ \Omega
\]

Where: - \(\eta_0\) = Impedance of free space \$\textbackslash approx\$
120\$\textbackslash pi\$ \$\textbackslash Omega\$
\$\textbackslash approx\$ 377 \$\textbackslash Omega\$ - \(\epsilon_0\)
= Permittivity of free space

\textbf{Practical form}:

\[
H = \frac{E}{377} \quad (\text{A/m})
\]

\textbf{Example}: E = 10 V/m \$\textbackslash rightarrow\$ H = 10/377
\$\textbackslash approx\$ 0.0265 A/m

\begin{center}\rule{0.5\linewidth}{0.5pt}\end{center}

\subsubsection{Near Field vs Far Field}\label{near-field-vs-far-field}

\paragraph{Near Field (Reactive Near
Field)}\label{near-field-reactive-near-field}

\textbf{Distance from antenna}: \(r < 0.62\sqrt{D^3/\lambda}\) (for
large antennas)

Or simpler: \(r < \lambda/(2\pi)\) (for small antennas)

\textbf{Characteristics}: - E and H not in simple ratio (reactive energy
dominates) - Energy oscillates between E-field and H-field storage -
Fields decay faster than \(1/r\) (typically \(1/r^2\) or \(1/r^3\))

\textbf{Example}: HF antenna (3 MHz, \$\textbackslash lambda\$ = 100 m)
at 10 m distance - Near field: E/H \$\textbackslash neq\$ 377
\$\textbackslash Omega\$ - Inductive or capacitive coupling dominates

\begin{center}\rule{0.5\linewidth}{0.5pt}\end{center}

\paragraph{Far Field (Radiating Far
Field)}\label{far-field-radiating-far-field}

\textbf{Distance from antenna}: \(r > 2D^2/\lambda\) (Fraunhofer
distance)

Where D = Largest antenna dimension

\textbf{Characteristics}: - E/H = 377 \$\textbackslash Omega\$ (plane
wave approximation valid) - Radiation pattern independent of distance
(shape constant) - Fields decay as \(1/r\) (power density as \(1/r^2\))

\textbf{Example}: WiFi 2.4 GHz (\$\textbackslash lambda\$ = 12.5 cm),
antenna size D = 5 cm

\[
r_{\text{far}} = \frac{2 \times (0.05)^2}{0.125} = 0.04\ \text{m} = 4\ \text{cm}
\]

\textbf{Far field begins at 4 cm} (very close for WiFi!)

\begin{center}\rule{0.5\linewidth}{0.5pt}\end{center}

\subsection{Power Density (Poynting
Vector)}\label{power-density-poynting-vector}

\textbf{Poynting vector} \(\vec{S}\) represents \textbf{power flow per
unit area}:

\[
\vec{S} = \vec{E} \times \vec{H} \quad (\text{W/m}^2)
\]

\textbf{Magnitude} (for plane wave with E \$\textbackslash perp\$ H):

\[
S = E \cdot H = \frac{E^2}{\eta_0} = \frac{E^2}{377}
\]

Or in terms of H:

\[
S = \eta_0 H^2 = 377 H^2
\]

\begin{center}\rule{0.5\linewidth}{0.5pt}\end{center}

\subsubsection{Time-Averaged Power
Density}\label{time-averaged-power-density}

\textbf{For sinusoidal wave}, instantaneous power oscillates at 2f. Use
\textbf{time-average}:

\[
\langle S \rangle = \frac{1}{2} \frac{E_0^2}{\eta_0} = \frac{E_{\text{rms}}^2}{\eta_0} = \frac{E_{\text{rms}}^2}{377}
\]

\textbf{Example}: E\_rms = 10 V/m

\[
\langle S \rangle = \frac{100}{377} \approx 0.265\ \text{W/m}^2
\]

\begin{center}\rule{0.5\linewidth}{0.5pt}\end{center}

\subsection{Power Density from Isotropic
Source}\label{power-density-from-isotropic-source}

\textbf{Isotropic radiator} distributes power uniformly over sphere:

\[
S = \frac{P_t}{4\pi r^2}
\]

Where: - \(P_t\) = Transmitted power (W) - \(r\) = Distance from source
(m) - \(4\pi r^2\) = Surface area of sphere

\textbf{Inverse square law}: Power density decreases as \(1/r^2\)

\textbf{Example}: 100 W isotropic source at 10 m

\[
S = \frac{100}{4\pi (10)^2} = \frac{100}{1257} \approx 0.0796\ \text{W/m}^2
\]

\begin{center}\rule{0.5\linewidth}{0.5pt}\end{center}

\subsection{Power Density from Directional
Antenna}\label{power-density-from-directional-antenna}

\textbf{Antenna with gain} G concentrates power:

\[
S = \frac{P_t \cdot G}{4\pi r^2}
\]

\textbf{Effective Isotropic Radiated Power (EIRP)}:

\[
\text{EIRP} = P_t \cdot G
\]

\textbf{Power density becomes}:

\[
S = \frac{\text{EIRP}}{4\pi r^2}
\]

\begin{center}\rule{0.5\linewidth}{0.5pt}\end{center}

\subsubsection{Example: WiFi Router}\label{example-wifi-router}

\textbf{Specs}: - Transmit power: 100 mW = 0.1 W - Antenna gain: 2 dBi
(linear gain \$\textbackslash approx\$ 1.58) - Distance: 10 m

\textbf{EIRP}:

\[
\text{EIRP} = 0.1 \times 1.58 = 0.158\ \text{W}
\]

\textbf{Power density at 10 m}:

\[
S = \frac{0.158}{4\pi (10)^2} = \frac{0.158}{1257} \approx 0.000126\ \text{W/m}^2 = 0.126\ \text{mW/m}^2
\]

\textbf{Convert to E-field}:

\[
E_{\text{rms}} = \sqrt{S \times 377} = \sqrt{0.000126 \times 377} \approx 0.218\ \text{V/m}
\]

\begin{center}\rule{0.5\linewidth}{0.5pt}\end{center}

\subsection{Relationship Between Power Density and
E-field}\label{relationship-between-power-density-and-e-field}

\textbf{Summary formulas} (far field, plane wave):

\[
S = \frac{E_{\text{rms}}^2}{377} \quad (\text{W/m}^2)
\]

\[
E_{\text{rms}} = \sqrt{377 \times S} \approx 19.4\sqrt{S} \quad (\text{V/m})
\]

\[
E_0 = \sqrt{2} \times E_{\text{rms}} = \sqrt{2 \times 377 \times S} \approx 27.5\sqrt{S}
\]

\begin{center}\rule{0.5\linewidth}{0.5pt}\end{center}

\subsubsection{Quick Conversion Table}\label{quick-conversion-table}

{\def\LTcaptype{} % do not increment counter
\begin{longtable}[]{@{}lll@{}}
\toprule\noalign{}
Power Density (W/m\textbackslash textsuperscript\{2\}) & E\_rms (V/m) &
E\_peak (V/m) \\
\midrule\noalign{}
\endhead
\bottomrule\noalign{}
\endlastfoot
0.001 (1 mW/m\textbackslash textsuperscript\{2\}) & 0.61 & 0.87 \\
0.01 (10 mW/m\textbackslash textsuperscript\{2\}) & 1.94 & 2.75 \\
0.1 & 6.14 & 8.68 \\
1 & 19.4 & 27.5 \\
10 & 61.4 & 86.8 \\
100 & 194 & 275 \\
\end{longtable}
}

\begin{center}\rule{0.5\linewidth}{0.5pt}\end{center}

\subsection{Power Delivered to Receiving
Antenna}\label{power-delivered-to-receiving-antenna}

\textbf{Effective aperture} \(A_e\) captures power from incident wave:

\[
P_r = S \cdot A_e
\]

Where:

\[
A_e = \frac{G_r \lambda^2}{4\pi}
\]

\begin{itemize}
\tightlist
\item
  \(G_r\) = Receive antenna gain (linear)
\item
  \(\lambda\) = Wavelength
\end{itemize}

\textbf{Combining}:

\[
P_r = \frac{P_t G_t G_r \lambda^2}{(4\pi r)^2}
\]

\textbf{This is the Friis transmission equation} (see
{[}{[}Free-Space-Path-Loss-(FSPL){]}{]})

\begin{center}\rule{0.5\linewidth}{0.5pt}\end{center}

\subsubsection{Example: Satellite
Downlink}\label{example-satellite-downlink}

\textbf{Specs}: - Satellite EIRP: 50 dBW = 100 kW - Frequency: 12 GHz
(\$\textbackslash lambda\$ = 0.025 m) - Distance: 36,000 km (GEO) - RX
antenna gain: 40 dBi (10,000 linear)

\textbf{Power density at ground}:

\[
S = \frac{10^5}{4\pi (3.6 \times 10^7)^2} = \frac{10^5}{1.63 \times 10^{16}} \approx 6.1 \times 10^{-12}\ \text{W/m}^2
\]

\textbf{E-field}:

\[
E_{\text{rms}} = \sqrt{377 \times 6.1 \times 10^{-12}} \approx 1.5 \times 10^{-3}\ \text{V/m} = 1.5\ \text{mV/m}
\]

\textbf{Received power} (1 m\textbackslash textsuperscript\{2\} dish,
A\_e \$\textbackslash approx\$ 0.5
m\textbackslash textsuperscript\{2\}):

\[
P_r = 6.1 \times 10^{-12} \times 0.5 \approx 3 \times 10^{-12}\ \text{W} = 3\ \text{pW}
\]

\textbf{In dBm}: \(10\log_{10}(3 \times 10^{-12} / 10^{-3}) = -115\) dBm

\textbf{Using Friis equation}:

\[
P_r = \frac{100,000 \times 10,000 \times (0.025)^2}{(4\pi \times 3.6 \times 10^7)^2} \approx 3 \times 10^{-12}\ \text{W}
\]

\textbf{Consistent!}

\begin{center}\rule{0.5\linewidth}{0.5pt}\end{center}

\subsection{RF Safety Standards}\label{rf-safety-standards}

\textbf{Exposure limits} protect against thermal and non-thermal
effects:

\subsubsection{FCC Limits (USA)}\label{fcc-limits-usa}

\textbf{Occupational/Controlled Exposure} (aware workers):

{\def\LTcaptype{} % do not increment counter
\begin{longtable}[]{@{}llll@{}}
\toprule\noalign{}
Frequency & E-field (V/m) & H-field (A/m) & Power Density
(W/m\textbackslash textsuperscript\{2\}) \\
\midrule\noalign{}
\endhead
\bottomrule\noalign{}
\endlastfoot
0.3-3 MHz & 614 & 1.63 & - \\
3-30 MHz & 1842/f & 4.89/f & - \\
30-300 MHz & 61.4 & 0.163 & 1.0 \\
300-1500 MHz & - & - & f/300 \\
1500-100,000 MHz & - & - & 5.0 \\
\end{longtable}
}

Where f is in MHz

\textbf{General Population/Uncontrolled Exposure} (public):

Limits are \textbf{5\$\textbackslash times\$ lower} (e.g., 0.2
W/m\textbackslash textsuperscript\{2\} @ 30-300 MHz)

\begin{center}\rule{0.5\linewidth}{0.5pt}\end{center}

\subsubsection{ICNIRP Limits
(International)}\label{icnirp-limits-international}

\textbf{General Public} (6-minute average):

{\def\LTcaptype{} % do not increment counter
\begin{longtable}[]{@{}lll@{}}
\toprule\noalign{}
Frequency & E-field (V/m) & Power Density
(W/m\textbackslash textsuperscript\{2\}) \\
\midrule\noalign{}
\endhead
\bottomrule\noalign{}
\endlastfoot
10-400 MHz & 28 & 2 \\
400-2000 MHz & 1.375\$\textbackslash sqrt\{\}\$f & f/200 \\
2-300 GHz & 61 & 10 \\
\end{longtable}
}

Where f is in MHz

\begin{center}\rule{0.5\linewidth}{0.5pt}\end{center}

\subsubsection{Example: WiFi Router
Compliance}\label{example-wifi-router-compliance}

\textbf{WiFi 2.4 GHz, 100 mW, gain 2 dBi}

\textbf{At 20 cm} (typical human distance):

\[
S = \frac{0.1 \times 1.58}{4\pi (0.2)^2} = \frac{0.158}{0.503} \approx 0.314\ \text{W/m}^2
\]

\textbf{FCC limit @ 2.4 GHz}: 5 W/m\textbackslash textsuperscript\{2\}
(controlled), 1 W/m\textbackslash textsuperscript\{2\} (uncontrolled)

\textbf{ICNIRP limit}: f/200 = 2400/200 = 12
W/m\textbackslash textsuperscript\{2\}

\textbf{Result}: WiFi at 20 cm = 0.314
W/m\textbackslash textsuperscript\{2\} \textbf{\textless{} 1
W/m\textbackslash textsuperscript\{2\}} (OK for public exposure, but
close!)

\textbf{At 1 m}: \(S = 0.0126\) W/m\textbackslash textsuperscript\{2\}
(much safer)

\begin{center}\rule{0.5\linewidth}{0.5pt}\end{center}

\subsection{Radar Power Budget}\label{radar-power-budget}

\textbf{Radar equation} relates transmitted power to received echo:

\[
P_r = \frac{P_t G^2 \lambda^2 \sigma}{(4\pi)^3 R^4}
\]

Where: - \(\sigma\) = Target radar cross-section
(m\textbackslash textsuperscript\{2\}) - R = Range to target (m)

\textbf{Power density at target}:

\[
S_{\text{target}} = \frac{P_t G}{4\pi R^2}
\]

\textbf{Reflected power density back at radar}:

\[
S_{\text{return}} = \frac{S_{\text{target}} \cdot \sigma}{4\pi R^2} = \frac{P_t G \sigma}{(4\pi)^2 R^4}
\]

\textbf{Notice}: \(1/R^4\) dependence (power travels to target and back)

\begin{center}\rule{0.5\linewidth}{0.5pt}\end{center}

\subsubsection{Example: Weather Radar}\label{example-weather-radar}

\textbf{Specs}: - Transmit power: 1 MW (peak) - Antenna gain: 45 dBi
(\$\textbackslash approx\$ 31,600 linear) - Frequency: 3 GHz
(\$\textbackslash lambda\$ = 0.1 m) - Target: Raindrop,
\$\textbackslash sigma\$ =
10\textbackslash textsuperscript\{-\}\textbackslash textsuperscript\{6\}
m\textbackslash textsuperscript\{2\} (light rain) - Range: 100 km

\textbf{Power density at raindrop}:

\[
S_{\text{target}} = \frac{10^6 \times 31,600}{4\pi (10^5)^2} = \frac{3.16 \times 10^{10}}{1.26 \times 10^{11}} \approx 0.25\ \text{W/m}^2
\]

\textbf{Received power}:

\[
P_r = \frac{10^6 \times (31,600)^2 \times (0.1)^2 \times 10^{-6}}{(4\pi)^3 (10^5)^4} \approx 1.6 \times 10^{-13}\ \text{W} = -98\ \text{dBm}
\]

\textbf{Weak but detectable} with sensitive receiver (noise floor
\textasciitilde{} -110 dBm)

\begin{center}\rule{0.5\linewidth}{0.5pt}\end{center}

\subsection{Electromagnetic Interference
(EMI)}\label{electromagnetic-interference-emi}

\textbf{Field strength limits} for conducted and radiated emissions:

\subsubsection{FCC Part 15 Radiated Emission
Limits}\label{fcc-part-15-radiated-emission-limits}

\textbf{Class B} (residential):

{\def\LTcaptype{} % do not increment counter
\begin{longtable}[]{@{}lll@{}}
\toprule\noalign{}
Frequency & E-field @ 3 m (\$\textbackslash mu\$V/m) &
dB\$\textbackslash mu\$V/m \\
\midrule\noalign{}
\endhead
\bottomrule\noalign{}
\endlastfoot
30-88 MHz & 100 & 40 \\
88-216 MHz & 150 & 43.5 \\
216-960 MHz & 200 & 46 \\
Above 960 MHz & 500 & 54 \\
\end{longtable}
}

\textbf{Measurement}: Use calibrated antenna + spectrum analyzer

\begin{center}\rule{0.5\linewidth}{0.5pt}\end{center}

\subsubsection{Example: Spurious Emission
Check}\label{example-spurious-emission-check}

\textbf{Digital device @ 300 MHz, measured 180 \$\textbackslash mu\$V/m
@ 3 m}

\textbf{Limit @ 300 MHz}: 200 \$\textbackslash mu\$V/m

\textbf{Result}: 180 \textless{} 200 \$\textbackslash rightarrow\$
\textbf{Pass}

\textbf{Margin}: \(20\log_{10}(200/180) = 0.9\) dB

\begin{center}\rule{0.5\linewidth}{0.5pt}\end{center}

\subsection{Field Strength in Different
Media}\label{field-strength-in-different-media}

\textbf{In dielectric medium} (not free space):

\[
\eta = \sqrt{\frac{\mu}{\epsilon}} = \frac{\eta_0}{\sqrt{\epsilon_r}}
\]

Where: - \(\epsilon_r\) = Relative permittivity - \(\eta_0 = 377\)
\$\textbackslash Omega\$ (free space)

\textbf{Example}: Water (\(\epsilon_r \approx 80\) @ low freq)

\[
\eta_{\text{water}} = \frac{377}{\sqrt{80}} \approx 42\ \Omega
\]

\textbf{Power density for same E-field}:

\[
S = \frac{E^2}{42}
\]

\textbf{9\$\textbackslash times\$ higher power density} than free space
(for same E-field)

\textbf{Implication}: Underwater communications have different impedance
matching requirements

\begin{center}\rule{0.5\linewidth}{0.5pt}\end{center}

\subsection{Antenna Gain and
Directivity}\label{antenna-gain-and-directivity}

\textbf{Gain} increases power density in preferred direction:

\[
G = \eta_{\text{ant}} \cdot D
\]

Where: - \(\eta_{\text{ant}}\) = Antenna efficiency (0-1) - D =
Directivity (ratio of max to average power density)

\textbf{Directivity}:

\[
D = \frac{S_{\text{max}}}{S_{\text{avg}}} = \frac{4\pi S_{\text{max}} r^2}{P_t}
\]

\textbf{Example}: Isotropic antenna - \(D = 1\) (0 dBi) - Power
uniformly distributed

\textbf{Half-wave dipole}: - \(D = 1.64\) (2.15 dBi) - Power
concentrated in broadside direction

\textbf{Parabolic dish} (diameter D, wavelength
\$\textbackslash lambda\$):

\[
G \approx \eta_{\text{ant}} \left(\frac{\pi D}{\lambda}\right)^2
\]

With \(\eta_{\text{ant}} \approx 0.5-0.7\) (typical)

\begin{center}\rule{0.5\linewidth}{0.5pt}\end{center}

\subsection{Skin Depth and Field
Penetration}\label{skin-depth-and-field-penetration}

\textbf{In conductors}, field decays exponentially:

\[
E(z) = E_0 e^{-z/\delta}
\]

\textbf{Skin depth} \(\delta\):

\[
\delta = \sqrt{\frac{2}{\omega \mu \sigma}} = \sqrt{\frac{1}{\pi f \mu \sigma}}
\]

Where: - \(\sigma\) = Conductivity (S/m) - Copper:
\(\sigma = 5.8 \times 10^7\) S/m

\textbf{Example}: Copper @ 1 GHz

\[
\delta = \sqrt{\frac{1}{\pi \times 10^9 \times 4\pi \times 10^{-7} \times 5.8 \times 10^7}} \approx 2.1\ \mu\text{m}
\]

\textbf{Implication}: At microwave frequencies, current flows in thin
surface layer (\textless{} 2 \$\textbackslash mu\$m)

\begin{center}\rule{0.5\linewidth}{0.5pt}\end{center}

\subsection{Summary Table}\label{summary-table}

{\def\LTcaptype{} % do not increment counter
\begin{longtable}[]{@{}
  >{\raggedright\arraybackslash}p{(\linewidth - 8\tabcolsep) * \real{0.1852}}
  >{\raggedright\arraybackslash}p{(\linewidth - 8\tabcolsep) * \real{0.1481}}
  >{\raggedright\arraybackslash}p{(\linewidth - 8\tabcolsep) * \real{0.1296}}
  >{\raggedright\arraybackslash}p{(\linewidth - 8\tabcolsep) * \real{0.2778}}
  >{\raggedright\arraybackslash}p{(\linewidth - 8\tabcolsep) * \real{0.2593}}@{}}
\toprule\noalign{}
\begin{minipage}[b]{\linewidth}\raggedright
Quantity
\end{minipage} & \begin{minipage}[b]{\linewidth}\raggedright
Symbol
\end{minipage} & \begin{minipage}[b]{\linewidth}\raggedright
Units
\end{minipage} & \begin{minipage}[b]{\linewidth}\raggedright
Typical Range
\end{minipage} & \begin{minipage}[b]{\linewidth}\raggedright
Relationship
\end{minipage} \\
\midrule\noalign{}
\endhead
\bottomrule\noalign{}
\endlastfoot
\textbf{Electric field} & E & V/m & 0.01-1000 &
\(E = \sqrt{377 \times S}\) \\
\textbf{Magnetic field} & H & A/m & 0.00003-3 & \(H = E/377\) \\
\textbf{Power density} & S & W/m\textbackslash textsuperscript\{2\} &
10\textbackslash textsuperscript\{-\}\textbackslash textsuperscript\{6\}
- 10 & \(S = E^2/377\) \\
\textbf{Transmitted power} & \(P_t\) & W & 0.001-100,000 &
\(S = P_t G / (4\pi r^2)\) \\
\textbf{Distance} & r & m & 0.01-10\textbackslash textsuperscript\{8\} &
\(S \propto 1/r^2\) \\
\textbf{Antenna gain} & G & - & 1-10\textbackslash textsuperscript\{6\}
& \(S \propto G\) \\
\end{longtable}
}

\begin{center}\rule{0.5\linewidth}{0.5pt}\end{center}

\subsection{Related Topics}\label{related-topics}

\begin{itemize}
\tightlist
\item
  \textbf{{[}{[}Free-Space-Path-Loss-(FSPL){]}{]}}: Uses power density
  to derive path loss
\item
  \textbf{{[}{[}Antenna-Theory-Basics{]}{]}}: Gain, effective aperture,
  directivity
\item
  \textbf{{[}{[}Maxwell\textquotesingle s-Equations-\&-Wave-Propagation{]}{]}}:
  E and H field derivation
\item
  \textbf{{[}{[}Signal-to-Noise-Ratio-(SNR){]}{]}}: Received power from
  power density
\item
  \textbf{{[}{[}Weather-Effects-(Rain-Fade,-Fog-Attenuation){]}{]}}:
  Power density reduction mechanisms
\end{itemize}

\begin{center}\rule{0.5\linewidth}{0.5pt}\end{center}

\textbf{Key takeaway}: \textbf{Power density S =
E\textbackslash textsuperscript\{2\}/377 in far field}. Follows inverse
square law (\(1/r^2\)) from isotropic source. Directional antennas
concentrate power (multiply by gain). E-field strength and power density
determine link performance and safety compliance. Far field (E/H = 377
\$\textbackslash Omega\$) begins at \(2D^2/\lambda\) from antenna.
Safety limits typically 0.2-10 W/m\textbackslash textsuperscript\{2\}
depending on frequency and exposure type.

\begin{center}\rule{0.5\linewidth}{0.5pt}\end{center}

\emph{This wiki is part of the {[}{[}Home\textbar Chimera Project{]}{]}
documentation.}
