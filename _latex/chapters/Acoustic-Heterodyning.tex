\chapter{Acoustic Heterodyning}
\label{ch:acoustic-heterodyning}

\begin{nontechnical}
\textbf{Imagine mixing colors}---when you mix blue and yellow paint, you get green. Acoustic heterodyning is similar, but with sound waves.

\textbf{The basic idea:}
\begin{itemize}
\item Take two ultrasound beams (too high-pitched for humans to hear, like dog whistles)
\item Aim them at the same spot
\item When they overlap, they ``mix'' together in a nonlinear way
\item This creates a new, audible sound at the \emph{difference} between their frequencies
\end{itemize}

\textbf{Real-world example:} If you send 40,000~Hz and 42,000~Hz ultrasound beams (both inaudible), they can create a 2,000~Hz tone (clearly audible) where they meet.

\textbf{Why this matters:} Museums use directional speakers that beam sound to just one person. Doctors use it for clearer ultrasound images. Scientists are exploring brain stimulation applications.

\textbf{The catch:} Only about 0.1--1\% of the ultrasound energy converts to audible sound, so you need powerful sources.
\end{nontechnical}

\section{Overview}

\textbf{Acoustic heterodyning} (also called parametric acoustic arrays) exploits \textbf{nonlinear acoustics} to generate new frequencies through wave mixing. When two ultrasound beams at frequencies $f_1$ and $f_2$ interact in a nonlinear medium, they produce a difference frequency $f_\Delta = |f_1 - f_2|$ that can be in the audible range.

\begin{keyconcept}
The key principle: \textbf{Nonlinear wave propagation} causes high-frequency ultrasound beams to interact and generate lower-frequency components. This enables highly directional low-frequency sound without requiring large transducer arrays.
\end{keyconcept}

\textbf{Established applications:}
\begin{itemize}
\item Parametric loudspeakers for directional audio
\item Underwater sonar systems
\item Medical harmonic imaging
\end{itemize}

\textbf{Emerging research:}
\begin{itemize}
\item Neural stimulation via focused ultrasound heterodyning
\item Targeted drug delivery using acoustic pressure modulation
\end{itemize}

\subsection{\texorpdfstring{1. Physical Mechanism
}{1. Physical Mechanism }}\label{physical-mechanism}

\textbf{Nonlinear wave equation}:
\[\frac{\partial^2 p}{\partial t^2} - c^2 \nabla^2 p = \frac{\beta}{\rho_0 c^2} \frac{\partial^2 (p^2)}{\partial t^2}\]
where \(\beta\) is nonlinear parameter (\textasciitilde5 for water,
\textasciitilde3.5-6 for tissue).

\textbf{Two-tone input}: \(p_1 \cos\omega_1 t + p_2 \cos\omega_2 t\)\\
\textbf{Nonlinear term} \(p^2\) contains: \(\cos(\omega_1 - \omega_2)t\)
\$\textbackslash rightarrow\$ difference frequency

\textbf{Amplitude}: \(p_\Delta \propto \beta k_1 k_2 A_1 A_2 L\)
(proportional to interaction length)

\textbf{Efficiency}: \textasciitilde0.01-1\% (ultrasound power
\$\textbackslash rightarrow\$ audible power)

\begin{center}\rule{0.5\linewidth}{0.5pt}\end{center}

\subsection{\texorpdfstring{2. Parametric Loudspeakers
}{2. Parametric Loudspeakers }}\label{parametric-loudspeakers}

\textbf{Principle}: Ultrasound beam (e.g., 40 kHz) is highly directional
(\textasciitilde5\$\^{}\textbackslash circ\$ beamwidth). Audible
difference frequency inherits directionality.

\textbf{Commercial products}: Audio Spotlight (Holosonics) -\/-\/-
targeted sound for museums, retail

\textbf{Advantages}: Directional low-frequency sound without large
transducers\\
\textbf{Limitations}: Low efficiency, nonlinear distortion

\begin{center}\rule{0.5\linewidth}{0.5pt}\end{center}

\subsection{\texorpdfstring{3. Medical Harmonic Imaging
}{3. Medical Harmonic Imaging }}\label{medical-harmonic-imaging}

\textbf{Technique}: Transmit ultrasound at \(f_0\), receive at \(2f_0\)
(second harmonic generated by tissue nonlinearity).

\textbf{Clinical use}: Echocardiography, liver imaging (standard
practice)\\
\textbf{Advantage}: Improved resolution and reduced clutter

\begin{center}\rule{0.5\linewidth}{0.5pt}\end{center}

\subsection{\texorpdfstring{4. Biological Tissue Nonlinearity
}{4. Biological Tissue Nonlinearity }}\label{biological-tissue-nonlinearity}

\textbf{Tissue nonlinear parameter}: - Muscle: \(\beta \approx 5.5\)\\
- Fat: \(\beta \approx 10\) (highly nonlinear!)\\
- Liver: \(\beta \approx 6.5\)

\textbf{Consequence}: Tissue is as nonlinear as or more than water
\$\textbackslash rightarrow\$ heterodyning feasible.

\begin{center}\rule{0.5\linewidth}{0.5pt}\end{center}

\subsection{\texorpdfstring{5. Neuromodulation Hypothesis
}{5. Neuromodulation Hypothesis }}\label{neuromodulation-hypothesis}

\textbf{Concept}: Two focused ultrasound beams (\(f_1 \approx f_2\))
cross in brain \$\textbackslash rightarrow\$ difference frequency
\(f_\Delta\) stimulates mechanosensitive ion channels.

\textbf{Advantages} (theoretical): - Spatial selectivity (only overlap
region affected)\\
- Tunable frequency\\
- Non-invasive

\textbf{Challenges}: 1. \textbf{Low efficiency}
\$\textbackslash rightarrow\$ high intensity needed (\textgreater1
W/cm\textbackslash textsuperscript\{2\}) \$\textbackslash rightarrow\$
safety concerns\\
2. \textbf{Skull distortion} \$\textbackslash rightarrow\$ adaptive
focusing required\\
3. \textbf{Mechanism unclear} \$\textbackslash rightarrow\$ thermal
vs.~mechanical?

\textbf{Experimental status} : - \emph{In vitro}: Some calcium signaling
observed (Ye et al., 2018)\\
- \emph{In vivo}: Behavioral changes in mice (mechanism debated)\\
- \emph{Humans}: No transcranial studies published

\begin{center}\rule{0.5\linewidth}{0.5pt}\end{center}

\subsection{\texorpdfstring{6. Ultrasonic Hearing?
}{6. Ultrasonic Hearing? }}\label{ultrasonic-hearing}

\textbf{Hypothesis}: Dual ultrasound beams \$\textbackslash rightarrow\$
heterodyning in cochlea \$\textbackslash rightarrow\$ auditory
perception

\textbf{Evidence}: Mixed\\
- Lin \& Wang (1978): Reported effect\\
- Foster et al.~(1982): Could not replicate (attributed to equipment
artifacts)

\textbf{Consensus}: \textbf{No robust evidence} for ultrasonic
heterodyning in human hearing (confounded by bone conduction,
subharmonics).

\begin{center}\rule{0.5\linewidth}{0.5pt}\end{center}

\subsection{\texorpdfstring{7. Underwater Sonar
}{7. Underwater Sonar }}\label{underwater-sonar}

\textbf{Parametric sonar}: Transmit high-frequency beams
\$\textbackslash rightarrow\$ ocean nonlinearity generates directional
low-frequency sound.

\textbf{Applications}: Submarine communication, bathymetric mapping,
marine mammal deterrence.

\begin{center}\rule{0.5\linewidth}{0.5pt}\end{center}

\subsection{8. Safety}\label{safety}

\textbf{FDA diagnostic ultrasound limits}: \textless720
mW/cm\textbackslash textsuperscript\{2\} (ISPTA)\\
\textbf{Parametric arrays}: Typically \textgreater1
W/cm\textbackslash textsuperscript\{2\} (above diagnostic, below
therapeutic \textasciitilde100 W/cm\textbackslash textsuperscript\{2\})

\textbf{Risks}: Heating (thermal index), cavitation (mechanical index)\\
\textbf{Mitigation}: Lower pressures, avoid gas-filled tissues, monitor
with passive cavitation detection

\begin{center}\rule{0.5\linewidth}{0.5pt}\end{center}

\subsection{9. Mathematical Model: Westervelt
Equation}\label{mathematical-model-westervelt-equation}

\textbf{Analytical solution} (planar wave):
\[p_\Delta(x) = \frac{\beta k_1 k_2 A_1 A_2}{4 \rho_0 c^2 \alpha_\Delta} \left(1 - e^{-\alpha_\Delta x}\right)\]
where \(\alpha_\Delta\) is absorption at difference frequency.

\begin{center}\rule{0.5\linewidth}{0.5pt}\end{center}

\subsection{10. Connections}\label{connections}

\begin{itemize}
\tightlist
\item
  {[}{[}Intermodulation-Distortion-in-Biology{]}{]} -\/-\/- General
  nonlinear mixing\\
\item
  {[}{[}Frey-Microwave-Auditory-Effect{]}{]} -\/-\/- EM-to-acoustic
  (different mechanism)\\
\item
  {[}{[}Non-Linear-Biological-Demodulation{]}{]} -\/-\/- Overview page
\end{itemize}

\begin{center}\rule{0.5\linewidth}{0.5pt}\end{center}

\subsection{11. Key References}\label{key-references}

\begin{enumerate}
\def\labelenumi{\arabic{enumi}.}
\tightlist
\item
  \textbf{Westervelt, \emph{J. Acoust. Soc. Am.} 35, 535 (1963)} -\/-\/-
  Theory\\
\item
  \textbf{Yoneyama et al., \emph{J. Acoust. Soc. Am.} 73, 1532 (1983)}
  -\/-\/- Parametric speaker\\
\item
  \textbf{Duck, \emph{Ultrasound Med. Biol.} 28, 1 (2002)} -\/-\/-
  Tissue nonlinearity\\
\item
  \textbf{Ye et al., \emph{Neuron} 98, 1020 (2018)} -\/-\/-
  Dual-frequency FUS neuromodulation
\end{enumerate}

\begin{center}\rule{0.5\linewidth}{0.5pt}\end{center}

\textbf{Last updated}: October 2025
