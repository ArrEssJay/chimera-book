\chapter{Convolutional Codes \& Viterbi Decoding}
\label{ch:convolutional-codes}

\begin{nontechnical}
\textbf{Convolutional codes + Viterbi decoding is like having a GPS that considers your entire journey to figure out where you really are---even if some GPS samples are noisy!}


\textbf{The Problem:} 
\begin{itemize}
\item Noise corrupts transmitted bits: some 0s become 1s, some 1s become 0s
\item How do you figure out what was actually sent?
\end{itemize}

\textbf{The Convolutional Code Solution---Add memory:}
\begin{enumerate}
\item Instead of encoding each bit independently, the encoder ``remembers'' previous bits
\item Each output bit depends on current + past few input bits
\item This creates patterns---if one bit gets corrupted, the pattern breaks and decoder notices!
\end{enumerate}

\textbf{The GPS Analogy:}
\begin{itemize}
\item \textbf{Bad GPS:} Each position reading is independent. Get noisy reading? Can't tell if it's wrong!
\item \textbf{Smart GPS:} Considers your speed, direction, previous positions. Get noisy reading that says you teleported 5 miles? ``That's impossible, ignore it!''
\end{itemize}

\textbf{Viterbi Decoding---Find the most likely path:}
\begin{itemize}
\item Looks at entire received sequence
\item Considers all possible paths the data could have taken
\item Picks the path that best matches what was received (even with errors!)
\end{itemize}

\textbf{Real-world example---Space probes:}
\begin{itemize}
\item \textbf{Voyager spacecraft:} 15+ billion miles away, incredibly noisy signal
\item Uses convolutional code with Viterbi decoding
\item Can correct errors even when 30--40\% of bits are corrupted!
\item This is why we still get photos from interstellar space!
\end{itemize}

\textbf{Everyday uses:}
\begin{itemize}
\item \textbf{WiFi:} 802.11a/g use convolutional codes
\item \textbf{Satellite TV:} DVB-S uses convolutional + Viterbi
\item \textbf{GSM (2G):} Your old cell phone used this
\item \textbf{GPS signals:} Navigation satellites use convolutional codes
\end{itemize}

\textbf{Why it works:}
\begin{itemize}
\item Pattern-based: Errors break patterns, decoder spots them
\item Context-aware: Uses past data to correct current data
\item Optimal: Viterbi finds the single best answer (mathematically proven!)
\end{itemize}

\textbf{Trade-off:} More memory = better error correction BUT more complex decoder. Most systems use 3--7 bits of memory (called ``constraint length'').

\textbf{Fun fact:} Andrew Viterbi invented this algorithm in 1967 for deep space communications---then co-founded Qualcomm, making billions from the algorithm used in every cell phone!
\end{nontechnical}

\section{Overview}

\textbf{Convolutional codes} are a class of error-correcting codes that encode data \textbf{continuously} rather than in fixed blocks. Unlike block codes that treat each codeword independently, convolutional codes maintain \textbf{memory} of previous input bits, enabling powerful error correction through pattern-based decoding.

\begin{keyconcept}
The fundamental distinction between block codes and convolutional codes lies in \textbf{memory}:
\begin{itemize}
\item \textbf{Block codes:} Encode $k$ bits $\rightarrow$ $n$ bits independently (memoryless)
\item \textbf{Convolutional codes:} Each output depends on current + \textbf{previous} input bits (memory)
\end{itemize}

This memory creates \textbf{structural dependencies} that enable the Viterbi algorithm to perform optimal maximum-likelihood (ML) decoding efficiently.
\end{keyconcept}

\textbf{Applications:} Convolutional codes are ubiquitous in modern communications:
\begin{itemize}
\item \textbf{Satellite:} DVB-S, GPS navigation signals
\item \textbf{Wireless:} WiFi (802.11a/g), LTE control channels
\item \textbf{Deep space:} Voyager spacecraft, Mars rovers
\item \textbf{Mobile:} GSM (2G), early 3G systems
\end{itemize}

\textbf{Key advantages:}
\begin{itemize}
\item Excellent performance with soft-decision decoding
\item Low latency (suitable for streaming applications)
\item Optimal ML decoding via Viterbi algorithm
\item Easily punctured to achieve variable code rates
\end{itemize}

\section{Fundamental Parameters}

\subsection{Constraint Length ($K$)}

The \textbf{constraint length} $K$ defines the number of input bits that influence each output bit:
\begin{equation}
K = \text{number of input bits affecting output}
\label{eq:constraint-length}
\end{equation}

The encoder maintains \textbf{memory} through a shift register with:
\begin{equation}
m = K - 1 \quad \text{(shift register stages)}
\label{eq:memory-order}
\end{equation}
where:
\begin{itemize}
\item $K$ = constraint length (total influencing bits)
\item $m$ = memory order (number of previous bits stored)
\end{itemize}

\textbf{Example:} For $K = 3$, the current bit plus 2 previous bits (3 total) determine each output.

The number of encoder states is:
\begin{equation}
N_{\text{states}} = 2^{K-1} = 2^m
\label{eq:num-states}
\end{equation}

\subsection{Code Rate ($r$)}

The \textbf{code rate} quantifies redundancy:
\begin{equation}
r = \frac{k}{n}
\label{eq:code-rate}
\end{equation}
where:
\begin{itemize}
\item $k$ = number of input bits per time step
\item $n$ = number of output bits per time step
\item $r$ = code rate (fraction of information bits)
\end{itemize}

\textbf{Common code rates:}
\begin{itemize}
\item $r = 1/2$: 1 input bit $\rightarrow$ 2 output bits (strong coding, deep space)
\item $r = 1/3$: 1 input bit $\rightarrow$ 3 output bits (very strong coding, low SNR)
\item $r = 2/3$: 2 input bits $\rightarrow$ 3 output bits (punctured from $r = 1/2$)
\item $r = 3/4$: 3 input bits $\rightarrow$ 4 output bits (high throughput)
\end{itemize}

Lower code rates provide more redundancy and stronger error correction at the cost of reduced data throughput.

\section{Encoder Structure}

A convolutional encoder consists of:
\begin{enumerate}
\item \textbf{Shift register:} Stores $K-1$ previous input bits
\item \textbf{Modulo-2 adders (XOR gates):} Combine register contents
\item \textbf{Generator polynomials:} Define connections from register to XORs
\end{enumerate}

\subsection{Generic Encoder Architecture}

\begin{center}
\begin{tikzpicture}[
  block/.style={rectangle, draw, minimum width=1cm, minimum height=0.8cm, font=\sffamily\small},
  node distance=1.5cm,
  font=\small
]

% Shift register stages
\node[block] (d0) {$D_0$};
\node[block, right of=d0] (d1) {$D_1$};
\node[block, right of=d1] (d2) {$D_2$};

% Input
\node[left of=d0, node distance=2cm] (input) {\sffamily Input};
\draw[->,thick] (input) -- node[above,font=\scriptsize] {$u_k$} (d0);

% Connections between stages
\draw[->,thick] (d0) -- (d1);
\draw[->,thick] (d1) -- (d2);

% XOR gates for output 1
\node[below of=d1, node distance=2cm, font=\small] (xor1) {$\bigoplus$};
\node[circle, draw, inner sep=0pt, minimum size=6pt, fill=black] at ($(d0.south)+(0,-0.3)$) (tap1a) {};
\node[circle, draw, inner sep=0pt, minimum size=6pt, fill=black] at ($(d1.south)+(0,-0.3)$) (tap1b) {};
\node[circle, draw, inner sep=0pt, minimum size=6pt, fill=black] at ($(d2.south)+(0,-0.3)$) (tap1c) {};
\draw[-] (d0.south) -- (tap1a);
\draw[-] (d1.south) -- (tap1b);
\draw[-] (d2.south) -- (tap1c);
\draw[-] (tap1a) |- (xor1);
\draw[-] (tap1b) -- (xor1);
\draw[-] (tap1c) |- (xor1);

% Output 1
\node[below of=xor1, node distance=0.8cm] (out1) {\sffamily $v_1$};
\draw[->,thick] (xor1) -- (out1);

% XOR gates for output 2
\node[below of=d0, node distance=3.5cm, font=\small] (xor2) {$\bigoplus$};
\node[circle, draw, inner sep=0pt, minimum size=6pt, fill=black] at ($(d0.south)+(0,-2.2)$) (tap2a) {};
\node[circle, draw, inner sep=0pt, minimum size=6pt, fill=black] at ($(d2.south)+(0,-2.2)$) (tap2c) {};
\draw[-] (tap1a) -- (tap2a);
\draw[-] (tap1c) -- (tap2c);
\draw[-] (tap2a) |- (xor2);
\draw[-] (tap2c) |- (xor2);

% Output 2
\node[below of=xor2, node distance=0.8cm] (out2) {\sffamily $v_2$};
\draw[->,thick] (xor2) -- (out2);

% Labels
\node[above=3pt of d0, font=\scriptsize] {Current};
\node[above=3pt of d1, font=\scriptsize] {Previous};
\node[above=3pt of d2, font=\scriptsize] {Previous};

% Title
\node[above=15pt of d1, font=\sffamily\bfseries] {Rate $r=1/2$, Constraint Length $K=3$};

\end{tikzpicture}
\end{center}

\textbf{Generator polynomials} define the connections:
\begin{itemize}
\item $g_1 = 111_2$ (binary): All three stages connected to XOR$_1$ $\rightarrow$ Output $v_1$
\item $g_2 = 101_2$ (binary): Stages $D_0$ and $D_2$ connected to XOR$_2$ $\rightarrow$ Output $v_2$
\end{itemize}

The generator polynomial determines which shift register taps feed each modulo-2 adder.

\section{Industry Standard: NASA Code}

\subsection{NASA Standard ($r=1/2$, $K=7$)}

The \textbf{NASA standard convolutional code} is one of the most widely deployed error-correcting codes, used in deep-space missions including Voyager, Cassini, and Mars rovers. This code achieves excellent performance with manageable complexity.

\textbf{Parameters:}
\begin{itemize}
\item \textbf{Code rate:} $r = 1/2$ (1 input bit $\rightarrow$ 2 output bits)
\item \textbf{Constraint length:} $K = 7$ (7-bit influence span)
\item \textbf{Memory:} $m = 6$ (6 shift register stages)
\item \textbf{Number of states:} $2^6 = 64$ states
\end{itemize}

\textbf{Generator polynomials} (octal notation):
\begin{align}
g_1 &= 171_8 = 1111001_2 \label{eq:nasa-g1} \\
g_2 &= 133_8 = 1011011_2 \label{eq:nasa-g2}
\end{align}
where:
\begin{itemize}
\item $g_1$ = generator for output $Y_1$ (stages 0,1,2,3,6 connected)
\item $g_2$ = generator for output $Y_2$ (stages 0,2,3,5,6 connected)
\end{itemize}

\begin{calloutbox}{Octal to Binary Conversion}
The octal notation is standard in convolutional code specifications:
\begin{itemize}
\item $171_8 = 001\ 111\ 001_2 = 1111001_2$ (trim leading zeros)
\item $133_8 = 001\ 011\ 011_2 = 1011011_2$
\end{itemize}

Each octal digit (0--7) represents exactly 3 binary bits, making it convenient for generator polynomial specification.
\end{calloutbox}

\subsection{Encoding Example (Simplified $K=3$ Code)}

To illustrate the encoding process, consider a simpler $K=3$ code with generators $g_1 = 111_2$ and $g_2 = 101_2$.

\textbf{Input data:} $101_2$ (3 bits)

\textbf{Initial state:} All zeros $[00]$ (2-stage shift register for $K=3$)

\begin{center}
\begin{tabular}{@{}cccccc@{}}
\toprule
\textbf{Time} & \textbf{Input} & \textbf{State} & \textbf{$Y_1$} & \textbf{$Y_2$} & \textbf{Output} \\
\midrule
0 & 1 & 10 & 1 & 1 & 11 \\
1 & 0 & 01 & 1 & 0 & 10 \\
2 & 1 & 10 & 0 & 1 & 01 \\
3 (flush) & 0 & 01 & 1 & 0 & 10 \\
4 (flush) & 0 & 00 & 1 & 1 & 11 \\
\bottomrule
\end{tabular}
\end{center}

\textbf{Encoded output:} $11\ 10\ 01\ 10\ 11$ (10 bits for 3 input bits + 2 flush bits)

\textbf{Code rate achieved:} $r = 3/(3+2) \times 2 = 3/10 = 0.3$ (effectively, including flush overhead)

\begin{warningbox}
\textbf{Termination (flushing) is critical.} The encoder must be returned to the all-zeros state by appending $K-1$ zero bits. This ensures the Viterbi decoder can reliably identify the final state, preventing decoding errors at the end of the sequence.

For $K=7$ codes, 6 flush bits are required. For short messages, this overhead can be significant.
\end{warningbox}

\subsection{State Diagram}\label{state-diagram}

\textbf{States}: All possible shift register contents

\textbf{For K=3}: \(2^{K-1} = 2^2 = 4\) states - State 00, State 01,
State 10, State 11

\textbf{Transitions}: Input bit determines next state

\textbf{Example (r=1/2, K=3, g1=111, g2=101)}:

\begin{verbatim}
State diagram:

   00 --0/00--> 00
    |  --1/11--> 10
    
   01 --0/11--> 00
    |  --1/00--> 10
    
   10 --0/10--> 01
    |  --1/01--> 11
    
   11 --0/01--> 01
    |  --1/10--> 11
\end{verbatim}

\textbf{Notation}: Input/Output (e.g., ``1/11'' = input 1 produces
output 11)

\begin{center}\rule{0.5\linewidth}{0.5pt}\end{center}

\section{Trellis Diagram}

The \textbf{trellis diagram} is the state diagram \textbf{unrolled in time}, showing all possible encoder state transitions over multiple time steps. It is the fundamental structure for Viterbi decoding.

\subsection{Trellis Structure ($K=3$ Example)}

\begin{center}
\begin{tikzpicture}[
  scale=1.2,
  every node/.style={font=\small},
  state/.style={circle, draw, minimum size=6pt, inner sep=1pt, fill=white},
  solid/.style={thick, ->,>=stealth},
  dashed/.style={thick, ->,>=stealth, dashed}
]

% Time labels
\foreach \t in {0,1,2,3,4} {
  \node[font=\scriptsize] at (\t*2.5, 3.5) {$t=\t$};
}

% State labels
\node[font=\scriptsize, left] at (-0.3, 3) {00};
\node[font=\scriptsize, left] at (-0.3, 2) {01};
\node[font=\scriptsize, left] at (-0.3, 1) {10};
\node[font=\scriptsize, left] at (-0.3, 0) {11};

% Draw states at each time
\foreach \t in {0,1,2,3,4} {
  \foreach \s in {0,1,2,3} {
    \node[state] (s\s t\t) at (\t*2.5, 3-\s) {};
  }
}

% Draw transitions (input 0 = solid, input 1 = dashed)
% From state 00
\foreach \t in {0,1,2,3} {
  \pgfmathtruncatemacro{\tnext}{\t+1}
  \draw[solid] (s0t\t) -- (s0t\tnext) node[midway,above,font=\tiny] {0/00};
  \draw[dashed] (s0t\t) -- (s2t\tnext) node[midway,below,font=\tiny] {1/11};
}

% From state 01
\foreach \t in {0,1,2,3} {
  \pgfmathtruncatemacro{\tnext}{\t+1}
  \draw[solid] (s1t\t) -- (s0t\tnext) node[midway,above,font=\tiny] {0/11};
  \draw[dashed] (s1t\t) -- (s2t\tnext) node[midway,below,font=\tiny] {1/00};
}

% From state 10
\foreach \t in {0,1,2,3} {
  \pgfmathtruncatemacro{\tnext}{\t+1}
  \draw[solid] (s2t\t) -- (s1t\tnext) node[midway,above,font=\tiny] {0/10};
  \draw[dashed] (s2t\t) -- (s3t\tnext) node[midway,below,font=\tiny] {1/01};
}

% From state 11
\foreach \t in {0,1,2,3} {
  \pgfmathtruncatemacro{\tnext}{\t+1}
  \draw[solid] (s3t\t) -- (s1t\tnext) node[midway,above,font=\tiny] {0/01};
  \draw[dashed] (s3t\t) -- (s3t\tnext) node[midway,below,font=\tiny] {1/10};
}

% Legend
\node[font=\scriptsize, align=left] at (5, -0.8) {
  \tikz{\draw[solid] (0,0) -- (0.4,0);} Input = 0\\
  \tikz{\draw[dashed] (0,0) -- (0.4,0);} Input = 1\\
  Labels: input/output
};

\end{tikzpicture}
\end{center}

\textbf{Key observations:}
\begin{itemize}
\item Each \textbf{state node} represents the shift register contents at time $t$
\item Each \textbf{branch} represents a state transition caused by an input bit
\item \textbf{Branch labels} show input/output (e.g., ``1/11'' = input 1 produces output 11)
\item A \textbf{path through the trellis} represents a complete encoded sequence
\item The encoder must start at state 00 (after reset/flush)
\end{itemize}

\textbf{Decoding problem:} Given received sequence $r_1, r_2, \ldots, r_N$, find the \textbf{most likely path} through the trellis. The Viterbi algorithm solves this optimally.

\section{Viterbi Algorithm}

The \textbf{Viterbi algorithm}, invented by Andrew J. Viterbi in 1967, performs \textbf{optimal maximum-likelihood (ML) decoding} for convolutional codes. It efficiently searches the exponentially large space of possible paths through the trellis.

\begin{keyconcept}
The Viterbi algorithm exploits the \textbf{Markov property} of convolutional codes: the optimal path to any state at time $t+1$ must include the optimal path to some state at time $t$.

This enables \textbf{dynamic programming}: instead of examining all $2^L$ possible sequences, we only track $2^{K-1}$ survivor paths at each time step.
\end{keyconcept}

\subsection{Algorithm Complexity}

\textbf{Computational complexity:}
\begin{equation}
\mathcal{O}(2^{K-1} \cdot L \cdot n)
\label{eq:viterbi-complexity}
\end{equation}
where:
\begin{itemize}
\item $K$ = constraint length
\item $L$ = sequence length (number of time steps)
\item $n$ = number of outputs per time step (code rate denominator)
\end{itemize}

\textbf{Practical limits:} The algorithm is efficient for $K \leq 9$. Beyond $K=10$, the exponential state space becomes prohibitive ($2^9 = 512$ states vs. $2^{10} = 1024$ states).

\subsection{Core Principle}

At each time step $t$, for each possible state $s$:
\begin{enumerate}
\item Compute \textbf{branch metrics} for all incoming transitions
\item Select the \textbf{survivor path} with minimum cumulative metric
\item Store the survivor path and its metric
\end{enumerate}

After processing all received symbols, \textbf{traceback} from the best final state to recover the decoded sequence.

\subsection{Complete Algorithm Steps}

\textbf{Step 1: Initialization}
\begin{itemize}
\item Set $\text{PM}_0(00) = 0$ (start at all-zeros state)
\item Set $\text{PM}_0(s) = \infty$ for all other states $s \neq 00$
\item Initialize survivor path storage
\end{itemize}

\textbf{Step 2: Forward Pass (for $t = 1$ to $L$)}

For each time step $t$:
\begin{enumerate}
\item \textbf{For each state $s$:}
   \begin{enumerate}
   \item Identify all valid predecessor states $s'$
   \item Compute branch metrics $\text{BM}_t(s' \to s)$
   \item Compute candidate path metrics: $\text{PM}_{t-1}(s') + \text{BM}_t(s' \to s)$
   \item Select minimum: $\text{PM}_t(s) = \min_{s'} [\text{PM}_{t-1}(s') + \text{BM}_t(s' \to s)]$
   \item Store survivor path: record which $s'$ achieved the minimum
   \end{enumerate}
\end{enumerate}

\textbf{Step 3: Termination}
\begin{itemize}
\item If encoder was flushed: select state 00 at time $L$
\item If not flushed: select state with minimum $\text{PM}_L(s)$
\end{itemize}

\textbf{Step 4: Traceback}
\begin{enumerate}
\item Start at selected final state $s_L$
\item Follow survivor paths backward: $s_L \to s_{L-1} \to \cdots \to s_1 \to s_0$
\item Extract input bits that caused each state transition
\item Reverse the sequence to obtain decoded bits
\end{enumerate}

\textbf{Step 5: Output}

Return decoded bit sequence (excluding flush bits if applicable).

\subsection{Branch Metrics}

The \textbf{branch metric} quantifies the ``distance'' between the received symbol and the expected output for a particular state transition.

\subsubsection{Hard-Decision Decoding (Hamming Distance)}

For binary demodulator output ($r_i \in \{0, 1\}$):
\begin{equation}
\text{BM}_{\text{hard}}(s' \to s) = \sum_{i=1}^{n} (r_i \oplus c_i)
\label{eq:branch-metric-hard}
\end{equation}
where:
\begin{itemize}
\item $r_i$ = received bit (0 or 1)
\item $c_i$ = expected output bit for transition $s' \to s$
\item $\oplus$ = modulo-2 addition (XOR)
\item $n$ = number of output bits per symbol
\end{itemize}

This counts the number of bit errors (\textbf{Hamming distance}).

\subsubsection{Soft-Decision Decoding (Euclidean Distance)}

For analog demodulator output ($r_i \in \mathbb{R}$):
\begin{equation}
\text{BM}_{\text{soft}}(s' \to s) = \sum_{i=1}^{n} (r_i - \hat{c}_i)^2
\label{eq:branch-metric-soft}
\end{equation}
where:
\begin{itemize}
\item $r_i \in \mathbb{R}$ = soft received value (e.g., $+3.2$ or $-1.7$)
\item $\hat{c}_i \in \{+1, -1\}$ = expected output mapped to antipodal signals
\end{itemize}

For BPSK modulation: $\hat{c}_i = 2c_i - 1$ maps $\{0,1\}$ to $\{-1,+1\}$.

\begin{keyconcept}
\textbf{Soft-decision decoding provides approximately 2--3 dB coding gain} over hard-decision at the same BER. This is ``free'' performance improvement requiring only analog values from the demodulator instead of binary decisions.

The key insight: preserving \textbf{reliability information} (how confident the demodulator is) enables better decoding decisions.
\end{keyconcept}

\subsection{Path Metric and Survivor Selection}

The \textbf{path metric} is the cumulative distance from the starting state to state $s$ at time $t$:
\begin{equation}
\text{PM}_t(s) = \min_{s'} \left[ \text{PM}_{t-1}(s') + \text{BM}_t(s' \to s) \right]
\label{eq:path-metric}
\end{equation}
where:
\begin{itemize}
\item $\text{PM}_t(s)$ = cumulative path metric to state $s$ at time $t$
\item $\text{PM}_{t-1}(s')$ = path metric to previous state $s'$ at time $t-1$
\item $\text{BM}_t(s' \to s)$ = branch metric for transition $s' \to s$ at time $t$
\item $\min_{s'}$ = minimize over all valid predecessor states $s'$
\end{itemize}

The \textbf{survivor path} to state $s$ is the path achieving the minimum in Equation \ref{eq:path-metric}.

\textbf{Add-Compare-Select (ACS) operation:}
\begin{enumerate}
\item \textbf{Add:} Compute $\text{PM}_{t-1}(s') + \text{BM}_t(s' \to s)$ for each predecessor
\item \textbf{Compare:} Find the minimum among all candidates
\item \textbf{Select:} Choose the survivor path and update $\text{PM}_t(s)$
\end{enumerate}

This ACS operation is the computational core of the Viterbi algorithm, performed $2^{K-1}$ times per time step (once for each state).

\subsection{Worked Example: Hard-Decision Viterbi Decoding}

\textbf{Given:}
\begin{itemize}
\item Code: $K=3$, $r=1/2$, generators $g_1=111$, $g_2=101$
\item Received sequence: $11\ 10\ 01\ 11\ 00$
\item Assumptions: Start state 00, end state 00 (flushed)
\end{itemize}

\textbf{Step-by-step decoding:}

\textbf{Time 0: Initialization}
\begin{itemize}
\item $\text{PM}_0(00) = 0$
\item $\text{PM}_0(s) = \infty$ for all $s \neq 00$
\end{itemize}

\textbf{Time 1: Received $11$}
\begin{itemize}
\item From state 00, input 0 produces 00: Hamming distance from 11 = 2
\item From state 00, input 1 produces 11: Hamming distance from 11 = 0
\item Update: $\text{PM}_1(00) = 0 + 2 = 2$, $\text{PM}_1(10) = 0 + 0 = 0$
\item Survivor: State 10 is most likely
\end{itemize}

\textbf{Continue for all time steps...}

The Viterbi algorithm continues this process:
\begin{enumerate}
\item At each time, compute branch metrics for all possible transitions
\item Select survivor path to each state (minimum cumulative metric)
\item Store survivor paths
\end{enumerate}

\textbf{Final: Traceback}

After processing all received symbols, traceback from the state with minimum $\text{PM}_L$ to recover the most likely transmitted sequence.

\section{Free Distance and Error Correction Capability}

\subsection{Definition of Free Distance}

The \textbf{free distance} $d_{\text{free}}$ is the minimum Hamming distance between \textbf{any two distinct encoded sequences} that diverge from and later remerge to the same state in the trellis.

\begin{equation}
d_{\text{free}} = \min_{i \neq j} d_H(c^{(i)}, c^{(j)})
\label{eq:free-distance}
\end{equation}
where:
\begin{itemize}
\item $c^{(i)}$ and $c^{(j)}$ are any two distinct codeword sequences
\item $d_H(\cdot, \cdot)$ is the Hamming distance
\end{itemize}

\subsection{Error Correction Capability}

The free distance determines the \textbf{guaranteed error correction capability}:
\begin{equation}
t_{\text{correct}} = \left\lfloor \frac{d_{\text{free}} - 1}{2} \right\rfloor
\label{eq:error-correction-capability}
\end{equation}

The code can \textbf{always correct} up to $t_{\text{correct}}$ errors. It can \textbf{detect} up to $d_{\text{free}} - 1$ errors.

\subsection{Common Convolutional Codes}

\begin{center}
\begin{tabular}{@{}lccccc@{}}
\toprule
\textbf{Code} & \textbf{$K$} & \textbf{Rate} & \textbf{$d_{\text{free}}$} & \textbf{$t$} & \textbf{States} \\
\midrule
(5, 7) & 3 & 1/2 & 5 & 2 & 4 \\
(171, 133) & 7 & 1/2 & 10 & 4 & 64 \\
(561, 753) & 9 & 1/2 & 12 & 5 & 256 \\
(1167, 1375, 1545) & 9 & 1/3 & 18 & 8 & 256 \\
\bottomrule
\end{tabular}
\end{center}

\textbf{Observations:}
\begin{itemize}
\item \textbf{Larger $K$}: Higher $d_{\text{free}}$ $\rightarrow$ better error correction
\item \textbf{Lower rate}: Higher $d_{\text{free}}$ at same $K$ (more redundancy)
\item \textbf{Trade-off}: Larger $K$ $\rightarrow$ more states $\rightarrow$ higher decoder complexity
\end{itemize}

\begin{calloutbox}{Why NASA Chose $K=7$}
The NASA standard $(171, 133)$ with $K=7$ represents an optimal balance:
\begin{itemize}
\item \textbf{Performance:} $d_{\text{free}} = 10$ provides excellent error correction
\item \textbf{Complexity:} 64 states is manageable even in 1970s hardware
\item \textbf{Memory:} Only 6 bits of shift register required
\item \textbf{Proven:} Decades of successful deployment in deep-space missions
\end{itemize}

Moving to $K=9$ would double the number of states (256) for only a 20\% improvement in $d_{\text{free}}$ (10 $\to$ 12).
\end{calloutbox}

\section{Performance Analysis}

\subsection{Bit Error Rate (BER)}

For BPSK modulation over AWGN channel with hard-decision Viterbi decoding:
\begin{equation}
P_b \approx \sum_{d=d_{\text{free}}}^{\infty} \beta_d \cdot Q\left(\sqrt{2 d r \frac{E_b}{N_0}}\right)
\label{eq:ber-convolutional}
\end{equation}
where:
\begin{itemize}
\item $P_b$ = bit error probability
\item $\beta_d$ = number of information bit errors for paths at Hamming distance $d$
\item $r$ = code rate
\item $E_b/N_0$ = energy per information bit to noise spectral density ratio
\item $Q(x) = \frac{1}{\sqrt{2\pi}} \int_x^\infty e^{-t^2/2} dt$ = Gaussian Q-function
\end{itemize}

\textbf{High SNR approximation:} At large $E_b/N_0$, the BER is dominated by the $d_{\text{free}}$ term:
\begin{equation}
P_b \approx \beta_{d_{\text{free}}} \cdot Q\left(\sqrt{2 d_{\text{free}} r \frac{E_b}{N_0}}\right)
\label{eq:ber-high-snr}
\end{equation}

\subsection{Coding Gain}

The \textbf{asymptotic coding gain} (compared to uncoded BPSK) is:
\begin{equation}
G_c = 10 \log_{10}(r \cdot d_{\text{free}}) \quad \text{(dB)}
\label{eq:coding-gain}
\end{equation}

\textbf{Worked Example: NASA Code $(171, 133)$}

Given:
\begin{itemize}
\item $K = 7$, $r = 1/2$, $d_{\text{free}} = 10$
\end{itemize}

Coding gain:
\begin{equation}
G_c = 10 \log_{10}(0.5 \times 10) = 10 \log_{10}(5) = 6.99 \approx 7.0 \text{ dB}
\end{equation}

\textbf{With soft-decision decoding:} Add approximately 2--2.5 dB additional gain:
\begin{equation}
G_{c,\text{soft}} \approx 7.0 + 2.0 = 9.0 \text{ dB}
\end{equation}

This means at BER $= 10^{-5}$, the coded system requires 9 dB \textbf{less} $E_b/N_0$ than uncoded BPSK.

\subsection{BER Performance Comparison}

\begin{center}
\begin{tabular}{@{}lrrr@{}}
\toprule
\textbf{$E_b/N_0$ (dB)} & \textbf{Uncoded BPSK} & \textbf{Conv (hard)} & \textbf{Conv (soft)} \\
\midrule
2 & $2.4 \times 10^{-2}$ & $7 \times 10^{-3}$ & $2 \times 10^{-3}$ \\
4 & $1.2 \times 10^{-3}$ & $3 \times 10^{-4}$ & $5 \times 10^{-5}$ \\
6 & $2.4 \times 10^{-5}$ & $2 \times 10^{-6}$ & $1 \times 10^{-7}$ \\
8 & $1.9 \times 10^{-7}$ & $5 \times 10^{-9}$ & $5 \times 10^{-10}$ \\
\bottomrule
\end{tabular}
\end{center}

\begin{keyconcept}
\textbf{Soft-decision Viterbi decoding provides approximately 2--2.5 dB gain} over hard-decision at the same BER. This is achieved by utilizing the \textbf{reliability information} from the demodulator (analog samples) rather than making hard binary decisions.

For systems where every dB matters (satellite, deep-space), this ``free'' gain is critical.
\end{keyconcept}

\section{Puncturing: Variable Rate Codes}

\textbf{Puncturing} is a technique to increase code rate by systematically \textbf{deleting} (not transmitting) selected output bits from a low-rate parent code.

\subsection{How Puncturing Works}

\textbf{Example:} Transform $r=1/2$ code into $r=2/3$ code.

\textbf{Original code:} 1 input bit $\rightarrow$ 2 output bits $(Y_1, Y_2)$

\textbf{Puncturing pattern (period 2):}

\begin{center}
\begin{tabular}{@{}ccccc@{}}
\toprule
\textbf{Period} & \textbf{Input} & \textbf{$Y_1$} & \textbf{$Y_2$} & \textbf{Transmitted} \\
\midrule
1 & bit 1 & \checkmark & \checkmark & $Y_1, Y_2$ \\
2 & bit 2 & \checkmark & \texttimes & $Y_1$ only \\
\bottomrule
\end{tabular}
\end{center}

\textbf{Result:} 2 input bits $\rightarrow$ 3 output bits (rate $r = 2/3$)

\textbf{Puncturing matrix:}
\begin{equation}
P = \begin{bmatrix} 1 & 1 \\ 1 & 0 \end{bmatrix}
\label{eq:puncturing-matrix}
\end{equation}
where:
\begin{itemize}
\item 1 = transmit this bit
\item 0 = delete (puncture) this bit
\item Rows correspond to outputs $(Y_1, Y_2)$
\item Columns correspond to time periods
\end{itemize}

\subsection{Common Punctured Rates}

From $r=1/2$ parent code:

\begin{center}
\begin{tabular}{@{}lccc@{}}
\toprule
\textbf{Target Rate} & \textbf{Period} & \textbf{Puncturing Matrix} & \textbf{Used In} \\
\midrule
2/3 & 2 & $\begin{bmatrix} 1 & 1 \\ 1 & 0 \end{bmatrix}$ & WiFi, LTE \\[0.3em]
3/4 & 3 & $\begin{bmatrix} 1 & 1 & 0 \\ 1 & 0 & 1 \end{bmatrix}$ & WiFi, DVB \\[0.3em]
5/6 & 5 & (5-column pattern) & DVB-S2 \\
7/8 & 7 & (7-column pattern) & DVB-S \\
\bottomrule
\end{tabular}
\end{center}

\textbf{Trade-off:} Higher code rates increase throughput but reduce error correction capability (lower effective $d_{\text{free}}$).

\section{Advanced Techniques}

\subsection{Tail-Biting}

\textbf{Problem with standard encoding:} Termination requires appending $K-1$ zero bits to flush the encoder to state 00. For a message of length $L$ bits, this creates overhead:
\begin{equation}
\text{Overhead} = \frac{K-1}{L} \times 100\%
\end{equation}

For short packets ($L < 100$), this overhead becomes significant.

\textbf{Tail-biting solution:} Initialize the encoder in a \textbf{non-zero state} chosen such that the ending state equals the starting state after encoding $L$ bits.

\textbf{Encoding:}
\begin{enumerate}
\item Pre-compute the required initial state (circular property)
\item Encode $L$ data bits starting from this state
\item Final state automatically equals initial state (no flush needed)
\end{enumerate}

\textbf{Decoding:} Try all $2^{K-1}$ possible starting states, run Viterbi for each, select the path with minimum metric.

\textbf{Benefits:}
\begin{itemize}
\item Zero termination overhead
\item Useful for short packets (control channels)
\end{itemize}

\textbf{Cost:} Decoder complexity increases by factor of $2^{K-1}$

\textbf{Applications:} LTE control channels, WiFi management frames

\subsection{Recursive Systematic Convolutional (RSC) Codes}

Standard convolutional codes are \textbf{non-systematic} (input bits not directly present in output). RSC codes modify this structure.

\textbf{Properties:}
\begin{itemize}
\item \textbf{Systematic:} One output equals the input bit directly
\item \textbf{Recursive:} Encoder includes feedback path
\end{itemize}

\textbf{Structure:}
\begin{center}
\begin{tikzpicture}[
  block/.style={rectangle, draw, minimum width=1.2cm, minimum height=0.8cm},
  node distance=2cm,
  font=\small
]

\node (input) {Input};
\node[circle, draw, right of=input, node distance=2cm] (sum1) {$\bigoplus$};
\node[block, right of=sum1, node distance=2cm] (sr) {Shift Reg};
\node[below of=sr, node distance=1.5cm] (xor) {XOR};
\node[right of=sr, node distance=2.5cm] (sys) {Systematic};
\node[right of=xor, node distance=2.5cm] (par) {Parity};

\draw[->,thick] (input) -- (sum1);
\draw[->,thick] (sum1) -- (sr);
\draw[->,thick] (sr) -- (sys);
\draw[->,thick] (sr) -- (xor);
\draw[->,thick] (xor) -- (par);
\draw[->,thick] (sr.south) -- ++(0,-0.5) -| (sum1.south);

\node[below=2cm of input, align=center, font=\scriptsize] {Feedback creates\\recursive structure};

\end{tikzpicture}
\end{center}

\textbf{Advantage for Turbo codes:} RSC codes have \textbf{infinite impulse response} (IIR) due to feedback. This spreads the effect of each input bit over the entire codeword, improving performance when used in parallel concatenation (Turbo codes).

\textbf{Applications:} LTE Turbo codes, 3GPP standards

\section{Practical Applications}

Convolutional codes with Viterbi decoding have been deployed in countless communication systems since the 1970s. Here we examine several prominent applications.

\subsection{Deep Space Communications: Voyager Spacecraft}

\textbf{System Parameters:}
\begin{itemize}
\item \textbf{Code:} $(171, 133)$, $K=7$, $r=1/2$
\item \textbf{Modulation:} BPSK
\item \textbf{Operating $E_b/N_0$:} $\sim 1$ dB (extremely weak signal at 15+ billion miles)
\item \textbf{BER after Viterbi:} $\sim 5 \times 10^{-3}$
\end{itemize}

\textbf{Concatenated Coding:}
\begin{itemize}
\item \textbf{Inner code:} Convolutional $(171, 133)$
\item \textbf{Outer code:} Reed-Solomon RS(255,223)
\item \textbf{Final BER:} $< 10^{-10}$ (virtually error-free)
\end{itemize}

This concatenated scheme enables reliable communication across interstellar distances where signal power is incredibly weak.

\subsection{WiFi: IEEE 802.11a/g}

\textbf{Convolutional Code:}
\begin{itemize}
\item \textbf{Base code:} $K=7$, $r=1/2$
\item \textbf{Punctured rates:} 1/2, 2/3, 3/4 (adaptive based on channel)
\item \textbf{Combined with:} OFDM (Orthogonal Frequency Division Multiplexing)
\end{itemize}

\textbf{Example: 54 Mbps Mode}
\begin{itemize}
\item 64-QAM modulation (6 bits/symbol)
\item Rate 3/4 convolutional code
\item Effective: $6 \times 3/4 = 4.5$ coded bits/symbol/subcarrier
\item 52 data subcarriers $\times$ 250 ksymbols/s = 54 Mbps
\end{itemize}

\subsection{GPS Navigation Signals}

\textbf{L1 C/A Code:}
\begin{itemize}
\item \textbf{Convolutional code:} $K=7$, $r=1/2$ (NASA-type)
\item \textbf{Navigation message:} 50 bps (low rate for robustness)
\item \textbf{After encoding:} 100 symbols/s
\item \textbf{Modulation:} BPSK
\item \textbf{Spreading:} CDMA at 1.023 Mcps (20.46× processing gain)
\end{itemize}

The combination of convolutional coding and spread spectrum enables GPS to work with signals \textbf{below the noise floor} ($\sim -130$ dBm received power).

\subsection{Satellite Television: DVB-S}

\textbf{Concatenated Coding Architecture:}
\begin{itemize}
\item \textbf{Inner code:} Convolutional $K=7$, $r=1/2$
\item \textbf{Puncturing:} Adaptive rates 2/3, 3/4, 5/6, 7/8 based on link quality
\item \textbf{Outer code:} Reed-Solomon RS(204,188)
\item \textbf{Interleaving:} Convolutional interleaver (handles burst errors)
\end{itemize}

\textbf{Error Handling Strategy:}
\begin{itemize}
\item \textbf{Inner convolutional code:} Corrects random errors from AWGN
\item \textbf{Outer RS code:} Corrects burst errors from rain fade
\item Result: Robust satellite link even during heavy rain
\end{itemize}

\subsection{GSM Mobile Telephony (2G)}

\textbf{Channel Coding:}
\begin{itemize}
\item \textbf{Speech codec:} 260 bits per 20 ms frame
\item \textbf{Convolutional code:} $K=5$, $r=1/2$ (simple for mobile handset)
\item \textbf{Interleaving:} 8-frame diagonal interleaver
\item \textbf{Modulation:} GMSK (Gaussian Minimum Shift Keying)
\end{itemize}

The relatively simple $K=5$ code keeps decoder complexity low for battery-powered handsets while still providing adequate error correction for voice quality.

\section{Implementation Considerations}

\subsection{Computational Complexity}

\textbf{Per time step operations:}
\begin{itemize}
\item \textbf{Branch metric computation:} $2^K$ calculations (one per possible previous state and input combination)
\item \textbf{Add-Compare-Select (ACS):} $2^{K-1}$ operations (one per current state)
\item \textbf{Path storage:} Store survivor path history for each state
\end{itemize}

\textbf{Memory requirements:}
\begin{itemize}
\item Path metrics: $2^{K-1}$ values
\item Survivor paths: $2^{K-1} \times \text{traceback\_depth}$ bits
\item Total memory: Approximately $2^{K-1} \times 5K$ bits
\end{itemize}

\textbf{Example: NASA $K=7$ decoder}
\begin{itemize}
\item States: 64
\item Traceback: $5 \times 7 = 35$ steps
\item Memory: $64 \times 35 = 2240$ bits $\approx$ 280 bytes
\end{itemize}

\subsection{Traceback Depth}

The \textbf{traceback depth} determines how far back the decoder looks when making decisions.

\textbf{Typical range:} $5K$ to $7K$ times the constraint length

\textbf{Examples:}
\begin{itemize}
\item $K=3$: Traceback 15--21 steps
\item $K=7$: Traceback 35--50 steps
\item $K=9$: Traceback 45--63 steps
\end{itemize}

\textbf{Trade-offs:}
\begin{itemize}
\item \textbf{Longer traceback:} Better decoding decisions, higher accuracy
\item \textbf{Shorter traceback:} Lower latency, less memory
\end{itemize}

Rule of thumb: Beyond $5K$ steps, performance improvement is marginal ($< 0.1$ dB).

\subsection{Fixed-Point vs. Floating-Point}

\textbf{Fixed-point implementation:}
\begin{itemize}
\item \textbf{Metric representation:} 6--8 bits sufficient
\item \textbf{Benefits:} Faster computation, lower power consumption
\item \textbf{Performance loss:} Negligible ($< 0.1$ dB)
\item \textbf{Ideal for:} Embedded systems, ASICs, FPGAs
\end{itemize}

\textbf{Quantization effects:}
\begin{itemize}
\item Branch metrics: 4--6 bits (soft inputs)
\item Path metrics: 8--12 bits (accumulated sum)
\item Metric normalization prevents overflow
\end{itemize}

\textbf{Floating-point implementation:}
\begin{itemize}
\item \textbf{Benefits:} Simpler algorithm, no overflow concerns
\item \textbf{Drawbacks:} Higher power, slower (unless GPU/SIMD)
\item \textbf{Ideal for:} Software-defined radio, simulation
\end{itemize}

\section{Advantages and Disadvantages}

\subsection*{Advantages}

\begin{enumerate}
\item \textbf{Optimal ML decoding:} Viterbi algorithm achieves maximum-likelihood decoding efficiently
\item \textbf{Soft-decision natural:} Easily exploits analog demodulator outputs for 2--2.5 dB gain
\item \textbf{Low latency:} Suitable for streaming applications (no need to buffer entire block)
\item \textbf{Excellent AWGN performance:} Well-suited for random error correction
\item \textbf{Flexible rates:} Easy puncturing to achieve variable code rates
\item \textbf{Proven reliability:} Decades of successful deployment in critical systems
\item \textbf{Hardware efficient:} Simple shift register + XOR implementation
\end{enumerate}

\subsection*{Disadvantages}

\begin{enumerate}
\item \textbf{Exponential state growth:} $2^{K-1}$ states limits practical constraint length to $K \leq 9$
\item \textbf{Termination overhead:} Requires $K-1$ flush bits (significant for short packets)
\item \textbf{Burst error vulnerability:} Single long burst can overwhelm decoder
\item \textbf{Suboptimal at high rates:} Performance degrades when heavily punctured ($r > 3/4$)
\item \textbf{Surpassed by modern codes:} Turbo codes and LDPC codes achieve better performance
\item \textbf{Fixed structure:} Less flexible than modern graph-based codes
\end{enumerate}

\section{Comparison: Block Codes vs. Convolutional Codes}

\begin{center}
\begin{tabular}{@{}p{3cm}p{4.5cm}p{4.5cm}@{}}
\toprule
\textbf{Property} & \textbf{Block Codes} & \textbf{Convolutional Codes} \\
\midrule
\textbf{Encoding} & Fixed-length blocks $(n,k)$ & Continuous stream \\
\textbf{Memory} & None (memoryless) & Yes (shift register) \\
\textbf{Structure} & Matrix multiplication & State machine \\
\textbf{Decoding} & Algebraic (syndrome) & Viterbi (trellis search) \\
\textbf{Complexity} & $O(n^2)$ to $O(n^3)$ & $O(2^K \cdot L)$ \\
\textbf{Latency} & Block delay & Traceback depth ($\sim 5K$) \\
\textbf{Soft-decision} & Possible but complex & Natural in Viterbi \\
\textbf{Best use} & Burst errors (RS) & Random errors (AWGN) \\
\textbf{Rate flexibility} & Fixed $(n,k)$ & Easy puncturing \\
\textbf{Examples} & Hamming, BCH, RS & NASA $(171,133)$ \\
\bottomrule
\end{tabular}
\end{center}

\begin{keyconcept}
\textbf{Concatenated coding} combines the strengths of both:
\begin{itemize}
\item \textbf{Inner code:} Convolutional (corrects random AWGN errors)
\item \textbf{Outer code:} Reed-Solomon (corrects burst errors)
\item \textbf{Result:} Robust against both error types (used in Voyager, DVB-S, etc.)
\end{itemize}

This concatenation approach dominated space and satellite communications for decades before Turbo and LDPC codes emerged.
\end{keyconcept}

\subsection{Design Guidelines}\label{design-guidelines}

\textbf{Choose K}: - \textbf{K=3-5}: Low complexity, embedded systems -
\textbf{K=7}: Standard (NASA, WiFi), good performance - \textbf{K=9}:
Better performance, higher complexity

\textbf{Choose rate}: - \textbf{1/2}: Strong coding (deep space) -
\textbf{1/3}: Very strong (low SNR) - \textbf{2/3, 3/4}: High throughput
(punctured)

\textbf{Soft-decision}: Always use if demodulator provides LLRs (+2 dB
free gain!)

\begin{center}\rule{0.5\linewidth}{0.5pt}\end{center}

\section{Summary}

\begin{center}
\begin{tabular}{@{}ll@{}}
\toprule
\textbf{Parameter} & \textbf{Value/Description} \\
\midrule
\textbf{Encoding type} & Continuous (not block-based) \\
\textbf{Memory} & $K-1$ shift register stages \\
\textbf{Number of states} & $2^{K-1}$ \\
\textbf{Common rates} & 1/2, 1/3, 2/3, 3/4 (punctured) \\
\textbf{Decoding} & Viterbi algorithm (optimal ML) \\
\textbf{Complexity} & $O(2^{K-1} \cdot L)$ \\
\textbf{Free distance} & Determines error correction capability \\
\textbf{Coding gain} & $10\log_{10}(r \cdot d_{\text{free}})$ dB \\
\textbf{Soft-decision gain} & $\sim$2--2.5 dB over hard-decision \\
\textbf{NASA standard} & $(171, 133)$, $K=7$, $d_{\text{free}}=10$ \\
\textbf{Typical applications} & Satellite, deep-space, WiFi, GPS \\
\textbf{Best for} & Random errors (AWGN channels) \\
\textbf{Limitation} & Exponential state growth, termination overhead \\
\bottomrule
\end{tabular}
\end{center}

\textbf{Key Insights:}
\begin{itemize}
\item \textbf{Memory is fundamental:} Output depends on current + past $K-1$ inputs
\item \textbf{Viterbi is optimal:} Achieves ML decoding efficiently via dynamic programming
\item \textbf{Free distance determines performance:} Higher $d_{\text{free}}$ $\rightarrow$ better correction
\item \textbf{Soft-decision is crucial:} Utilizing analog values provides significant gain
\item \textbf{Puncturing enables flexibility:} Easy to adapt code rate to channel conditions
\item \textbf{Trade-off exists:} Larger $K$ gives better performance but higher complexity
\end{itemize}

\section{Further Reading}

\textbf{Prerequisites:}
\begin{itemize}
\item \textbf{Chapter 5:} Binary Phase-Shift Keying (BPSK)---modulation for coded systems
\item \textbf{Chapter 8:} Bit Error Rate (BER)---performance metrics
\item \textbf{Chapter 9:} Signal-to-Noise Ratio (SNR)---link budget fundamentals
\item \textbf{Chapter 11:} Additive White Gaussian Noise (AWGN)---channel model
\end{itemize}

\textbf{Related error correction:}
\begin{itemize}
\item \textbf{Chapter 21:} Forward Error Correction (FEC)---general overview
\item \textbf{Chapter 22:} Block Codes (Hamming, BCH, Reed-Solomon)---alternative approach
\item \textbf{Chapter 24:} Turbo Codes---parallel concatenated convolutional codes
\item \textbf{Chapter 25:} LDPC Codes---modern capacity-approaching codes
\item \textbf{Chapter 23:} Hamming Distance \& Error Detection---fundamental concepts
\end{itemize}

\textbf{System integration:}
\begin{itemize}
\item \textbf{Chapter 15:} Complete Link Budget Analysis---incorporating coding gain
\item \textbf{Chapter 17:} Shannon's Channel Capacity---fundamental limits
\item \textbf{Chapter 19:} Spectral Efficiency---throughput with coding
\end{itemize}

\textbf{Advanced topics:}
\begin{itemize}
\item \textbf{Chapter 28:} OFDM \& Multicarrier Modulation---used with convolutional codes in WiFi
\item \textbf{Chapter 30:} Adaptive Modulation \& Coding---dynamic rate selection
\item \textbf{Chapter 31:} Synchronization---carrier and timing recovery for coded systems
\end{itemize}
